\chapter{行列式}
\lectureinfo{2015年9月16日 1限}

\section{行列式とは}

まず「行列式とは何か」について、簡単に説明しておきましょう。

一般に$n$次正方行列$A$に対し、その行列式$\det A$という数が定義されます。列ベクトルを並べて$A = (\bm{a}_1 \ \cdots \ \bm{a}_n)$と表せば、$n$本のベクトル$(\bm{a}_1, \ldots, \bm{a}_n)$から行列式が定まると考えてもよいです。この行列式は「$\det A$の計算で、$A$が正則かどうか判定できる」という大変すごいご利益を持っています。この事実から、$\det A = 0$かどうかで
\begin{itemize}
\item $(\bm{a}_1, \ldots, \bm{a}_n)$が$1$次独立かどうか
\item 連立$1$次方程式$A\bm{x} = \bm{b}$の解がただ一つに決まるかどうか
\end{itemize}
の判定ができます。また行列式を使うと、連立$1$次方程式$A\bm{x} = \bm{b}$の解を書き下す、Cramerの公式と呼ばれる式が作れます。

\section{$2$次と$3$次の行列式}

さて、$n$次正方行列$A = (a_{ij})_{1\leq i, j\leq n}$の行列式の定義は、一応
\[
\det A = \sum_{\sigma \in \mathfrak{S}_n} \sgn(\sigma) a_{1 \sigma(1)} a_{2 \sigma(2)} \cdots a_{n \sigma(n)}
\]
という式で与えられます。最終的にはこの式を理解して、$n$次の行列式の計算ができるようにならないといけません。ですがこの式には初見の記号が出てきており、いきなり理解するのは大変です。なのでこの式は一旦脇に置いといて、まずは$n = 2, 3$の簡単な場合だけ調べてみましょう。

なお$n = 2, 3$の場合だけを先に扱うのは、決して単なる「逃げ」ではありません。実務上、$2$次や$3$次の行列式の計算は非常にたくさん出てきます。というのも行列式には
\begin{itemize}
\item $n$次の行列式の計算は、$(n - 1)$次の行列式の計算に帰着できる
\item 行列式の計算の手間は大体$n!$に比例してややこしくなるので、高次の行列式をそのまま定義通りに計算することはまずない\footnote{これは手計算に限った話ではありません。コンピュータであっても、定義通りに計算できるのはせいぜい$10$次くらいまでだと思います。また「同価格帯のコンピュータは$1$年半で性能が大体$2$倍になる」というMooreの法則によれば、コンピュータは時間を追うごとに指数函数的に性能が上がります。かたや行列式はサイズが大きくなると階乗の勢いで計算の手間が増えるので、どんな明るい未来がやってきても、今の汎用コンピュータ (より正確には、Turing機械と呼ばれるタイプに属する機械) が行列式を定義通りに計算できる見込みはありません。基本変形と組み合わせたり近似公式を使ったりして、なんとか計算しているのが実情です。}
\end{itemize}
という性質があります。高次の行列式を計算する場合であっても、実際には$2$次や$3$次の行列式をたくさん計算することになるのです。ですから上に書いた行列式の定義式を理解するのとは別個に、$2$次、$3$次の行列式は暗記して、すらすら使えるようになっておく必要があります。

\subsection{$2$次の行列式}

$2$次の行列式は
\[
\det
\begin{pmatrix}
a & b \\
c & d \\
\end{pmatrix}
:= ad-bc
\]
で定義されます。既に知っている人も、そこそこいるかと思います。知らなかった人は、この式を今すぐ暗記して、脊髄反射で書けるようにしておいてください。

\paragraph{面積との関係}

\subsection{$3$次の行列式}

$3$次の行列式も、まず定義式を書いてみます。
\[
\det
\begin{pmatrix}
a & b & c \\
d & e & f \\
g & h & i
\end{pmatrix}
:=  aei + bfg + chd - ceg - bdi - ahf
\]
この式には$2$つの覚え方があります。$1$つ目は\textbf{Sarrusの方法}と呼ばれるものです。次の図のように、行列式を取る行列と曲がった矢印を重ねます。そして最初右下に向かう矢印はそのまま、左下に向かう矢印にはマイナスの符号をつけて、全て足すのです。

% Sarrus

もう$1$つの覚え方は、\textbf{余因子展開}というものです。さっきの行列式をよく見ると
\begin{align*}
\det
\begin{pmatrix}
a & b & c \\
d & e & f \\
g & h & i
\end{pmatrix}
&=  aei + bfg + chd - ceg - bdi - ahf
= a(ei - hf) - b(di - fg) + c(hd - eg) \\
&=
a
\begin{pmatrix}
e & f \\
h & i
\end{pmatrix}
- b
\begin{pmatrix}
d & f \\
g & i
\end{pmatrix}
+ c
\begin{pmatrix}
d & e \\
g & h
\end{pmatrix}
\end{align*}
となっています。ルールが分かるような書き方をすると、こうなります:
\begin{itemize}
\item $1$行目と$1$列目を取り去ってできる小行列の行列式を、$(1, 1)$成分の$a$にかける
\item $1$行目と$2$列目を取り去ってできる小行列の行列式を、$(1, 2)$成分の$b$にかける
\item $1$行目と$3$列目を取り去ってできる小行列の行列式を、$(1, 3)$成分の$c$にかける
\end{itemize}
ということをやって、$\pm$の符号を交互にして足しているのです。要となる成分が$1$行目を左から順に動くので、これを\textbf{$1$行目に関する余因子展開}といいます。

今と同じことを、列に関しても考えられます。行列式の定義式は
\begin{align*}
\det
\begin{pmatrix}
a & b & c \\
d & e & f \\
g & h & i
\end{pmatrix}
&=  aei + bfg + chd - ceg - bdi - ahf
= a(ei - hf) - d(bi - ch) + g(bf - ce) \\
&=
a
\begin{pmatrix}
e & f \\
h & i
\end{pmatrix}
- d
\begin{pmatrix}
b & c \\
h & i
\end{pmatrix}
+ g
\begin{pmatrix}
b & c \\
e & f
\end{pmatrix}
\end{align*}
と書き直せます。この式は
\begin{itemize}
\item $1$行目と$1$列目を取り去ってできる小行列の行列式を、$(1, 1)$成分の$a$にかける
\item $2$行目と$1$列目を取り去ってできる小行列の行列式を、$(2, 1)$成分の$b$にかける
\item $3$行目と$1$列目を取り去ってできる小行列の行列式を、$(3, 1)$成分の$c$にかける
\end{itemize}
ということをやって、出てくる数に$\pm$を交互につけて足しています。こっちは要となる成分が$1$列目を縦に動くので、\textbf{$1$列目に関する余因子展開}といいます。同じようにして$i$行目あるいは$j$列目に関する余因子展開をすることもできますが、まずは$1$行目 / $1$列目での余因子展開を確実にできるようになりましょう。これさえできれば、行 / 列を動かしてもやることは同じです。

\subsection{高次の行列式}

$4$次より大きい行列式に対しても、やることは同じです。高次の行列式は、余因子展開で計算することができます。たとえば$4$次の行列式の場合
\[
&\det
\begin{pmatrix}
a_{11} & a_{12} & a_{13} & a_{14} \\
a_{21} & a_{22} & a_{23} & a_{24} \\
a_{31} & a_{32} & a_{33} & a_{34} \\
a_{41} & a_{42} & a_{43} & a_{44}
\end{pmatrix} \\
&= a_{11} \det
\begin{pmatrix}
a_{22} & a_{23} & a_{24} \\
a_{32} & a_{33} & a_{34} \\
a_{42} & a_{43} & a_{44}
\end{pmatrix}
- a_{12} \det
\begin{pmatrix}
a_{21} & a_{23} & a_{24} \\
a_{31} & a_{33} & a_{34} \\
a_{41} & a_{43} & a_{44}
\end{pmatrix}
+ a_{13} \det
\begin{pmatrix}
a_{21} & a_{22} & a_{24} \\
a_{31} & a_{32} & a_{34} \\
a_{41} & a_{42} & a_{44}
\end{pmatrix}
- a_{14} \det
\begin{pmatrix}
a_{21} & a_{22} & a_{23} \\
a_{31} & a_{32} & a_{33} \\
a_{41} & a_{42} & a_{43}
\end{pmatrix}
\]
という式が成り立ちます。一般に$n$次の行列式は余因子展開によって$n - 1$次の行列式の計算に帰着できます。これを使えば、再帰的に行列式の計算を$2$次まで落としこめるというわけです。

\section{置換群}

\subsection{置換の定義}

\subsection{置換の符号}

