\chapter{線型写像と行列}
\lectureinfo{2015年7月8日 1限}

\section{授業の締めくくりに向けて}

レポートの解説プリントは次回分・次々回分が残っていますが、授業自体は今回が最後です。そこで一度、ここまでに勉強したことをまとめておきましょう。

\subsection{簡単なまとめ}

\paragraph{線型代数の二本柱}

だいたいS1タームの最後$2$回くらいからずっと、授業では「線型代数」と呼ばれる内容を扱ってきました。その内容は、ざっくりと
\begin{itemize}
\item 連立$1$次方程式の解法と行列の計算
\item 線型空間と線型写像の理論
\end{itemize}
の$2$つに分かれます。で、これらの理論と計算はもちろん無関係ではありません。線型空間や線型写像の理論は、行列計算を抽象化したものです。たとえば線型空間の理論の中で出てきた諸々の概念は、全て行列とか連立$1$次方程式の話の中に対応物があります。抽象化する$1$つの理由は、適用範囲を広げるためです。
\begin{table}[h!tbp]
\centering
\begin{tabular}{ll} \hline
\textbf{抽象的な理論での概念} & \textbf{行列の話で対応するもの} \\ \hline
線型空間 & 数ベクトル空間$\mathbb{R}^n$ \\
線型空間の次元 & $\mathbb{R}^n$の$n$ \\
線型写像 & 行列 \\
線型写像$A$の核 & $A\bm{x} = \bm{0}$の解集合 \\
$A$の像の次元 & 行列$A$の階数 \\
$A$の核の次元 & $A\bm{x} = \bm{b}$の解のパラメータの個数 \\ \hline
\end{tabular}
\end{table}

\paragraph{理論と計算の関わり}

線型代数の理論の中に「基底」というものがありました。これは非常に強力なツールです。というのも、基底を取るということは、線型空間に座標を入れることそのものでした。ですから基底を取ってしまえば、線型空間は数ベクトル空間$\mathbb{R}^n$と同一視できます。そして線型写像$f\colon U\rightarrow V$の定義域$U$と値域$V$のそれぞれで基底を取ってしまえば、$f$は数ベクトル空間の間の線型写像と同一視できます。これは行列そのものです。

このように「基底」を使えば、線型空間や線型写像の話は、出自とは無関係に数ベクトル空間や行列の話に落とし込めます。世の中には行列とは無関係に線型写像が飛び出てくることがままあります。たとえば線型微分方程式を解くときや、特定の漸化式を満たす数列を求めるときなどです。でもそれらのややこしい話は、基底を使って行列の話にすり替えてしまう\footnote{行列の話が使えるのは、基底を取った後です。基底を取るところまでは、個々の問題に応じて別途頑張る必要があります。}ことで、連立$1$次方程式の話に帰着させられます。こうすれば、問題が解けますね。線型代数は、実際に生じる問題を解くために非常に有用なツールなのです。こうして、線型代数をよく理解しておくと
\begin{itemize}
\item 抽象論を使って、色々な問題に適応できるようにしておいて
\item その後で具体的な行列の問題に直し、問題を解く
\end{itemize}
という方法で、色々な問題を解けるようになります。

また、線型代数の抽象論をよく理解していると、具体的な問題にも取り組みやすくなります。たとえば今回、再び連立$1$次方程式の問題に取り組みます。でも前回と全く同じ扱いでは面白くないですから、今回は次元の理論を使い、いかに手際よく解を求めるかという話を中心にします。具体的な計算をする上でも、抽象的な理論は計算の見通しを綺麗にする、非常に便利な道具なのです。このことも、抽象論を展開する一つのモチベーションです。

具体的な計算と抽象的な理論の両方に習熟して、線型代数を使いこなせるようになってください。

% 「行列の基本変形で変わらないもの」とは何か?

\subsection{なぜ線型代数を学ぶのか?}

さて、ここまで「線型代数が数学の問題を解く時に便利だ」という話をしてきましたが、皆さんの中には「本当に線型代数使う場面なんてあるんかいな?」と思っている人がいるかもしれません。そこで線型代数が現実に必要そうな場面を、$2$つほど紹介します\footnote{なお、ここまでの授業では扱われていない中で、非常に重要なテーマが$2$つあります。それは正方行列の「行列式」と「固有値」です。これらを知っていると、線型代数の重要性が一段と身に染みて理解できるようになります。}。

\paragraph{統計}

おそらく、ほとんど全ての人が今後何らかの「データ」を扱うことになるでしょう。それは実験データかもしれませんし、社会調査の結果かもしれませんが、何にせよデータを基にした推論が必要になります。その際変数が複数個あると、しばしば行列計算が必要になります。たとえば主成分分析や重回帰分析といった手法を使うと、そもそも問題の記述に行列が不可欠になります。そして統計の問題では、最終的には計算を行わなければいけないものの、抽象的な線型代数の議論を援用することが多々あります。

\paragraph{微分方程式}

もう$1$つの重要な例は、微分方程式です。

何を勉強するにしても、大抵の場合は微分方程式と格闘する必要性に迫られます。たとえば物理を勉強すると、基本法則の多くが微分方程式で書かれているはずです。化学をするなら、電子の軌道を支配するSchr\"odinger方程式\index{Schr\"odingerほうていしき@Schr\"odinger方程式}という微分方程式のことを知らないといけません。また化学反応の速度は、反応機構に応じた微分方程式に従います。感染症の流行モデルを記述する手法の$1$つにSIRモデル\index{SIRもでる@SIRモデル}というものがありますが、これも微分方程式で記述されます。その他色々例があると思いますが、何をするにしても、とにかく微分方程式は必要なのです。そして線型微分方程式が相手なら、線型代数を使うことができます。この辺の事情は、解空間のことを過去にちょこっとだけ紹介しました。

ただ世の中の微分方程式は、解けることの方が稀です。大体の場合は解けない微分方程式が出てくるので、近似で頑張る必要などが出てきます。その際に良く使う手法の一つが「線型化」と呼ばれるものです。これは「平衡状態」に対応する微分方程式の解があるとき、「平衡状態からのズレが従う微分方程式」を調べるというものです。こうするとズレの主要部分だけに着目した、線型微分方程式が得られます。こちらを解けば、元の方程式が完全に解けなくてもある程度の情報が得られます。たとえば「物をちょっと動かしたときに元の場所に戻るかどうか」とかくらいなら、確かめられます。だから線型微分方程式は、そのまま解ける場合でも近似方程式として出てくる場合でも、重要なのです。

こういう感じで、線型代数というのは非常に重要です。実際の問題に出くわしてみないとあまり実感が分からないかもしれませんが、後々使うということだけは、知っておいてください。

\section{いくつかの便利な補題}

問題を解く前に、脇道に逸れますが便利な話を紹介します。

\subsection{$1$次独立性の判定法}

$1$次独立性には便利な判定法があるので紹介します\footnote{今までも暗黙のうちに使っていますし、みなさんもきっと当然に感じていることなので、今更言う必要がないかもしれませんが……}。それは「一部の成分だけを抜き出して$1$次独立だったら、元のベクトルの組も$1$次独立」ということです。たとえば
\[
\bm{v}_1 :=
\begin{pmatrix}
1 \\
0 \\
1
\end{pmatrix}, \quad
\bm{v}_2 :=
\begin{pmatrix}
0 \\
1 \\
1
\end{pmatrix}
\]
というベクトルを考えます。このとき$\bm{v}_1$の$1, 2$行目は${}^t(1, 0)$、$\bm{v}_2$の$1, 2$行目は${}^t (0, 1)$なので、ここを見るだけで直ちに「$\alpha_1 \bm{v}_1 + \alpha \bm{v}_2 = \bm{0}$ならば$\alpha_1 = \alpha_2 = 0$」が分かります。今は一番分かり易い形を例として示しましたが、ベクトルがもう少しややこしい形でも、あるいは本数が増えても、$1$次独立性を示すには「一部の成分さえ見て示せれば十分」というわけです。たとえば$(2, 2)$行列の場合、$2$本の列ベクトルが$1$次独立なことと行列式が$0$でないことは同値です。こういうのを組み合わせると、ラクに$1$次独立性が示せます。

証明はもちろん愚直にやってもできますが、ここでは「射影」\index{しゃえい@(ベクトルの) 射影}というテクニックを使って示してみます。正整数の増加列$1\leq j_1 < j_2 < \cdots < j_p \leq n$に対し、$(p, n)$型行列$\Proj_{j_1, j_2, \ldots, j_p}\in\Mat_{p, n}(\mathbb{R})$を
\begin{align*}
\Proj_{j_1, j_2, \ldots, j_p} :=
\bordermatrix{
 & & \scalebox{0.8}{${j_1}^{\text{th}}$} &  & \scalebox{0.8}{${j_2}^{\text{th}}$} &  & \scalebox{0.8}{${j_p}^{\text{th}}$} \cr
 & & 1 &  &  \cr
 & &  &  & 1 \cr
 & & & & & \ddots \cr
 & & & & & & 1 &
}
\end{align*}
で定めます。つまり各$1\leq i \leq p$に対し「$i$行目は$j_i$列目だけ$1$で、他は$0$」という条件で定まる行列が$\Proj_{j_1, j_2, \ldots, j_p}$です。これを$\mathbb{R}^n$のベクトルに当てると、$j_1, j_2, \ldots, j_p$番目の成分だけをちょうど抜き出せます。たとえば
\[
\Proj_{2, 3, 5}
\begin{pmatrix}
x_1 \\
x_2 \\
x_3 \\
x_4 \\
x_5
\end{pmatrix}
=
\begin{pmatrix}
0 & 1 & 0 & 0 & 0 \\
0 & 0 & 1 & 0 & 0 \\
0 & 0 & 0 & 0 & 1
\end{pmatrix}
\begin{pmatrix}
x_1 \\
x_2 \\
x_3 \\
x_4 \\
x_5
\end{pmatrix}
=
\begin{pmatrix}
x_2 \\
x_3 \\
x_5
\end{pmatrix}
\]
となっていますね。

これを使えば、証明はすぐ終わります。前回の問$3$で、線型写像$\varphi$に対し「$\varphi(\bm{f}_1), \ldots, \varphi(\bm{f}_n)$が$1$次独立なら、$\bm{f}_1, \ldots, \bm{f}_n$も$1$次独立である」という事実を示しています。これを$\Proj_{j_1, j_2, \ldots, j_p}$に適用するだけです。

\subsection{正則行列と$1$次独立性/生成系}

逆行列を持つ正方行列のことを正則行列と呼ぶのでした。行列に左右から正方行列をかけたとき、核や像がどう変わるかを見てみます。$A$を$(m, n)$型行列とします。僕たちはこれから、$m$次の正則行列$X$と$n$次の正則行列$Y$に対し、$XAY$の核や像が$A$の核や像と比べてどうズレるかを調べます。特に念頭にあるのは、$X$や$Y$が基本行列の場合です。つまり「基本変形で核や像がどうズレるか」を調べます。

証明に入る前に、この問題を「線型写像」の立場からざっくり眺めてみましょう。行列$A$はベクトル$\bm{x} \in \mathbb{R}^n$に$A\bm{x} \in \mathbb{R}^m$を対応させる写像なので、$A\colon \mathbb{R}^n \rightarrow \mathbb{R}^m$と表せます。同じようにして$m$次正方行列$X$は$X\colon \mathbb{R}^m\rightarrow\mathbb{R}^m$という線型写像を、$n$次正方行列$Y$は$Y\colon \mathbb{R}^n\rightarrow\mathbb{R}^n$という写像を定めます。こう思えば、$\bm{x}$に$XAY\bm{x}$を対応させる写像は$\bm{x}\mapsto Y\bm{x} \mapsto AY\bm{x} \mapsto XAY\bm{x}$という合成写像だと分かります。これを図式で書くと
\[
\begin{tikzcd}
\mathbb{R}^n \arrow{r}{Y} & \mathbb{R}^n \arrow{r}{A} & \mathbb{R}^m \arrow{r}{X} & \mathbb{R}^m
\end{tikzcd}
\]
となります。元々の写像$A\colon \mathbb{R}^n\rightarrow\mathbb{R}^m$に、写像$Y$を右から、写像$X$を左からかませているわけです\footnote{写像の合成を書く順序は、作用する順序と逆向きなので気を付けましょう。写像$XAY$は、ベクトルに「$Y$, $A$, $X$の順」で作用します。}。 %座標の取り換え

そして$X$や$Y$は正則行列なので、全単射を定めています。特に$Y$は全射なので、$\bm{x}$が$\mathbb{R}^n$の全ての元を動き回るとき、$Y\bm{x}$も$\mathbb{R}^n$の全ての元を動きます。こう思えば、$\Im AY$と$\Im A$は同じになりそうですね。また$X$は単射なので、$X$が作用して$\bm{0}$になるベクトルは$\bm{0}$以外にありません。だから$\Ker XA$と$\Ker A$は同じになりそうです。

また$X$が全射だとはいえ、一般には$\Im A$のベクトルたちを$X$で移して得られる$\Im XA$は、当然元の$\Im XA$とは異なります。でも$X$で潰れるベクトルがないので、$\dim \Im XA = \dim \Im A$は期待してよさそうです。また$\Ker A$のベクトルを$Y^{-1}$で動かすと$\Ker AY$になるので、一般には$\Ker AY \neq \Ker A$です。でも$Y$で潰れるベクトルはないので、$\dim \Ker AY = \dim \Ker A$も期待してよさそうです。

こんなイメージを頭の中に置きながら、ちゃんと証明をしてみましょう。

\paragraph{左からの正則行列の作用}

まず$X$が$m$次の正則行列のとき、$\Ker XA = \Ker A$を示します。実際$A\bm{x} = \bm{0}$ならば$XA\bm{x} = \bm{0}$は当たり前なので$\Ker A\subset \Ker XA$です。そして$X^{-1}$が存在するので、$XA\bm{x} = \bm{0}$が成り立てば両辺に左から$X^{-1}$をかけて、$A\bm{x} = X^{-1}\bm{0} = \bm{0}$を得ます。よって$\Ker XA \subset \Ker A$です。こうして$\Ker XA = \Ker A$が分かりました。

% X は単射 だけで OK

\paragraph{右からの正則行列の作用}

次に$Y$が$n$次の正則行列のとき、$\Im AY = \Im A$を示します。勝手に$\bm{y} \in \Im AY$を取ると$\bm{y} = AY\bm{x}$なる$\bm{x} \in \mathbb{R}^n$が存在します。この式を$\bm{y} = A(Y\bm{x})$と読めば、$\bm{y} \in \Im A$が従います。故に$\Im AY \subset \Im A$です。逆に$\bm{y} \in \Im A$とすると、$\bm{y} = A\bm{x}$となる$\bm{x} \in  \mathbb{R}^n$が取れます。そうすると$Y^{-1}$の存在から$\bm{y} = AY(Y^{-1}\bm{x})$と書けるので$\bm{y} \in \Im AY$です。よって$\Im A \subset \Im AY$です。これで$\Im A = \Im AY$が示せました。

% Y は全射 だけで OK

\paragraph{次元について} 今の状況で、$\Ker AY = \Ker A$や$\Im XA = \Im A$は成り立ちません。ですが$\dim \Ker AY = \dim \Ker A$や$\dim \Im XA = \dim \Im A$は成り立ちます。さらにまとめると$X, Y$が正則行列のとき、$\dim \Ker XAY = \dim \Ker A$, $\dim \Im XAY = \dim \Im A$です。なぜなら$\dim \Ker XAY = \dim \Ker AY = n - \dim \Im AY = n - \dim A = \dim \Ker A$です。これより$\dim \Im XAY = n - \dim \Ker XAY = n - \dim \Ker A = \dim \Im A$となります。

\subsection{基本変形で変わらないもの}

今の話を基本行列に当てはめてみましょう。基本行列には$3$つの系列がありましたが、どれも正則なことに変わりありません。そして基本行列を左からかけると行基本変形に、右からかけると列基本変形\footnote{列基本変形はまだちゃんと扱っていませんが、次の節で$1$つ例を出します。}になるのでした。ですから
\begin{itemize}
\item $A$を行基本変形しても、$\Ker$は変化しない
\item $A$を列基本変形しても、$\Im$は変化しない
\end{itemize}
と分かります。「行基本変形で$\Ker$は変化しない」というのは単に「行基本変形による掃き出しで方程式が解ける」というだけですから、既に知っていたことです。でも列基本変形で$\Im$が求まるというのは、新しい情報です\footnote{問題2のヒントに「与えられた列ベクトルを転置で行ベクトルにして、それを並べた行列を行基本変形すれば答えが求まる」と書いてあるのは、今述べた原理によるものです。${}^t(AX) = {}^t\! X\, {}^t\! A$なので「$A$に列基本変形をすること」は「$A$の転置に行基本変形をすること」と同じです。行基本変形だけで話を済ませるために、転置を取ったのだと思います。

ですが個人的には、列基本変形を使った方が「像が保たれる」という事実が見やすくて良いように思いました。この\textbf{問2は間違える人が非常に多かった}ですので、計算の原理を今一度復習しておいてください。}。

また$\Ker A$や$\Im A$を完全に求めることはできなくても、次元の情報を組み合わせて上手く解を作れることもあります。たとえば一般に$\Im XA \neq \Im A$なので、$A$を行基本変形したら$\Im A$は分からなくなってしまいます。でも$\dim XA = \dim A$までは分かるので、方程式$A\bm{x} = \bm{0}$を解いて$\Ker A$を求めれば、次元定理$\dim \Im A = n - \dim \Ker A$から$\Im A$の基底の本数が分かります。あとはたとえば$A$の見た目を利用するとかして、$A$の列ベクトルの中から$\dim \Im A$本の$1$次独立なベクトルを持ってくれば、それで$\Im A$の基底が得られたことになります。

こんな風に「次元の情報」を使うことで、$\Ker A$や$\Im A$の基底が何本あるかを知ることができます。だから「$\Ker A$と$\Im A$を両方求める」という問題の場合、大体の場合「行基本変形で$\Ker A$を求め、その後列基本変形で$\Im A$を求める」という二度手間をする必要はあまりありません。さらに言えば、行基本変形と列基本変形を組み合わせても良いのです。どちらを併用しても$\Ker$と$\Im$の次元は保たれますから、単に基底の本数を求めるだけなら、行/列のやりやすい基本変形を両方使って大丈夫です。たとえば「行基本変形をしてたら、$2$つ同じ列ベクトルが出てきた」なんて場合は、列基本変形で重複する列ベクトルを消してから、行基本変形を進めたりできます。

どういう手順で計算を進めるのが一番良いかは、問題によってまちまちですし、一発で最善策を見抜くことは難しいでしょう。でも最善策にたどり着けないとしても、手持ちの手法を上手く組み合わせることで、行き当たりばったりよりは大分マシな解き方ができるはずです。

\paragraph{問5の解答}

行列の階数 ($\Im$の次元) や$\Ker$の次元は基本変形で変わらない。

\subsection{列基本変形の定義と例} \label{subsection:column_elementary_operation}

それでは早速、列基本変形を使ってみましょう。行基本変形を定義する文章中で、「行」を「列」に置き換えると、そのまま列基本変形\index{れつきほんへんけい@列基本変形}の定義になります。すなわち
\begin{itemize}
\item $2$つの列を入れ替えること
\item ある列に別の列の定数倍を加えること
\item ある列に$0$でない定数をかけること
\end{itemize}
が、列基本変形です。そして行基本変形が「基本行列の左からの積」で実現できたのと同様に、列基本変形は「基本行列の右からの積」で実現できます。

この辺の事情は、実際に手を動かして理解する方がよいでしょう。「列基本変形が像を保つ」という事実を使い、問2を列基本変形で解いてみます。計算を追いかけてみてください。繰り返し言いますが、\textbf{問2は間違えた人が非常に多い}です。必ず復習してください。

\paragraph{問2の解答}
\[
A := 
\begin{pmatrix}
\bm{a}_1 & \bm{a}_2 & \bm{a}_3
\end{pmatrix}
=
\Biggl(
\begin{array}{rrr}
-1 & 3 & 5 \\
1 & -1 & 1 \\
2 & 0 & 8
\end{array}
\Biggr)
\]
とおくと$W = \Im A$である。そこで
\begin{align*}
\Biggl(
\begin{array}{rrr}
-1 & 3 & 5 \\
1 & -1 & 1 \\
2 & 0 & 8
\end{array}
\Biggr)
\xrightarrow[\text{$2$, $3$列目の掃き出し}]{\text{$(1,1)$成分を要に}}
\Biggl(
\begin{array}{rrr}
-1 & 0 & 0 \\
1 & 2 & 6 \\
2 & 6 & 18
\end{array}
\Biggr)
&= 
\Biggl(
\begin{array}{rrr}
-1 & 3 & 5 \\
1 & -1 & 1 \\
2 & 0 & 8
\end{array}
\Biggr)
\Biggl(
\begin{array}{rrr}
1 & 3 & 5 \\
0 & 1 & 0 \\
0 & 0 & 1
\end{array}
\Biggr) \\
\xrightarrow[\text{\hbox to 9.5zw{\hfil $2$列目を$1/2$倍\hfil}}]{\text{\hbox to 9.5zw{\hfil $1$列目を$(-1)$倍\hfil}}}
\Biggl(
\begin{array}{rrr}
1 & 0 & 0 \\
-1 & 1 & 6 \\
-2 & 3 & 18
\end{array}
\Biggr)
&=
\Biggl(
\begin{array}{rrr}
-1 & 0 & 0 \\
1 & 2 & 6 \\
2 & 6 & 18
\end{array}
\Biggr)
\Biggl(
\begin{array}{rrr}
-1 & 0 & 0 \\
0 & \frac{1}{2} & 0 \\
0 & 0 & 1
\end{array}
\Biggr) \\
\xrightarrow[\text{\hbox to 11.5zw{\hfil $1$, $3$列目を掃き出し\hfil}}]{\text{\hbox to 11.5zw{\hfil $(2, 2)$成分を要に\hfil}}}
\Biggl(
\begin{array}{rrr}
1 & 0 & 0 \\
0 & 1 & 0 \\
1 & 3 & 0
\end{array}
\Biggr)
&=
\Biggl(
\begin{array}{rrr}
1 & 0 & 0 \\
-1 & 1 & 6 \\
-2 & 3 & 18
\end{array}
\Biggr)
\Biggl(
\begin{array}{rrr}
1 & 0 & 0 \\
1 & 1 & -6 \\
0 & 0 & 1
\end{array}
\Biggr)
\end{align*}
となる。これで$W = \Im A$の基底が${}^t(1, 0, 1)$と${}^t(0, 1, 3)$だと分かる。また、これらのベクトルの$\bm{0}$でない$1$次結合は、必ず第$1$成分または第$2$成分のいずれかが$0$でない。よって${}^t(0, 0, 1)$を加えると、$\mathbb{R}^3$の基底になる。 \qed

\section{連立$1$次方程式再訪}

% 行列のランクと像の次元
% 行基本変形が Ker を保つこと
% 列基本変形が Im を保つこと

さて今回の問題の多くは、再び連立$1$次方程式です。「またかよ」と思った人がいるかもしれませんが、前とは一つ違うことがあります。それは「線型空間の次元」という道具があることです。前回は掃き出し法の様々な使い方を知ることがテーマでしたが、今度は次元定理\footnote{$A$が$(m, n)$型行列のとき、$n = \dim \Ker A + \dim \Im A$が成り立つというやつです。証明は前回のプリントの\pageref{paragraph:dimension_theorem}ページに書いてありますが、まずは式を認めて使ってみるというのでも良いでしょう。とにかく使い道のある公式なので、使い方は絶対に覚えてください。}という強力な道具が加わるので、これを使っていっそうサクサクと解を求めにいくことができます。

\paragraph{問1の解答}

まず問$1$の解答をまとめて載せておきます。例によって、解の表示方法は$1$つだけ与えています。

\noindent (1) $y$は任意で$x = 1 + 2y$ (2) $y$は任意で$x = 2y$ (3) $z$は任意で$(x, y) = (2 + 3z, 1 + 2z)$ (4) $x$は任意で$(y, z) = (2x, x)$ (5) $(x, y, z) = (1, 2, 4)$ (6) $(x, y, z) = (\frac{1}{2}, \frac{1}{2}, \frac{1}{2})$ (7) $x$は任意で$(y, z) = (x - 1, x - 3)$ (8) $x, y$は任意で$z = 3 - x - y$ (9) $x$は任意で$y = z = -1 + x$ (10) $z$は任意で$(x, y) = (-2 + 5z, -3 - 3z)$ (11) $z$は任意で$(x, y) = (1 - 2z, 1 + 2z)$


\subsection{線型代数の言葉による連立$1$次方程式の解釈}

連立$1$次方程式
\begin{align*}
a_{11} x_1 + a_{12} x_2 + \cdots + a_{1n} x_n &= b_1 \\
a_{21} x_1 + a_{12} x_2 + \cdots + a_{1n} x_n &= b_2 \\
\vdots \\
a_{m1} x_1 + a_{m2} x_2 + \cdots + a_{mn} x_n &= b_m
\end{align*}
を、いつものように$A\bm{x} = \bm{b}$と表します。ここで$\bm{x} \in \mathbb{R}^n$, $\bm{b} \in \mathbb{R}^m$です。

\paragraph{係数行列の表す線型写像}

係数行列$A$は$(m, n)$型行列なので、$A\colon \mathbb{R}^n \rightarrow \mathbb{R}^m$という線型写像を定めます\footnote{問題プリントでは$f_A$と書いていました。}。この見方に立つと、$A\bm{x} = \bm{b}$を満たす$\bm{x}$を見つけるということは、$A$による$\bm{b}$の逆像$\{\bm{x} \in \mathbb{R}^n\mid A\bm{x} = \bm{b}\}$を全て求めることに他なりません。特に$\bm{b} = \bm{0}$の場合、方程式を解くことは$\Ker A$を求めることに他なりません。

\paragraph{解のパラメータ}

線型写像の場合、単射性の破れ方は核で支配されるのでした。もう一度復習しておくと、$\bm{x}$と$\bm{x}'$が共に解であれば、$\bm{x} - \bm{x}' \in \Ker A$です。実際$A\bm{x} = A\bm{x}' = \bm{b}$なら$A(\bm{x} - \bm{x}') = \bm{b} - \bm{b} = \bm{0}$です。また$\bm{x}$が解で$\bm{y} \in \Ker A$なら、$A(\bm{x} + \bm{y}) = \bm{b} + \bm{0} = \bm{b}$が成り立ちます。だから$A\bm{x} = \bm{b}$の解$\bm{x}_0 \in \mathbb{R}^n$を$1$つ見つけてしまえば、残りの解は全て$\bm{x}_0 + \bm{y}$ ($\bm{y} \in \Ker A$)と表せるし、逆にこの形以外の解は存在しません。

そしていま$\bm{x}_0\in\mathbb{R}^n$を$A\bm{x} = \bm{b}$の$1$つの解とし、さらに$(\bm{x}_1, \bm{x}_2, \ldots, \bm{x}_l)$を$\Ker A$の基底とします。このとき全ての$\Ker A$の元は$\bm{x}_1, \ldots, \bm{x}_l$の$1$次結合でただ$1$通りに表せます。したがって全ての解は、$\bm{x}_0 + t_1 \bm{x}_1 + \cdots + t_l \bm{x}_l$ ($t_1, \ldots, t_l \in \mathbb{R}$) の形に一意的に表せます。逆に解が$\bm{x}_0 + t_1 \bm{x}_1 + \cdots + t_l \bm{x}_l$の形にパラメータ表示されるとき、$\bm{x}_1, \ldots, \bm{x}_l$は$\Ker A$の元です。このパラメータ表示が全ての解を尽くし、そして一意的であることが、$(\bm{x}_1, \ldots, \bm{x}_l)$が$\Ker A$の基底であることを導きます。

このように解のパラメータは、$\Ker A$の元を基底の$1$次結合で表すときの係数だったのです。だから解に現れるパラメータは、ちょうど$\dim \Ker A$個だと分かります。

\paragraph{像の次元}

$A$の列ベクトルのうち、$1$次独立なものの最大本数を$\rank A$というのでした。これは$\dim \Im A$と同じです。というのも$A = (\bm{a}_1\ \bm{a}_2\ \cdots\ \bm{a}_n)$とおくと、
\[
A
\begin{pmatrix}
x_1 \\
\vdots \\
x_n
\end{pmatrix}
= 
\begin{pmatrix}
\bm{a}_1 & \cdots & \bm{a}_n
\end{pmatrix}
\begin{pmatrix}
x_1 \\
\vdots \\
x_n
\end{pmatrix}
= x_1 \bm{a}_1 + x_2 \bm{a}_2 + \cdots + x_n \bm{a}_n
\]
より$\Im A = \mathbb{R}\bm{a}_1 + \mathbb{R}\bm{a}_2 + \cdots + \mathbb{R}\bm{a}_n$でした。ここで前回の結果を思い出しましょう。もし$\bm{a}_1, \bm{a}_2, \ldots, \bm{a}_n$が$1$次独立でなければ、このうち上手く$1$本を削っても、$\Im A$の生成系が得られるのでした。そうして$1$次独立になるまで上手くベクトルを削り続けると、基底ができます。ここで$\rank A$の定義は「$1$次独立な$A$の列ベクトルの組の最大本数」であり、$A$の列ベクトルを上手く選ぶと$\Im A$の基底ができるのだから、$\rank A \geq \Im A$です。でも次元の性質から、$\Im A$の中に$\dim \Im A$本以上の$1$次独立なベクトルを取ってくることはできません。だから$(\dim \Im A + 1)$本以上からなる$A$の列ベクトルの組は、必ず$1$次従属になります。これで$\rank A = \Im A$が成り立つことが分かりました。

\subsection{次元定理の利用例}

実際に次元定理を使って、問3と問4を解いてみましょう。問3の (2) は「次元を考えるだけで解が分かる」という問題の典型例です。また問4では行基本変形と列基本変形のどちらを用いても解くことができます。なんとなく (1) では列基本変形、(2) では行基本変形を使うと解きやすい気がする\footnote{(1) は$1$行目の$2$列目だけに$1$があり、(2) は$1$列目にたくさん$1$があるから、こう思いました。これ以上の深い理由はありません。}ので、両方使ってみます。

\paragraph{問3の解答}
\noindent (2) $A := (1 \quad 1 \quad 1 \quad 1)$とおくと、$A$は線型写像$\mathbb{R}^4\rightarrow\mathbb{R}$を定める。このとき$W = \Ker A$である。$A \neq O$より$\dim \Im A = \rank A \geq 1$である。かたや$\Im A \subset\mathbb{R}$より$\dim \Im A \leq 1$なので、これで$\dim \Im A = 1$と分かる。よって次元定理から$\dim \Ker A = 4 - 1 = 3$が従う。

これより$\Ker A$の基底を求めるには、$3$本の$1$次独立なベクトルを見つければ良い。$A$の形から、$1$次独立なベクトル$3$本が次のように取れると分かる: 
\[
\bm{v}_1 :=
\begin{pmatrix}
-1 \\
1 \\
0 \\
0
\end{pmatrix}, \quad
\bm{v}_2 :=
\begin{pmatrix}
-1 \\
0 \\
1 \\
0
\end{pmatrix}, \quad
\bm{v}_3 :=
\begin{pmatrix}
-1 \\
0 \\
0 \\
1
\end{pmatrix}
\]

\noindent (1)
\[
A := 
\Biggl(
\begin{array}{rrrr}
1 & 1 & -1 & 1 \\
-2 & -1 & 0 & 1 \\
3 & 1 & 1 & -3
\end{array}
\Biggr)
\]
とおくと、$A$は線型写像$\mathbb{R}^4\rightarrow\mathbb{R}^3$を定める。このとき$W = \Ker A$である。行基本変形で$\Ker A$と$\rank A$は不変なので、掃き出しを行う。すると
\[
\Biggl(
\begin{array}{rrrr}
1 & 1 & -1 & 1 \\
-2 & -1 & 0 & 1 \\
3 & 1 & 1 & -3
\end{array}
\Biggr)
\xrightarrow[]{}
\Biggl(
\begin{array}{rrrr}
1 & 1 & -1 & 1 \\
0 & 1 & -2 & 3 \\
0 & -2 & 4 & -6
\end{array}
\Biggr)
\xrightarrow[]{}
\Biggl(
\begin{array}{rrrr}
1 & 0 & 1 & -2 \\
0 & 1 & -2 & 3 \\
0 & 0 & 0 & 0
\end{array}
\Biggr)
\]
となる。これより$\rank A = 2$が分かる。よって$\dim \Ker A = 4 - \dim \Im A = 2$である。あとは$\Ker A$の中に$2$本の$1$次独立なベクトルを探せばよいが、掃き出し後の表式から解は
\[
\begin{pmatrix}
-z + 2w \\
2z - 3w \\
z \\
w
\end{pmatrix}
=
z
\begin{pmatrix}
-1 \\
2 \\
1 \\
0
\end{pmatrix}
+
w
\begin{pmatrix}
2 \\
-3 \\
0 \\
1
\end{pmatrix}
\]
と表せる。よって
\[
\bm{v}_1 :=
\begin{pmatrix}
-1 \\
2 \\
1 \\
0
\end{pmatrix}, \quad
\bm{v}_2 :=
\begin{pmatrix}
2 \\
-3 \\
0 \\
1
\end{pmatrix}
\]
が基底である。\qed

\paragraph{問4の解答}
(1) 与えられた行列を列基本変形すると
\begin{align*}
\Biggl(
\begin{array}{rrr}
2 & 1 & 4 \\
4 & 3 & 10 \\
2 & 3 & 8
\end{array}
\Biggr)
\xrightarrow[\text{\hbox to 8zw{\hfil 入れ替え\hfil}}]{\text{\hbox to 8zw{\hfil $1$列目と$2$列目を\hfil}}}
& \Biggl(
\begin{array}{rrr}
1 & 2 & 4 \\
3 & 4 & 10 \\
3 & 2 & 8
\end{array}
\Biggr)
\xrightarrow[\text{\hbox to 10zw{\hfil $2$, $3$列目を掃き出し\hfil}}]{\text{\hbox to 10zw{\hfil $(1, 1)$成分を要に\hfil}}}
\Biggl(
\begin{array}{rrr}
1 & 0 & 0 \\
3 & -2 & -2 \\
3 & -4 & -4
\end{array}
\Biggr) \\
\xrightarrow[\text{\hbox to 8zw{\hfil $(-1/2)$倍\hfil}}]{\text{\hbox to 8zw{\hfil $2$列目を\hfil}}}
& \Biggl(
\begin{array}{rrr}
1 & 0 & 0 \\
3 & 1 & -2 \\
3 & 2 & -4
\end{array}
\Biggr)
\xrightarrow[\text{\hbox to 10zw{\hfil $1$, $3$列目を掃き出し\hfil}}]{\text{\hbox to 10zw{\hfil $(2, 2)$成分を要に\hfil}}}
\Biggl(
\begin{array}{rrr}
1 & 0 & 0 \\
0 & 1 & 0 \\
-3 & 2 & 0
\end{array}
\Biggr)
\end{align*}
となる。これで$\Im A$の基底は$\bigl({}^t(1, 0, -3), {}^t(0, 1, 2)\bigr)$で、$\dim \Im A = 2$と分かる。これより$\dim \Ker A = 3 - 2  = 1$と分かる。だから$\Ker A$に属する$\bm{0}$でないベクトルを$1$本見つければ、それが基底になる。たとえば$A$を眺めると${}^t(1, 2, -1) \in \Ker A$が分かる\footnote{これは、$1$行目を眺めて何となく「$1$列目と$2$列目の$2$倍を足したら$3$列目になりそうだなあ」と思ってたら答えが見つかりました。今回はうまく答えを見つけましたが、一般の場合は行基本変形による掃き出しで攻めるのが正攻法です。$\dim \Ker A = 1$と分かってるので、掃き出し法を途中まで行えば答えに当たりがつくはずです。}。

\noindent (2) 与えられた行列を行基本変形すると
\begin{align*}
& \left(
\begin{array}{rrrrr}
1 & 1 & 3 & 0 & 0 \\
1 & 2 & 5 & -2 & -5 \\
1 & 1 & 3 & 1 & 2 \\
2 & 0 & 2 & 4 & 10
\end{array}
\right)
\xrightarrow[\text{\hbox to 10zw{\hfil $2$, $3$, $4$行目を掃き出し\hfil}}]{\text{\hbox to 10zw{\hfil $(1, 1)$成分を要に\hfil}}}
\left(
\begin{array}{rrrrr}
1 & 1 & 3 & 0 & 0 \\
0 & 1 & 2 & -2 & -5 \\
0 & 0 & 0 & 1 & 2 \\
0 & -2 & -4 & 4 & 10
\end{array}
\right) \\
\xrightarrow[\text{\hbox to 10zw{\hfil $1$, $4$行目を掃き出し\hfil}}]{\text{\hbox to 10zw{\hfil $(2, 2)$成分を要に\hfil}}}
& \left(
\begin{array}{rrrrr}
1 & 0 & 1 & 2 & 5 \\
0 & 1 & 2 & -2 & -5 \\
0 & 0 & 0 & 1 & 2 \\
0 & 0 & 0 & 0 & 0
\end{array}
\right)
\xrightarrow[\text{\hbox to 10zw{\hfil $(1, 2)$行目を掃き出し\hfil}}]{\text{\hbox to 10zw{\hfil $(3, 4)$成分を要に\hfil}}}
\left(
\begin{array}{rrrrr}
1 & 0 & 1 & 0 & 1 \\
0 & 1 & 2 & 0 & 1 \\
0 & 0 & 0 & 1 & 2 \\
0 & 0 & 0 & 0 & 0
\end{array}
\right)
\end{align*}
となる。これで$\Ker A$の基底が求まる: 
\[
\bm{v}_1 :=
\begin{pmatrix}
-1 \\
-2 \\
1 \\
0 \\
0
\end{pmatrix}, \quad
\bm{v}_2 :=
\begin{pmatrix}
-1 \\
-1 \\
0 \\
-2 \\
1
\end{pmatrix}
\]
$\dim \Ker A = 2$だから$\dim \Im A = 5 - 2 = 3$である。そこで$\Im A$を張る$3$本の列ベクトル、列基本変形で探す。
\begin{align*}
\left(
\begin{array}{rrrrr}
1 & 1 & 3 & 0 & 0 \\
1 & 2 & 5 & -2 & -5 \\
1 & 1 & 3 & 1 & 2 \\
2 & 0 & 2 & 4 & 10
\end{array}
\right)
\xrightarrow[\text{\hbox to 10zw{\hfil $5$列目に足す\hfil}}]{\text{\hbox to 10zw{\hfil $4$列目の$(-3)$倍を\hfil}}}
\left(
\begin{array}{rrrrr}
1 & 1 & 3 & 0 & 0 \\
1 & 2 & 5 & -2 & 1 \\
1 & 1 & 3 & 1 & -1 \\
2 & 0 & 2 & 4 & -2
\end{array}
\right)
\xrightarrow[\text{\hbox to 9zw{\hfil $4$列目に足す\hfil}}]{\text{\hbox to 9zw{\hfil $5$列目の$2$倍を\hfil}}}
\left(
\begin{array}{rrrrr}
1 & 1 & 3 & 0 & 0 \\
1 & 2 & 5 & 0 & 1 \\
1 & 1 & 3 & -1 & -1 \\
2 & 0 & 2 & 0 & -2
\end{array}
\right)
\end{align*}
とすると、上$3$行を見ることで$1$, $4$, $5$列目のベクトルが$1$次独立だと分かる\footnote{もうちょっとだけ補足すると、$1$列目、$5$列目、$4$列目の順番にベクトルを並べると下三角行列ができます。ここまでくれば$1$次独立なことはほとんど明らかです。}。$\dim \Im A = 3$なので、これらが$\Im A$の基底をなす。

