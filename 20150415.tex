\chapter{2015年4月15日}

\begin{flushright}
担当教員: 植野 義明 / TA: 穂坂 秀昭
\end{flushright}

\section{はじめに}

\paragraph{ごあいさつ}

みなさん、はじめまして。この授業のTA (ティーチング・アシスタント)をすることになりました、数理科学研究科博士課程の穂坂といいます。これから$1$セメスターの間、よろしくお願いします。

この授業では基本的に毎回レポート問題が課され、それをTAの穂坂が添削して返却することになっています。また答案返却に合わせて、この文書のような解説プリントを配付していく予定です。何か疑問要望等があれば、随時質問をしてください。

\paragraph{プリントのオンライン公開}

このプリントは毎回教室で配布するのに加え、インターネット上でも配布されます。そのURLは下記の通りですので、必要な方はアクセスしてください。

\paragraph{このプリントの作り方について}

せっかくなので、このプリントをどう作っているかについて説明しておきます。

ふつう「コンピュータで文書を作る」というと、大抵の人がMicrosoft Wordとか一太郎といったワープロソフトを連想すると思います。ところが残念なことに、市販のワープロソフトでは数式を入力するのに大変な苦労を強いられてしまいます。そこで数学が専門の人はどうするかというと、そういったワープロソフトの代わりに``\LaTeX''というソフトウェアを使います。これはD.~E.~Knuthという非常に有名な数学者・計算機科学者が作った``\TeX''というソフトウェアを、色々な人が改良してできあがったものです。

\LaTeX はワープロソフトとはちょっと違い、文字のスタイルを変えたり見出しをつけたりするのに「コマンド」というものを使います。ですから\LaTeX を使うにはコマンドの使い方を覚えなければいけません。加えてキーボードが打ち込んだものが、見た目通りに出てくるわけでもありません。一旦コマンドも含めて打ち込んだ文書を\LaTeX というプログラムに処理させることによって、やっと整形された文書がでてきます。ですから使い始めるにはちょっとハードルが高いのですが、使い慣れるとワープロソフトよりも手際よく文書が書けるし、また数式を中心として文書の仕上がりが美しいというメリットもあります。

もしかしたら皆さんの中には\LaTeX を既に知っている人がいるかもしれませんし、また将来\LaTeX を使う必要に迫られる人がいるかもしれません。そこでこの文書の\LaTeX ソースコードを
\begin{center}
\url{https://github.com/HideakiHosaka/2015_linear_algebra}
\end{center}
に置いておきます。もし\LaTeX の方に興味がある人は、こちらを見てください。また東京大学が持つ情報処理システムのオンライン自習教材「はいぱーワークブック」の第27章\footnote{\url{http://hwb.ecc.u-tokyo.ac.jp/current/applications/latex/}}に、\LaTeX の説明があります。\LaTeX を使う人は、一度読んでおくと良いと思います。

ちなみにソースコードの公開には``GitHub''というサービス\footnote{もしあなたが既に``GitHub''を知っているなら、きっと``pull request''の機能も知っているはずです。プリントに対して何か意見があれば、積極的にpull requestを送ってください。\textsf{\raisebox{1pt}{:}D}}を利用しています。上に貼ったURLを開くと、古いバージョンのプリントや、そうしたプリントがどう更新されていったかも見ることができます。授業自体の役には立たないと思いますが、興味があれば見てみてください。

\section{複素(数)平面の幾何}

\subsection{複素数と複素平面}

\paragraph{複素数の定義}

まず、最初に複素数の定義をおさらいしましょう。$i^2=-1$という規則で$i$という「数」\footnote{誰もが一度は「$-1$の平方根を数と呼んでいいのか」という疑問を抱いたことがあると思います。その疑問に対する答えをまだ書いていなかったので、ここでは括弧つきで「数」と書きました。でも一々こう書くと面倒なので、以下では括弧をつけないことにします。}を定めます。このとき、$2$つの実数$x,y\in\mathbb{R}$を用いて$z=x+yi$と表される数を複素数と言うのでした。また$z=x+yi$の$x$を実部、$y$を虚部と言うことも知っているはずです。足し算と掛け算はそれぞれ、分配法則などが上手く成り立つよう
\begin{align*}
(x+yi) + (x'+y'i) &= (x+x')+(y+y')i, &  (x+yi)(x'+y'i) &= (xx' - yy') + (xy'+x'y)i
\end{align*}
と決めていました。また、これらのルールがあれば
\begin{align*}
\frac{x+yi}{x'+y'i}
= \frac{(x+yi)(x'-y'i)}{(x'+y'i)(x'-y'i)} = \frac{(xx'+yy')+(x'y-xy')i}{(x')^2+(y')^2}
= \frac{xx'+yy'}{(x')^2+(y')^2} + \frac{x'y-xy'}{(x')^2+(y')^2}i
\end{align*}
というように割り算もできます。こうして複素数では四則演算が全部できると分かりました。

\paragraph{複素平面}

さて、全ての複素数は$x+yi$の形に、$(x,y)$という二つの実数$x,y$のペアを用いて表せます。また複素数を二つの実数$x,y$で$x+yi$の形に表す方法がただ一通りなことも明らかでしょう。これらの事実から、\textbf{複素数$z=x+yi$と座標平面上の点$(x,y)$とを$1:1$に対応させられる}ことが分かります。このように平面$\mathbb{R}^2$上の点一つ一つを複素数と見なしたとき、平面$\mathbb{R}^2$のことを\textbf{複素(数)平面}\footnote{複素平面と複素数平面は、どっちの言葉も同じ意味です。複素平面の方を使う人が多いですが、複素数平面と言っても誤解を招くことはないですし、昔は複素数平面という言葉も割と良く使われていたそうです。関西学院大学の示野信一先生のブログに詳しい事情が書いてありますので、気になる人は読んでみてください: \url{http://mathsci.blog41.fc2.com/blog-entry-60.html}}と呼びます。

\begin{figure}[h!tbp]
\begin{center}
\begin{picture}(200,100)
\put(15,0){\vector(0,1){100}}
\put(0,15){\vector(1,0){200}}
\put(15,15){\dashbox{1.2}(100,50)}
\put(7,6){$0$}
\put(112,8){$x$}
\put(7,63){$y$}
\put(115,65){\circle*{3}}
\put(118,67){$x+yi$}
\end{picture}
\caption{複素平面における数と点の対応}
\end{center}
\end{figure}

少し大げさに言うと、我々は「複素数という代数的なもの」と「平面という幾何的なもの」を対応付けたと言うことができます。このことには、非常に重要な意味があります。なぜなら\textbf{代数の観点と幾何の観点を行ったりきたりすることで、色々なことが分かるようになる}からです。たとえば長さなどの幾何的な情報を代数的な操作で捉えたり、逆に掛け算などの代数的な操作を図形的に捉えたりというように。これから、それをやってみましょう。

\subsection{幾何を代数で捉える}

\paragraph{複素数の大きさ}

\begin{comment}
\paragraph{複素数と複素数平面: 問1 の解答}

\paragraph{複素数と複素数平面: 問4 の解答}

(2)は(1)を使えば$|z_1-z_2| = |z_1 + (-z_2) | \leq |z_1|+|{-z_2}| = |z_1| + |z_2|$と分かる。
\end{comment}

\subsection{代数を幾何で捉える}

続いて、四則演算という代数的な操作を複素平面で見てみましょう。複素数の足し算や引き算がベクトルの足し算や引き算と全く同じであることは、すぐに分かると思います。非自明なのは\textbf{平面上の点やベクトルには掛け算が定義されていないのに対し、複素数には掛け算がある}という点です\footnote{「ベクトルにも内積や外積があるじゃん」という声が聞こえてきそうですが、内積や外積は、いわゆる普通の「積」とは違う性質を持ちます。$2$つのベクトルの内積は数になってしまい、ベクトルにはなりません。また$2$つのベクトルの外積はベクトルになりますが、ベクトルの外積は順序を入れ替わると結果が変わります。この辺が数の掛け算と全然違うところです。}。そこで「$2$つの複素数を掛け算した結果は、複素平面上ではどのように見えるか」が問題となってきます。まずは問題を$1$つ解いてみましょう。

\paragraph{複素数と複素数平面: 問2の解答} $z=2+i$とおくと
\begin{align*}
z^2 &= 3+4i, & z^3 &= 2 + 11i, & z^4 &= -7+24i, & z^5 &= -38+41i, & z^6 &= -117+44i
\end{align*}
である。これらをプロットした結果は次の通り。

% TODO: ここに図をはさむ

\paragraph{極形式表示}

いま複素数$z\neq 0$に対し$z^0=1$, $z^1=1$, $z^2$, $z^3$, $z^4$, $z^5$, $z^6$を平面上にプロットし、かつこれらの点を原点と結んだ結果、隣り合う角が全て同じ大きさであるように見えます。それを実際に確かめてみましょう。

実軸からの角度を計算したいので、極座標を使うのが筋がよさそうです。そこで$z=x+yi$の点$(x,y)$を、極座標$(r,\theta)$を用いて$x=r\cos\theta$, $y=r\sin\theta$と表します。これを対応する複素数の方で表すと
\begin{align*}
z =x+yi = r\cos\theta + (r\sin\theta)i = r(\cos\theta+i\sin\theta)
\end{align*}
となります。この書き方を、複素数の\textbf{極形式表示}と呼びます。この書き方のもとで$r=|z|$です。また$\theta$は、複素平面の半直線$0z$と実軸の非負の部分がなす角を、実軸から反時計回りに測った角度となっています。この$\theta$を複素数$z$の\textbf{偏角}と言い、$\theta=\arg z$と書きます。$\arg z$は一通りでなく$2\pi$の整数倍だけずらせますが、今は気にしないでおきます。

極形式で表された$2$つの複素数を掛け算すると
\begin{align*}
r(\cos\theta+i\sin\theta) \times r'(\cos\theta'+i\sin\theta')
&= rr'\bigr\{ (\cos\theta\cos\theta' - \sin\theta\sin\theta') + i(\sin\theta\cos\theta'+\cos\theta\sin\theta') \bigl\} \\
&= rr'\bigl(\cos(\theta+\theta') + i\sin(\theta+\theta')\bigr)
\end{align*}
となります。最後の式変形は、もちろん三角函数\footnote{函数と関数は同じ意味です。少々古臭い言い回しですが、好みの問題でこちらを使います。}の加法定理を使っています。この式は非常に重要なことを示唆しています。それは複素数$z$, $z'$の積$zz'$について
\begin{itemize}
\item 大きさは、$|zz'|=|z||z'|$で与えられる
\item 偏角は$\arg zz' = \arg z + \arg z'$で与えられる
\end{itemize}
ということです。言い換えれば、複素数$z$に対して別の複素数$z'$を掛け算する操作は
\begin{itemize}
\item $z$の大きさを$|z'|$倍し
\item $z$の偏角に$\arg z'$を足し算する
\end{itemize}
ということに他ならないからです。このように極形式を使うことによって、複素数の掛け算が「拡大縮小」と「回転」の組み合わせという図形的意味を持つことが読み取れるのです。

% TODO: ここに図を挟む

なお、一々$\cos\theta+i\sin\theta$と書いているのは長ったらしくて大変なので、以下では実数$\theta$に対して$e^{i\theta} = \cos\theta + i \sin\theta$や$\exp i\theta = \cos\theta + i \sin\theta$と表すことにします。たとえば
\[
e^{i\pi} = \cos\pi + i\sin\pi = -1
\]
という感じです。指数函数$e^x$と同じ記法を用いることには実は意味がある\footnote{そのうち微分積分学の授業で、函数のTaylor展開というものを習うはずです。その後で巾(べき)級数で指数函数$e^x$を定義し直すと、元々の「$e$のなんとか乗」という意味を越えて、$e^x$の$x$に複素数を代入できるようになります。そうして初めて$e^{i\theta}=\cos\theta+i\sin\theta$という式に正しく意味を与えることができます。}のですが、今は「単なる記号」だと思っていてください。この記号を使うと、極形式表示での掛け算は
\[
re^{i\theta} \times r'e^{i\theta'} = rr'e^{i(\theta+\theta')}
\]
と書けます。スッキリしてていいですね。

極形式表示は計算面でも、非常に威力を発揮することがあります。

\begin{comment}
\paragraph{複素数と複素数平面: 問3の解答} $z=e^{2k\pi i/n}$ ($k=0,1,\ldots,n-1$)とおくと、
\[
z^n = \bigl(e^{\frac{2k\pi i}{n}}\bigr)^n = \exp\Bigl(\frac{2k\pi i}{n}+\frac{2k\pi i}{n}+\cdots+\frac{2k\pi i}{n}\Bigr) = \exp \Bigl(\frac{2k\pi i}{n} \times n\Bigr) = e^{2k\pi i} = (e^{2\pi i})^k = 1
\]
となる。ここで$k$が$k=0,1,\ldots,n-1$を動けば、$n$個の異なる複素数が得られる\footnote{複素平面上にプロットすれば、異なることが直ちに分かります。}。また多項式$z^n-1$は$n$次式だから、$n+1$個以上の根を持つことはない。ゆえに$z=e^{2k\pi i/n}$ ($k=0,1,\ldots,n-1$)が全ての根を与える。
\end{comment}

\section{多項式の性質}

\subsection{多項式の割り算}

\begin{comment}
\paragraph{多項式: 問1の解答}

\paragraph{多項式: 問2の解答}

\paragraph{多項式: 問3の解答} 	$F(x)$を$2x^2+x-1$で割った余りを$px+q$と書くと、何か多項式$P(x)$を用いて
\[
F(x) = (2x^2+x-1)P(x) + px + q
\]
と書ける。$F(x)$を$x+1$で割った余りが$6$なので$F(-1)=6$である、よって上式に$x=-1$を代入して$6=-p+q$を得る。同様に$F(x)$を$2x-1$で割った余りが$3$なので、$F(\frac{1}{2})=3$である。これより$3=\frac{1}{2}p+q$を得る。こうして$p,q$の連立$1$次方程式が得られたので、解くと$p=-2$, $q=4$が得られる。よって余りは$-2x+4$である。

\paragraph{多項式: 問6の解答}
\end{comment}

\subsection{複素数について}
\begin{citation}
{}``Was sind und was sollen die Zahlen?''
\end{citation}

これは、ドイツの数学者Richard Dedekindが記した本\footnote{ちなみに、1893年に出版されたドイツ語原著の第2刷が、東大の数理科学図書室にあります。}のタイトルです。『数とは何か、また何であるべきか?』この問題を、考えてみましょう。

\subsection{代数学の基本定理}

