\chapter{対角化の計算練習}
\lectureinfo{2015年11月4日 1限}

\section{行列の対角化の計算}

\subsection{対角化のレシピ}

前回「行列の固有ベクトルで基底を作れば、対角化ができる」ということを確認しました。今回は実際にその計算を色々やってみます。まずは正方行列$A$を対角化する計算方法を確認しておきましょう。

\begin{enumerate}
\item $A$の固有多項式$\varphi_A(t) := \det(tI - A)$を計算する。
\item 方程式$\varphi_A(t) = 0$を解いて、固有値を全て求める。ここで方程式が解けなかったら諦める。
\item それぞれの固有値$\lambda$に対し、連立$1$次方程式$A \bm{x} = \lambda\bm{x}\ (\Leftrightarrow (A - \lambda I)\bm{x} = \bm{0})$を解いて、$\Ker (A - \lambda I)$の基底をつくる。
\item 各$\lambda$毎に$\Ker (A - \lambda I)$の基底を集めてきて、幸いなことに$n$本のベクトルが得られていたら、それを一列に並べて正方行列$P$を作る。このとき$P^{-1} A P$が対角行列になる。もし$n$本の固有ベクトルが集まらなかったら諦める。
\end{enumerate}

この方針に従って、計算をやってみましょう。その後で「諦める」という箇所について、いつ諦めるべきなのか、諦める場合はどうするのかをもう少し深く考察します。

\subsection{計算例}

$2$次と$3$次の場合に$1$つずつ、計算例を見てみましょう。

\paragraph{問1 (1) の解答} 固有多項式は
\[
\det
\begin{pmatrix}
t - 1 & -2 \\
-2 & t - 1
\end{pmatrix}
= (t - 1)^2 - 4 = t^2 - 2t -3 = (t + 1)(t - 3)
\]
なので、固有値は$3, -1$だと分かる。このそれぞれに対応する固有ベクトルを求める。

まず
\[
\begin{pmatrix}
1 & 2 \\
2 & 1
\end{pmatrix}
\begin{pmatrix}
x \\
y
\end{pmatrix}
=
3
\begin{pmatrix}
x \\
y
\end{pmatrix}
\]
を解くと、$(x, y) = \alpha(1, 1)$ ($\alpha \in \mathbb{R}$)と分かる。次に
\[
\begin{pmatrix}
1 & 2 \\
2 & 1
\end{pmatrix}
\begin{pmatrix}
x \\
y
\end{pmatrix}
=
-1
\begin{pmatrix}
x \\
y
\end{pmatrix}
\]
を解くと、$(x, y) = \alpha(-1, 1)$ ($\alpha \in \mathbb{R}$)となる。よって固有ベクトル$2$本で$\mathbb{R}^2$の基底が張れるので、この問題は対角化可能である。固有ベクトル$2$本を並べて
\[
P :=
\begin{pmatrix}
1 & 1 \\
1 & -1
\end{pmatrix}
\]
と定めると
\[
P^{-1}
\begin{pmatrix}
1 & 2 \\
2 & 1
\end{pmatrix}
P
=
\frac{1}{-2}
\begin{pmatrix}
-1 & -1 \\
-1 & 1
\end{pmatrix}
\begin{pmatrix}
3 & -1 \\
3 & 1
\end{pmatrix}
=
\begin{pmatrix}
3 & 0 \\
0 & -1
\end{pmatrix}
\]
となる。 \qed

\paragraph{問4 (3) の解答} 固有多項式は
\begin{align*}
\varphi(t) &=
\det
\begin{pmatrix}
t - 1 & -1 & 1 \\
-2 & t + 3 & -4 \\
-1 & 2 & t - 3
\end{pmatrix} \\
&= (t - 1)(t + 3)(t - 3) + (-2) \cdot 2 \cdot 1 + (-1) (-4) (-1) - 1 (t + 3) (-1) - (-1) (-2) (t - 3) - (t - 1)\cdot 2 \cdot (-4) \\
&= (t - 1)(t^2 - 9) - 4 - 4 + (t + 3) - 2(t - 3) + 8(t - 1) \\
&= (t^3 - t^2 - 9t + 9) - 8 + 7t + 1 = t^3 - t^2 - 2t + 2 \\
&= (t - 1)(t^2 - 2) = (t - 1)(t + \sqrt{2})(t - \sqrt{2})
\end{align*}
なので、固有値は$1$と$\pm\sqrt{2}$である。このそれぞれに対応する固有ベクトルを求める。

まず
\[
\begin{pmatrix}
1 & 1 & -1 \\
2 & -3 & 4 \\
1 & -2 & 3
\end{pmatrix}
\begin{pmatrix}
x \\
y \\
z
\end{pmatrix}
=
\begin{pmatrix}
x \\
y \\
z
\end{pmatrix}
\]
を解くと$(x, y, z) = \alpha(0, 1, 1)$ ($\alpha \in \mathbb{R}$) となる。次に
\[
\begin{pmatrix}
1 & 1 & -1 \\
2 & -3 & 4 \\
1 & -2 & 3
\end{pmatrix}
\begin{pmatrix}
x \\
y \\
z
\end{pmatrix}
=
\sqrt{2}
\begin{pmatrix}
x \\
y \\
z
\end{pmatrix}
\]
を解くと$(x, y, z) = \alpha(1 + \sqrt{2}, 2, 1)$ ($\alpha \in \mathbb{R}$) となる。最後に
\[
\begin{pmatrix}
1 & 1 & -1 \\
2 & -3 & 4 \\
1 & -2 & 3
\end{pmatrix}
\begin{pmatrix}
x \\
y \\
z
\end{pmatrix}
=
-\sqrt{2}
\begin{pmatrix}
x \\
y \\
z
\end{pmatrix}
\]
を解くと$(x, y, z) = \alpha(1 - \sqrt{2}, 2, 1)$ ($\alpha \in \mathbb{R}$) である。よって$3$本の固有ベクトルを並べて
\[
P :=
\begin{pmatrix}
0 & 1 + \sqrt{2} & 1 - \sqrt{2} \\
1 & 2 & 2 \\
1 & 1 & 1
\end{pmatrix}
\]
と定めると
\[
P^{-1} A P 
=
\frac{1}{2\sqrt{2}}
\begin{pmatrix}
0 & 1 & -1 \\
-2\sqrt{2} & -1 + \sqrt{2} & 1 + \sqrt{2} \\
4\sqrt{2} & 1 - \sqrt{2} & -1 - \sqrt{2}
\end{pmatrix}
\begin{pmatrix}
1 & 1 & -1 \\
2 & -3 & 4 \\
1 & -2 & 3
\end{pmatrix}
\begin{pmatrix}
0 & 1 + \sqrt{2} & 1 - \sqrt{2} \\
1 & 2 & 2 \\
1 & 1 & 1
\end{pmatrix}
=
\begin{pmatrix}
1 \\
& \sqrt{2} \\
& & -\sqrt{2}
\end{pmatrix}
\]
となり、対角化ができる。 \qed

\subsection{問題の解答}

他の問題に対する解答を載せておきます。一々細かい計算を書くのは省きますが、固有値と対応する固有ベクトル、それに対角化する行列$P$の逆行列$P^{-1}$を書いておきます。間違えた人は、どこで計算ミスをしたのかチェックしてください。

\paragraph{問1の解答} (2) 固有値は$\pm2$である。対応する固有ベクトルと、対角化行列$P$の逆行列は
\[
A =
\begin{pmatrix}
0 & 4 \\
1 & 0
\end{pmatrix}, 
\bm{v}_2 = 
\begin{pmatrix}
2 \\
1
\end{pmatrix}, 
\bm{v}_{-2} = 
\begin{pmatrix}
-2 \\
1
\end{pmatrix}, \quad
P^{-1} = 
\frac{1}{4}
\begin{pmatrix}
1 & 2 \\
-1 & 2
\end{pmatrix}, \quad
P^{-1} AP =
\begin{pmatrix}
2 & 0 \\
0 & -2
\end{pmatrix}
\]

(3) 固有値は$\pm\sqrt{2}$である。対応する固有ベクトルと、対角化行列$P$の逆行列は
\[
\bm{v}_{\sqrt{2}} = 
\begin{pmatrix}
1 + \sqrt{2} \\
1
\end{pmatrix}, 
\bm{v}_{-\sqrt{2}} = 
\begin{pmatrix}
1 - \sqrt{2} \\
1
\end{pmatrix}, \quad
P^{-1} = 
\frac{1}{2\sqrt{2}}
\begin{pmatrix}
1 & -1 + \sqrt{2} \\
-1 & 1 + \sqrt{2}
\end{pmatrix}, \quad
P^{-1} AP =
\begin{pmatrix}
\sqrt{2} & 0 \\
0 & -\sqrt{2}
\end{pmatrix}
\]

\paragraph{問2の解答}

(1) 固有値は$2, 0, -1$である。対応する固有ベクトルと、対角化行列$P$の逆行列は
\[
\bm{v}_{2} = 
\begin{pmatrix}
1 \\
3 \\
3
\end{pmatrix}, 
\bm{v}_{0} = 
\begin{pmatrix}
1 \\
0 \\
0
\end{pmatrix}, 
\bm{v}_{-1} = 
\begin{pmatrix}
1 \\
-1 \\
1
\end{pmatrix}, \quad
P^{-1} = 
\frac{1}{6}
\begin{pmatrix}
0 & 1 & 1 \\
0 & -3 & 3 \\
6 & 2 & -4
\end{pmatrix}, \quad
P^{-1} AP =
\begin{pmatrix}
2 \\
 & 0 \\
 & & -1
\end{pmatrix}
\]

\noindent (2) 固有値は$2, 1, -1$である。対応する固有ベクトルと、対角化行列$P$の逆行列は
\[
\bm{v}_{2} = 
\begin{pmatrix}
0 \\
1 \\
0
\end{pmatrix}, 
\bm{v}_{1} = 
\begin{pmatrix}
2 \\
0 \\
3
\end{pmatrix}, 
\bm{v}_{-1} = 
\begin{pmatrix}
3 \\
0 \\
4
\end{pmatrix}, \quad
P^{-1} = 
\begin{pmatrix}
0 & 1 & 0 \\
-4 & 0 & 3 \\
3 & 0 & -2
\end{pmatrix}, \quad
P^{-1} AP =
\begin{pmatrix}
2 \\
 & 1 \\
 & & -1
\end{pmatrix}
\]

\noindent (3) 固有値は$3, 2, 1$である。対応する固有ベクトルと、対角化行列$P$の逆行列は
\[
\bm{v}_{3} = 
\begin{pmatrix}
1 \\
1 \\
1
\end{pmatrix}, 
\bm{v}_{2} = 
\begin{pmatrix}
1 \\
1 \\
0
\end{pmatrix}, 
\bm{v}_{1} = 
\begin{pmatrix}
1 \\
0 \\
0
\end{pmatrix}, \quad
P^{-1} = 
\begin{pmatrix}
0 & 0 & 1 \\
0 & 1 & -1 \\
1 & -1 & 0
\end{pmatrix}, \quad
P^{-1} AP =
\begin{pmatrix}
3 \\
 & 2 \\
 & & 1
\end{pmatrix}
\]

\paragraph{問3の解答}
(1) 固有値は$6, 1$である。対応する固有ベクトルと、対角化行列$P$の逆行列は
\[
\bm{v}_{6} = 
\begin{pmatrix}
1 \\
2
\end{pmatrix}, 
\bm{v}_{1} = 
\begin{pmatrix}
-2 \\
1
\end{pmatrix}, \quad
P^{-1} = 
\frac{1}{5}
\begin{pmatrix}
1 & 2 \\
-2 & 1
\end{pmatrix}, \quad
P^{-1} AP =
\begin{pmatrix}
6 & 0 \\
0 & 1
\end{pmatrix}
\]

\noindent (2) 固有値は$7, 0$である。対応する固有ベクトルと、対角化行列$P$の逆行列は
\[
\bm{v}_{7} = 
\begin{pmatrix}
1 \\
3
\end{pmatrix}, 
\bm{v}_{0} = 
\begin{pmatrix}
-2 \\
1
\end{pmatrix}, \quad
P^{-1} = 
\frac{1}{7}
\begin{pmatrix}
1 & 2 \\
-3 & 1
\end{pmatrix}, \quad
P^{-1} AP =
\begin{pmatrix}
7 & 0 \\
0 & 0
\end{pmatrix}
\]

\noindent (3) 固有値は$7, -1$である。対応する固有ベクトルと、対角化行列$P$の逆行列は
\[
\bm{v}_{7} = 
\begin{pmatrix}
1 \\
2
\end{pmatrix}, 
\bm{v}_{-1} = 
\begin{pmatrix}
-3 \\
2
\end{pmatrix}, \quad
P^{-1} = 
\frac{1}{8}
\begin{pmatrix}
2 & 3 \\
-2 & 1
\end{pmatrix}, \quad
P^{-1} AP =
\begin{pmatrix}
7 & 0 \\
0 & -1
\end{pmatrix}
\]

\noindent (4) 固有値は$\pm i$である。対応する固有ベクトルと、対角化行列$P$の逆行列は
\[
\bm{v}_{i} = 
\begin{pmatrix}
i \\
1
\end{pmatrix}, 
\bm{v}_{-i} = 
\begin{pmatrix}
-i \\
1
\end{pmatrix}, \quad
P^{-1} = 
\frac{1}{2}
\begin{pmatrix}
-i & 1 \\
i & 1
\end{pmatrix}, \quad
P^{-1} AP =
\begin{pmatrix}
i & 0 \\
0 & -i
\end{pmatrix}
\]

\paragraph{問4の解答}
(1) 固有値は$2, 1, -1$である。対応する固有ベクトルと、対角化行列$P$の逆行列は
\[
\bm{v}_{2} = 
\begin{pmatrix}
0 \\
0 \\
1
\end{pmatrix}, 
\bm{v}_{1} = 
\begin{pmatrix}
-1 \\
-1 \\
3
\end{pmatrix}, 
\bm{v}_{-1} = 
\begin{pmatrix}
1 \\
-1 \\
-1
\end{pmatrix}, \quad
P^{-1} = 
\frac{1}{2}
\begin{pmatrix}
2 & 4 & 2 \\
-1 & -1 & 0 \\
1 & -1 & 0
\end{pmatrix}, \quad
P^{-1} AP =
\begin{pmatrix}
2 \\
 & 1 \\
 & & -1
\end{pmatrix}
\]

\noindent (2) 固有値は$2, 1, 0$である。対応する固有ベクトルと、対角化行列$P$の逆行列は
\[
\bm{v}_{2} = 
\begin{pmatrix}
1 \\
2 \\
2
\end{pmatrix}, 
\bm{v}_{1} = 
\begin{pmatrix}
1 \\
2 \\
1
\end{pmatrix}, 
\bm{v}_{0} = 
\begin{pmatrix}
1 \\
1 \\
0
\end{pmatrix}, \quad
P^{-1} = 
\begin{pmatrix}
1 & -1 & 1 \\
-2 & 2 & -1 \\
2 & -1 & 0
\end{pmatrix}, \quad
P^{-1} AP =
\begin{pmatrix}
2 \\
 & 1 \\
 & & 0
\end{pmatrix}
\]

\section{対角化可能性の判定条件}

今回の問題はいずれも、上手く対角化ができるものばかりでした。しかし先週見た通り、世の中には対角化不可能な行列も存在します。そこでもう少し、対角化可能性について考えてみましょう。

\subsection{対角化可能なための十分条件}

前回
\[
\begin{pmatrix}
0 & 1 \\
0 & 0
\end{pmatrix}
\]
という行列が対角化できないことを確認しました。この行列の固有多項式は$t^2$だから、固有値は$0$のみです。しかし固有値$0$に属する固有ベクトルが$1$本しかないため、対角化ができないのでした。そこで「固有ベクトルが足りなくなる状況を回避できるのは、どういうときか」という問題が生じます。

\paragraph{異なる固有値に属する固有ベクトルの$1$次独立性}

これから「固有ベクトルで基底を作る」という話をするにあたり、一つ基本的な事実を示しておきます。それは、異なる固有値に属する固有ベクトルは$1$次独立になるということです。

$n$次正方行列$A \in \Mat_n(\mathbb{R})$の固有ベクトル$\bm{p}_1, \bm{p}_2, \ldots, \bm{p}_k$が、それぞれ異なる固有値$\lambda_1, \lambda_2, \ldots, \lambda_k$に属しているとします。このとき
\[
\alpha_1 \bm{p}_1 + \alpha_2 \bm{p}_2 + \cdots + \alpha_n \bm{p}_n = \bm{0}
\]
を仮定して、$\alpha_1 = \alpha_2 = \cdots = \alpha_n = 0$を示しましょう。

固有ベクトルの定義から、$1 \leq i \leq k$に対し$A\bm{p}_i = \lambda_i \bm{p}_i$が成り立ちます。よって$\alpha_1 \bm{p}_1 + \alpha_2 \bm{p}_2 + \cdots + \alpha_n \bm{p}_n = \bm{0}$の両辺に$A^l$をかけると
\begin{align*}
\bm{0} &= A^l \bm{0} = A^l (\alpha_1 \bm{p}_1 + \alpha_2 \bm{p}_2 + \cdots + \alpha_n \bm{p}_n)
= \lambda_1^l \alpha_1 \bm{p}_1 +  \lambda_2^l \alpha_2 \bm{p}_2 + \cdots + \lambda_k^l \alpha_n \bm{p}_n
\end{align*}
が得られます。この$l$を$l = 0, 1, \ldots, k - 1$と動かした式を並べると
\begin{align*}
\begin{array}{r@{\,}c@{\,}r@{\,}c@{\,}c@{\,}c@{\,}r@{\,}c}
\alpha_1 \bm{p}_1 & + & \alpha_2 \bm{p}_2 & + & \cdots & + & \alpha_n \bm{p}_n  & = \bm{0} \\
\lambda_1 \alpha_1 \bm{p}_1 & + & \lambda_2 \alpha_2 \bm{p}_2 & + & \cdots & + & \lambda_k \alpha_n \bm{p}_n  &= \bm{0} \\
\lambda_1^2 \alpha_1 \bm{p}_1 & + & \lambda_2^2 \alpha_2 \bm{p}_2 & + & \cdots & + & \lambda_k^2 \alpha_n \bm{p}_n  &= \bm{0} \\
& & \vdots \\
\lambda_1^{k - 1} \alpha_1 \bm{p}_1 & + & \lambda_2^{k - 2} \alpha_2 \bm{p}_2 & + & \cdots & + & \lambda_k^{k - 1} \alpha_n \bm{p}_n  &= \bm{0}
\end{array}
\end{align*}
となります。行列の形で書くと
\[
\begin{pmatrix}
\alpha_1 \bm{p}_1 & \alpha_2 \bm{p}_2 & \cdots & \alpha_n \bm{p}_n
\end{pmatrix}
\begin{pmatrix}
1 & \lambda_1 & \lambda_1^2 & \cdots & \lambda_1^{k - 1} \\
1 & \lambda_2 & \lambda_2^2 & \cdots & \lambda_2^{k - 1} \\
\vdots & \vdots & \vdots & \ddots & \vdots \\
1 & \lambda_n & \lambda_n^2 & \cdots & \lambda_n^{k - 1} \\
\end{pmatrix}
=
\begin{pmatrix}
\bm{0} & \bm{0} & \cdots & \bm{0}
\end{pmatrix}
\]
となっています。この左辺に出てきた$k$次正方行列の行列式は、Vandermonde行列式なので
\[
\det
\begin{pmatrix}
1 & \lambda_1 & \lambda_1^2 & \cdots & \lambda_1^{k - 1} \\
1 & \lambda_2 & \lambda_2^2 & \cdots & \lambda_2^{k - 1} \\
\vdots & \vdots & \vdots & \ddots & \vdots \\
1 & \lambda_n & \lambda_n^2 & \cdots & \lambda_n^{k - 1} \\
\end{pmatrix}
=
\pm \prod_{1 \leq i, j \leq k} (\lambda_j - \lambda_i)
\]
と計算できます。ここで行列$A$の固有値$\lambda_1, \lambda_2, \ldots, \lambda_k$は全て異なるとしたので、今出てきた行列式の値は$0$ではなく、行列が正則だと分かります。したがって逆行列をかけて
\[
\begin{pmatrix}
\alpha_1 \bm{p}_1 & \alpha_2 \bm{p}_2 & \cdots & \alpha_n \bm{p}_n
\end{pmatrix}
=
\begin{pmatrix}
\bm{0} & \bm{0} & \cdots & \bm{0}
\end{pmatrix}
\begin{pmatrix}
1 & \lambda_1 & \lambda_1^2 & \cdots & \lambda_1^{k - 1} \\
1 & \lambda_2 & \lambda_2^2 & \cdots & \lambda_2^{k - 1} \\
\vdots & \vdots & \vdots & \ddots & \vdots \\
1 & \lambda_n & \lambda_n^2 & \cdots & \lambda_n^{k - 1} \\
\end{pmatrix}^{-1}
=
\begin{pmatrix}
\bm{0} & \bm{0} & \cdots & \bm{0}
\end{pmatrix}
\]
を得ます。つまり全ての$1 \leq i \leq k$に対し$\alpha_i \bm{p}_i = \bm{0}$が分かります。ここで固有ベクトルは$\bm{p}_i \neq \bm{0}$を満たしているので、$\alpha_i = 0$が従います。これで固有ベクトルの$1$次独立性が言えました。

\paragraph{固有値が重根を持たない場合}

さて、話を対角化問題に戻します。

固有値があれば、それに対応する固有ベクトルが少なくとも$1$本は存在します。実際$\lambda$が$n$次正方行列$A \in \Mat_n(\mathbb{R})$の固有値なら$\Ker (A - \lambda I) \neq \{\bm{0}\}$なので、$1$本は固有ベクトルが取れます。ということは、$A$が$n$個の異なる固有値$\lambda_1, \lambda_2, \ldots, \lambda_n$を持っていれば、確実に$n$本の固有ベクトルが得られます。各$\lambda_i$毎に少なくとも$1$本は固有ベクトルが存在するのだから、それを$1$本ずつ集めて来れば自動的に$n$本になってしまいます。異なる固有値に属する固有ベクトルが$1$次独立になることをさっき示しましたから、これで固有ベクトルからなる基底を作れたことになります。よって$A$は対角化可能です。

また$A$が異なる$n$個の固有値を持つことは、固有多項式が重根を持たずに
\[
\varphi_A(t) = (t - \lambda_1)(t - \lambda_2) \cdots (t - \lambda_n)
\]
と因数分解されることと同値です。というわけで、\textbf{固有多項式が重根を持たなければ、固有ベクトルからなる基底が作れて、必ず対角化可能}になります。

\paragraph{対称行列の場合}

対角化問題を考える上で、もう一つ都合のよいケースがあります。それは行列が実対称な場合です。この場合は固有多項式が重根を持ってしまったとしても、必ずうまく対角化できることが知られています。しかも対角化するための行列として、直交行列が取って来られるのです。これと同様にして複素数成分の場合、Hermite行列 ($A^* = A$を満たす行列\footnote{$A^* := \overline{{}^t\!A}$です。}) がいつもユニタリ行列 ($U^{-1} = U^*$を満たす行列) で対角化できることが知られています。

\subsection{対角化不可能な可能性}

今の話を裏返せば、対角化不可能な場合は「固有多項式が重根を持つ場合」に限られます。固有多項式が重根を持って初めて「固有ベクトルが足りなかったらどうしよう」という心配をする必要に迫られるのです。

ただ「固有多項式が重根だったら、必ず対角化不可能か」というと、そうでもありません。これが微妙なところです。たとえばさっきの対角化不可能な行列
\[
A =
\begin{pmatrix}
0 & 1 \\
0 & 0
\end{pmatrix}
\]
は固有多項式が$t^2$で、重根$0$を持っていました。また$2$次のゼロ行列$O$の固有多項式は$\varphi_O(t) = \det(tI - O) = t^2$なので、これも重根$0$を持ちます。でも$O$は既に対角化されているから、対角化可能です。

こんな感じで、固有多項式が重根を持つからといって、常に対角化できないとは限らないのです。しかもさっきの$A$と$O$は、固有多項式がどちらも$t^2$で同じであるにも関わらず、一方は対角化可能で他方は対角化不可能でした。したがって\textbf{固有多項式を見ているだけでは、対角化可能かどうかの判定はできない}ことも分かります。固有多項式が重根を持ってしまったとき、僕たちは対角化可能性を判定するために、より詳しい情報が必要なのです。
この問題を考えるにあたっては
\begin{itemize}
\item どういう行列の場合、固有多項式が重根を持つか
\item 同じ固有多項式を持つ行列について、対角化可能か不可能かは何によって左右されるのか
\end{itemize}
を順番に突き止める必要があります。そこで固有多項式が重根を持つ行列について、詳しく調べてみましょう。

\section{Cayley--Hamiltonの定理とスペクトル分解}

いかなる正方行列$A$に対しても、その固有多項式$\varphi_A(t)$に自分自身を代入すると$\varphi_A(A) = O$となることが知られています。これを\textbf{Cayley--Hamiltonの定理}と言います。この定理を上手く使うことで、行列に関する色々な性質を知ることができます。

\subsection{$2$次の場合}

まず$2$次の場合に確認してみましょう。
\[
A = 
\begin{pmatrix}
a & b \\
c & d
\end{pmatrix}
\]
の固有多項式は$\varphi_A(t) = t^2 - (a + d)t + (ad - bc)$でした。これに$A$を代入すると
\[
A^2 - (a + d)A + (\det A)I
= 
\begin{pmatrix}
a^2 + bc & ab + bd \\
ac + cd & bc + d^2
\end{pmatrix}
- (a + d)
\begin{pmatrix}
a & b \\
c & d
\end{pmatrix}
+ (ad - bc)
\begin{pmatrix}
1 & 0 \\
0 & 1
\end{pmatrix}
= O
\]
になっていますね。

\subsection{行列のスペクトル分解}

一旦Cayley--Hamiltonの定理を認めて、先にそこから得られる帰結を導いてしまいましょう。

いま$n$次正方行列$A$の固有多項式$\varphi_A(t)$が、既約な多項式の積として
\[
\varphi_A(t) = p_1(t) p_2(t) \cdots p_l(t)
\]
と書けていたとします。このとき$1/\varphi_A(t)$を部分分数分解することで
\[
\frac{1}{\varphi_A(t)} = \frac{f_1(t)}{p_1(t)} + \frac{f_2(t)}{p_2(t)} + \cdots + \frac{f_l(t)}{p_l(t)}
\]
と書けます。これの両辺に$\varphi_A(t) = p_1(t) p_2(t) \cdots p_l(t)$という恒等式をかけることで
\begin{align*}
1 &= \frac{f_1(t)\varphi_A(t)}{p_1(t)} + \frac{f_2(t)\varphi_A(t)}{p_2(t)} + \cdots + \frac{f_l(t)\varphi_A(t)}{p_l(t)} \\
&= f_1(t) p_2(t) \cdots p_l(t) + p_1(t) f_2(t) p_3(t) \cdots p_l(t) + p_1(t) p_2(t) \cdots p_{l - 1}(t) f_l(t)
\end{align*}
が得られます。ここで$t$に行列$A$を代入してみましょう。$1 \leq i \leq k$に対し
\[
P_i := p_1(A) \cdots p_{i - 1}(A) f_i(A) p_{i + 1}(A) \cdots p_k(A)
\]
と書けば
\[
I = P_1 + P_2 + \cdots + P_k
\]
となります。

\paragraph{スペクトル分解の性質}

今のスペクトル分解で出てきた行列の性質を調べましょう。まず$i \neq j$のとき、$P_iP_j = P_jP_i = O$となることを示します。定義から
\[
P_i P_j = p_1(A) \cdots p_{i - 1}(A) f_i(A) p_{i + 1}(A) \cdots p_k(A) p_1(A) \cdots p_{j - 1}(A) f_j(A) p_{j + 1}(A) \cdots p_l(A)
\]
です。ここで$i \neq j$なら、右辺に$p_1(A), \ldots, p_l(A)$が全部出揃って
\[
P_i P_j = \varphi_A(A) \times (\text{その他の項})
\]
と書けます。よってCayley--Hamiltonの定理から$P_i P_j = O$です。これが分かると、$I = P_1 + P_2 + \cdots + P_k$の両辺に左から$P_i$をかけることで$P_i = P_i^2$が従います。

\paragraph{スペクトル分解に応じた空間の分解}

$I = P_1 + P_2 + \cdots + P_k$より、全てのベクトル$\bm{u} \in \mathbb{R}^n$は$\bm{u} = P_1\bm{u} + P_2\bm{u} + \cdots + P_l\bm{u}$と書けます。よって$\mathbb{R}^n = \Im P_1 + \Im P_2 + \cdots + \Im P_l$です。また$\bm{u} \in \Im P_i \cap \Im P_j$なら$\bm{u} = P_i \bm{v} = P_j \bm{w}$と書けます。すると$\bm{u} = P_i \bm{u} = P_i^2 \bm{u} = P_i P_j \bm{w} = \bm{0}$となるので、この分解は直和分解になっています。かくして$\mathbb{R}^n = \Im P_1 \oplus \Im P_2 \oplus \cdots \oplus \Im P_l$です。そこで各$\Im P_i$の基底を集めて$\mathbb{R}^n$の基底を作れば、$A$は
\[
\begin{pmatrix}
AP_1 \\
& AP_2 \\
& & \ddots \\
& & AP_n
\end{pmatrix}
\]
と、ブロック対角行列に分解されます。

\subsection{固有値問題の細分}

スペクトル分解を使うと、行列の対角化の問題が「固有値が全て同じである行列の対角化問題」に帰着できます。さらに単位行列のスカラー倍を適当に足せば固有値はずらせるので、対角化問題が最終的に、「固有値が全て$0$である行列の対角化問題」まで帰着できます。これはすなわち、巾零行列の対角化問題に他なりません。

\subsection{Cayley--Hamitonの定理の証明}

せっかくなので、Cayley--Hamiltonの定理の証明を与えておきます。Cayley自身は「自分では一般の次数の場合に証明しようと思わない」と言っちゃってる\footnote{\url{https://archive.org/stream/philtrans05474612/05474612\#page/n7/mode/2up}に書いてあります。}のですが、僕たちは頑張って証明をつけます。

\paragraph{固有多項式が重根を持たない場合}

まず最初に、固有多項式が重根を持たない場合を示しましょう。$n$次正方行列$A \in \Mat_n(\mathbb{R})$の固有多項式$\varphi_A(t)$が
\[
\varphi_A(t) = (t - \lambda_1) (t - \lambda_2) \cdots (t - \lambda_n)
\]
と分解したとします。このとき$A$は対角化可能です。
\[
P^{-1} A P = 
\begin{pmatrix}
\lambda_1 \\
& \lambda_2 \\
& & \ddots \\
& & & \lambda_n
\end{pmatrix}
\]
とすると、
\begin{align*}
\varphi_A(A) &= P \varphi_A(P^{-1} A P ) P^{-1}
= P (P^{-1} A P - \lambda_1 I) (P^{-1} A P - \lambda_2 I) \cdots (P^{-1} A P - \lambda_n I) P^{-1} \\
&= 
P 
\begin{pmatrix}
0 \\
& \lambda_2 - \lambda_1 \\
& & \ddots \\
& & & \lambda_n - \lambda_1
\end{pmatrix}
\begin{pmatrix}
\lambda_1 - \lambda_2 \\
& 0 \\
& & \ddots \\
& & & \lambda_n - \lambda_2
\end{pmatrix}
\cdots
\begin{pmatrix}
\lambda_1 - \lambda_n \\
& \lambda_2 - \lambda_n \\
& & \ddots \\
& & & 0
\end{pmatrix}
P^{-1} \\
&= O
\end{align*}
となります。

\paragraph{固有多項式が重根を持つような行列が「あまりない」こと}

さて、行列$A$に対して固有多項式$\varphi_A(t)$を対応させる写像を$\varphi\colon \Mat_n(\mathbb{R}) \rightarrow \mathbb{R}[t]$と書きます。このとき$\varphi_A(t)$が重根を持つような$A$全体のなす集合を$\Mat_n(\mathbb{R})^{\text{rs}}$と書くことにします。すると実は、$\Mat_n(\mathbb{R})$の中で$\Mat_n(\mathbb{R})^{\text{rs}}$の元は「あまりない」ことが示せます。

たとえば$2$次の場合、
\[
A = 
\begin{pmatrix}
a & b \\
c & d
\end{pmatrix}
\]
の固有多項式$\varphi_A(t) = t^2 - (a + d)t + (ad - bc)$が重根を持つ条件は、判別式を使って
\[
(a + d)^2 - 4(ad - bc) = 0
\]
と書けます。すなわち
\[
\Mat_2(\mathbb{R})^{\text{rs}} =
\biggl\{
A = 
\begin{pmatrix}
a & b \\
c & d
\end{pmatrix}
\in \Mat_2(\mathbb{R})
\mid
(a + d)^2 - 4(ad - bc) = 0
\biggr\}
\]
これは$a, b, c, d$という$4$つの変数の多項式の零点ですね。

実は$3$次以上の多項式に対しても「判別式」を作ることができます。それによって$\varphi_A(t)$の判別式は、$A$の成分の多項式として表せます。$\Mat_n(\mathbb{R})^{\text{rs}}$は、$\Mat_n(\mathbb{R})$における多項式の零点集合です。

ここで「多項式の零点集合」とはどんなものかを考えてみます。たとえば平面$\mathbb{R}^2$上で、$2$変数多項式$f(x, y) = y - x^2$の零点と言えば、放物線ですよね。平面全部の点に比べれば、放物線上に乗っている点の数は明らかに少ないです。もうちょっと正確に言えば、放物線上の点のどんな点でどんなに小さい半径の円を描いても、必ず$f(x, y) \neq 0$となる点が円の中に紛れ込みます。

$\varphi$の場合も、これと全く同じ状況が成り立っています。$\Mat_n(\mathbb{R})$は実数を$4$つ並べた行列全体の集合なので、線型空間としては$\mathbb{R}^n$と同型です。そこで$\varphi$を$\mathbb{R}^4$上の写像と同一視すれば、$\varphi_A(t) = 0$となる行列の零点は多項式の零点として書けます。次元は少々高いものの、さっきの放物線のように「空間全体の中で見れば潰れている」と言うことができます。こうして、固有多項式が重根を持つような行列は少ないことが示せます。

\paragraph{多項式写像の性質}

