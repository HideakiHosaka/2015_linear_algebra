\chapter{おまけ}
\lectureinfo{2015年7月22日 1限}

\section{最初に}
みなさんこんにちは。夏休み、いかがでしたか?

今回のテーマはS2タームの補足ということで、まずは行列の核と像について簡単に復習をします。夏休みを挟んで色々忘れてしまった人は、ちょっとリハビリをしてください。Aセメスターの頭に行列式を勉強する際にも、後々で基本変形を使います。また、S2ターム最後の授業で出された課題には、実は数学的に面白いネタが潜んでいました。そこで夏休み明け初回のこのプリントでは、ちょっと発展的なお話をいくつか紹介したいと思います。その後で最後に、この先の授業の展望を紹介します。

なお、今回は過去のプリントを参照する箇所が何ヶ所かあります。繰り返しになりますが、プリントをなくしてしまった人は
\begin{center}
\url{https://github.com/HideakiHosaka/2015_linear_algebra/raw/master/2015linear_algebra.pdf}
\end{center}
からダウンロードしてください。ちょっと重たいので、PCからダウンロードすることをお勧めします。


\section{復習: 行列の核と像の求め方}

最初に、問1から問3までの計算問題について復習しておきましょう。これらの問題は、いずれも「与えられた行列$A$に対し、$A$の核と像の基底を求める」という問題です。そして夏学期に勉強したのは
\begin{itemize}
\item 行列$A$を\uline{行}基本変形すると\uline{核}の基底が求まる
\item 行列$A$を\uline{列}基本変形すると\uline{像}の基底が求まる
\end{itemize}
という事実でした。列基本変形をすべき場面で行基本変形をしてしまっていた人が少なからずいたので、気を付けてください。行基本変形と列基本変形のやり方については、昔配った「掃き出し法」の回のプリントp. \pageref{section:swipe}と「線型写像と行列」の回のプリントp. \pageref{subsection:column_elementary_operation}を復習してください。

「行と列どっちだっけ」とこんがらがったときは、
\begin{itemize}
\item 核を求めること$=$連立一次方程式を解くこと
\item 連立一次方程式に対する消去法$=$ (拡大) 係数行列の行基本変形
\end{itemize}
を思い出しましょう。これを覚えていれば、「核を求めるときは行基本変形だ」と確信をもって言えるようになります。

また核と像両方の基底を求める問2, 3では、次元定理を使って計算を避けている人がいました。良い作戦だと思います。ただ次元定理は、次元に関する情報しか与えてくれません。たとえば問3なら、(1)で$\Ker A$を求めれば$\dim \Im A = 3$は分かります。$\Im A$は$A$の$4$本の列ベクトルで生成されますから、そのうち$1$次独立な$3$本を見つければ、それが$\Im A$の基底になります。でも「どの$3$本が当たりか」は次元を見ていても分かりません。きちんと$1$次独立性を確かめる必要があります。

\paragraph{問1--問3の解答}
\begin{itemize}
\item[問1] $(1, -2, 1)$
\item[問2] (1) $(1, -3, 1)$ (2) $(1, 0, 1), (0, 1, 2)$
\item[問3] (1) $(2, 2, -3, -3)$ (2) $(1, 0, 0 ,1), (0, 1, 0 ,0), (0, 0, 1 ,-2)$ \qed
\end{itemize}

\section{線型写像のなす線型空間}

今回はいくつかの問題で、行列のなす線型空間や、それらを定義域とする線型写像が登場します。「線型写像のなす線型空間で定義された線型写像」とか書いていると訳が分からなくなりそうですが、よく出てくるので、この機会にぜひ扱いに慣れてください。

\subsection{行列のなす線型空間}

線型空間の中で一番典型的なものは、列ベクトルの空間$\mathbb{R}^n$です。これが線型空間であることは既に知っているはずです。一方$(m, n)$型行列全体のなす集合を$\Mat_{m,n}(\mathbb{R})$と書いていました。これもまた線型空間になります。というのも、$\Mat_{m, n}(\mathbb{R})$には和とスカラー倍が定義されています。そして$\Mat_{m, n}(\mathbb{R})$と$\mathbb{R}^{mn}$とを見比べたとき、和とスカラー倍に関してだけいえば、成分が長方形に並んでいるか縦一列に並んでいるかの違いしかありません。ですから$\Mat_{m, n}(\mathbb{R})$が線型空間の公理を満たすことは明らかでしょう。

そして$\mathbb{R}^n$には\textbf{標準基底}と呼ばれる基底がありました。$i$番目の成分だけが$1$で他の成分が全て$0$であるようなベクトルを$\bm{e}_i$と書くとき、$(\bm{e}_1, \bm{e}_2, \ldots, \bm{e}_n)$が標準基底でした。各軸の方向に「単位ベクトル」を取ってきて、それを集めたものですね。同様にして$\Mat_{m, n}(\mathbb{R})$の場合も「$1$つの成分だけが$1$で他は$0$」という行列\footnote{こういう行列を\textbf{行列単位}といいます。}を全部集めてきて$1$列に並べれば、基底がます。そのことを確認しましょう。

\paragraph{問8の解答} $(i, j)$成分だけが$1$で他の成分が全て$0$であるような行列を$E_{ij} \in \Mat_{m, n}(\mathbb{R})$と書く\footnote{元の問8は$n$次正方行列が問題になっていますが、別に縦横のサイズが違っても問題の解き方は全く変わらないので、一般化して解きます。}。一般に行列$A = (a_{ij})_{\substack{1\leq i \leq m\\ 1\leq j\leq n}} \in \Mat_{m, n}(\mathbb{R})$に対し
\[
A=
\begin{pmatrix}
a_{11} & a_{12} & \cdots & a_{1n} \\
a_{21} & a_{22} & \cdots & a_{2n} \\
\vdots & \vdots & \ddots & \vdots \\
a_{m1} & a_{n2} & \cdots & a_{mn}
\end{pmatrix}
= 
\sum_{i = 1}^m \sum_{j = 1}^n a_{ij} E_{ij}
\]
が成り立つ。この式から、全ての行列は$E_{ij}\ (1\leq i\leq m, 1\leq j\leq n)$の$1$次結合で書けることが分かる。つまり$E_{ij}$たちは$\Mat_{m, n}(\mathbb{R})$を生成する。また$A = O$であれば、全ての$a_{ij}$が$0$になる。よって$E_{ij}$たちは$1$次独立である。よって$E_{ij}$たちを$1$列に並べた$(E_{11}, \ldots, E_{1n}, E_{21}, \ldots, E_{2n}, \ldots, E_{m1}, \ldots, E_{mn})$は$\Mat_{m, n}(\mathbb{R})$の基底である。\qed

\subsection{線型写像のなす線型空間}

次に、行列と線型写像の関係を復習しておきましょう。線型空間$V, W$の間の写像$f\colon V \rightarrow W$が線型写像であるとは、任意の$\bm{u}, \bm{v} \in V$と$\alpha \in \mathbb{R}$に対し$f(\bm{u} + \bm{v}) = f(\bm{u}) + f(\bm{v})$, $f(\alpha \bm{u}) = \alpha f(\bm{u})$が成り立つことでした\footnote{S2ターム初回のプリントp. \pageref{subsection:linear_map}にて、図式つきで線型写像の定義を説明しています。どうして「線型写像が、線型空間の間の良い写像と言えるのか」を忘れてしまった人は、参考にしてください。}。そして
\begin{itemize}
\item 行列のかけ算は数ベクトル空間の間の線型写像である
\item 線型空間を基底によって数ベクトル空間と同一視すると、線型写像は行列として表される
\end{itemize}
という事実がありました。基底によって、線型写像は行列とぴったり対応しています。だから$V$から$W$への線型写像全体の集合を$\Hom(V, W)$と書くことにすれば、行列の集合$\Mat_{m, n}(\mathbb{R})$が線型空間だったのと全く同様に、$\Hom(V, W)$も線型空間になります。

この線型空間の構造についても、復習しましょう。行列の和とスカラー倍は、$A, B$を行列、$\bm{u}$を数ベクトル、$\alpha$をスカラーとして
\[
(A + B) \bm{u} = A\bm{u} + B\bm{u}, (\alpha A)(\bm{u}) = \alpha (A\bm{u})
\]
という式を満たしていました。ですから一般の線型写像$f, g \in \Hom(V, W)$の場合にも、
\[
(f + g)(\bm{u}) := f(\bm{u}) + g(\bm{u}), (\alpha f)(\bm{u}) := \alpha f(\bm{u})
\]
によって写像$f + g, \alpha f \colon V \rightarrow W$を定めれば、これらは線型写像になります。こうして$\Hom(V, W)$に和とスカラー倍が定まります。線型写像と行列が大体同じだと思えば、$\Hom(V, W)$が線型空間になることは納得いくでしょう\footnote{S2タームの一番最初に配ったプリントのp.~\pageref{subsection:vector_space_of_linear_map}には$\Hom(V, W)$が線型空間になることのきちんとした証明もつけていますので、合わせて参考にしてください。}。

\subsection{行列の線型空間を定義域とする線型写像}

ここでもう一度確認しますが、$(m, n)$型行列全体の集合$\Mat_{m, n}(\mathbb{R})$は線型空間でした。また線型写像は、線型空間を定義域として、和とスカラー倍について良い振る舞いをする写像のことでした。したがって「$\Mat_{m, n}(\mathbb{R})$を定義域とする」線型写像を考えることができます。その例をいくつか見てみましょう。

なお、いつもは「線型写像が分からなくなったら行列に戻ることが大事だ」と言っていましたが、ここでは\textbf{一旦線型写像を行列と切り離して考える}方がいいです。線型写像の定義だけを使って、話を進めます。

\paragraph{問6 (1) の解答}
行列のトレースは
\[
\tr
\begin{pmatrix}
x_{11} & x_{12} \\
x_{21} & x_{22}
\end{pmatrix}
:= x_{11} + x_{22}
\]
で定まっていた。これは$2$次正方行列に対して数を対応させているので、$\tr\colon \Mat_2(\mathbb{R}) \rightarrow \mathbb{R}$という写像が定まっていることになる\footnote{見ればすぐ分かりますが、以下での議論は$n$次元でも全く同様に成り立ちます。}。そして
\begin{align*}
\tr
\Biggl\{
\begin{pmatrix}
x_{11} & x_{12} \\
x_{21} & x_{22}
\end{pmatrix}
+
\begin{pmatrix}
y_{11} & y_{12} \\
y_{21} & y_{22}
\end{pmatrix}
\Biggr\}
&=
\tr
\begin{pmatrix}
x_{11} + y_{11} & x_{12} + y_{12} \\
x_{21} + y_{22} & x_{22} + y_{22}
\end{pmatrix} = x_{11} + y_{11} + x_{22} + y_{22} \\
&= (x_{11} + x_{22}) + (y_{11} + y_{22})
= 
\tr
\begin{pmatrix}
x_{11} & x_{12} \\
x_{21} & x_{22}
\end{pmatrix}
+
\tr
\begin{pmatrix}
y_{11} & y_{12} \\
y_{21} & y_{22}
\end{pmatrix} \\
\tr
\Biggl\{
\alpha
\begin{pmatrix}
x_{11} & x_{12} \\
x_{21} & x_{22}
\end{pmatrix}
\Biggr\}
&=
\tr
\begin{pmatrix}
\alpha x_{11} & \alpha x_{12} \\
\alpha x_{21} & \alpha x_{22}
\end{pmatrix}
= \alpha x_{11} + \alpha x_{22} = \alpha (x_{11} + x_{22})
= \alpha
\tr
\begin{pmatrix}
x_{11} & x_{12} \\
x_{21} & x_{22}
\end{pmatrix}
\end{align*}
となっているので、$\tr\colon \Mat_2(\mathbb{R}) \rightarrow \mathbb{R}$は線型写像である。\qed

\paragraph{問9の解答} $n$次正方行列$A$を左からかける写像$m_A\colon \Mat_n(\mathbb{R}) \rightarrow \Mat_n(\mathbb{R})$は線型写像である。実際、任意の$X, Y \in \Mat_n(\mathbb{R})$と$\alpha \in \mathbb{R}$に対し
\[
m_A(X + Y ) = A (X + Y) = AX + AY = m_A(X) + m_A(Y) , m_A(\alpha X) = A (\alpha X) = \alpha AX = \alpha m_A(X)
\]
が成り立っている。またトレース$\tr\colon \Mat_n(\mathbb{R}) \rightarrow \mathbb{R}$も線型写像である。$t_A$は合成写像として$t_A := \tr \circ m_A$のように書けるので、やはり線型である。 \qed

\paragraph{問7の解答}
(1) $2$次正方行列$H \in \Mat_2(\mathbb{R})$を
\[
H := 
\begin{pmatrix}
1 & 0 \\
0 & -1
\end{pmatrix}
\]
で定め、写像$h$を$h(X) = [H, X] := HX - XH$で定める。$X$が$2$次正方行列なら$HX$, $XH$は共に$2$次正方行列だから$HX - XH$もそうであり、結果として$h$は$h\colon \Mat_2(\mathbb{R}) \rightarrow \Mat_2(\mathbb{R})$という写像を定めている。そして任意の$X \in \Mat_2(\mathbb{R})$と$\alpha \in \mathbb{R}$に対し
\begin{align*}
h(X + Y) &= H (X + Y) - (X + Y) H = HX - XH + HY - YH = h(X) + h(Y) \\
h(\alpha X) &= H (\alpha X) - (\alpha X) H = \alpha (HX - XH) = \alpha h(X)
\end{align*}
が成り立つので、$h$は線型写像である。

\noindent (3) $2$次の単位行列を$I$と書く。このとき$h(I) = HI - IH = H - H = O$である。また$h(H) = H^2 - H^2 = O$である。これより$I, H \in \Ker h$が従う。一方
\[
E := 
\begin{pmatrix}
0 & 1 \\
0 & 0
\end{pmatrix}, \quad
F := 
\begin{pmatrix}
0 & 0 \\
1 & 0
\end{pmatrix}
\]
と定めると、
\[
\begin{pmatrix}
a & b \\
c & d
\end{pmatrix}
= 
bE + \frac{a + d}{2} I + \frac{a - d}{2} H + cF
\]
と書ける。これと$h(E) = [H, E] = 2E$, $h(F) = [H, F] = -2F$より、$h$の線型性を使って
\[
h
\Biggl(
\begin{pmatrix}
a & b \\
c & d
\end{pmatrix}
\Biggr)
= 2b E - 2cF
\]
と計算できる。よって$\Im h$は$E, F$の張る$\Mat_2(\mathbb{R})$の部分空間に一致している。また$E, F$は$1$次独立なので、これらが$\Im h$の基底になっていることが分かる。

\noindent (2) 次元定理と (3) から$\dim \Ker h = \dim \Mat_2(\mathbb{R}) - \dim \Im h = 4 - 2 = 2$と分かる。一方$I, H \in \Ker h$は既に示しており、これらは$1$次独立である。故に$I, H$が$\Ker h$の基底をなす。\qed

\paragraph{問7の$h$の行列表示}

さて、今の問7を解く途中で出てきた
\[
\begin{pmatrix}
a & b \\
c & d
\end{pmatrix}
= 
bE + \frac{a - d}{2} H + cF + \frac{a + d}{2} I
\]
を思い出しましょう。この式より$(E, H, F, I)$は$\Mat_2(\mathbb{R})$の生成系です\footnote{基底として$(E, H, F, I)$を取ったのは、TAの個人的な数学的趣味によるものです。上手い基底を取ったおかげで、行列表示したときに対角行列でています。こういうとき、線型写像が\textbf{対角化されている}といいます。さらに言うとLie環$\mathfrak{sl}_2$の「表現の既約分解」「最高ウェイト表現」というものに対応した基底の並べ方をしているのですが、今はここまで気にしなくていいです。}。一方$\Mat_2(\mathbb{R})$は$4$次元ですから、この$(E, H, F, I)$は$\Mat_2(\mathbb{R})$の基底になります。この基底$(E, H, F, I)$について、$h$を行列表示してみましょう。

線型写像の行列表示とは、基底を用いて定義域と値域の線型空間を数ベクトル空間と同一視したとき、$h$が行列としてどう表されるかを見たものです。今の場合$\Mat_2(\mathbb{R})$の基底として$(E, H, F, I)$を取っていますから
\[
E
\longleftrightarrow
\begin{pmatrix}
1 \\
0 \\
0 \\
0
\end{pmatrix}, 
H
\longleftrightarrow
\begin{pmatrix}
0 \\
1 \\
0 \\
0
\end{pmatrix}, 
F
\longleftrightarrow
\begin{pmatrix}
0 \\
0 \\
1 \\
0
\end{pmatrix}, 
I
\longleftrightarrow
\begin{pmatrix}
0 \\
0 \\
0 \\
1
\end{pmatrix}
\]
という読み替えで$\Mat_2(\mathbb{R})$と$\mathbb{R}^4$を対応させます。これに合わせて$h(E) = 2E$, $h(F) = -2F$, $h(H) = h(I) =O$を読み替えると、$h$は
\begin{align*}
\begin{pmatrix}
1 \\
0 \\
0 \\
0
\end{pmatrix}
\longleftrightarrow
E
\xmapsto{\text{\hbox to 2zw{\hfil$h$\hfil}}}
2E
&\longleftrightarrow
\begin{pmatrix}
2 \\
0 \\
0 \\
0
\end{pmatrix}, 
&
\begin{pmatrix}
0 \\
1 \\
0 \\
0
\end{pmatrix}
\longleftrightarrow
H
\xmapsto{\text{\hbox to 2zw{\hfil$h$\hfil}}}
O
&\longleftrightarrow
\begin{pmatrix}
0 \\
0 \\
0 \\
0
\end{pmatrix}
\\
\begin{pmatrix}
0 \\
0 \\
1 \\
0
\end{pmatrix}
\longleftrightarrow
F
\xmapsto{\text{\hbox to 2zw{\hfil$h$\hfil}}}
-2F
&\longleftrightarrow
\begin{pmatrix}
0 \\
0 \\
-2 \\
0
\end{pmatrix}, 
&
\begin{pmatrix}
0 \\
0 \\
0 \\
1
\end{pmatrix}
\longleftrightarrow
I
\xmapsto{\text{\hbox to 2zw{\hfil$h$\hfil}}}
O
&\longleftrightarrow
\begin{pmatrix}
0 \\
0 \\
0 \\
0
\end{pmatrix}
\end{align*}
という対応を与えることが分かります。この対応を実現する行列は、$h$の行先となる列ベクトルを並べた
\[
\left(
\begin{array}{rrrr}
2 & 0 & 0 & 0 \\
0 & 0 & 0 & 0 \\
0 & 0 & -2 & 0 \\
0 & 0 & 0 & 0
\end{array}
\right)
\]
となります。

ここで$h$は、元々$\Mat_2(\mathbb{R})$の
\[
\begin{pmatrix}
1 & 0 \\
0 & -1
\end{pmatrix}
\]
という行列を使い、$h(X) = HX - XH$という式で定義された写像でした。ところが上で導いた$h$の行列表示は、定義に使った行列$H$や、あるいは基底の行列$E, F, I$と特に関係を見出せません。これは偶然というわけではなく、一般に行列の空間$\Mat_{m, n}(\mathbb{R})$を定義域とする線型写像$f$について、\textbf{「$f$の中に放り込まれる行列」と「$f$の行列表示」との間に関係はありません}。

さっき「線型写像と行列を切り離した方が良い」といったのは、関係ない$2$つの行列がごっちゃになるのを防ぐためでした。一度切り分け方が分かってしまえば、こんがらがることは無くなると思います。もし途中でつっかえたら、落ち着いて問7を最初から追いかけてください。

また今は$h$の行列表示を求めてみましたが、一般の場合に行列表示を求める必要があるかというと、そうでもありません。たとえば$n$次行列のなす線型空間の次元は$\dim \Mat_n(\mathbb{R}) = n^2$なので、$n$が$3$とか$4$になるだけで空間の次元はすごく大きくなります。そうした空間を定義域とする線型写像を考えると、行列表示をしたら行列の$1$辺が$9$列とか$16$列になってしまい、書くだけで大変です。その上そもそも上に書いた通り、行列表示をしたからといって意味のある情報が出てくるわけでもありません\footnote{行列表示の例として問7の$h$を選んだのは、「対角化」という意味のある例だったからです。}。ですので行列のなす線型空間が絡む残りの問題についても、行列表示は使わずに問題を解いてみます。

\paragraph{問6 (2), (3) の解答} $2$次正方行列$X = (x_{ij})$について
\[
\tr
\begin{pmatrix}
x_{11} & x_{12} \\
x_{21} & x_{22}
\end{pmatrix}
= x_{11} + x_{22}
\]
なので、トレースが$0$であることと$x_{22} = -x_{11}$は同値である。よって全ての$X \in \Ker \tr$は
\[
\begin{pmatrix}
x_{11} & x_{12} \\
x_{21} & x_{22}
\end{pmatrix}
=
x_{11} H + x_{12} E + x_{21} F
\]
と表せる\footnote{$H, E, F$は問$7$の解答に出てきた行列です。}。また$x_{11} H + x_{12} E + x_{21} F = O$ならば$x_{11} = x_{12} = x_{21} = 0$となる。よって$(H, E, F)$は$1$次独立でもある。これより$\Ker \tr$の基底として$(H, E, F)$が取れる。

また、任意の$a \in \mathbb{R}$に対して
\[
\tr
\begin{pmatrix}
a & 0 \\
0 & 0
\end{pmatrix}
=
a
\]
が成り立つので、$\tr\colon \Mat_2(\mathbb{R}) \rightarrow \mathbb{R}$は全射である。特に$\Im \tr = \mathbb{R}$は$1$次元で、その基底として$1$が取れる。 \qed

\section{線型空間の直和分解}

今回の問4, 5には、線型空間の直和分解というものが見えています。これは一言で言えば「空間を$2$つの部分空間の方向に分ける」という操作に相当します。たとえば線型空間の基底を取ってくれば、線型空間を各基底の方向にバラすことができます。でも時と場合によっては、基底を取ってまで細かくしなくても、ほどほどに空間をバラしたいことがあります。そこで登場するのが、直和分解という考え方です。

\subsection{直和の定義}

前回のプリントの再掲になりますが、直和の定義と性質を確認しておきましょう。$U$を線型空間とし、$V, W\subset U$をその部分空間とします。そして「任意の$\bm{u} \in U$が、$\bm{u} = \bm{v} + \bm{w}$ ($\bm{v} \in V, \bm{w}\in W$)の形にただ一通りに書ける」という条件が成り立つとき、$U$は$V$と$W$の\textbf{直和}である\footnote{正確には、ここで定義した直和は「内部直和」といいます。この他に「外部直和」というものがあって、慣れてくると内部直和と外部直和をどっちも「直和」と略すようになります。事実としては内部直和と外部直和は自然に同型になるので、そんな深く気にすることはありません。でも他の場所で「この場所は、僕の知ってる直和と違うな?」と思ったときは、どっちの直和なのかを考えてみてください。}といい、$U = V \oplus W$と書きます。

直感的に言えば、これは「$U$を$V$方向と$W$方向に分ける」ことを表しています。ですから直和の条件は、次のようにも言い換えられます。
\begin{itemize}
\item $V$の基底と$W$の基底を連結すると、$U$全体の基底が得られる。
\item $U = V + W$かつ$V\cap W = \{\bm{0}\}$が成り立つ。
\item $V \cap W = \{\bm{0}\}$かつ$\dim V + \dim W = \dim U$が成り立つ。
\end{itemize}
$3$つ以上の空間の直和についても定義は同様です。

この中でも最も典型的な例は基底でしょう。線型空間$V$について、$(\bm{f}_1, \ldots, \bm{f}_n)$が基底であることは、全ての$\bm{v} \in V$が$(\bm{f}_1, \ldots, \bm{f}_n)$の一次結合でただ一通りに書けることでした。一方、$\bm{f}_i$の張る$V$の部分空間を$\mathbb{R}\bm{f}_i$と書いていました。ですから$(\bm{f}_1, \ldots, \bm{f}_n)$が基底であるという事実は、直和の記号を使って表せば$V = \mathbb{R}\bm{f}_1 \oplus \cdots \oplus \mathbb{R}\bm{f}_n$となります。こんな感じで、問1から問3の答えを書き直してみましょう。

\paragraph{問1--問3の解答}
\begin{itemize}
\item[問1] $\Ker f = \mathbb{R}\,{}^t(1, -2, 1)$
\item[問2] (1) $\Ker f = \mathbb{R}\,{}^t(1, -3, 1)$ (2) $\Im f = \mathbb{R}\,{}^t(1, 0, 1) \oplus \mathbb{R}\,{}^t(0, 1, 2)$
\item[問3] (1) $\Ker f = \mathbb{R}\,{}^t(2, 2, -3, -3)$ (2) $\Im f = \mathbb{R}\,{}^t(1, 0, 0 ,1) \oplus \mathbb{R}\,{}^t(0, 1, 0 ,0) \oplus \mathbb{R}\,{}^t(0, 0, 1 ,-2)$ \qed
\end{itemize}

\subsection{直和分解の例}

さて既に述べた通り、直和分解が真に役立つのは「基底よりもう少し大きいブロックに空間を分ける」という場合です。基底を持ち出して話が進むなら、わざわざ直和を持ち出す理由がないですからね。そこで行列のなす線型空間$\Mat_n(\mathbb{R})$で、典型的な直和分解の例を確認します。

\paragraph{問4, 問5の解答}

$f, g\colon \Mat_2(\mathbb{R}) \rightarrow \Mat_2(\mathbb{R})$を、$f(X) := X + {}^t X$, $g(X) := X - {}^t X$と定める。このとき
\begin{itemize}
\item $f(X) = O$と${}^t X = -X$は同値なので、$\Ker f = \Alt_2(\mathbb{R})$
\item $g(X) = O$と${}^t X = X$は同値なので、$\Ker g = \Sym_2(\mathbb{R})$
\end{itemize}
である\footnote{$\Sym_n(\mathbb{R})$は$n$次対称行列、$\Alt_n(\mathbb{R})$は$n$次交代行列のなす線型空間です。}。一方
\begin{itemize}
\item 任意の$X \in \Mat_2(\mathbb{R})$に対し${}^t f(X) = {}^t(X + {}^t X) = {}^t X + X = f(X)$だから、$\Im f \subset \Sym_2(\mathbb{R})$
\item $X \in \Sym_2(\mathbb{R})$に対し、$f(X/2) = (X + {}^t X)/2 = X$となるので、$\Sym_2(\mathbb{R}) \subset \Im f$
\end{itemize}
だから、$\Im f = \Sym_2(\mathbb{R})$である。同様にして
\begin{itemize}
\item 任意の$X \in \Mat_2(\mathbb{R})$に対し${}^t g(X) = {}^t(X - {}^t X) = {}^t X - X = g(X)$だから、$\Im f \subset \Alt_2(\mathbb{R})$
\item $X \in \Alt_2(\mathbb{R})$に対し、$f(X/2) = (X - {}^t X)/2 = X$となるので、$\Alt_2(\mathbb{R}) \subset \Im f$
\end{itemize}
だから、$\Im f = \Alt_2(\mathbb{R})$である。

これより、残りの問題は$\Sym_2(\mathbb{R})$, $\Alt_2(\mathbb{R})$の基底を求めることに帰着された。一般に行列
\[
A =
\begin{pmatrix}
a & b \\
c & d
\end{pmatrix}
\]
が対称行列であることは$b = c$と、交代行列であることは「$b = -c$かつ$a = d = 0$」とそれぞれ同値である。よって全ての対称行列は
\[
\begin{pmatrix}
a & b \\
b & d
\end{pmatrix}
=
a
\begin{pmatrix}
1 & 0 \\
0 & 0
\end{pmatrix}
+ b
\begin{pmatrix}
0 & 1 \\
1 & 0
\end{pmatrix}
+ d
\begin{pmatrix}
0 & 0 \\
0 & 1
\end{pmatrix}
\]
の形に表せ、全ての交代行列は
\[
\begin{pmatrix}
0 & -c \\
c & 0
\end{pmatrix}
= c
\begin{pmatrix}
0 & -1 \\
1 & 0
\end{pmatrix}
\]
の形に表せる。これらの表式は一意的なので、基底が求まったことになる。\qed

\paragraph{対称行列と交代行列による分解}

今の問4, 5に出てきた$f, g\colon \Mat_2(\mathbb{R}) \rightarrow \Mat_2(\mathbb{R})$は、全ての$n$次正方行列$X$に対し$f(X) + g(X) = 2X$を満たしています。この両辺を$2$で割れば
\[
X = \frac{1}{2} f(X) + \frac{1}{2} g(X)
= \frac{X + {}^tX}{2} + \frac{X - {}^tX}{2}
\]
という式が得られます。注目すべきは、右辺で第$1$項が$\Im f = \Sym_2(\mathbb{R})$に、第$2$項が$\Im g = \Alt_2(\mathbb{R})$に入っていることです。つまり全ての行列は、対称行列と交代行列の和に書けます。これより$\Mat_2(\mathbb{R}) = \Sym_2(\mathbb{R}) + \Alt_2(\mathbb{R})$だと分かります。

さらに$X \in \Sym_2(\mathbb{R}) \cap \Alt_2(\mathbb{R})$だとしたら、${}^tX = X$かつ${}^tX = -X$が成り立つので、$X = O$です。よって$\Sym_2(\mathbb{R}) \cap \Alt_2(\mathbb{R}) = \{O\}$なので、$\Mat_2(\mathbb{R})$は$\Sym_2(\mathbb{R})$と$\Alt_2(\mathbb{R})$の直和で$\Mat_2(\mathbb{R}) = \Sym_2(\mathbb{R}) \oplus \Alt_2(\mathbb{R})$が成り立っています。つまり\textbf{全ての行列は、対称行列と交代行列の和にただ一通りに書ける}というわけです。この分解は知っておいて損はありませんので、心の片隅に留めておいてください。

\section{行列のなすLie環}

今回の問7には「交換子」と呼ばれるものが登場します。問7では行列の交換子だけを計算しますが、実は行列に限らず、交換子は世の中の色々な場面で登場することが知られています。その性質を、少し突っ込んで調べましょう。

なお、以下の話では具体例としてちょくちょく量子力学が顔を出します。別に「量子力学を知らなければ線型代数が理解できない」というわけではないですが、これを機に量子力学と線型代数を関連付けて勉強するのも良いでしょう。たとえば、駒場キャンパスにいらっしゃる総合文化研究科の清水明先生が書かれた『新版 量子論の基礎』(サイエンス社) は手頃で良い本だと思います。このプリントに出てくる量子力学の話は、ほとんどこの本に書いてあります。

\subsection{交換子}

$V$を線型空間とし、$X, Y \in \End(V)$を$V$上の線型変換とします。このとき$[X, Y] := XY - YX$を$X$と$Y$の\textbf{交換子}といいます。

最も典型的な例は、行列の交換子です。たとえばS1タームの終わりの方に、一度演習問題で
\[
H := 
\begin{pmatrix}
1 & 0 \\
0 & -1
\end{pmatrix}, \quad
E := 
\begin{pmatrix}
0 & 1 \\
0 & 0
\end{pmatrix}, \quad
F := 
\begin{pmatrix}
0 & 0 \\
1 & 0
\end{pmatrix}
\]
の交換子を調べる問題が登場しました\footnote{\pageref{paragraph:commutator}ページにあるので、思い出してみてください。}。また$X$や$Y$は行列で書かれている必要もありません。たとえば実数値の無限回微分可能な函数全体の集合$V = C^{\infty}(\mathbb{R})$の上で、$X$は微分$\frac{d}{dx}$、$Y$は$x$をかける線型写像だとします。このとき$f \in C^{\infty}(\mathbb{R})$に対し
\[
[X, Y]f(x) = XY f(x) - YX f(x) = \frac{d}{dx} \bigl(x f(x)\bigr) - x\Bigl(\frac{d}{dx} f(x)\Bigr) = f(x) + x f'(x) - x f'(x) =  f(x)
\]
が成り立ちます。

\paragraph{交換子の性質}

交換子は次の性質を満たします。
\begin{itemize}
\item 双線型性: $[\alpha X_1 + \beta X_2, Y] = \alpha[X_1, Y] + \beta[X_2, Y]$, $[X, \alpha Y_1 + \beta Y_2] = \alpha [X, Y_1] + \beta [X, Y_2]$
\item 交代性: $[Y, X] = -[X, Y]$
\item Jacobi恒等式: $\bigl[X, [Y, Z]\bigr] + \bigl[Y, [Z, X]\bigr] + \bigl[Z, [X, Y]\bigr] = 0$
\end{itemize}
これは定義式$[X, Y] = XY - YX$を逐一代入して計算するだけで証明ができます。試してみてください。

\subsection{Lie環}

さて、いまは行列などの掛け算を使って交換子$[, ]$を定義しました。ですが色々計算をしてみると、時と場合によっては\textbf{交換子の性質だけを使って議論を進められたりします}。そこで一般に線型空間$\mathfrak{g}$\footnote{`$\mathfrak{g}$'は、アルファベット`g'のフラクトゥール体です。Lie環の研究の創始者である数学者Sophus LieがLie環を表すのにフラクトゥール体を使ったことが、今では数学者の間で標準的な習わしになっています。}の上でJacobi恒等式を満たす交代双線型写像$[, ]\colon \mathfrak{g}\times\mathfrak{g}\rightarrow\mathfrak{g}$が定義されているとき、$\mathfrak{g}$を\textbf{Lie環}といいます。$n$次正方行列の集合$\Mat_n(\mathbb{R})$はLie環の一例です。

\paragraph{なぜLie環を考えるのか?}

僕たちはなぜ、わざわざLie環などという対象を定義し、その性質を調べるのでしょうか?

答え方には色々ありますが、一つの理由としては\textbf{Lie環に現れる関係式が、色々なところで登場する}という事実が挙げられます。たとえば複素数を成分とする$2$次正方行列\footnote{授業中に複素数成分の行列を表だって扱ったことはないですが、計算の仕方は実数成分のときと全く同じです。}
\[
\sigma_x := 
\frac{1}{2}
\begin{pmatrix}
0 & 1 \\
1 & 0
\end{pmatrix}, \quad
\sigma_y := 
\frac{1}{2}
\begin{pmatrix}
0 & -i \\
i & 0 
\end{pmatrix}, \quad
\sigma_z := 
\frac{1}{2}
\begin{pmatrix}
1 & 0 \\
0 & -1
\end{pmatrix}
\in \Mat_2(\mathbb{C})
\]
は$[\sigma_x, \sigma_y] = i\sigma_z, [\sigma_y, \sigma_z] = i\sigma_x, [\sigma_z, \sigma_x] = i\sigma_y$という式を満たします。一方で$3$変数の函数に対する微分作用素$L_x, L_y, L_z$を
\[
L_x := y\frac{\partial}{\partial z} - z\frac{\partial}{\partial y}, 
L_y := z\frac{\partial}{\partial x} - x\frac{\partial}{\partial x}, 
L_z := x\frac{\partial}{\partial y} - y\frac{\partial}{\partial z}
\]
で定める\footnote{つまり「$3$変数函数$f(x, y, z)$に対し$L_x f := yf_z - zf_y$などと定める」という意味です。}と、これらもまた$[L_x, L_y] = L_z, [L_y, L_z] = L_z, [L_z, L_x] = L_y$という関係式を満たします。これは$\sigma_x, \sigma_y, \sigma_z$の交換関係と大体同じです\footnote{$\sigma'_x := -i\sigma_x$, $\sigma'_y := -i\sigma_y$, $\sigma'_z := -i\sigma_z$とおくと、本当に$L_x$, $L_y$, $L_z$の交換関係と同じ式を満たすようになります。}。

また量子力学の勉強をしていると、必ず$[\hat{x}, \hat{p}] = 1$という式が登場します\footnote{この式を\textbf{正準交換関係}といいます。量子力学の授業だと普通$[\hat{x}, \hat{p}] = i\hbar$という形で出てきますが、右辺が$1$でも$i\hbar$でもLie環を考えるにあたっては大差ないので、数学では$[x, p] = 1$としてしまうことが多いです。}。たとえば交換子の例として微分作用素$X = \frac{d}{dx}$と$x$倍作用素$Y$を挙げましたが、これらは$[X, Y] = 1$を満たしていました。この$X$と$Y$は、正準交換関係を満たす演算子の\textbf{Schr\"odinger表現}といいます。他にも正準交換関係を満たす演算子は色々ありますが、とにかく$[\hat{x}, \hat{p}] = 1$を満たすことが大事なのです。

このように「同じ交換関係式が、時と場所を変えて色々な状況で出現する」という実態を知ると「一々個別に調べるのは面倒だから、交換関係だけに注目して、その性質を調べ上げてしまえ」という発想に至るのは自然なことでしょう。こうして「特定の交換関係だけ」を抽出したものがLie環で、その交換関係を満たす行列や演算子のことを「Lie環の表現」といいます。

\paragraph{Lie環の由緒}

さらに言うと、Lie環の役割を一段と良く理解するには「なぜLie環の交換関係が色々なところに登場するか」を知らないといけません。そのキーワードが「Lie群」と「対称性」です。

たとえば水素原子を考えると、原子核は陽子$1$個だけからなります。陽子の作る電場は、原点を通るあらゆる軸に対する回転対称性を持ちます。回転対称性以外にも対称性には色々な種類がありますが、数学では対称性を表すのに「群」という言葉を使います。Lie群はその中でも特別な群であり、Lie群の対称性を「無限小」のレベルで見ようとするとLie環が登場します。いよいよLie環が重要そうな気がしてきますね\footnote{なおLie環の対称性は、いつもLie群に付随して現れるわけではありません。Lie環の対称性が単独で現れる状況もあります。}。

ですがLie群とLie環の対応を述べるには、行列の指数函数\footnote{指数函数$e^x = \exp x$をTaylor展開した式には、正方行列を代入することができます。その式で行列の指数函数を定義します。}の性質を詳しく調べなければいけません。事実としては、全ての行列$X, Y \in \Mat_n(\mathbb{R})$に対し
\[
e^{tX} e^{tY} e^{-tX} e^{-tY} = \exp\Bigl( t^2 [X, Y] + (\text{$t^3$より高次の項})\Bigr)
\]
が成り立つことが、Lie環の交換子$[, ]$が重要な理由です。でもこの書き方では「何で$e^{tX} e^{tY} e^{-tX} e^{-tY}$という式が出てくるのか」を説明できていません。これを理解するには少し「群論」というものを勉強する必要があります。行列式の話をする時に少し群論が出てくると思いますので、Lie群とLie環の対応を述べるのはその機会に譲りたいと思います。

\subsection{Lie環$\mathfrak{sl}_n$}

ここまで出てきた具体的なLie環は$\Mat_n(\mathbb{R})$しかありませんが、これ以外にも$\mathfrak{sl}_n$という重要なLie環があるので、それを紹介します。中でも$\mathfrak{sl}_2$は、Lie環の理論を展開する上でも、あるいは量子力学等への応用を考えるにしても、非常に重要な役割を果たします。

\paragraph{定義}

一般に$\tr\colon \Mat_n(\mathbb{R}) \rightarrow \mathbb{R}$の核を$\mathfrak{sl}_n(\mathbb{R})$と書きます\footnote{この$\mathfrak{sl}$も、``sl''のフラクトゥール体です。}。任意の$X, Y \in \Mat_n(\mathbb{R})$に対して$\tr XY = \tr YX$なので、特に$X, Y\in\mathfrak{sl}_n(\mathbb{R})$に対しても$[X, Y] \in \mathfrak{sl}_n(\mathbb{R})$が成り立ちます。よって$\mathfrak{sl}_n(\mathbb{R})$は$[, ]$で閉じており、Lie環になります。問7の議論を使えば、$\dim \mathfrak{sl}_n(\mathbb{R}) = n - 1$と分かります。

\paragraph{$\mathfrak{sl}_2(\mathbb{C})$とPauli行列}

ここで数の範囲を広げて、複素係数の行列で$\mathfrak{sl}_2(\mathbb{C})$を考えます。既に$E, H, F$という基底が$\mathfrak{sl}_2(\mathbb{C})$に取れることを知っていますが、複素係数の場合は特に
\[
\sigma_x := 
\frac{1}{2}
\begin{pmatrix}
0 & 1 \\
1 & 0
\end{pmatrix}, \quad
\sigma_y := 
\frac{1}{2}
\begin{pmatrix}
0 & -i \\
i & 0 
\end{pmatrix}, \quad
\sigma_z := 
\frac{1}{2}
\begin{pmatrix}
1 & 0 \\
0 & -1
\end{pmatrix}
\]
という別の基底が取れます。これらの行列を\textbf{Pauli行列}といいます。

Pauli行列の特徴的な点は、Hermite行列であること\footnote{複素数を成分とする行列$X$が$\overline{{}^tX} = X$を満たすとき、$X$を\textbf{Hermite行列}といいます。}です。Hermite行列はいつも「固有値」と呼ばれるものが実数になります。そして量子力学では
\begin{itemize}
\item 物理系の状態は、複素線型空間のベクトルで表される
\item 物理量はHermite行列で表される
\item ある状態における物理量の観測値は、物理量を表すHermite行列の固有値のどれかから確率的に選択される
\end{itemize}
という事情があります。物理量の観測値が実数になることを、Hermite性が保証してくれるのです。なので$\mathfrak{sl}_2(\mathbb{C})$の基底としては$(E, H, F)$の方が便利でも、物理量を考えるときは$(\sigma_x, \sigma_y, \sigma_z)$が好まれたりします。たとえば電子の持つスピンの観測値は、そのまま$\sigma_x, \sigma_y, \sigma_z$の固有値になることが知られています。

\subsection{随伴表現}
最後に、問7に出てきた線型写像$h$がどうして大事なのか、軽く紹介します。

$\mathfrak{g}$をLie環とします\footnote{慣れなければ$\mathfrak{g} = \Mat_n(\mathbb{R})$と思っていいです。}。交換子$[, ]$は双線型なので、片側については線型になっています。そこで$X \in \mathfrak{g}$に対して$\ad X(Y) := [X, Y]$と定義すれば、$\ad X \colon \mathfrak{g} \rightarrow \mathfrak{g}$という線型写像、つまり$\ad X \in \End(\mathfrak{g})$が得られます。これには\textbf{随伴表現}という名前がついています。

交換子がJacobi恒等式を満たすことから、$\ad X$について次が成り立ちます。
\begin{itemize}
\item $\ad X\bigl([Y, Z]\bigr) = [\ad X(Y), Z] + [X, \ad Y(Z)]$
\item $\ad {[X, Y]} (Z) = [\ad X, \ad Y] (Z)$
\end{itemize}
そして$\ad X \in \End(\mathfrak{g})$の$X \in \mathfrak{g}$を動かすことで、$\ad\colon \mathfrak{g} \rightarrow \End(\mathfrak{g})$という写像が得られます。$\mathfrak{g}$は線型空間なので、$\End(\mathfrak{g})$もまたLie環です。そして今の$2$つ目の式$\ad {[X, Y]} = [\ad X, \ad Y]$は、写像$\ad$がLie環の構造と整合的であることを表しています。
\[
\begin{tikzcd}
\mathfrak{g} \times \mathfrak{g} \arrow{r}{\ad\times\ad} \arrow{d}[swap]{[,]} & \End(\mathfrak{g}) \times \End(\mathfrak{g}) \arrow{d}{[,]} \\
\mathfrak{g} \arrow{r}{\ad} & \End(\mathfrak{g})
\end{tikzcd}
\]
ここで線型写像が「線型空間の構造と整合的な写像」だったことを思い出せば、Lie環の間の写像を考えるときも、整合的な写像を考えるのは自然だと思えるはずです。特に$\ad$は、どんなLie環$\mathfrak{g}$にももれなくついてくるという点で、とりわけ重要なものです。

Lie環論と呼ばれる分野では、Lie環の構造に整合的な線型写像をうまく使うことで、Lie環の構造を調べます。その中でも$\ad - \alpha \id$ ($\alpha \in \mathbb{R}$) の形をした写像の核は$\ad$の固有空間と呼ばれ、極めて重要な役割を果たします。線型代数を使いこなすことの良い具体例だったので、今回ちょっと詳しく取り上げてみました。

\section{双対空間}

一般に線型空間$V$があると、もれなく「$V$の双対空間」と呼ばれる線型空間$V^*$が定義できます。$V^*$のことを知っていると線型代数の理解が一段と深まりますし、また相対論を勉強する時に出てくる「共変/反変テンソル」というものを理解するのにも役立ちます。今回のプリントの最初に説明した「線型写像の線型空間」の例だから、慣れないと多少こんがらがるかもしれません。でも双対空間を考えてみることは、抽象論の良い訓練になるでしょう。

\subsection{双対空間の定義}

線型空間$V, W$があったとき、$V$から$W$への線型写像全体の集合$\Hom(V, W)$も再び線型空間になるのでした。ここで特に$W = \mathbb{R}$のとき、$V^* := \Hom(V, \mathbb{R})$と書き、$V$の\textbf{双対 (そうつい) 空間}といいます。

\paragraph{$\mathbb{R}^n$の双対空間} 一番簡単な$\mathbb{R}^2$の場合に、双対空間$(\mathbb{R}^2)^*$は何かを考えてみましょう。今の場合、$\mathbb{R}^2$から$\mathbb{R}$への線型写像は$(1, 2)$型行列、つまり行ベクトルと同じです。よって$\Hom(\mathbb{R}^2, \mathbb{R}) = \Mat_{1, 2}(\mathbb{R})$です。$\mathbb{R}^n$の場合でも全く話は同じで、列ベクトル全体のなす数ベクトル空間$\mathbb{R}^n$の双対空間$(\mathbb{R}^n)^*$は、同じ長さの行ベクトル全体がなす空間です。

\subsection{双対基底}

線型空間$V$に基底$(\bm{f}_1, \bm{f}_2, \ldots, \bm{f}_n)$が与えられると、それに応じて自然に$V^*$の基底$(\bm{f}^{\vee}_1, \bm{f}^{\vee}_2, \ldots, \bm{f}^{\vee}_n)$が定義されます。この作り方を今から説明します。

まず$V$を定義域とする線型写像は、基底の行先を決めればただ一通りに決まります。そこで$\bm{f}^{\vee}_i\colon V \rightarrow\mathbb{R}$を
\begin{align*}
\bm{f}^{\vee}_i(\bm{f}_j) := \delta_{ij} =
\begin{cases}
1 & (i = j) \\
0 & (i \neq j)
\end{cases}
\end{align*}
と定めます。これから示すように、$(\bm{f}^{\vee}_1, \ldots, \bm{f}^{\vee}_n)$は$V^*$の基底になります。これを$(\bm{f}_1, \ldots, \bm{f}_n)$の\textbf{双対基底}といいます。まず具体例を見てみましょう。

\paragraph{$\mathbb{R}^n$の場合} $\mathbb{R}^n$の標準基底が相手の場合、双対基底も非常に分かり易いものになります。$\bm{e}_i$を転置してできる行ベクトルを$\bm{e}^{\vee}_i$と書くと
\[
\bm{e}^{\vee}_i (\bm{e}_j)
=
\bordermatrix{
& & & & \scalebox{0.8}{$i^{\text{th}}$} \cr
& 0 & \cdots & 0 & 1 & 0 & \cdots & 0
}
\begin{pmatrix}
0 \\
\vdots \\
1 \\
\vdots \\
0
\end{pmatrix}
= \delta_{ij}
\]
が成り立つので、$(\bm{e}_1, \ldots, \bm{e}_n)$の双対基底$(\bm{e}^{\vee}_1, \ldots, \bm{e}^{\vee}_n)$は、各$\bm{e}_i$を行列だと思って転置することで得られます。この例では、双対基底がちゃんと双対空間の基底をなしていることがほとんど明らかです	。

\paragraph{双対基底が基底になること} 例を確認したところで、一般の場合に話を戻しましょう。$V$を線型空間とし、その基底$(\bm{f}_1, \bm{f}_2, \ldots, \bm{f}_n)$を取ります。

さっき定義した$\bm{f}^{\vee}_1, \bm{f}^{\vee}_2, \ldots, \bm{f}^{\vee}_n$は$1$次独立です。実際$\alpha_1 \bm{f}^{\vee}_1 + \alpha_2 \bm{f}^{\vee}_2 + \cdots + \alpha_n \bm{f}^{\vee}_n = 0 \in V^*$とおく\footnote{全てのベクトルに$0$を対応させる写像は、線型写像です。これを\textbf{零写像}といい、$0$で表します。}と、この写像に何を代入しても値は$0$です。そこで$\bm{f}_j$ ($j = 1, 2, \ldots, n$) を代入すると
\[
0 = \sum_{i = 1}^n \alpha_i \bm{f}^{\vee}_i (\bm{f}_j) = \sum_{i = 1}^n \alpha_i \delta_{ij} = \alpha_j
\]
となるので、$\alpha_1 = \alpha_2 = \cdots = \alpha_n = 0$となります。これで$1$次独立性がいえました。

また、全ての$f \in V^*$は$\bm{f}^{\vee}_1, \bm{f}^{\vee}_2, \ldots, \bm{f}^{\vee}_n$の$1$次結合で書けます。実際$f \in V^*$に対し、$f' := f(\bm{f}_1)\bm{f}^{\vee}_1 + f(\bm{f}_2)\bm{f}^{\vee}_2 + \cdots + f(\bm{f}_n)\bm{f}^{\vee}_n$とおくと、各$\bm{f}_j$ ($j = 1, 2, \ldots, n$) を$f'$に代入して
\[
f'(\bm{f}_j) = \sum_{i = 1}^n f(\bm{f}_i)\bm{f}^{\vee}_i(\bm{f}_j) = \sum_{i = 1}^n f(\bm{f}_i) \delta_{ij} = f(\bm{f}_j)
\]
が得られます。すなわち基底$(\bm{f}_1, \bm{f}_2, \ldots, \bm{f}_n)$の全ての行先が$f$と$f'$とで同じなので、写像として$f = f'$が従います。これで$V^*$が$f \in V^*$は$\bm{f}^{\vee}_1, \bm{f}^{\vee}_2, \ldots, \bm{f}^{\vee}_n$で生成されることが示せました。以上より$(\bm{f}^{\vee}_1, \bm{f}^{\vee}_2, \ldots, \bm{f}^{\vee}_n)$は$V$の基底です。

このことより、特に$\dim V^* = \dim V$が従います。また一般の線型空間$V$は「標準的」と呼ぶべき基底を持つとは限りませんが、$V$の基底を決めたなら、それに応じて$V^*$に双対基底という自然な基底が定まります。言い方を変えれば「$V$に座標を入れたら、それに応じて$V^*$に自然な座標の入れ方が決まる」と言えます。

\paragraph{問10の解答} 双対基底の本数を数えれば$\dim V^* = \dim V$が従う。 \qed

\subsection{$2$回双対}

線型空間$V$から双対空間$V^*$を作ったのと同じようにして、今度は$V^*$の双対空間$V^{**} := (V^*)^*$を考えることができます。$\mathbb{R}^n$の場合を思い出すと、双対$*$を取ることは行列としての転置と同じでした。ですから$*$を$2$回取ったら、元に戻りそうな気がします。そこで一般の場合に、この$V^{**}$の構造を調べましょう。

\paragraph{評価写像} $V^{**}$の元は「$V^{*}$の元に対して、実数$\mathbb{R}$を対応させる」という写像です。そして$V^{*}$は$V\rightarrow\mathbb{R}$という写像です。だから$\bm{v} \in V$を$1$つ固定すると、$\varphi\mapsto \varphi(\bm{v})$という方法で$\ev_{\bm{v}}:V^{*}\rightarrow \mathbb{R}$という写像が作れます。これを\textbf{評価写像}というのでした。\pageref{paragraph:evaluation_map}ページでは多項式に対する評価写像を考えましたが、$V^*$の元に対する評価写像についても全く同様に、$\ev_{\bm{v}}$の線型性が示せます。よって$\ev_{\bm{v}}\in V^{**}$です。

そして$\ev_{\bm{v}}$の$\bm{v}$を動かすことで、$\ev\colon V\rightarrow V^{**}; \bm{v}\mapsto \ev_{\bm{v}}$という写像ができます。これも線型写像になっています。実際、任意の$\varphi \in V^*$に対し
\begin{itemize}
\item $\ev_{\bm{v} + \bm{w}}(\varphi) = \varphi(\bm{v} + \bm{w}) = \varphi(\bm{v}) + \varphi(\bm{w}) = (\ev_{\bm{v}} + \ev_{\bm{w}})(\varphi)$
\item $\ev_{\alpha\bm{v}}(\varphi) = \varphi(\alpha\bm{v}) = \alpha \varphi(\bm{v}) = \alpha \ev_{\bm{v}}(\varphi)$
\end{itemize}
なので、$\ev_{\bm{v} + \bm{w}} = \ev_{\bm{v}} + \ev_{\bm{w}}$かつ$\ev_{\alpha\bm{v}} = \alpha \ev_{\bm{v}}$です。

\paragraph{$V$と$V^{**}$の同型} この$\ev\colon V\rightarrow V^{**}$が同型になることを示しましょう。$\dim V^{**} = \dim V^* = \dim V$なので、単射性だけ示せば自動的に全射性が従います。

いま$\bm{v} \neq \bm{0} \in V$とすると、$\bm{v}$を延長することで$V$の基底$(\bm{f}_1 := \bm{v}, \bm{f}_2, \ldots, \bm{f}_n)$が作れます。この基底の双対基底$(\bm{f}^{\vee}_1, \bm{f}^{\vee}_2, \ldots, \bm{f}^{\vee}_n)$を取ると、$\bm{f}^{\vee}_1(\bm{v}) = 1 \neq 0$となります。よって$\varphi \in V^*$で$\varphi(\bm{v})\neq 0$となるものが存在します。

この対偶を取ると「任意の$\varphi \in V^*$に対して$\varphi(\bm{v}) = 0$なら、$\bm{v} = \bm{0}$」となります。よって$\bm{v} \in V$が$\ev_{\bm{v}} = 0 \in V^{**}$を満たせば、任意の$\varphi \in V^*$に対し$\varphi(\bm{v}) = \ev_{\bm{v}}(\varphi) = 0$となるから、$\bm{v} = \bm{0}$です。よって$\ev$は単射です\footnote{ここの証明は$\dim V = \infty$でも大体有効です。無限次元の場合に破綻するのは「単射ならば全射」の部分だけであって、$\ev\colon V\rightarrow V^{**}$が単射なことはいつでも正しいです。}。

だから双対空間を考えると$V \rightarrow V^* \rightarrow V^{**} \rightarrow V^{***} \rightarrow \cdots$というように、いくらでも線型空間を作ることはできるのですが、結局は$1$つおきに同型になってしまうのです。また今は$V$から$V^*$を作りましたが、$V^{*}$から$V$と同型な$V^{**}$を作れることを考えると「$V$と$V^*$の立場は対等なのではないか」という気がしてきます。そこで$\bm{v} \in V$と$f \in V^*$に対し、$f(\bm{v})$のことを$\langle \bm{v}, f \rangle$と内積のように書くことがあります。また「函数」という言葉を使うと対等性が崩れるので、$\langle \bm{v}, f \rangle$のことを「標準的なペアリング」と言ったりもします。

\subsection{$\Hom$の分解}

$V$, $W$が線型空間のとき、$V^*$の元と$W$の元を組み合わせることで$V$から$W$への線型写像を作ることができます。

いきなり抽象論をやる前に、数ベクトル空間の例を見てみましょう。既に見たとおり、$(\mathbb{R}^n)^*$は$n$次元の横ベクトルのなす線型空間でした。また$\mathbb{R}^m$は$m$次元の縦ベクトルの掛け算でした。すると$\mathbb{R}^m$の縦ベクトルは$(m, 1)$型行列、$(\mathbb{R}^n)^*$の横ベクトルは$(1, n)$型行列と同一視できるので、これらを掛け算\footnote{サイズの違うベクトルが出てくるので、以下では「$\mathbb{R}^n$における$i$番目の標準基底」を表すのに$\bm{e}_{n, i}$と書くことにします。}して$(m, n)$型行列が作れます:
\[
\bm{e}_{n, i}\bm{e}^{\vee}_{m, j} =
\bordermatrix{
& \cr
& 0 \cr
& \vdots \cr
& 0 \cr
\scalebox{0.8}{$i^{\text{th}}$} & 1 \cr
& 0 \cr
& \vdots \cr
& 0
}
\bordermatrix{
& & & & \scalebox{0.8}{$j^{\text{th}}$} \cr
& 0 & \cdots & 0 & 1 & 0 & \cdots & 0
}
=
\bordermatrix{
 &	& \scalebox{0.8}{$j^{\text{th}}$} \cr
 \cr
\scalebox{0.8}{$i^{\text{th}}$}	& & 1 & \ & \  \cr
& \cr
& \cr
}
= E_{ij}
\]
そして$E_{ij}$に$\bm{e}_{n, k}$を当てると$E_{ij}\bm{e}_{n, k} = \delta_{jk}\bm{e}_{m, i}$となりますが、この式は$\bm{e}_{m, i}\bm{e}^{\vee}_{n, j}(\bm{e}_{n, k})$と同じです。図式にすると
\begin{align*}
\begin{array}{c@{\,}@{\,}c@{\,}c@{\,}c@{\,}c}
\mathbb{R}^m \times (\mathbb{R}^n)^*	& \xrightarrow{\text{\hbox to 3zw{\hfil\hfil}}}	& \Hom(\mathbb{R}^n, \mathbb{R}^m)	& = & \Mat_{m, n}(\mathbb{R}) \\
\rotatebox{90}{$\in$}					& 				& \rotatebox{90}{$\in$}				&   & \rotatebox{90}{$\in$} \\
(\bm{e}_{m, i}, \bm{e}^{\vee}_{n, j})	& \xmapsto{\text{\hbox to 3zw{}}}		& \bm{e}_{m, i} \bm{e}^{\vee}_{n, j}	& = & E_{ij}
\end{array}
\end{align*}
という感じです。一般の場合にも同じことができます。$V, W$が線型空間で$\bm{w} \in W$, $\varphi \in V^*$とします。このとき写像$\bm{w}\otimes\varphi\colon V \rightarrow W$を\footnote{本当は、記号$\otimes$は「テンソル積」と呼ばれるものを表すのですが、今は$\bm{w}\otimes\varphi$を「まとめて一つの記号」だと思って差支えありません。}
\[
(\bm{w}\otimes\varphi)(\bm{v}) := \varphi(\bm{v})\bm{w}
\]
で定義できます。これも図式にすると
\begin{align*}
\begin{array}{c@{\,}@{\,}c@{\,}c@{\,}c@{\,}c}
W \times (V)^*			& \xrightarrow{\text{\hbox to 3zw{\hfil\hfil}}}	& \Hom(V, W) \\
\rotatebox{90}{$\in$}	& 												& \rotatebox{90}{$\in$} \\
(\bm{w}, \varphi)		& \xmapsto{\text{\hbox to 3zw{}}}				& \bm{w} \otimes \varphi
\end{array}
\end{align*}
となります。

そして$\Hom(V, W)$の元は、$\bm{w}\otimes\varphi$の形の元の$1$次結合で表すことができます。たとえば$V = \mathbb{R}^n$, $W = \mathbb{R}^m$の場合、$\bm{e}_{m, i}\otimes \bm{e}^{\vee}_{n, j} = E_{ij}$の$1$次結合によって、どんな行列も作れますね。これと同じことが抽象的な線型空間についても成り立つのです。

\paragraph{トレースとの関係} 最後に、$\Hom(V, W)$の元が$\bm{w}\otimes\varphi$の形の写像の和で書けることのご利益を一つ紹介します。それは\textbf{行列表示によらないトレースの定義ができる}という事実です。

$\bm{v} \in V$, $\varphi \in V^*$があると、線型写像$\bm{v} \otimes \varphi \colon V \rightarrow V$が定義できるのでした。一方$\varphi(\bm{v}) \in \mathbb{R}$でもあります。実はこのとき、$\tr \bm{v} \otimes \varphi = \varphi(\bm{v})$という式が成り立っています。

$V = \mathbb{R}^n$の場合に確かめてみましょう。$\bm{e}_i \otimes \bm{e}^{\vee}_j = E_{ij}$だったので、$\tr \bm{e}_i \otimes \bm{e}^{\vee}_j = \delta_{ij}$です。かたや双対基底の定義から$\bm{e}^{\vee}_j(\bm{e}_i) = \delta_{ij}$なので、両者は確かに一致しています。そして$\tr$は線型写像だったので、$\tr \bm{e}_i \otimes \bm{e}^{\vee}_j = \bm{e}^{\vee}_j(\bm{e}_i)$から、全ての$\bm{v} \in \mathbb{R}^n$, $\varphi \in (\mathbb{R}^n)^*$に対し$\tr \bm{v} \otimes \varphi = \varphi(\bm{v})$が成り立つことが従います。

まだ「(正方行列とは限らない、一般の) 線型変換のトレース」は定義していませんでしたが、普通は「線型変換を行列表示したときのトレース」を考えます。実は線型変換をどのように行列表示しても、トレースが変わらないことが証明できるのです。ですが行列は、線型変換を基底で表示したときに見えてくる一つの姿でしかありません。その表示によらず、線型変換の言葉だけでトレースを捉えようとすると、今のような議論が必要になるのです。

\section{Aセメスターの展望}

最後に、Aセメスターに何をやるはずかについて、簡単に述べておきます。

\paragraph{行列式}

まず最初に扱うのは行列式です。これは$n$次元空間における「立体の体積」にあたるもので、$n$本のベクトルが一次独立性かどうかを判定することに使ったり、連立$1$次方程式を解いたりするのに非常に役立ちます。理論上も実務上も極めて大事です。ですから
\begin{itemize}
\item 行列式の理論的な意味を知ること
\item 行列式を手際よく計算できるようになること
\end{itemize}
を心がけてください。

\paragraph{計量線型空間}

これまでの線型代数の授業では「線型性」について突っ込んだ議論を行ってきましたが、「内積」にあたるものは登場していません。たとえば我々の空間$\mathbb{R}^3$では、$\bm{u}, \bm{v} \in \mathbb{R}^3$のなす角を$\theta$とすれば$\bm{u} \cdot \bm{v} = \|\bm{u}\| \|\bm{v}\| \cos \theta$が成り立ちます。行列の積を定義するときに$\mathbb{R}^n$の内積っぽい計算をしていても、「角度」や「長さ」といった幾何学的な話は、線型空間の中で出てきませんでした。こういう「角度」や「長さ」にあたるものを\textbf{計量}といいます。この計量の扱いを、勉強することになります。

その際僕たちは、「座標」と「計量」を切り離して考えます。これはたとえば「$\mathbb{R}^3$のベクトル$\bm{e}_i$の長さを$1$ではなくす」ということに相当します。なぜこんなことをするのか分からないかもしれませんが、「長さの目盛りを取り換える」と思えば、そんな不自然ではないはずです。たとえば僕たちは長さを測るのに僕たちは1cmとか1mmとか、いくつか単位を使い分けます。そうすると$\bm{e}_1$の長さが1cmだとして、mm単位で測れば$\|\bm{e}_1\| = 10\textrm{mm}$になったりするわけです。ですから、座標と長さは必ずしも連動する必要はありません。僕たちは後で、単なる拡大 / 縮小にとどまらず、もっと広い範疇で「長さ」に相当する概念を定式化します。

\paragraph{行列の固有値と対角化}

$f$を線型空間$V$上の線型変換としましょう。このとき$V$に基底を取れば$f$は行列によって表示されました。基底を色々取り換えると$f$の行列表示も色々変わるわけですが、運が良いと$f$が対角行列で表示されることがあります。このとき、$f$の行列表示の対角成分に並ぶ数を$f$の\textbf{固有値}といいます。

後々証明しますが、$f$の対角化のやり方は (存在すれば) 本質的に$1$通りしかありません。違う方法で対角化できたとしても、対角成分に数が並ぶ順番が変わるだけなのです。ここで、抽象的に線型写像を扱う上で大事なのは「基底による見かけの表示の向こう側にある、本質的なもの」を取り出すことでした。$f$の固有値も行列表示に依存しないので、$f$にとって本質的な量だと言えます。実際、微分方程式など色々な場面に線型代数を応用する時に、固有値の計算がカギになる場面は山ほどあります。そうした固有値の計算方法や利用法を、$1$年生の最後で学ぶことになるでしょう。

