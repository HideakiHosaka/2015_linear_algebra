\documentclass{jsbook}

% パッケージ類の読み込み

\usepackage[hidelinks]{hyperref}				% ハイパーリンクと \namelink に使う
\usepackage{otf}								% 丸囲みの①を出すのに使う
\usepackage{title}								% タイトル等の見た目を調整する手製スタイルファイル
\usepackage{comment}							% 複数行にわたるコメントアウト

\usepackage{amsmath,amssymb,amsfonts}			% AMS パッケージ
\usepackage[dvipdfmx]{pict2e}					% picture 環境の拡張
\usepackage[dvipdfmx]{graphicx}					% 外部画像の取り込みなど

\def\qed{\hfill\raisebox{-1pt}{\scalebox{0.9}[1.4]{$\blacksquare$}}}

\begin{document}

\frontmatter

\title{数理科学基礎(線形代数学) / 線形代数学\\レポート解答解説}
\author{穂坂 秀昭}

\maketitle

\tableofcontents

\mainmatter

\part{数理科学基礎(線形代数学)}\label{part1}

\chapter{複素数と代数学の基本定理}

\lectureinfo{2015年4月15日 1限}

\section{はじめに}

\paragraph{ごあいさつ}

みなさん、はじめまして。この授業のTA (ティーチング・アシスタント)をすることになりました、数理科学研究科博士課程の穂坂といいます。これから$1$セメスターの間、よろしくお願いします。

この授業では毎回レポート問題が課され、それをTAの穂坂が添削して返却します。また答案返却に合わせて、この文書のような解説プリントを配付していく予定です。何か疑問要望等があれば、提出するレポートの片隅にメッセージを書くなり、メールを \url{hosaka@ms.u-tokyo.ac.jp} に送るなり、授業後に聞くなりしてください。

\paragraph{課題提出時のお願い}
答案が消えたら困りますので、次の$2$点は必ず守ってください。\vspace{-0.5zw}
\begin{itemize}
\item \textbf{氏名、学生証番号の両方}を書いてください。
\item 答案が複数枚に渡るときは、\textbf{左上をホチキス止め}してください。
\end{itemize}

\vspace{-1zw}

\paragraph{問題を解くにあたって}

毎週出題される問題はレポート課題ですから、とにもかくにも期限までに提出しないといけません。とは言っても、どうせ解くなら$1$問$1$問からなるべく多くの教訓を引きずり出したいし、何よりなるべく楽しい問題を解きたいものです。レポートに取り組むときは、次のようなことを意識してください。\vspace{-0.5zw}
\begin{itemize}
\item 簡単な計算問題は、授業で扱った定理などを確かめるための具体例です。常に「どの定理を、どう使っているのか」を考えながら解きましょう。
\item どんな問題であっても、ただ解くだけでなく、見通しの良い解法を探すべきです。計算問題ならなるべく手間を減らし、証明問題なら本質的な部分を捉えるよう努力しないといけません。
\end{itemize}\vspace{-1zw}

%\paragraph{プリントのオンライン公開}

%このプリントは毎回教室で配布するのに加え、インターネット上でも配布されます。そのURLは下記の通りですので、必要な方はアクセスしてください。

\paragraph{このプリントの作り方について}

せっかくなので、このプリントをどう作っているかについて説明しておきます。

ふつう「コンピュータで文書を作る」というと、大抵の人がMicrosoft Wordとか一太郎といったワープロソフトを連想すると思います。ところが残念なことに、市販のワープロソフトでは数式を入力するのに大変な苦労を強いられてしまいます。そこで数学が専門の人はどうするかというと、そういったワープロソフトの代わりに``\LaTeX''というソフトウェアを使います。これはD.~E.~Knuth\footnote{\href{http://www-cs-faculty.stanford.edu/~uno/}{\url{http://www-cs-faculty.stanford.edu/~uno/}}}という非常に有名な数学者・計算機科学者が作った``\TeX''というソフトウェアを、色々な人が改良してできあがったものです。

\LaTeX はワープロソフトとはちょっと違い、文字のスタイルを変えたり見出しをつけたりするのに「コマンド」というものを使います。ですから\LaTeX を使うにはコマンドの使い方を覚えなければいけません。加えてキーボードが打ち込んだものが、見た目通りに出てくるわけでもありません。一旦コマンドも含めて打ち込んだ文書を\LaTeX というプログラムその他色々に処理させることによって、やっと整形された文書がでてきます。ですから使い始めるにはちょっとハードルが高いのですが、使い慣れるとワープロソフトよりも手際よく文書が書けるし、また数式を中心として文書の仕上がりが美しいというメリットもあります。

もしかしたら皆さんの中には\LaTeX を既に知っている人がいるかもしれませんし、また将来\LaTeX を使う必要に迫られる人がいるかもしれません。そこで
%\begin{center}
\ \url{https://github.com/HideakiHosaka/2015_linear_algebra}
%\end{center}
に、この文書のPDFファイルと\LaTeX ソースコードを置いておきます。もし\LaTeX の方に興味がある人は、ページ右下の``Download ZIP''のボタンから一式をダウンロードしてください。また東京大学が持つ情報処理システムのオンライン自習教材「はいぱーワークブック」の第27章\footnote{\url{http://hwb.ecc.u-tokyo.ac.jp/current/applications/latex/}}に、\LaTeX の説明があります。\LaTeX を使う人は、一度読んでおくと良いと思います。

ちなみにソースコードの公開には``GitHub''というサービス\footnote{もしあなたが既に``GitHub''を知っているなら、きっと``pull request''の機能も知っているはずです。プリントに対して何か意見があれば、積極的にpull requestを送ってください。\textsf{\raisebox{1pt}{:}D}}を利用しています。上に貼ったURLを開くと、古いバージョンのプリントや、そうしたプリントがどう更新されていったかも見ることができます。授業自体の役には立たないと思いますが、興味があれば見てみてください。

\section{複素(数)平面の幾何}

\subsection{複素数と複素平面}

\paragraph{複素数の定義}

まず、最初に複素数の定義をおさらいしましょう。$i^2=-1$という規則で$i$という「数」\footnote{誰もが一度は「$-1$の平方根を数と呼んでいいのか」という疑問を抱いたことがあると思います。その疑問に対する答えをまだ書いていなかったので、ここでは括弧つきで「数」と書きました。でも一々こう書くと面倒なので、以下では括弧をつけないことにします。}を定めます。このとき、$2$つの実数$x,y\in\mathbb{R}$を用いて$z=x+yi$と表される数を複素数と言うのでした。また$z=x+yi$の$x$を実部、$y$を虚部と言うことも知っているはずです。足し算と掛け算はそれぞれ、分配法則などが上手く成り立つよう
\begin{align*}
(x+yi) + (x'+y'i) &:= (x+x')+(y+y')i, &  (x+yi)(x'+y'i) &:= (xx' - yy') + (xy'+x'y)i
\end{align*}
と決めていました\footnote{ここで出てくる$:=$は「左辺のものを右辺で定義する」という意味です。式変形をするときの$=$とは意味が違うので、定義の際には$:=$が使われることがあります。また左右を入れ替えて$=:$とすると「右辺のものを左辺で定義する」という意味になります。}。また、これらのルールがあれば、$x'+y'i\neq 0$のとき
\begin{align*}
\frac{x+yi}{x'+y'i}
= \frac{(x+yi)(x'-y'i)}{(x'+y'i)(x'-y'i)} = \frac{(xx'+yy')+(x'y-xy')i}{(x')^2+(y')^2}
= \frac{xx'+yy'}{(x')^2+(y')^2} + \frac{x'y-xy'}{(x')^2+(y')^2}i
\end{align*}
というように割り算もできます。こうして複素数では四則演算が全部できると分かりました。

\paragraph{複素平面}

さて、全ての複素数は$x+yi$の形に、$(x,y)$という二つの実数$x,y$のペアを用いて表せます。また複素数を二つの実数$x,y$で$x+yi$の形に表す方法がただ一通りなことも明らかでしょう。これらの事実\footnote{集合と写像の言葉できちんと書くと、$(x,y)\in\mathbb{R}^2$に$x+yi\in\mathbb{C}$を対応させる写像$\mathbb{R}^2\rightarrow\mathbb{C}$が全単射、ということです。}から、\textbf{複素数$z=x+yi$と座標平面上の点$(x,y)$とを$1:1$に対応させられる}ことが分かります。このように平面$\mathbb{R}^2$上の点一つ一つを複素数と見なしたとき、平面$\mathbb{R}^2$のことを\textbf{複素(数)平面}\footnote{複素平面と複素数平面は、どっちの言葉も同じ意味です。複素平面の方を使う人が多いですが、複素数平面と言っても誤解を招くことはないですし、昔は複素数平面という言葉も割と良く使われていたそうです。関西学院大学の示野信一先生のブログに詳しい事情が書いてありますので、気になる人は読んでみてください: \url{http://mathsci.blog41.fc2.com/blog-entry-60.html}}と呼びます。また$x$軸, $y$軸をそれぞれ\textbf{実軸}(real axis)、\textbf{虚軸}(imaginary axis)と言います。

\begin{figure}[h!tbp]
\begin{center}
\begin{picture}(210,100)
\put(201,12){Re}
\put(11,92){Im}
\put(15,0){\vector(0,1){90}}
\put(0,15){\vector(1,0){200}}
\put(15,15){\dashbox{1.2}(100,50)}
\put(7,6){$0$}
\put(112,8){$x$}
\put(7,63){$y$}
\put(115,65){\circle*{3}}
\put(118,67){$x+yi$}
\end{picture}
\caption{複素平面における数と点の対応}
\end{center}
\end{figure}

少し大げさに言うと、我々は「複素数という代数的なもの」と「平面という幾何的なもの」を対応付けました。このことには、非常に重要な意味があります。なぜなら\textbf{代数の観点と幾何の観点を行ったりきたりすることで、色々なことが分かるようになる}からです。たとえば長さなどの幾何的な情報を代数的な操作で捉えたり、逆に掛け算などの代数的な操作を図形的に捉えたりというように。これから、それをやってみましょう。

\subsection{幾何を代数で捉える}

\paragraph{複素数の大きさと共役}

幾何的な情報の最も典型的なものとして「$2$点間の距離」が挙げられます。複素平面の場合、原点$0$と$z\in\mathbb{C}$との距離を$z$の\textbf{絶対値}または\textbf{大きさ}と言い、$|z|$で表します。三平方の定理から、すぐに$|x+yi|=\sqrt{x^2+y^2}$が従います。またベクトルのときと同様、複素数$z,z'\in\mathbb{C}$の間の距離は$|z-z'|$となります。

幾何的な観点からは、「線/点対称移動」といった操作を考えられるというメリットもあります。たとえば「実軸に対する線対称移動」で$z$が写る点を$\overline{z}$と書くと、$\overline{x+yi}=x-yi$です。この$\overline{z}$を$z$の\textbf{共役}と言います\footnote{この部分は嘘ではないですが、若干語弊があります。今は「実軸に関する線対称移動」という幾何学的な操作として共役を定義しましたが、本来「共役」とは、実数$\mathbb{R}$から複素数$\mathbb{C}$を作る「拡大」という代数的な操作に伴って定義されるものです。}。共役は$\overline{z+w}=\overline{z}+\overline{w}$, $\overline{zw}=\overline{z}\,\overline{w}$という性質を満たすことが、計算で確かめられます。また共役を用いて、複素数の大きさを表すことができます。


\paragraph{複素数と複素数平面: 問1の解答}
$z=x+iy$のとき、$z\overline{z}=(x+yi)(x-yi)=x^2+y^2=|z|^2$となる。$z=0$であることは$z$と原点$0$との距離が$0$であることと同値なので、直ちに$z=0\Leftrightarrow|z|=0$を得る。 \qed

\paragraph{複素数と複素数平面: 問4の解答}

(1) $z_1 = x_1+y_1 i$, $z_2 = x_2 + y_2 i$とおく。絶対値は$0$以上の実数だから、$|z_1+z_2|\leq |z_1|+|z_2|$の代わりに$|z_1+z_2|^2\leq (|z_1|+|z_2|)^2$を示せばよい。実際計算すると
\begin{align*}
(|z_1|+|z_2|)^2 - |z_1+z_2|^2 
&= |z_1|^2 + 2|z_1||z_2| + |z_2|^2 - |z_1+z_2|^2 \\
&= x_1^2+y_1^2 + 2\sqrt{x_1^2+y_1^2}\sqrt{x_2^2+y_2^2} + x_2^2 + y_2^2 - \bigl((x_1+x_2)^2+(y_1+y_2)^2\bigr) \\
&= 2\Bigl(\sqrt{x_1^2+y_1^2}\sqrt{x_2^2+y_2^2} -(x_1x_2+y_1y_2) \Bigr)
\end{align*}
となる。そして括弧の中は、平方根が非負であることと次の計算とを組み合わせれば、$0$以上と分かる。
\begin{align*}
\Bigl(\sqrt{x_1^2+y_1^2}\sqrt{x_2^2+y_2^2}\Bigr)^2 - (x_1x_2+y_1y_2)^2
&= x_1^2y_2^2+y_1^2x_2^2 - 2x_1x_2y_1y_2 = (x_1y_2-x_2y_1)^2 \geq 0
\end{align*}
これで示すべきことが言えた。(2)は(1)を使えば$|z_1-z_2| = |z_1 + (-z_2) | \leq |z_1|+|{-z_2}| = |z_1| + |z_2|$となる。 \qed


\subsection{代数を幾何で捉える}

続いて、四則演算という代数的な操作を複素平面で見てみましょう。複素数の足し算や引き算がベクトルの足し算や引き算と全く同じであることは、すぐに分かると思います。非自明なのは\textbf{平面上の点やベクトルには掛け算が定義されていないのに対し、複素数には掛け算がある}という点です\footnote{「ベクトルにも内積や外積があるじゃん」という声が聞こえてきそうですが、内積や外積は、いわゆる普通の「積」とは違う性質を持ちます。$2$つのベクトルの内積は数になってしまい、ベクトルにはなりません。また$2$つのベクトルの外積はベクトルになりますが、ベクトルの外積は順序を入れ替わると結果が変わります。この辺が数の掛け算と全然違うところです。}。そこで「$2$つの複素数を掛け算した結果は、複素平面上ではどのように見えるか」が問題となってきます。まずは問題を$1$つ解いてみましょう。

\paragraph{複素数と複素数平面: 問2の解答} $z=2+i$とおくと
\begin{align*}
z^2 &= 3+4i, & z^3 &= 2 + 11i, & z^4 &= -7+24i, & z^5 &= -38+41i, & z^6 &= -117+44i \\[-2zw]
\end{align*}
である。これらをプロットした結果\footnote{$z^5$と$z^6$は図から激しくはみ出るので描いていません。}は次の図の通り。

\begin{figure}[h!tbp]
\begin{center}
\begin{picture}(200,130)
\put(201,12){Re}
\put(96,127){Im}
\put(100,0){\vector(0,1){125}}
\put(0,15){\vector(1,0){200}}
\put(93,6){$0$}
\put(104,15){\circle*{2}}
\put(106,6){$1$}
\put(108,19){\circle*{2}}
\put(111,19){$z$}
\put(112,31){\circle*{2}}
\put(113,32){$z^2$}
\put(108,59){\circle*{2}}
\put(109,60){$z^3$}
\put(72,111){\circle*{2}}
\put(70,113){$z^4$}
\put(100,15){\line(2,1){8}}
\put(100,15){\line(3,4){12}}
\put(100,15){\line(2,11){8}}
\put(100,15){\line(-7,24){28}}
\end{picture}
\caption{$z=2+i$のべき乗のプロット}
\end{center}
\end{figure}

\paragraph{極形式表示}

いま複素数$z\neq 0$に対し$z^0=1$, $z^1=1$, $z^2$, $z^3$, $z^4$を平面上にプロットし、これらの点を原点と結んだ結果を眺めると、隣り合う角が全て同じ大きさであるように見えます。それを実際に確かめてみましょう。

角度を計算したいので、極座標を使うのが筋がよさそうです。そこで$z=x+yi$の表す点$(x,y)$を、極座標$(r,\theta)$を用いて$x=r\cos\theta$, $y=r\sin\theta$と表します。これを対応する複素数の方で表すと
\begin{align*}
z =x+yi = r\cos\theta + (r\sin\theta)i = r(\cos\theta+i\sin\theta)
\end{align*}
となります。この書き方を、複素数の\textbf{極形式表示}と呼びます。極形式表示のもとで$r=|z|$です。また$\theta$は、複素平面の半直線$0z$と実軸の非負の部分がなす角を、反時計回りに測った角度となっています。この$\theta$を複素数$z$の\textbf{偏角}と言い、$\theta=\arg z$と書きます。$\arg z$は一通りでなく$2\pi$の整数倍だけずらせますが、今は気にしないでおきます。


\begin{figure}[h!tbp]
\begin{center}
\begin{picture}(200,90)
\put(201,12){Re}
\put(11,87){Im}
\put(15,0){\vector(0,1){86}}
\put(0,15){\vector(1,0){200}}
\put(15,15){\line(2,1){100}}
\put(15,15){\dashbox{1.2}(100,50)}
\put(0,0){\qbezier(30,15)(30,20)(27,21)}
\put(7,6){$0$}
\put(102,7){$r\cos\theta$}
\put(-13,62){$r\sin\theta$}
\put(115,65){\circle*{3}}
\put(118,67){$r(\cos\theta+i\sin\theta)$}
\put(60,43){$r$}
\put(32,16){$\theta$}
\end{picture}
\caption{複素数の極形式表示}
\end{center}
\end{figure}\vspace{-0.5zw}

さて、極形式で表された$2$つの複素数を掛け算すると
\begin{align*}
r(\cos\theta+i\sin\theta) \times r'(\cos\theta'+i\sin\theta')
&= rr'\bigr\{ (\cos\theta\cos\theta' - \sin\theta\sin\theta') + i(\sin\theta\cos\theta'+\cos\theta\sin\theta') \bigl\} \\
&= rr'\bigl(\cos(\theta+\theta') + i\sin(\theta+\theta')\bigr)
\end{align*}
となります。最後の式変形は、もちろん三角函数\footnote{函数と関数は同じ意味です。少々古臭い言い回しですが、好みの問題でこちらを使います。}の加法定理を使っています。この式は非常に重要なことを示唆しています。それは複素数$z$, $z'$の積$zz'$について\vspace{-0.5zw}
\begin{itemize}
\item 大きさは、$|zz'|=|z||z'|$で与えられる
\item 偏角は$\arg zz' = \arg z + \arg z'$で与えられる
\end{itemize}\vspace{-0.5zw}
ということです。言い換えれば、複素数$z$に対して別の複素数$z'$を掛け算する操作は\vspace{-0.5zw}
\begin{itemize}
\item $z$の大きさを$|z'|$倍し
\item $z$の偏角に$\arg z'$を足し算する
\end{itemize}
ということに他ならないからです。このように極形式を使うことによって、複素数の掛け算が「拡大縮小」と「回転」の組み合わせという図形的意味を持つことが読み取れるのです。

% TODO: ここに図を挟む

なお、一々$\cos\theta+i\sin\theta$と書いているのは長ったらしくて大変なので、以下では実数$\theta$に対して$e^{i\theta} = \cos\theta + i \sin\theta$や$\exp i\theta = \cos\theta + i \sin\theta$と表すことにします。たとえば
$e^{i\pi} = \cos\pi + i\sin\pi = -1$
という感じです。指数函数$e^x$と同じ記法を用いることには実は意味がある\footnote{そのうち微分積分学の授業で、函数のTaylor展開というものを習うはずです。その後に巾(べき)級数で指数函数$e^x$を定義し直すと、元々の「$e$のなんとか乗」という意味を越えて、$e^x$の$x$に複素数を代入できるようになります。そうして初めて$e^{i\theta}=\cos\theta+i\sin\theta$という式に正しく意味を与えることができます。}のですが、今は「単なる記号」だと思っていてください。この記号を使うと、極形式表示での掛け算は
\[
re^{i\theta} \times r'e^{i\theta'} = rr'e^{i(\theta+\theta')}
\]
と書けます。スッキリしてていいですね。

極形式表示は計算面でも、非常に威力を発揮することがあります。


\paragraph{複素数と複素数平面: 問3の解答} $z=e^{2k\pi i/n}$ ($k=0,1,\ldots,n-1$)とおくと、
\[
z^n = \bigl(e^{\frac{2k\pi i}{n}}\bigr)^n = \exp\Bigl(\frac{2k\pi i}{n}+\frac{2k\pi i}{n}+\cdots+\frac{2k\pi i}{n}\Bigr) = \exp \Bigl(\frac{2k\pi i}{n} \times n\Bigr) = e^{2k\pi i} = (e^{2\pi i})^k = 1
\]
となる。ここで$k$が$k=0,1,\ldots,n-1$を動けば、$n$個の異なる複素数が得られる\footnote{複素平面上にプロットすれば、異なることが直ちに分かります。}。また多項式$z^n-1$は$n$次式だから、$n+1$個以上の根を持つことはない。ゆえに$z=e^{2k\pi i/n}$ ($k=0,1,\ldots,n-1$)が全ての根を与える。 \qed

\subsection{残りの問題}

ここまでで紹介していなかった問題の解答を記します。

\paragraph{複素数と複素数平面: 問6の解答} 

$x,y\in M$とすると$x=a^2+b^2$, $y=c^2+d^2$となる整数$a,b,c,d\in\mathbb{Z}$が取れる。このとき$x=|a+bi|^2$, $y=|c+di|^2$なので$xy=|a+bi|^2|c+di|^2=|(a+bi)(c+di)|^2=|(ac-bd)+(ad+bc)i|^2=(ac-bd)^2+(ad+bc)^2$となる。$a,b,c,d\in\mathbb{Z}$より$ac-bd,ad+bc\in\mathbb{Z}$である。よって$xy\in M$である。 \qed

\paragraph{複素数と複素数平面: 問7の解答} 
次の式の$t$に好きな有理数を代入すればよい。
\[
\Bigl(\frac{2t}{1+t^2}\Bigr)^2 + \Bigl(\frac{1-t^2}{1+t^2}\Bigr)^2 = 1
\]
$t=\tan\theta$とおくと、この式は$\sin^2 2\theta+\cos^2 2\theta=1$に化ける。したがって$0\leq t \leq 1$の範囲で$2t/(1+t^2)$は短調増加する。これと$0$以上$1$以下の有理数が無限個存在することから、有理点も無限個存在することが分かる。 \qed 

\section{多項式の性質}

\subsection{多項式の割り算}

実数係数や複素数係数\footnote{有理数係数でも大丈夫です。より一般に、係数が「体」と呼ばれるものであれば、複素数係数と同様に割り算ができます。}の多項式では、割り算と余りの計算ができます。$f(x)$を$m$次多項式、$g(x)$を$n$次多項式、$m\geq n$とすると、$f(x)=g(x)q(x)+r(x)$を満たす多項式$g(x)$と$n$次未満の多項式$r(x)$がただ一つだけ存在します。実際$f(x) = a_m x^m + a_{m-1}x^{m-1} + \cdots + a_0$, $g(x) = b_n x^n + b_{n-1} x^{n-1} +\cdots + b_0$とおくと、$f(x) - b_n x^{m-n} g(x) / a_m$の次数は$m-1$次以下になります。こうやって「$g(x)$に上手い数と$x$のべき乗をかけ、$f(x)$の項を次数が高い順に消していく」という操作をすれば、商と余りの計算ができます。実際の計算には筆算を使ったり、あるいは$g(x)$が$1$次式のときは「組立除法」という技が使えたりしますが、その辺は割愛します。

特に$g(x)$が$1$次式$x-\alpha$のとき、$f(x)=(x-\alpha)q(x)+r(x)$に出てくる$r(x)$は$0$次以下\footnote{$0$でない定数は$0$次式ですが、$0$だけは次数を$-\infty$と定めます。これは多項式の次数を$\deg$で表すとき、$\deg fg=\deg f + \deg g$が常に成り立つようにするためです。}の式、つまり定数です。なので$r(x)$の代わりに$r$と書くと、$f(x)=(x-\alpha)q(x)+r$の両辺に$x=\alpha$を代入して$r=f(\alpha)$が得られます。よって、$f$が$x-\alpha$で割り切れることと$f(\alpha)=0$が同値になります。この事実を\textbf{因数定理}と呼ぶのでした。

因数定理などを用いて解ける問題を、まとめて片付けてしまいましょう。


\paragraph{多項式: 問1の解答}
$f(x)$が$x-\alpha$で割り切れるので、$f(x)=(x-\alpha)g(x)$と書ける。また$f(x)$は$x-\beta$でも割り切れるので、$f(\beta)=0$である。よって$0=f(\beta)=(\beta-\alpha)g(\beta)$となるが、$\alpha\neq\beta$より$g(\beta)=0$でないといけない。よって$g(x)$は$(x-\beta)$で割り切れる。これより$f(x)$は$(x-\alpha)(x-\beta)$で割り切れる。 \qed

\paragraph{多項式: 問2の解答}
$f(x)=x^3+10x^2+ax-2$に$x=2,-3$を代入した値が等しい。よって$f(2)=46+2a$と$f(-3)=-3a+61$が等しいのだから、$a=3$が得られる。求める余りは$52$である。 \qed

\paragraph{多項式: 問3の解答}
$f(x)=mx^3+nx^2-5$とおく。$f(-\frac{1}{2})=0$より$-\frac{m}{8}+\frac{n}{4}-5=0$である。また$f(\frac{2}{3})=7$より$\frac{8}{27}m+\frac{4}{9}n-5=7$である。これより$m=6$, $n=23$である。 \qed

\paragraph{多項式: 問4の解答} 	$F(x)$を$2x^2+x-1$で割った余りを$px+q$と書くと、何か多項式$P(x)$を用いて
\[
F(x) = (2x^2+x-1)P(x) + px + q
\]
と書ける。$F(x)$を$x+1$で割った余りが$6$なので$F(-1)=6$である、よって上式に$x=-1$を代入して$6=-p+q$を得る。同様に$F(x)$を$2x-1$で割った余りが$3$なので、$F(\frac{1}{2})=3$である。これより$3=\frac{1}{2}p+q$を得る。こうして$p,q$の連立$1$次方程式が得られたので、解くと$p=-2$, $q=4$が得られる。よって余りは$-2x+4$である。 \qed

\paragraph{多項式: 問6の解答}
$3$次多項式$f(x)$は、適当な多項式$g(x)$と$h(x)$によって$f(x)=g(x)(x^2-1)+5x-8=h(x)(x^2-x-6)+17x+4$と書ける。これに$x=1,-1,-2,3$を代入すると、それぞれ$f(1)=-3$, $f(-1)=-13$, $f(-2)=-30$, $f(3)=55$が得られる。そこで$f(x)=ax^3+bx^2+cx+d$とおくと
\begin{align*}
\begin{cases}
a+b+c+d &= -3 \\
-a+b-c+d &= -13 \\
-8a+4b-2c+d &= -30 \\
27a + 9b + 3c + d &= 55
\end{cases}
\end{align*}
という連立一次方程式が得られる。これを$a,b,c,d$について解けば$f(x) = 2x^3+3x-8$と分かる。 \qed

\subsection{有名な多項式}

今回の問題の中にはいくつか有名な多項式が出てくるので、問題を解きつつ紹介します。

\paragraph{複素数と複素数平面: 問5の解答} $f(z)=z^4+z^3+z^2+z+1$である。

\noindent (1) $z$を$f(z)=0$の根とする。$z^5-1=(z-1)f(z)=0$なので、$z\neq 0$である。よって$f(z)/z^2=0$である。また$f(z)/z^2=z^2+z+1+z^{-1}+z^{-2} = (z+z^{-1})^2 + (z+z^{-1}) - 1 = t^2+t-1$である。よって$t^2+t-1=0$となるので、これを解いて$t=\frac{-1\pm\sqrt{5}}{2}$を得る。

\noindent (2) $t=z+z^{-1}$より$z^2-tz+1=0$である。$t=\frac{-1+\sqrt{5}}{2}$のとき、この方程式を解くと
\[
z = \frac{t\pm\sqrt{t^2-4}}{2} = \frac{-1+\sqrt{5}\pm\sqrt{(1-\sqrt{5})^2-16}}{4}
= \frac{-1+\sqrt{5}\pm\sqrt{-10-2\sqrt{5}}}{4}
= \frac{-1+\sqrt{5}\pm\sqrt{10+2\sqrt{5}}\,i}{4}
\]
となる。同様に$t=\frac{-1-\sqrt{5}}{2}$のとき
\[
z = \frac{t\pm\sqrt{t^2-4}}{2} = \frac{-1-\sqrt{5}\pm\sqrt{(-1-\sqrt{5})^2-16}}{4}
= \frac{-1-\sqrt{5}\pm\sqrt{-10+2\sqrt{5}}}{4}
= \frac{-1-\sqrt{5}\pm\sqrt{10-2\sqrt{5}}\,i}{4}
\]
が得られる。これらが全ての解である。

\noindent (3) $z = e^{2k\pi i/5}$ ($k=0,1,2,3,4$)が$z^5=1$の全ての解である。複素平面にプロットすれば、(2)で求めた解のうち実部と虚部がともに正なものが$e^{2\pi i/5}$だと分かる。これと$\sin(\frac{\pi}{2}-\theta) = \cos\theta$, $\cos(\frac{\pi}{2}-\theta) = \sin\theta$より
\[
\frac{-1+\sqrt{5}+\sqrt{10+2\sqrt{5}}\,i}{4} = e^{2\pi i/5}
= \cos\frac{2\pi}{5} + i \sin \frac{2\pi}{5} = \sin\frac{\pi}{10} + i \cos \frac{\pi}{10}
\]
となる。この式の実部と虚部を見ればよい。 \qed

\paragraph{多項式: 問5の解答}
\noindent (1) $x^n-a^n = (x-a)(x^{n-1}+ax^{n-2}+a^2x^{n-3}+\cdots+a^{n-1})$

\noindent (2) もし$x^n-a^n$が$x+a$で割り切れることは、$x$に$-a$を代入した結果が$0$になることと同値である。すなわち$0=(-a)^n-a^n=\bigl((-1)^n-1\bigr)a^n$より、$(-1)^n=1$が必要十分条件である。これは$n$が偶数であることに他ならない。

\noindent (3) $x^n+a^n$に$x=-a$を代入すると$(-a)^n+a^n = \bigl((-1)^n+1\bigr)a^n$となる。これが$0$になることは$n$が奇数であることと同値である。よって$n$が奇数なら$x^n+a^n$は$x+a$で割り切れる。 \qed

\paragraph{円周等分多項式} 今の問題で出てきた因数分解$x^n-a^n=(x-a)(x^{n-1}+ax^{n-2}+\cdots+a^{n-1})$は非常に良く見かけます。特に$a=1$と置いてできる多項式$x^n-1$の複素根は、極形式で考えれば直ちに$e^{2k\pi i/n}$ ($k=0,1,\ldots,n-1$)と分かります。これらの解をプロットすると、原点を中心とする半径$1$の円周が$n$等分されます。そういう理由で$n$が素数のとき\footnote{$n$が素数でないときも円周等分多項式は定義されますが、諸々の事情で$z^{n-1}+z^{n-2}+\cdots+1$そのものにはなりません。}、$z^n-1$を$z-1$で割ってできる多項式$z^{n-1}+z^{n-2}+\cdots+1$のことを円周等分多項式と呼びます。

\paragraph{多項式: 問7の解答} 答えのみ記す。
\begin{itemize}
\item[(1)] %\ \\[-4zw]
%\begin{align*}
$4(ab+cd)^2-(a^2+b^2-c^2-d^2)^2
%&= (2ab+2cd+a^2+b^2-c^2-d^2)(2ab+2cd-a^2-b^2+c^2+d^2) \\
%&= \bigl((a+b)^2-(c-d)^2\bigr)\bigl((c+d)^2-(a-b)^2\bigr) \\
= (a+b+c-d)(a+b-c+d)(a-b+c+d)(-a+b+c+d)$
%\end{align*}
\item[(2)] $x^3-(a+b+c)x^2+(ab+bc+ca)x-abc = (x-a)(x-b)(x-c)$ \qed
\end{itemize}

\paragraph{基本対称式}

今の問題の(2)で出てきた$a+b+c$, $ab+bc+ca$, $abc$はいずれも$a$, $b$, $c$について対称な多項式です。このような多項式を$a$, $b$, $c$の対称式と言う\footnote{変数の数が増えても同様で、多項式$f(x_1,\ldots,x_n)$がどの$2$つの$x_i$, $x_j$を入れ替えても変わらないとき、$f$を$n$変数の対称式と言います。}のでした。特に、ここに出てきた$3$つの対称式は\textbf{基本対称式}という名前がついています。これは「全ての対称式は基本対称式たちの定数倍の和と積で書ける」という重要な定理があるからです。遠くない将来にまた登場すると思いますので、その時に改めて詳しく解説します。

\subsection{複素数について}
\begin{citation}
{}``Was sind und was sollen die Zahlen?''
\end{citation}

これは、ドイツの数学者Richard Dedekindが記した本\footnote{ちなみに、1893年に出版されたドイツ語原著の第2刷が、東大駒場キャンパスの数理科学図書室にあります。}のタイトルです。『数とは何か、また何であるべきか?』この問題を、考えてみましょう。

\paragraph{複素数は数なのか?}

我々が日常生活の中で出会う「数」にはどんなものがあるでしょうか?たとえば物の個数を数えるときは自然数を使いますし、お金の計算をするときは収入と支出を表すのに正の数と負の数を使います。また料理をすればレシピの中に分数が出てきますし、円周の長さを測ろうとしたら$\pi$のような無理数も現れます。これらに登場する数は、いずれも実数の範囲に収まっていますね。

一方、複素数は実数の範囲を超えるものです。そのため$i^2=-1$となる数$i$を我々の世界で目にすることはありません。たとえばおつかいに行った子供が「ママー!\negthinspace おつりで$300+250\,i$円もらったよ!」なんて言うわけないですね。複素数を「気持ち悪い」と感じる主たる理由は、おそらくここにあるのではないでしょうか。英語では空想上の数``imaginary number''と呼ばれますし、日本語ではさらにネガティブな含みを持つ「虚」数という呼び名もあります。

ですが「我々の身の回りに見当たらないから」というだけの理由で、複素数を数と呼ぶべきではないのでしょうか。ここで一度、我々の身近にある数について「何が数たらしめているのか」を考えてみましょう。数を考える上で何よりも大事なことは「計算」です。たとえば自然数だったら足し算と掛け算ができます。整数なら、引き算がいつでもできます。有理数や実数なら、$0$以外の数による割り算もできます。また計算とは別に「大小の比較」ができることも、数の大きな特徴でしょう。

複素数は残念なことに「大小の比較」をすることはできません。ですが既に見てきたとおり、複素数では四則演算の全てを行うことができます。これをもって「数」と呼んでも良いのではないでしょうか。また「$-1$の平方根」というと気味が悪いかもしれませんが、「形式的に$i$という数を付け加え、$i^2=-1$というルールで計算を行う」という風に思えば、$i$の存在も受け入れられる気がします。

実際、現代数学では今のような方法で複素数を捉えています。一般に実数や有理数のように「四則演算ができる数の範囲」を、数学では\textbf{体(たい)}と呼びます。そしていま、実数係数の$1$変数多項式全体を$\mathbb{R}[x]$と書きます。このとき多項式$f(x)$に対して「$x^2$に$-1$を代入する」という操作、より正確には「多項式$f(x)$を$x^2+1$で割った余りを考える」という操作\footnote{代数学の用語では「多項式環$\mathbb{R}[x]$をイデアル$(x^2+1)$で割る」という操作に相当します。}をすれば、多項式の変数だった$x$が$-1$の平方根であるかのように振る舞います。これを「多項式$x^2+1$の根を添加して実数体$\mathbb{R}$を拡大する」と言います。このように体が与えられたとき、その中にない元を付け加えて大きい体を作る\textbf{拡大}という操作を定義することによって、きちんと複素数体$\mathbb{C}$を定義できるのです。

\paragraph{複素数: 問8の解答}

大体の人が「実数の範疇を飛び出る」とか「大小がなくなる」と書いてくれました。もちろん、どちらも数の性質を踏まえた、真っ当な感じ方だと思います。 \qed

\subsection{代数学の基本定理}

今回最後の話題は、複素数係数の多項式の根です。多項式$P(x)$に対し、方程式$P(x)=0$の解を根と言うのでした。まずは「係数が全て実数」という特別な場合に、複素数の根が「共役とペアで現れる」という事実を確認しましょう。

\paragraph{代数学の基本定理: 問1の解答}

$f(z) = a_0 + a_1 z + \cdots + a_n z^n$とおく。このとき$a_0,a_1,\ldots,a_n\in\mathbb{R}$である。よって$\alpha\in\mathbb{C}$が$f(z)$の解であるとき、$f(\alpha)=0$の共役を取ると
\begin{align*}
0 &= \overline{f(\alpha)}
= \overline{a_0 + a_1 \alpha + \cdots + a_n \alpha^n}
= \overline{a_0} +\overline{a_1 \alpha} + \cdots + \overline{a_n \alpha^n}
= a_0 + a_1 \overline{\alpha} + \cdots + a_n \overline{\alpha}^n
= f(\overline{\alpha})
\end{align*}
となる。よって$\overline{\alpha}$も解である。 \qed

\paragraph{代数学の基本定理}

さて多項式が根を持つかどうかは、「考える数の範囲」によって変わってきます。たとえば$P(x)=x^2+1$は実数の範囲で根を持ちませんが、複素数の範囲に広げると$x=\pm i$という根を持ちます。このように「多項式がいつも根を持つか」は係数の範囲に依存するのですが、実はなんと、複素数で考える限り\textbf{定数でない多項式は必ず$1$個は根を持つ}のです\footnote{ただし「解を持つ」ことと「解が計算で求まる」ことは全く別の問題です。たとえば$f$を連続函数とします。このとき$a<b$とすると、$f$は閉区間$[a,b]$上のどこかの点$c$で最大値を取ります。この事実はグラフを描けばすぐ納得できますが、でも「最大値を取る$c$がどこにあるか」について、全く教えてくれません。代数学の基本定理も、これと同じ状況になっています。

ちなみに$2$次方程式は解の公式を使えば常に解けますが、Galois理論というものを使うと、$5$次以上の多項式に対する「解の公式」が存在しないことまで証明できます。}。これが代数学の基本定理と呼ばれる内容です。

そして「根が$1$個は存在する」ことが分かると、そこから「定数でない$n$次多項式はいつも$n$個の根を持つ」ことまで言えてしまいます。このことを問題で確かめてから、最後に代数学の基本定理の完全な証明を与えましょう\footnote{この証明には微分積分学の知識が必要なので、今は読み解くのが難しいかもしれません。連続函数の性質を習った後で読むと、手頃な勉強になるでしょう。}。

\paragraph{代数学の基本定理: 問2の解答}
$f$の次数に関する帰納法で示す。まず$\deg f=1$のときは$1$次式$1$個の積である。次に、$n-1$次以下の全ての複素係数多項式が$1$次式の積に分解すると仮定する。このとき$f(z)$を$n$次多項式とすると、代数学の基本定理より$f(z)$の根が存在する。それを$\alpha$とすると$f(z)=(z-\alpha)g(z)$と書け、$g(z)$は$n-1$次多項式となる。帰納法の過程から$g(z)$は$1$次式の積に分解するので、$f(z)$全体も$1$次式の積となる。


\paragraph{代数学の基本定理: 問3の解答}

$g(0)\neq 0$より、$g(z)$の定数項は$0$でない。定数項の次に低い次数の項が$a_k z^k$だとして、$g(z) = a_0 + a_k z^k + a_{k+1} z^{k+1} + \cdots + a_n z^n$とおく。うまく$z$の偏角を調節すれば$a_0$と$a_k z^k$の偏角が$\pi$だけずれ、原点から見て逆向きになるようにできる。その上で$|z|$を十分小さくすれば、$|a_0+a_k z^k|$は$|a_0|$より小さくなるはずである。さらに$|z|$を小さくすれば、$z$の$(k+1)$乗以上の項の絶対値は、$z^k$の項の絶対値に比べていくらでも小さくなる。そうすれば$|g(z_0)|<|g(0)|$となるはずである。この考え方に基づき、$z$の適切な偏角と絶対値を見出そう。

いま$a_0 = r_0 e^{i\theta_0}$, $a_k = r_k e^{i\theta_k}$ ($r_0,r_k > 0$)と書ける。これらを用いて$z_0:=re^{i(\theta_0-\theta_k+\pi)/k}$と定めると
\begin{align*}
g(z_0) &= a_0 + a_k z_0^k \Bigl( 1 + \frac{a_{k+1}}{a_k} z_0 + \cdots + \frac{a_n}{a_k} z_0^{n-k}\Bigr)
= r_0e^{i\theta_0} + r_k e^{i\theta_k} \Bigl(re^{i(\theta_0-\theta_k+\pi)/k}\Bigr)^k \Bigl( 1 + \frac{a_{k+1}}{a_k} z_0 + \cdots + \frac{a_n}{a_k} z_0^{n-k}\Bigr)\\
&= r_0e^{i\theta_0} + r_k r^k e^{i(\theta_0+\pi)} \Bigl( 1 + \frac{a_{k+1}}{a_k} z_0 + \cdots + \frac{a_n}{a_k} z_0^{n-k}\Bigr)
= (r_0-r_kr^k) e^{i\theta_0} - r_k r^k e^{i\theta_0} \Bigl( \frac{a_{k+1}}{a_k} z_0 + \cdots + \frac{a_n}{a_k} z_0^{n-k}\Bigr)
\end{align*}
となる。よって$|e^{i\theta_0}|=1$と三角不等式より、$|f(z_0)|$は
\begin{align*}
|g(z_0)| &\leq |r_0-r_k r^k | + r_k r^k \Bigl| \frac{a_{k+1}}{a_k} z_0 + \cdots + \frac{a_n}{a_k} z_0^{n-k}\Bigr|
= |r_0-r_k r^k| + r_k r^k|z_0| \Bigl| \frac{a_{k+1}}{a_k} + \frac{a_{k+2}}{a_{k+1}}z_0  + \cdots + \frac{a_n}{a_k} z_0^{n-k-1}\Bigr| \\
&\leq |r_0-r_k r^k| + r_k r^{k+1}\Bigl\{\Bigl|\frac{a_{k+1}}{a_k}\Bigr| + \Bigl|\frac{a_{k+2}}{a_k} z_0\Bigr| + \cdots + \Bigl|\frac{a_{n}}{a_k} z_0^{n-k-1}\Bigr|\Bigr\}
\end{align*}
となる。この式で
\[
M := \max\Bigl\{\Bigl|\frac{a_{k+1}}{a_k}\Bigr|, \Bigl|\frac{a_{k+2}}{a_k} z_0\Bigr|, \ldots, \Bigl|\frac{a_{n}}{a_k} z_0^{n-k-1}\Bigr|\Bigr\}
\]
とおくと、$|g(z_0)|\leq |r_0-r_kr^k| + r_k r^{k+1}(n-k-1)M$が得られる。そこで
\[
r := \min\Bigl\{\Bigl|\frac{r_0}{r_k}\Bigr|^{\frac{1}{k}} , \frac{1}{2(n-k-1)M}\Bigr\}
\]
と取れば、$ r_0-r_k r^k \geq 0 $ かつ$ r \leq \frac{1}{2(n-k-1)M} $なので
\[
|g(z_0)|\leq r_0 - r_k r^k + r_k r^{k+1}(n-k-1)M \leq r_0 - r_k r^k \bigl( 1 - r(n-k-1)M\bigr)
\leq r_0 - \frac{r_k r^k}{2} < r_0
\]
である。これより$|g(z_0)|<r_0=|g(0)|$となることが分かった。 \qed


\paragraph{代数学の基本定理の証明}

せっかく問$3$の解答を与えたので、代数学の基本定理の証明を与えておきます。以下、$f(z)$を複素係数の多項式とします。証明のアイデアは次の通りです。
\begin{enumerate}
\item $f(z)=0$となる$z$が存在することを「$z$が複素平面全体を動く時の$|f(z)|$の最小値が$0$」と言い換える。
\item 原点を中心とする十分大きい半径$R$の円板を考え、その外側には$|f(z)|$の最小値が決して現れないことを言う。
\item 半径$R$の円板内のどこかで、$|f(z)|$が最小値を取ることを言う。
\item $|f(z)|$の最小値が$0$でないと、矛盾が生じることを示す。
\end{enumerate}


\noindent \underline{step 1.} $|z|>R$なる全ての複素数$z$に対して$|f(z)|>|f(0)|$が成り立つような正の実数$R>0$が取れることを示す。

$f(z)=a_n z^n + a_{n-1} z^{n-1} + \cdots + a_0$とおく。このとき
\[
|f(z)| = |a_n||z|^n\Bigl|1+\frac{a_{n-1}}{a_n z} + \cdots + \frac{a_0}{a_n z^n}\Bigr|
\]
である。ここで三角不等式$|z_1|-|z_2|\leq|z_1-z_2|$を使うと、$|z_1+z_2|\geq|z_1|-|-z_2|=|z_1|-|z_2|$なので
\begin{align*}
\Bigl|1+\frac{a_{n-1}}{a_n z} + \cdots + \frac{a_0}{a_n z^n}\Bigr|
&\geq \Bigl|1+\frac{a_{n-1}}{a_n z} + \cdots + \frac{a_1}{a_n z^{n-1}}\Bigr| - \Bigl|\frac{a_0}{a_n z^n}\Bigr| \geq \cdots
\geq 1-\Bigl|\frac{a_{n-1}}{a_n z}\Bigr| - \cdots -\Bigl|\frac{a_{0}}{a_n z^n}\Bigr|
\end{align*}
となる。ここで
\[
R_0:=\max\Biggl\{\frac{n|a_{n-1}|}{0.1|a_n|},\Biggl(\frac{n|a_{n-2}|}{0.1|a_n|}\Biggr)^{\frac{1}{2}},\ldots,\Biggl(\frac{n|a_{0}|}{0.1|a_n|}\Biggr)^{\frac{1}{n}}\Biggr\}
\]
とおく\footnote{$0.1$という数に本質的な意味はありません。この$0.1$は、$1$未満の任意の正の数$\varepsilon$に置き換えて構いません。}と、$|z|>|R_0|$のとき、全ての$1\leq k\leq n-1$に対し
\[
\Bigl|\frac{a_{n-k}}{a_n z^k}\Bigr| = \frac{|a_{n-k}|}{|a_n||z|^k} < \frac{|a_{n-k}|}{|a_n|R_0^k} \leq \frac{|a_{n-k}|}{|a_n|}\Biggl(\frac{0.1|a_n|}{n|a_{n-k}|}\Biggr)^{\frac{1}{k}\cdot k} = \frac{0.1}{n}
\]
となる。よって
\[
\Bigl|1+\frac{a_{n-1}}{a_n z} + \cdots + \frac{a_0}{a_n z^n}\Bigr| \geq
1-\Bigl|\frac{a_{n-1}}{a_n z}\Bigr| - \cdots -\Bigl|\frac{a_{0}}{a_n z^n}\Bigr|
> 1-\frac{0.1}{n}-\cdots-\frac{0.1}{n} = 0.9
\]
となる。さらに正の実数$R$を
\[
R:=\max\Biggl\{\Biggl(\frac{|f(0)|}{0.9|a_n|}\Biggr)^{\frac{1}{n}},R_0\Biggr\}
\]
とおくと、$|z|>R$のとき
\[
|f(z)| = |a_n||z|^n\Bigl|1+\frac{a_{n-1}}{a_n z} + \cdots + \frac{a_0}{a_n z^n}\Bigr| > |a_n|\Biggl(\frac{|f(0)|}{0.9|a_n|}\Biggr)^{\frac{1}{n}\cdot n} \cdot 0.9 = |f(0)|
\]
となる。

\noindent \underline{step 2.} 複素平面上の実数値函数$|f(z)|$が最小値を持つことを示す。

半径$R$の閉円板$\Delta_R:=\{z\in\mathbb{C}\mid |z|\leq R\}$を考える。$|f(z)|$は$\mathbb{C}=\mathbb{R}^2$上の連続函数である。そして$\Delta_R$は平面$\mathbb{R}^2$内の有界閉集合だから、$|f(z)|$は$\Delta_R$上で最小値を取る。その点を$z_0$とする。いま$|z|>R$とすると、$R$の取り方から$|f(z)|>|f(0)|$が従う。一方$0\in\Delta_R$より$|f(z_0)|\leq |f(0)|$である。これらより$|z|>R$のときも$|f(z)|>|f(z_0)|$が従う。かくして$|f(z_0)|$は、複素平面全体における$|f(z)|$の最小値である。

\noindent \underline{step 3.} 方程式$f(z)=0$が解を持つことを示す。

もし$|f(z_0)|\neq 0$だとすると、$f(z_0)\neq 0$である。そこで新しい多項式$g(z)$を$g(z):=f(z+z_0)/f(z_0)$で定めることができる。このとき$g(0)=1$なので、問3の結果より、適当な複素数$z_1\in\mathbb{C}$を取ると$|g(z_1)|<1$とできる。ところが$|g(z_1)| = |f(z_1+z_0)| / |f(z_0)|$と合わせると$|f(z_1+z_0)| < |f(z_0)|$が導かれてしまう。これは$|f(z_0)|$が複素平面全体で$|f(z)|$の最小値になることに矛盾する。ゆえに$|f(z_0)|=0$である。つまり、$z_0$は$f(z)$の根である。 \qed



\part{線形代数学\CID{7555}}\label{part2}	% \CID{7555} は丸囲みの①

\end{document}
