\chapter{座標空間と数ベクトル}

\begin{flushright}
担当教員: 植野 義明 / TA: 穂坂 秀昭 \\
講義日時: 2015年5月7日 1限
\end{flushright}

\section{情報の調べ方}

\subsection{文献調査の一般論}

数学的なことではありませんが、今回の「vectorの語源を調べる」という問題に対して不適切な答案が多かったので、文献調査に関する最低限の基礎を述べておきます。きっと他の授業でも同じようなことを教わると信じていますが、念のため一読しておいてください。

まず\textbf{調査に使った文献を明記しないのは問答無用でアウト}です。今回は「調査すること」自体が課題ですから、どのような答案であってもどこかから引用されていることは分かります。ですが他のレポートで、出典を書かずにで他人の文章を取りこんだら、それは剽窃と呼ばれることになります。他人の文章を引っ張るときは「引用」という決まった作法に従い、自分の文章でないことを明らかにした上で、「誰がどこに書いたものなのか」を確実に分かるようにしてください\footnote{\url{http://www.bengo4.com/topics/1332/} に、著作権の視点からの分かりやすい説明があります。}。「たかが授業のレポートごときで」などと甘く見てはいけません。事が極限まで大きくなると、STAPの人がやらかしたように「学位論文の$1$つの章が丸ごとコピペ」という事態になるのですから\footnote{\url{http://stapcells.blogspot.jp/2014/02/blog-post_2064.html}}。

次に、出典を明記するにしても\textbf{信頼できる文献をきちんと選んで}引用しましょう。特にWikipediaは便利なもので、キーワードをインターネットでググると頻繁にヒットします。ですがWikipediaの記事は専門家が書いているとは限りません。大抵の場合Wikipediaの記事には出典が付いていますから、引用する場合は元の文献にきちんと当たり、孫引き (引用の引用) にならないようにしましょう。引用でなくとも[要出典]タグが付いた場所など、出典がはっきりしない部分を鵜呑みにしてはいけません。

Wikipedia以外のwebサイトでも同じです。「公的機関のwebサイトだから信頼できる」等の特別な事情がない限り、インターネット上の文書は「誰だか知らない人が書いたもの」に過ぎません。しばらく昔にYahoo!知恵袋を使って京都大学の受験生がカンニングしたことが話題になりましたが、その際Yahoo!知恵袋で「英文和訳問題の答」を教えた人は、翻訳ソフトを使っていたそうです\footnote{\url{http://www.47news.jp/CN/201102/CN2011022801000578.html}}。この程度の低品質な情報も平然と転がっているのですから、\textbf{インターネットで調べた情報を使うときは、その情報が信頼できるという裏付けを取ってください}。

実際``vector''の語源を一つとっても、インターネット上には不正確な情報があふれていることが垣間見えます。せっかくの機会ですから、少し気合を入れて情報を調べてみましょう。

\subsection{``vector''の語源}

\paragraph{ネット上の記事}

最終的には信頼できる文献を探さねばいけないとはいえ、調査の最初の段階ではインターネットを使うのが楽です。手始めに「vector 語源」などでググってみましょう。見つかったサイトの記述から、関連個所をいくつか引用してみます\footnote{以下でのwebサイトの引用は、いずれも2015/5/14に閲覧したときの情報に基づいています。}。

\begin{itemize}
\item \url{http://www.nn.iij4u.or.jp/~hsat/techterm/vector.html} \\
辞書を紐解くと, 数学用語の vector は 1843 年に数学者 William Rowen Hamilton (ハミルトン)(1805--1865) がラテン語 vectum から作ったとある。この vectum は「運ぶ」という意味であり, ギリシャ語の `\'oχoζ が語源。 更に遡ると veh\u{e}re に行き着くらしい。サンスクリットのヴァーハナは北インドで子供を守る女神の乗り物を意味するということである。
% カレッジクラウン英和辞典, 三省堂?
\item \url{ http://hooktail.sub.jp/vectoranalysis/restudyVector1/} \\
vectorという言葉は,1843年にHamiltonがラテン語のvectumから作った造語ですが,vectum とは運ぶもの,という意味です.更に遡ると,サンスクリット語で女神の乗り物を意味するヴァーハナと同根のようです.
\item \url{http://detail.chiebukuro.yahoo.co.jp/qa/question_detail/q1485274904}  \\
語源はラテン語です。ラテン語の意味は `carrier' 「運び屋」です。
\item \href{http://ja.wikipedia.org/wiki/\%E3\%83\%99\%E3\%82\%AF\%E3\%83\%88\%E3\%83\%AB}{\texttt{http://ja.wikipedia.org/wiki/ベクトル}} \\
ベクトルは ドイツ語: Vector\footnote{(引用者注) これはスペルミスです。ドイツ語での「ベクトル」は``Vector''ではなく``Ve\underline{k}tor''になります。}に由来し、ベクターは 英語: vector に由来する。…(中略)… vector は「運ぶ」を意味するラテン語: vehere に由来し、18世紀の天文学者によってはじめて使われた
\end{itemize}

これらの記事から、どうもラテン語の``vehere''とか``vectum''といった単語と関係があるようです。またHamiltonという数学者が関わっている気配も見てとれます。ところがvectorという言葉が生まれた時期について、1843年および18世紀という、$2$種類の食い違う記述が見られます。どちらが正しいのでしょう?もう少しWikipediaを調べてみましょう。\href{http://ja.wikipedia.org/wiki/\%E7\%A9\%BA\%E9\%96\%93\%E3\%83\%99\%E3\%82\%AF\%E3\%83\%88\%E3\%83\%AB}{\texttt{http://ja.wikipedia.org/wiki/空間ベクトル}} を見てみます。
\begin{quotation}
ハミルトンは1846年に四元数の複素数における実部と虚部に相当するものとしてスカラーとベクトルという用語を導入した(今日の用法とは異なることに注意されたい):
\end{quotation}

この記事は「ベクトルという用語を、Hamiltonが1846年に導入した」と書いてあります。18世紀でも1843年でもありません。ふと某名探偵の「真実はいつもひとつ!」という言葉が脳裏をよぎります。何が真実なのでしょうか?

\paragraph{書籍や論文の調査}

こういう時は、きちんとした文献の出番です。まず英単語としての語源を調べるなら、やはり使うべきは辞書でしょう。研究社から出ている『\href{http://webshop.kenkyusha.co.jp/book/978-4-7674-1016-6.html}{新英和大辞典 第6版}』\footnote{辞書のチョイスは個人的な好みによるものです。ちゃんとした辞書なら、何でもOKです。受講生の皆さんの中には、大修館書店の『\href{http://plaza.taishukan.co.jp/shop/product/detail/10177}{ジーニアス英和大辞典}』を引用した人もいました。}を引いてみると、p.~2727に大体こんなことが書いてあります\footnote{まるっきりこのままではありません。辞書の方では、凡例記号で色々省略がされています。}。この辞書は「18世紀にvectorが誕生」という説を支持するようです。
\begin{quotation}
vectorは元々ラテン語で `carrier'という意味の単語。そしてvectorはvehereの過去分詞形vectusから派生している\footnote{(引用者注) 水谷智洋編『改訂版 羅和辞典』(研究社) の記述と一致することを確認しました。またvectusは「動詞veh\=oの不定法現在形」だそうです。ラテン語を読めないので、この辺の事情を深く調べるのは諦めました。}。初出は1704年\footnote{(引用者注) ただし凡例によると、初出は「`利用可能な文献による限りの初出例の年代' ということで, 絶対的なものではない」とのこと。}。 
\end{quotation}

一方Wikipediaの「空間ベクトル」の記事には、Hamiltonが1846年に導入したという事実がきちんと出典付で書いてあります。それによれば``London, Edinburgh \& Dublin Philosophical Magazine 3rd series 29 27''が出典だそうです。皆さんはまだ見慣れないかもしれませんが、これは学術雑誌と呼ばれるものです。ですからこの文献を直接当たれば、1846年に``vector''という言葉をHamiltonが使ったか確認できますね。
%PDF http://www.maths.soton.ac.uk/EMIS/classics/Hamilton/OnQuat.pdf 16 ページ / articles 4, 15, 18, 物理図書に原論文蔵書あり

ここで皆さんの中に「1846年の文献なんてどこで探すんだよ」と思った人がいるかもしれません。最近は便利なもので、なんと昔の学術雑誌がスキャンされてインターネット上に公開されていたりするのです。実際、今回のも雑誌のタイトルでググれば見つかります\footnote{\url{http://www.biodiversitylibrary.org/bibliography/58679\#/summary} にある``ser.3:v.31 (1846:July-Dec)''です。}。その上、英語版WikipediaでHamiltonの記事 \footnote{\url{http://en.wikipedia.org/wiki/William_Rowan_Hamilton}} を見てみると、Referencesの節に``David R. Wilkins's collection of Hamilton's Mathematical Papers''\footnote{\url{http://www.maths.soton.ac.uk/EMIS/classics/Hamilton/}} というリンクがあります。なんとHamiltonの論文を、全てまとめて (しかも\TeX で打ち直して!\negthinspace) webサイトを作ってくれた人がいるのです。そこで \url{http://www.maths.soton.ac.uk/EMIS/classics/Hamilton/OnQuat.pdf} の16ページを見ると
\begin{quotation}
``... the algebraically imaginary part, being geometrically constructed by a straight line, or radius vector, which has, in general, for each determined quaternion, a determined length and determined direction in space, may be called the vector part ...''
\end{quotation}
という記述が見られます。Wikipediaの「空間ベクトル」の記事で引用されてたやつですね。これは1846年7月の記事らしいので、確かに1846年にHamiltonがvectorという用語を使ったことは間違いありません\footnote{さらに東京大学OPACで検索すると、本郷にある理学部の物理図書室にLondon, Edinburgh \& Dublin Philosophical Magazineの現物が眠っていることが分かります。1846年の文献でありながら、直接現物を確認することもできてしまうのです。}。

\paragraph{2種類の問題}

ここまでで、$2$つ確実に分かったことがあります。\vspace{-0.5zw}
\begin{itemize}
\item 新英和大辞典によると、vectorはラテン語の「運び屋」という意味の単語で、初出は1704年である
\item Hamiltonは1846年にvector partという用語を導入した
\end{itemize}\vspace{-0.5zw}
これらの事実は疑いようがありません。そこで浮かび上がってくるのは「vectorという英単語が誕生したのは1704年で、それを数学の文脈で最初に用いたのがHamiltonだった」という可能性です。

このような「一般的に使われている単語が、数学用語として用いられるようになる」という事案は、現代でも日常的に起こっています。たとえば ``cellular algebra'' という数学の言葉がありますが、これは1996年、J.~J.~Grahamと G.~I.~Lehrerという$2$人の数学者によって定義されたことがはっきりしています\footnote{正確な文献情報は次の通り: J.~J.~Graham, G.~I.~Lehrer, ``Cellular algebras'', \textit{Inventiones Mathematicae} \textbf{123}, 1--34.}。ですがcellularという単語は、1996年よりずっとずっと昔から存在していました\footnote{「Wikipediaによれば」1665年にRobert Hookeが細胞を``cell''と名付け、1674年にLeewenhoekが詳しく観察したそうです。気になる人はきちんと調べてみてください。}。そこでvectorについても、同じことが起きているのではないかと推測されるわけです。

インターネットで見つけた2つのwebページには、これらの記述と食い違う内容が書かれています。一番最初に挙げたwebサイトを見直すと「辞書を紐解いたら」Hamiltonがvectorを作ったと書かれているそうです。ところが参考文献として挙げられている、三省堂の『\href{http://www.sanseido-publ.co.jp/sinagire/col_cr_ej.html}{カレッジクラウン英和辞典}』\footnote{カレッジクラウン英和辞典第2版は1986年に出版された古い辞書ですが、頑張って現物を調達しました。}および岩波書店の『岩波理化学辞典 第5版』を実際に紐解いても、ベクトルの語源とか意味だけが書いてあって、「Hamiltonが作った」とはどこにも書いてありません。したがって、一番最初に挙げたwebサイトの著者はvectorについて「言葉が作られたこと」と「言葉を数学に持ち込んだこと」を混同していると推測されます\footnote{Hamiltonが1833年に``Introductory Lecture on Astronomy''という本を出版していることを考えると、Hamiltonが天文学の`vector'という言葉を知らなかった可能性はほとんど無いでしょう。}。また、1843年に出版されたHamiltonの論文全部を検索してもvectorという単語は出ず、1844年にやっと ``radius vector'' という単語が飛び出します。1843年説も間違いでしたね。このようにインターネットには、著者の勘違い等に基づく不正確な情報が転がっていることがしばしばあるのです。

ついでに言えばページが作成された日時と内容を見るに、「物理のかぎしっぽ」の記事は$1$つ目の文献を剽窃した上、精査していないようにも見えます。「クロ」とまで言い切る証拠はないですが、皆さんはこのような疑わしい文章を書かないように心がけましょう。

\paragraph{vectorの語源と初出は?}

これまでの調査で、vectorの語源がラテン語の動詞``vectus''であること、元々の意味が「運び屋」であることはほぼはっきりしました。ですが初出がまだはっきりしません。せっかくだから、もう少し頑張りましょう。こういう時に岩波書店の『数学辞典』が役立ちそうですが、残念ながら今回は欲しい情報が載っていませんでした。そこでWikipediaの「ベクトル」のページにある参考文献、Daniel Fleisch著、河辺哲次訳 『物理のためのベクトルとテンソル』 (岩波書店、2013年) のp.~1を開いてみます。
\begin{quotation}
\textbf{ベクトル}(vector)という言葉は「運ぶ」を意味するラテン語vehereに由来します.この言葉は,太陽の周りを「運ばれる」惑星の運動を研究していた18世紀の天文学者たちによって,初めて使われました.
\end{quotation}

新英和大辞典と合わせると、初出1704年で、かつ最初に天文学で使われたとみてよさようです。ですがこの本には「誰が使ったか」まで書いてないので、ここで\href{http://www.oed.com/}{Oxford English Dictionary} (OED) の第2版を持ち出してみましょう。19巻のp.~470で``vector''を引くと、こんな風に書いてあります。

\begin{quotation}
$\dagger$\textbf{1}. \textit{Astr.} (See quot. 1704) Also \textit{vector radius}, = \textit{radius vector} \textsc{Radius} 3 \textbf{e}. \textit{Obs.}
\textbf{1704} J. \textsc{Harris} \textit{Lex. Techn.} I. s.v., A Line supposed to be drawn from any Planet...
\end{quotation}

これでようやく、初出文献が分かりました。J.  Harris という人の ``Lex. Techn.'' という本だそうです。略されているため正確なタイトルが分からないので、東京大学OPACで検索してみましょう。「著者Harris, 出版年1704」で検索をかけると何も出ないのですが、他大学の図書まで含めた検索をかけると、放送大学図書館の蔵書がヒットします。そのレコードで著者とタイトルがJohn Harris, ``Lexicon technicum, or, An universal English dictionary of arts and sciences'' (1704) だと確定します。これを見れば、1704年に本当にvectorが登場したか分かります。

あとはどうやってこの本を調べるか、という問題が残ります。東京大学内の図書館に所蔵がありません。一つには、東京大学の図書館を経由して\href{http://lib.ouj.ac.jp/}{放送大学付属図書館}に文献の複写を依頼する手があります。また国立国会図書館の郵送複写サービス等も利用可能です。ところが今回「国立国会図書館サーチ」\footnote{\url{http://www.ndl.go.jp/index.html} の左上です。NDL-OPACとは別なので注意してください。}に書籍タイトルを突っ込むと、なんとこの本のスキャン画像が出てきます: \url{http://jairo.nii.ac.jp/0108/00002692} これの頭文字``V''のページを見れば、``Vector''の項目に\vspace{-0.5zw}
\begin{quotation}
A Line supposed to be drawn from any Planet moving round a Center, or the Focus of an Ellipsis, to that Center or Focus, is by some Writers of the New Astronomy, called the Vector; because 'tis that Line by which the Planet seems to be carried round its Center, and with which it describes proportional Areas in proportional Times.
\end{quotation}\vspace{-0.5zw}
と説明が書いてあります。確かに1704年に``vector''が登場したことがはっきりします\footnote{ちなみにvectorの初出については、Harrisの本よりさらに遡ろうとしてみました。しかしHarrisの本にある``the New Astronomy''がどこにあるのか、天文学専攻の友人に聞いても良く分からなかったので、諦めました。}。

ちなみにOEDには「数学の``vector''は、Hamiltonが1846年に使った」とも書かれていて、そこでは上で挙げたHamiltonの論文が引用されています。頑張れば、独自の調査でOEDの初出年代にたどり着くこともできるんですね。

% Michael J. Crowe, A History of Vector Analysis; see also his lecture notes on the subject
% http://ja.wikipedia.org/wiki/%E7%A9%BA%E9%96%93%E3%83%99%E3%82%AF%E3%83%88%E3%83%AB
% ハミルトンは1846年に四元数の複素数における実部と虚部に相当するものとしてスカラーとベクトルという用語を導入した: 代数的な虚部(ベクトル)は、幾何的には直線または半径ベクトルであり、それらは一般的には、各々の四元数によって決定され、空間における向きと長さが定まり、それをベクトル部(虚部)または単に四元数のベクトルと呼ぶ
% ベクトルの先祖は四元数であり、ハミルトンが1843年に複素数の一般化によって考案したものである。
% W. R. Hamilton (1846) London, Edinburgh & Dublin Philosophical Magazine 3rd series 29 27

\subsection{教訓}

このように、普段「ネットでちょこっと調べればいいや」と思いがちな情報検索も、頑張ろうとすると結構時間と手間がかかるものなのです。ただ、ここまで読んだ皆さんは「何か無駄が多くなかったか?」と思うかもしれません。それは正しいです。実際webサイトの記述を検証するために余分な手間がかかっていはいますが、それを差っ引いても
\begin{itemize}
\item 一番最初にOEDと『新英和大辞典』をセットで調べてみる
\item 出てきた参考文献の原本を、東京大学/国会図書館のOPACやHamiltonの業績コレクションを使って検索する
\end{itemize}
という順で作業してしまえば、もっと手早く正確な情報にたどり着けたはずです。また今回、「昔の書物がオンラインで見られること」が非常に重要な役割を果たしました。ですから、今回の例を教訓に
\begin{itemize}
\item 情報検索に「使える」文献や手段 (今回はOEDなど) を知っておくこと
\item オンラインで見られる文献を頑張って探すこと
\end{itemize}
の大事さを認識しておいてください。

最後に、これまでの``vector''の語源の話は全て検証可能です。ですからもしTAが嘘をついているとすれば、挙げた文献に当たればそれを見抜くことができます。今回そこまでするかどうかは皆さんにお任せしますが、先生やTAが言ったことも必要に応じて検証せねばならないし、もし検証不可能なことがあれば皆さんは考え、質問をしなければいけません。そのことを忘れないでください。

\section{空間内における直線と平面の取り扱い}

皆さんは既に、平面$\mathbb{R}^2$内の直線の扱いについては十分習熟しているはずです。たとえば$2$点を通る直線の方程式を求めるとか、直線の法線ベクトルを求めるといった計算は、すぐにできるでしょう。同じようなことを$1$次元上げもできるようになろう、というのが今回のお題の一つです。

まず最初に、平面を決定するための条件を確認しておきます。平面は
\begin{itemize}
\item $3$つの相異なる点
\item $1$つの点と、平面を張る$2$本の$1$次独立なベクトル
\item $1$つの点と、$1$本の法ベクトル
\end{itemize}
が与えられれば定まります。これは頭の中で想像すればすぐ分かります。また、平面の表し方には
\begin{itemize}
\item 方程式による表示
\item パラメータ表示
\end{itemize}
という$2$通りのやり方がありました。ですから
\begin{itemize}
\item 与えられた平面の決定条件から、平面の方程式やパラメータ表示を求めること
\item 平面の決定条件のうちどれか$1$つから、残りの$2$つの条件を導くこと
\item 平面の方程式とパラメータ表示をいったりきたりすること
\end{itemize}
がスムーズにできなければいけません。

それから、今回の問題では「$2$次元版と$3$次元版」が並列されていることに気づいた人が多いと思います。以下で見るように、問題の次元が$1$個変わっても、解き方はほとんど変わりません。そして皆さんが後で学ぶ「線型代数」とは、そのような\textbf{次元によらず同じように使えるテクニック}を身に着けることだと言えます。そのような「何で同じように問題が解けるのか」という原理に関しても、注意してみてください。

\subsection{直線と平面のパラメータ表示}

まず最初に、直線のパラメータ表示を復習しましょう。空間内の直線は、通る点$\bm{x}_0 = (x_0, y_0, z_0)$と方向ベクトル${}^t(a,b,c)$が\footnote{記号${}^t$は、縦ベクトルを横ベクトルに直す記号です。縦ベクトルを使うと紙面上で面積を消費するので、印刷物ではこんな書き方をすることがしばしばあります。また「別に横ベクトルでもいいじゃん」と思うかもしれませんが、後で行列の話をするとき、縦ベクトルと横ベクトルを区別する必要が出てきます。なので以下、空間内のベクトルは転置記号をつけて縦ベクトルにします。}与えられれば決まります。そのとき直線上の点$\bm{x}$に対し$\bm{x} - \bm{x}_0$は${}^t(a,b,c)$と平行なので、適当な実数$t\in\mathbb{R}$を用いて$\bm{x} - \bm{x}_0 = t\,{}^t(a,b,c)$ と書けます。逆に全ての実数$t$に対して$\bm{x}=\bm{x_0} + t\,{}^t(a,b,c)$は直線上の点です。こうして$\bm{x}=\bm{x_0} + t\,{}^t(a,b,c)$ ($t\in\mathbb{R}$) が直線のパラメータ表示を与えることが分かりました。平面の場合も同様です。通る点$\bm{x}_0$と平面を張るベクトル$\bm{u}$, $\bm{v}$が与えられると、$\bm{x} = \bm{x}_0 + a \bm{u} + b \bm{v}$ ($a,b\in\mathbb{R}$)がパラメータ表示です。

さて今度は、$2$点$\bm{p}_1$, $\bm{p}_2$が与えられたとします。この$2$点を通る直線の方向ベクトルは$\bm{p}_2 - \bm{p}_1$なので、直線のパラメータ表示は$\bm{x} = \bm{p}_1 + t(\bm{p}_2 - \bm{p}_1) = (1-t) \bm{p}_1 + t\bm{p}_2$となります。したがってベクトル$\bm{p}_1$と$\bm{p}_2$を係数が$1$になるよう足し合わせることで、直線の方程式が得られます。書き方を変えると、$s\bm{p}_1+t\bm{p}_2$ ($s + t = 1$) という表示もできます。

平面についても同じように、$\bm{p}_1$,  $\bm{p}_2$, $\bm{p}_3$を通る平面のパラメータ表示は$\bm{x} = \bm{p}_1 + s (\bm{p}_2 - \bm{p}_1) + t (\bm{p}_3 - \bm{p}_1) = (1-s-t) \bm{p}_1 + s \bm{p}_2 + t \bm{p}_3 $ ($s,t\in\mathbb{R}$)あるいは$\bm{x} = s \bm{p}_1 + t \bm{p}_2 + u \bm{p}_3$ ($s + t + u = 1$)となります。

これらの式は結構利用価値が高いので、覚えておいてください。

\subsection{平面の方程式と法ベクトル}

次に、平面の法ベクトルについて考えてみましょう。平面は、通る点と法ベクトルがあれば一通りに決定されます。そこで$\bm{x}_0={}^t(x_0, y_0, z_0)$を通り$\bm{n}={}^t(a,b,c)$を法ベクトルとする平面を考えます。すると平面上の任意の点$\bm{x}$に対し、$\bm{x}-\bm{x}_0$は$\bm{n}$と垂直になります。したがって
\[
0 = \bm{n} \cdot (\bm{x}-\bm{x}_0) = a(x - x_0) + b(y - y_0) + c(z - z_0)
\]
です。この式で$d := -a x_0 - b y_0 - c z_0$とおけば、平面上の点$\bm{x}$は方程式$ax+by+cz+d=0$を満たします。

逆に方程式$ax+by+cz+d=0$が与えられたとき、この方程式を満たす点全体の集合は平面になります。まず$(a,b,c)\neq(0,0,0)$なので、$ax+by+cz+d=0$の解が一つは存在します。たとえば$a\neq 0$なら、$\bm{x}_0 := {}^t(-d/a,0,0)$とすれば良いです。そして$\bm{x} = {}^t (x,y,z)$が方程式の解だとすれば、$ax+by+cz+d=0$なので
\[
0 = ax + by + cz + d  = {}^t (a,b,c) \cdot \bm{x} - {}^t (a,b,c) \cdot \bm{x}_0 = {}^t (a,b,c) \cdot (\bm{x}-\bm{x}_0)
\]
となります。こうして全ての解ベクトルは${}^t(a,b,c)$と垂直なので、方程式$ax+by+cz+d=0$の定める集合は平面だと分かりました。

この事実を踏まえれば、問6と問7が解けます。

\paragraph{問6, 7の解答} 平面$\Pi\colon ax+by+cz+d=0$\footnote{$\Pi$は、ギリシャ文字$\pi$ (パイ) の大文字です。}の法ベクトルは${}^t(a,b,c)$である。また点$(p, q, r)$が与えられたとき、この点を通り方向ベクトルが${}^t(a,b,c)$であるような直線$\ell$のパラメータ表示は${}^t(p, q, r)+\alpha\,{}^t(a, b, c)$ ($\alpha\in\mathbb{R}$)である。いま${}^t(a,b,c)$は平面$\Pi$の法ベクトルだから、平面$\Pi$と直線$\ell$は$1$点で交わる。すなわち平面$ax+by+cz+d=0$にベクトル${}^t(p, q, r)+\alpha\,{}^t(a, b, c)$が乗るような$\alpha\in\mathbb{R}$が唯一存在する。この$\alpha$を用いて、$(p, q, r)$から平面までの距離は$\|\alpha\,{}^t(a, b, c)\|=|\alpha|\sqrt{a^2+b^2+c^2}$と書ける。他方${}^t(p,q,r)+\alpha\,{}^t(a,b,c)$が方程式$ax+by+cz+d=0$を満たすので
\[
a(a\alpha + p) + b(b\alpha + q) + c(c\alpha + r ) + d = 0 \quad \text{すなわち} \quad \alpha(a^2+b^2+c^2) = -(ap+bq+cr+d)
\]
よって求める距離は
\[
|\alpha|\sqrt{a^2+b^2+c^2} = \frac{|ap+bq+cr+d|}{\sqrt{a^2+b^2+c^2}}
\]
で与えられる。この式で$c=r=0$と置けば、平面内の場合の式が得られる。 \qed

\paragraph{与えられた点を通る平面}

これまでの議論で、与えられた$3$点を通る平面を求める方法が分かりました。
\begin{itemize}
\item まずその方程式を$ax+by+cz+d=0$とおき
\item この方程式に$3$点の座標を代入して、$a,b,c$の連立$1$次方程式を得る
\item 得られた方程式を解く
\end{itemize}
とすればOKです。ただ、ここで出てくる方程式は$3$変数の連立$1$次方程式なので、一般の場合に解くのは面倒です。問2~問5のように、通る点の座標に$0$がたくさん出てくる場合は解くのが楽ですが、そうでない場合は
\begin{itemize}
\item パラメータ表示を使う
\item 後で説明するベクトルの外積を使い、法線ベクトルを求めてから方程式を得る
\end{itemize}
など別の手を使う方が賢いでしょう。

\paragraph{問2, 3の解答} 平面の方程式を$\alpha x + \beta y + \gamma z = \delta$とおき、この式に$(a,0,0)$, $(0,b,0)$, $(0,0,c)$を代入して、方程式
\[
\frac{x}{a} + \frac{y}{b} + \frac{z}{c} = 1
\]
が直ちに得られる。この式で$z/c$の項を落とせば、直線の式が得られる。 \qed

\paragraph{問4, 5の解答} 原点を通るので、方程式の定数項は$0$である。あとは座標$(1,0,a)$と$(1,0,b)$を代入すれば、求める式は$z = ax + by$と分かる。平面上の直線の場合は、同様にして$y = ax$が求まる。 \qed

\subsection{平面の方程式と決定条件}

% 平面の決定条件と方程式の解の存在

\paragraph{問8, 10の解答} $2$点$(p_1,q_1)$と$(p_2,q_2)$を通る直線がただ一つに決まる条件は$(p_1,q_1)\neq(p_2,q_2)$である。同様に$3$点$(p_1,q_1,r_1)$, $(p_2,q_2,r_2)$, $(p_3,q_3,r_3)$を通る平面がただ一つに決まる条件は、$3$点が同一直線上にないことである。 \qed

\paragraph{代数的な考察}
この問題の結果自体は、直感的にはほとんど明らかです。そもそもEuclidの『原論』にも、公理として「相異なる$2$点を通る直線がただ$1$つ存在する」と書かれています。が、ここでちょっと視点を変え「直線がただ一つに決まる」という事実を、代数的に捉えてみましょう。

平面$\mathbb{R}^2$内の直線$ax+by+c=0$が点$(p_1,q_1)$と$(p_2,q_2)$を両方通るとします。このとき
$a p_1 + b q_1 + c =0$および$a p_2 + b q_2 + c  = 0$という式が成り立ちます。この$2$つの式の差を取れば
\[
a(p_1-p_2) + b(q_1-q_2) = 0
\]
という式が得られます。逆にこの式を$a$と$b$の満たすべき方程式だと思えば、この$0$でない解の一つ一つが$(p_1,q_1)$と$(p_2,q_2)$の両方を通る直線に対応することになります。ただし$0$でない任意の実数$\alpha\in\mathbb{R}$に対し、方程式$ax+by+c=0$と$\alpha ax + \alpha by + \alpha c=0$は同じ直線を表します。ですから$(p_1-p_2) a + (q_1-q_2) b = 0$の解$(a,b)$に$0$でない定数$\alpha$をかけたとき、$(a,b)$と$(\alpha a,\alpha b)$は同じ直線に対応します。

いま$(p_1,q_1)\neq(p_2,q_2)$だとしたら、$(p_1-p_2 ) a + (q_1 - q_2) b = 0$は、原点を通る$(a,b)$平面の直線を表します。したがって、$(0,0)$以外のどの二つの解も定数倍でうつります。ゆえに直線はただ一つに定まります。ところが$(p_1,q_1)=(p_2,q_2)$なら、全ての$(a,b)$が解となってしまいます。これらの解のうちには定数倍でうつり合わないものが無数にありますから、$(p_1,q_1)$と$(p_2,q_2)$の両方を通る直線は無数に存在します。

このように「相異なる$2$点$(p_1,q_1)$, $(p_2,q_2)$を通る直線がただ一つ存在する」という条件が「方程式$(p_1 - p_2) a + (q_1 - q_2) b = 0$の$(0,0)$でない解が、定数倍を除いてただ一つに決まる」という代数的な言葉に言い換えられたわけです。若干言い方がまどろっこしいですが、少しして線型代数の勉強をすると「方程式$(p_1 - p_2) a + (q_1 - q_2) b = 0$の解空間が$1$次元」という、もっとすっきりした表現が使えるようになります。

空間の場合も話は同じです。平面の方程式$ax+by+cz+d = 0$に通る$3$点の座標を代入すると、$a,b,c,d$に関する$3$つの連立$1$次方程式が得られます。変数が$4$つで式が$3$つですから、この連立方程式の解はただ$1$つに決まらず、解を動かす自由度が残ります。ただ$0$でない定数を$(a,b,c,d)$にかけても平面自体は変わらないので、「$1$次元分の自由度」が「平面が$1$個に決まること」に相当します。

\subsection{直線の方程式}

ここまで使いませんでしたが、直線の方程式についても少しだけ触れておきます。パラメータ表示された直線$\ell\colon\,{}^t(x_0, y_0, z_0) + t\,{}^t(a, b, c)$が与えられたとします。このとき直線上の点$\bm{x}={}^t(x,y,z)$は、ある実数$t\in\mathbb{R}$に対し
\begin{align*}
\begin{cases}
x = x_0 + at \\
y = y_0 + bt \\
z = z_0 + ct
\end{cases}
\end{align*}
を満たします。この式を$t$について解くことで、直線上の点が満たす式
\[
\frac{x - x_0}{a} = \frac{y - y_0}{b} = \frac{z - z_0}{c}
\]
が得られるのでした。ちなみにこの式は$a,b,c$の中に$0$が混ざっているとマズいですが、そういう時も使えるような式を得るなら、分母を払っておけば大丈夫です。$a,b,c$の中に、少なくとも$1$つは$0$でないものがあります。それが$a$だった場合は、今の方程式を
\[
b(x - x_0) = a(y - y_0), c(x - x_0) = a(z - z_0)
\]
と書き換えれば問題ありません。逆にこの方程式が与えられていれば、$t = (x - x_0)/a$とおくことで、$a\neq 0$と合わせて$y - y_0=bt$, $z - z_0 = ct$が得られます。こうして直線の方程式からパラメータ表示を復元することもできました。

ここで、直線の方程式は\vspace{-1zw}
\begin{align*}
\begin{cases}
\Pi_1\colon bx - ay - (b x_0 - a y_0) = 0 \\
\Pi_2\colon cx - az - (c x_0 - a z_0) = 0
\end{cases}
\end{align*}
という$2$本の平面の方程式を連立して得られています。これらの平面は
\begin{itemize}
\item いずれも点$(x_0, y_0, z_0)$を通り
\item それぞれの法ベクトル$(b, -a, 0)$と$(c, 0, -a)$が共に$\ell$の方向ベクトル$(a,b,c)$と垂直
\end{itemize}
なことに気を付けましょう。したがって平面$\Pi_1$, $\Pi_2$は共に直線$\ell$を含んでいることが分かります。すなわち直線$\ell$の方程式は、$\ell$を\textbf{$2$枚の平面$\Pi_1$, $\Pi_2$の共通部分}として表しているわけです。

このような手は、他の場合にも応用できます。空間内の図形をつかむには、その図形をいくつかの分かりやすい図形の共通部分として表しておいて、それぞれのパーツの方程式を作れば良いのです。そうすれば連立方程式の形で、求める図形を表す方程式が得られます。

\section{ベクトルの$1$次独立性と行列式}

\subsection{ベクトルの$1$次独立性}

\paragraph{$1$次独立性}
空間内の$n$本のベクトル$\bm{v}_1, \ldots, \bm{v}_n$が$1$次独立であることの定義は、$\alpha_1 \bm{v}_1 + \cdots + \alpha_n \bm{v}_n$となる$(\alpha_1,\ldots,\alpha_n)$が$(0,\ldots,0)$以外に存在しないことでした。$1$次独立でなく、たとえばもし$\alpha_1=1$だったなら、$\bm{v}_1 = -\alpha_2 \bm{v}_2 - \cdots - \alpha_n \bm{v}_n$というように、$\bm{v}_1$が他のベクトルのスカラー倍と足し算で表されます。なので$1$次独立性は、大体「ベクトルたちがバラバラの方向を向いている」ということに相当します。また$1$次独立でないことを$1$次従属と呼びます。これを踏まえた上で、問9, 11の解答を見てみましょう。

\paragraph{共線条件と共面条件 (問9, 11の解答)}

$2$点$(p_1,q_1)$, $(p_2,q_2)$を通る直線のパラメータ表示は、$s\,{}^t(p_1,q_1)+t\,{}^t(p_2,q_2)$ ($s+t=1$) で与えられます。したがって$3$点が同一直線上にあるためには、$(p_3,q_3)=s\,{}^t(p_1,q_1)+t\,{}^t(p_2,q_2)$, $s+t=1$という連立方程式が解を持つか、$(p_1,q_1)=(p_2,q_2)$であることが必要十分です。

ただ、この方程式は「ベクトルのパラメータ表示」と「$1$次方程式」がバラバラに出てきてちょっと扱いづらいです。そこで平面ベクトルの話ですがわざと次元を上げて、$\,{}^t(1, p_1, q_1)$, $\,{}^t(1, p_2, q_2)$というベクトルを考えてみます。そうすると$s\,{}^t(1, p_1, q_1) + t\,{}^t(1, p_2, q_2) = \,{}^t( s+t, s p_1,+ t p_2, s q_1 + t q_2)$となるので、求める条件が「$s\,{}^t(1, p_1, q_1)+t\,{}^t(1, p_2, q_2) = \,{}^t(1, p_3, q_3)$となる$s$, $t$が存在すること」でまとまります\footnote{この「一度次元を上げてから、見やすいように書き直す」という手法は、他の場面でもしばしば用いられます。}。この式は移項すれば
\[
s\,{}^t(1, p_1, q_1)+t\,{}^t(1, p_2, q_2) - \,{}^t(1, p_3, q_3) = \bm{0}
\]
となります。すなわち「$\,{}^t(1,p_1,q_1)$, $\,{}^t(1, p_2, q_2)$, $\,{}^t(1, p_3, q_3)$が$1$次従属であること」が求める条件です。

空間の場合も、議論は全く同じです。求める共面条件は「ベクトル$\,{}^t(1,p_1,q_1,r_1)$, $\,{}^t(1,p_2,q_2,r_2)$, $\,{}^t(1,p_3,q_3,r_3)$, $\,{}^t(1,p_4,q_4,r_4)$が$1$次従属であること」となります。

\subsection{ベクトルの外積}

\paragraph{問12の解答} ${}^t(p_1,q_1)$に垂直なベクトルは、明らかに$\alpha\,{}^t(-q_1,p_1)$ ($\alpha\neq 0$)である。実際内積を取れば、
${}^t(p_1, q_1)\cdot\alpha \,{}^t(-q_1,p_1) = \alpha(- p_1 q_1 + q_1 p_1 ) =0$となる。

\paragraph{ベクトルの外積 (問13の解答)}

平面$\mathbb{R}^2$の場合はこのように垂直なベクトルはすぐ求まるわけですが、空間の場合はどうなるでしょうか。すなわち空間の場合、$2$本のベクトル$\bm{u}$, $\bm{v}$が与えられたとき、両方のベクトルに垂直なベクトルを作るにはどうすれば良いでしょうか。実は「ベクトルの外積」というものを使うと、それを求めることができます。

$\bm{u} = {}^t(p_1, q_1, r_1)$, $\bm{v} = {}^t(p_2, q_2, r_2)$に対し、$\bm{u}$と$\bm{v}$の外積は$\bm{u}\times\bm{v} := (q_1 r_2-q_2 r_1, r_1 p_2-r_2 p_1, p_1 q_2-p_2 q_1)$という式で与えられるベクトルです。そうすると
\begin{align*}
\bm{u}\cdot(\bm{u}\times\bm{v})
&= p_1(q_1 r_2-q_2 r_1) +  q_1(r_1 p_2-r_2 p_1) + r_1(p_1 q_2-p_2 q_1) = 0\\
\bm{v}\cdot(\bm{u}\times\bm{v})
&= p_2(q_1 r_2-q_2 r_1) + q_2(r_1 p_2-r_2 p_1) + r_2(p_1 q_2-p_2 q_1) = 0
\end{align*}
となるので、確かに$\bm{u}\times\bm{v}$は$\bm{u}$および$\bm{v}$と垂直になることが分かります。今は天下りな定義を与えましたが、これから行列式を使い、もう少しマシな説明をします。

\subsection{行列式}

ベクトルの$1$次独立性 / 従属性や外積の話は「行列式」というものを知っていると良く理解できます。本格的に勉強するのは後ですから、ここではラフに話をしておきます。

色々な定義がありますが、$\mathbb{R}^2$内の$2$本の縦ベクトル$\bm{u}$, $\bm{v}$の張る平行四辺形の符号付き面積\footnote{符号は、$\bm{u}$から$\bm{v}$に向かう方向が正の向きであるときに$+$とし、負の向き$-$とします。}を$\det( \bm{u} \  \bm{v} )$と書きます。また$\mathbb{R}^3$内の$3$本の縦ベクトル$\bm{u}$, $\bm{v}$, $\bm{w}$の張る平行六面体の符号付き体積\footnote{符号は、$\bm{u}$, $\bm{v}$, $\bm{w}$がこの順で右手系をなすときに$+$, 左手系をなすときに$-$とします。}を$\det( \bm{u} \ \bm{v} \ \bm{w})$と書きます。このとき、$\det$をベクトルの成分で書き下すと
\[
\det
\begin{pmatrix}
a & c \\
b & d 
\end{pmatrix}
= ad-bc, \quad
\det
\begin{pmatrix}
p_1 & p_2 & p_3 \\
q_1 & q_2 & q_3 \\
r_1 & r_2 & r_3
\end{pmatrix}
=
p_1 \det
\begin{pmatrix}
q_2 & q_3 \\
r_2 & r_3
\end{pmatrix}
-
q_1 \det
\begin{pmatrix}
p_2 & r_3 \\
r_2 & p_3
\end{pmatrix}
+
r_1 \det
\begin{pmatrix}
p_2 & p_3 \\
q_2 & q_3
\end{pmatrix}
\]
となることが知られています。

さて平面内に$2$本のベクトルがあったとき、これらが$1$次独立だったら平行四辺形が張れます。逆に$1$次従属だったら、平行四辺形を張ろうとすると面積が$0$になります。したがって$\bm{u}$, $\bm{v}$が$1$次独立であることと$\det( \bm{u} \ \bm{v} ) \neq 0$が同値になります。$3$次元以上の場合でも同様に$\det$を使えば、すごくややこしくなりますが「ベクトルが$1$次独立であるか」を成分計算する式が得られるのです。これを使えば共線・共面条件をたった$1$本の式で書き下せます。

さらに$3$次元の$\det$を注意深く観察すると、
\[
\det
\begin{pmatrix}
p_1 & p_2 & p_3 \\
q_1 & q_2 & q_3 \\
r_1 & r_2 & r_3
\end{pmatrix}
=
{}^t(p_1, q_1, r_1)
\cdot
{}^t
\Biggl(
\det
\begin{pmatrix}
q_2 & q_3 \\
r_2 & r_3
\end{pmatrix}
, 
-
\det
\begin{pmatrix}
p_2 & p_3 \\
r_2 & r_3
\end{pmatrix}
, 
\det
\begin{pmatrix}
p_2 & p_3 \\
q_2 & q_3
\end{pmatrix}
\Biggr)
\]
のように、$\det$が$2$本のベクトルの内積で書けることが分かります。ここで$2$次元のときの$\det$を見ると、列の入れ替えで符号が$(-1)$倍されると分かります。よって今の式で$\det$の前の$-$が消せて
\[
\det
\begin{pmatrix}
p_1 & p_2 & p_3 \\
q_1 & q_2 & q_3 \\
r_1 & r_2 & r_3
\end{pmatrix}
=
{}^t(p_1, q_1, r_1)
\cdot
{}^t
\Biggl(
\det
\begin{pmatrix}
q_2 & q_3 \\
r_2 & r_3
\end{pmatrix}
, 
\det
\begin{pmatrix}
r_2 & r_3 \\
p_2 & p_3
\end{pmatrix}
, 
\det
\begin{pmatrix}
p_2 & p_3 \\
q_2 & q_3
\end{pmatrix}
\Biggr)
\]
となります。この右辺に出てくる、成分に$\det$が入ったベクトルが、さっき外積と呼んだものです。したがって、ここまでの議論で$\det(\bm{u}\ \bm{v}\ \bm{w}) = \bm{u}\cdot(\bm{v}\times\bm{w})$が分かりました。この式を使えば、なぜ$\bm{v}\times\bm{w}$が$\bm{v}$と直交するのか一目瞭然ですね。$\bm{v}\cdot(\bm{v}\times\bm{w})$はベクトル$\bm{v}$, $\bm{v}$, $\bm{w}$の張る平行六面体の体積ですが、この平行六面体はぺっちゃんこなので、どう見ても体積$0$です。こうして内積の値が$0$となり、$\bm{v}$と$\bm{v}\times\bm{w}$が直交すると分かります。$\bm{w}$についても同じです。

このようにベクトルの$1$次独立性を$\det$と関連付けることによって、ベクトルが一段と便利な道具になります。また$\det$を用いてベクトルの情報を得る過程には、$\det$が持つ諸々の計算公式が重要な役割を果たします。そういう事も追々勉強しますので、楽しみにしていてください。

\section{残りの問題}

\paragraph{問14の解答} $3$つの点が同一直線上にないことが必要十分条件です。このとき$3$点は三角形をなすから、外接円が$3$点全てを通過します。続いて円の方程式を$(x-a)^2  + (y-b)^2 = r^2$とおく。これに$(x, y) = (p_1, q_1), (p_2, q_2), (p_3, q_3)$を代入すると、$a, b, r$に関する$3$変数の連立一次方程式が得られる…のですが、これを真面目に解いてもあまり徳しません。$\triangle\mathrm{ABC}$の$3$点の位置ベクトルを$\bm{a},\bm{b},\bm{c}$とすると、外心が$\displaystyle \frac{\sin 2\angle A \bm{a} + \sin 2\angle B \bm{b} + \sin 2\angle C \bm{c}}{\sin2\angle A + \sin 2\angle B + \sin 2 \angle C}$と表せます。こっちの式の方が使い勝手が良いと思います\footnote{頂点の取り方を工夫していくらか計算してみましたが、座標ベースではあんまり綺麗な式や面白い式が得られませんでした。すいません。}。

\paragraph{問15の解答}

存在しないことを背理法で示す。全ての頂点が格子点である正三角形があったとする。このとき
\begin{itemize}
\item 成分が整数であるようなベクトルに沿って格子点を平行移動したものは、再び格子点になる
\item 図形を平行移動すると、元の図形と合同になる
\end{itemize}
という事実があるので、正三角形の頂点の$1$つが原点であると仮定して差支えない。そこで正三角形の頂点を$(0,0)$, $(a,b)$, $(c,d)$ ($a,b,c,d\in\mathbb{Z}$)とする。

いま正三角形の一辺の長さは$\sqrt{a^2+b^2}$である。したがって正三角形の面積は
\[
\frac{1}{2}\times \sqrt{a^2+b^2} \times \frac{\sqrt{3}}{2}\sqrt{a^2+b^2} = \frac{\sqrt{3}(a^2+b^2)}{4}
\]
である。一方、座標を使って計算すると面積が$|ad-bc|/2$と求まる。したがって
\[
\frac{\sqrt{3}(a^2+b^2)}{4} = \frac{|ad-bc|}{2} \quad \text{より} \quad \sqrt{3} = \frac{2|ad-bc|}{a^2+b^2} \in \mathbb{Q}
\]
となる。これは$\sqrt{3}$が無理数である事実に反する。 \qed

