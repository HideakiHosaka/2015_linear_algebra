\chapter{線型写像と行列}

\begin{flushright}
担当教員: 植野 義明 / TA: 穂坂 秀昭 \\
講義日時: 2015年5月27日 1限
\end{flushright}

\section{問題訂正のお詫びと雑談}

最初に問題訂正のお知らせです。今回の問題の$4$番と$5$番に間違いがあります。すいませんでした。どこが間違えていたのかは、それぞれの問題の解説で詳しく書きます。$5$番に取り組んだ人については、多くの人が「何かがおかしい」ということに気づいていたようです。また何人か、問題が間違っている事実を的確に見抜いた人もいました。\\

ところで少々むちゃくちゃなことを言いますが、\textbf{先生やTAの言うことがいつも正しいなどと思ってはダメです}。もちろん (少なくとも、このプリントを書いているTAの穂坂は) 意図的に間違ったことを言おうと思っているわけではなく、なるべくミスをなくす努力をしています。とは言っても人間である以上、何かの拍子に間違いが入ることは避けられません。ですから最終的には、皆さん自身が物事が正しいかどうかを判断しなければいけません。

特に数学では命題の真偽が白黒はっきりします\footnote{残念ながら (?) ごく稀に白黒はっきりつかない問題もあります。それは「G\"odelの不完全性定理」によって与えられます。せっかくなので、少し「証明不可能な命題」について、雰囲気を紹介してみたいと思います。この部分は読み飛ばしても、本編には全く影響しません。

G\"odelの不完全性定理は「自然数の算術 (Peano算術) を含むいかなる体系にも、真偽のいずれも証明できない命題が存在する」ことを主張します。そして不完全性定理には「真偽の判定が不可能な命題の存在」を保証する第$1$不完全性定理と、これより一段と強く、証明不可能な命題を具体的に与える第$2$不完全性定理とがあります。僕たちがやっている数学では、そのうち真偽が判定できない命題に突き当たるのです。

もう少し詳しいことを言うと、僕たちが普段「数学」と呼んでいるものはZermelo--Fraenkelの公理系 (ZF) に選択公理 (Axiom of Choice; AC) という公理を付け加えた``ZFC''という公理系の上に構築されています。ZFCでは自然数の算術を扱えますので、G\"odelの不完全性定理により、この公理系で真偽の判定が不可能な命題$P$が存在します。この場合ZFCに証明不可能な命題$P$を公理として付け加えても、その否定を付け加えても、元のZFCより強い公理系ができます。僕たちは、このどちらに従って数学をしても良いのです。もちろん必要にならない限りは「証明不可能な命題に手を触れない」という立場で問題は起こりません。

このような「真偽のいずれも証明できない命題」の中で最も良く知られたものの一つが「連続体仮説 (Continuum Hypothesis; CH)」と呼ばれるものです。これを説明するには「無限集合の個数」に関する概念が必要なので、少しだけ説明しましょう。

$2$つの集合$A$と$B$が「同じ個数」であることは、どうやれば判定できるでしょうか?$A$と$B$が共に有限集合であるときは、元の個数を数えれば同じ個数かどうかが分かります。ところが無限集合の場合には「数える」ことができません。そこで無限集合$A$, $B$が「同じ個数」であることを「全単射$f\colon A\rightarrow B$が存在すること」で定義します。「集合の元の間に$1:1$の対応が付けば、同じ個数と呼んでいいだろう」というわけです。この「同じ個数である」ということを、数学では「$2$つの集合が同じ濃度である」と言います。また無限集合の「個数」に相当するものを、集合の「濃度」と言います。

さて、自然数全体の集合$\mathbb{N}$と実数全体の集合$\mathbb{R}$はどちらも無限集合です。また$\mathbb{N}\subset\mathbb{R}$ですから、少なくとも$\mathbb{R}$の濃度は$\mathbb{N}$の濃度以上であることは間違いありません。そして実は「Cantorの対角線論法」と呼ばれる手法により、「$\mathbb{N}$から$\mathbb{R}$への全単射が存在しない」ことが証明できます。ですから実数全体の集合$\mathbb{R}$の濃度は、自然数全体の集合$\mathbb{N}$の濃度より真に大きくなります。ところが色々なものを探しても「自然数と実数の間の濃度を持つ集合」が見つからないのです。そこで「実数と自然数の間の濃度を持つ集合は存在しないのではないか」という仮説が登場しました。これが「連続体仮説」です。

そして我々の数学の体系ZFCのもと、G\"odelが1940年に連続体仮説の否定が証明不可能なこと、Cohenが1963年に連続体仮説が証明不可能なことをそれぞれ示しています。ですから、この「連続体仮説」が、まさに「白とも黒ともつかない数学的命題」にあたります。数学というと「証明すること」ばかりだと思われがちですが、その中には「証明不可能なことを証明する」という不思議な話もあるのです。

ちなみに、この説明を書くにあたり
\begin{itemize}
\item G\"odelの不完全性定理に関しては、鹿島亮『\href{http://www.asakura.co.jp/G_12.php?isbn=ISBN978-4-254-11765-3}{数理論理学}』(朝倉書店) を
\item 連続体仮説に関しては、東大数理科学研究科の古田幹雄先生の解説 \url{http://www.ms.u-tokyo.ac.jp/~furuta/BernsteinZorn.pdf} を
\end{itemize}
それぞれ参考にしました。興味のある人は読んでみてください。
}から、問題や解説が間違っているときは「反例」が見つかるはずです。授業で何か怪しいことを見つけたときは「先生やTAが言ってることなんだから、自分の方が間違っているんだろう」などと思わず「本当に正しいのか」を疑ってかかってください。そして間違いを見抜くことができたら、それを正した上で問題に取り組むなどしてください。

\section{数ベクトルに対する線型写像}

前回の授業で天下りに行列とその演算を定義しましたが、今回はいよいよ「その心」を説明します。一言で言ってしまうと、行列とは「$\mathbb{R}^n$から$\mathbb{R}^m$への線型写像」と呼ばれる特別な写像を実現する道具なのです。その意味を、これから確認していきましょう。

\subsection{数ベクトルに対する線型写像の定義}

$f\colon\mathbb{R}^n \rightarrow \mathbb{R}^m$が線型写像であるとは、
\begin{itemize}
\item 任意の$\bm{u}$, $\bm{v}\in\mathbb{R}^n$に対し、$f(\bm{u} + \bm{v}) = f(\bm{u}) + f(\bm{v})$
\item 任意の$\bm{u}\in\mathbb{R}^n$, $\alpha\in\mathbb{R}$に対し、$f(\alpha\bm{u}) = \alpha f(\bm{u})$
\end{itemize}
が満たされることをいいます。難しいことを言っているような気もしますが、$m = n = 1$とおけば$f$は普通の函数になります。そして$u$がただの数なので、$f(u) = u f(1)$となります。つまり$f$は定数項のない$1$次函数 (正比例の式) です。だから線型写像は、\textbf{$1$次函数$y = ax$をベクトルに対して一般化したもの}と言えます。こう言えば、線型写像を考えるご利益がありそうだという気がしてくると思います。そして実際、線型写像は色々なところで出てきます。

線型写像の一つの例は、行列の積です。$A$を$(m,n)$型行列とします。このとき$\bm{v}\in\mathbb{R}^n$に対し、$A\bm{v}\in\mathbb{R}^m$です。したがって「$A$を$\mathbb{R}^n$のベクトルに左からかける」という操作によって、$A$は写像$\mathbb{R}^n\rightarrow\mathbb{R}^m$を定めます。そして
\begin{itemize}
\item 任意の$\bm{u}$, $\bm{v}\in\mathbb{R}^n$に対し、$A(\bm{u} + \bm{v}) = A\bm{u} + A\bm{v}$
\item 任意の$\bm{u}\in\mathbb{R}^n$, $\alpha\in\mathbb{R}$に対し、$A(\alpha\bm{u}) = \alpha A\bm{u}$
\end{itemize}
が成り立ちます。よって$A$倍写像は線型写像であることが分かりました。

\paragraph{問2の解答} 抵抗や起電力の具体的な値は本質的でないので、最も一般的な状況で議論する。図のように\footnote{問題中の図では抵抗がギザギザの折れ線で表されていました。この記号は今でもよく使われてはいますが古いですので、ここでは現行の規格JIS C 0617-4に従い、箱で抵抗を表しました。}電池の起電力$V_1, V_2$、抵抗値$r_0, r_1, r_2$および電流$I_0, I_1, I_2$を定める。このとき
\begin{itemize}
\item 上側の分岐点におけるKirchhoffの第$1$法則から$I_0 = I_1 + I_2$
\item 左右それぞれのサイクルに対するKirchhoffの第$2$法則から$V_1 = r_1 I_1 + r_0 I_0, V_2 = r_2 I_2 + r_0 I_0$
\end{itemize}
\begin{figure}[h!tbp]
\centering
\begin{picture}(240, 140)
% 横の導線: 下
\put(20,20){\line(1,0){200}}
% 横の導線: 上
\put(20,120){\line(1,0){35}}		% 左の線
\put(85,120){\line(1,0){70}}		% 真ん中の線
\put(185,120){\line(1,0){35}}		% 上の線
% 縦の導線: 一番左
\put(20, 20){\line(0,1){52}}		% 下の線
\put(20, 75){\line(0,1){45}}		% 上の線
% 縦の導線: 真ん中
\put(120, 20){\line(0,1){30}}		% 下の線
\put(120, 80){\line(0,1){40}}		% 上の線
% 縦の導線: 一番右
\put(220, 20){\line(0,1){52}}			% 下の線
\put(220, 75){\line(0,1){45}}			% 上の線
% 電池 V_1
\put(15, 72){\line(1,0){10}}			% マイナス極
\put(10, 75){\line(1,0){20}}			% プラス極
\put(0, 70){$V_1$}
% 電池 V_2
\put(215, 72){\line(1,0){10}}			% マイナス極
\put(210, 75){\line(1,0){20}}			% プラス極
\put(231, 70){$V_2$}
% 抵抗
\put(115, 50){\framebox(10, 30){}} 	% r_0
\put(100,60){$r_0$}
\put(55, 115){\framebox(30, 10){}}	% r_1
\put(65 , 130){$r_1$}
\put(155, 115){\framebox(30, 10){}}	% r_2
\put(165 , 130){$r_2$}
% 電流
\put(115, 115){\vector(0, -1){25}}		% I_0
\put(100 , 100){$I_0$}
\put(95, 125){\vector(1,0){25}}		% I_1
\put(100 , 130){$I_1$}
\put(150, 125){\vector(-1,0){25}}		% I_2
\put(130 , 130){$I_2$}
\end{picture}
\end{figure}
が成り立つ\footnote{要は「導線の接点において、流れ込む電流と流れ出る電流は釣り合う」「回路を一周すると、抵抗での電圧低下と電池の起電力が釣り合う」と言っているだけです。物理を習っていなくても、こう書けば「それはそうなるだろう」という気がしてきませんか?また、式さえ立ててしまえば、後は単なる連立一次方程式の問題です。}。これらの式から$I_0$を消去し、方程式を行列で書くと
\begin{align*}
\begin{pmatrix}
V_1 \\
V_2
\end{pmatrix}
=
\begin{pmatrix}
r_0 + r_1 & r_0 \\
r_0 & r_0 + r_2
\end{pmatrix}
\begin{pmatrix}
I_1 \\
I_2
\end{pmatrix}
\end{align*}
が成り立つ\footnote{$I_0$を消去せずに$3$次正方行列を使って方程式を書き下すこともできますが、計算方法を知っていたとしても、$3$次正方行列の逆行列を求めるのは若干面倒です。今の場合は簡単に変数$I_0$を消去できるので、$2$次正方行列を使った計算に持ち込みました。}。よって、逆行列を両辺に左からかけると
\begin{align*}
\begin{pmatrix}
I_1 \\
I_2
\end{pmatrix}
=
\frac{1}{r_1 r_2 + r_0 (r_1 + r_2)}
\begin{pmatrix}
r_0 + r_2 & -r_0 \\
-r_0 & r_0 + r_1
\end{pmatrix}
\begin{pmatrix}
V_1 \\
V_2
\end{pmatrix}
\end{align*}
だから
\begin{align*}
I_0 = I_1 + I_2
&= \frac{1}{r_1 r_2 + r_0 (r_1 + r_2)}
\begin{pmatrix}
1 & 1
\end{pmatrix}
\begin{pmatrix}
r_0 + r_2 & -r_0 \\
-r_0 & r_0 + r_1
\end{pmatrix}
\begin{pmatrix}
V_1 \\
V_2
\end{pmatrix}
= \frac{1}{r_1 r_2 + r_0 (r_1 + r_2)}
\begin{pmatrix}
 r_2 & r_1
\end{pmatrix}
\begin{pmatrix}
V_1 \\
V_2
\end{pmatrix} \\
&= \frac{1}{r_1 r_2 + r_0 (r_1 + r_2)} (r_2 V_1 + r_1 V_2)
\end{align*}
となる。確かにこれは、$V_1 = 0$のときの解と$V_2 = 0$のときの解を足し合わせたものになっている。数値の代入はやればできるので省略する。

\paragraph{線型性} いまの答えの式を
\[
I_0
=
\frac{1}{r_1 r_2 + r_0 (r_1 + r_2)}
\begin{pmatrix}
 r_2 & r_1
\end{pmatrix}
\begin{pmatrix}
V_1 \\
0
\end{pmatrix}
+
\frac{1}{r_1 r_2 + r_0 (r_1 + r_2)}
\begin{pmatrix}
 r_2 & r_1
\end{pmatrix}
\begin{pmatrix}
0 \\
V_2
\end{pmatrix}
\]
と書き換えてみます。こうすれば「$V_1 = 0$のときの$I_0$の値」「$V_2 = 0$のときの$I_0$の値」を足したものが最終的な$I_0$の値になることがよりはっきり見えます。そして、この式の書き換えで使ったのは紛れもなく\textbf{行列をかける操作の線型性}です。だからこの問題は「$I_0, I_1, I_2$が$V_1, V_2$に対して線型に依存すること」を確認していたことになります。

もう少し別の線型写像の例も見てみましょう。

\paragraph{問1の解答} (1) 任意の$\bm{x}$, $\bm{y}\in\mathbb{R}^n$と実数$\alpha$に対し
\begin{align*}
\overline{\bm{x} + \bm{y}}
&= \frac{1}{n} \sum_{i = 1}^n (x_i + y_i) = \frac{1}{n} \sum_{i = 1}^n x_i +\frac{1}{n} \sum_{i = 1}^n  y_i
= \overline{\bm{x}} + \overline{\bm{y}} \\
\overline{\alpha \bm{x}}
&= \frac{1}{n} \sum_{i = 1}^n \alpha x_i 
= \alpha \frac{1}{n} \sum_{i = 1}^n x_i 
= \alpha \overline{\bm{x}}
\end{align*}
が成り立つ。よって平均を取る写像は線型である。

\noindent (2) 元の問題では$n = 5$であるが、一般の場合で示す。$p_1,\ldots,p_n$を、$p_1 + \cdots + p_n = 1$かつ$p_i > 0$を満たす実数とする。また$\bm{x} = {}^t(x_1,\ldots, x_n)\in\mathbb{R}^n$とする。このとき期待値は
\[
E[\bm{x}] = p_1 x_1 + p_2 x_2 + \cdots + p_n x_n = \sum_{i = 1}^n p_i x_i
\]
で与えられる。よって任意の$\bm{x}, \bm{y}\in\mathbb{R}^n$と$\alpha\in\mathbb{R}$に対して
\begin{align*}
E[\bm{x} + \bm{y}] &= \sum_{i = 1}^ n p_i (x_i + y_i) = \sum_{i = 1}^ n p_i x_i + \sum_{i = 1}^ n p_i  y_i = E[\bm{x}] + E[\bm{y}] \\
E[\alpha \bm{x}] &= \sum_{i = 1}^ n p_i \alpha x_i = \alpha \sum_{i = 1}^ n p_i x_i = \alpha E[\bm{x}]
\end{align*}
が成り立つ。すなわち期待値を取る写像は線型である。

\paragraph{問1の補足}
上の解答では線型性を定義に従って示しました。ですが (2) で$\bm{p} = {}^t (p_1, \ldots, p_n)\in\mathbb{R}^n$とおくと、$E[\bm{x}] = {}^t \bm{p}\bm{x}$という式が成り立ちます。つまり期待値を取る写像$E$は横ベクトル${}^t \bm{p}$を左からかける写像と一致するので、ここから線型性が分かります。また (2) で$p_1 = \cdots = p_n := 1/n$とおけば、(1) の場合になります。

\subsection{線型写像の行列表示}

行列の積は線型写像でしたが、実は逆に、\textbf{全ての線型写像は行列の積で書くことができます}。

いま$f\colon \mathbb{R}^n \rightarrow \mathbb{R}^m$を線型写像とします。このとき$1\leq i\leq n$に対し、第$i$成分が$1$でそれ以外の成分が$0$である行列を$\bm{e}_i$と書きます。これを用いて$\bm{v}_i :=f(\bm{e}_i)\in\mathbb{R}^m$とおき、さらにベクトル$\bm{v}_1,\ldots,\bm{v}_n$たちを並べて$A = (\bm{v}_1 \  \bm{v}_2 \  \cdots \ \bm{v}_n)$という行列を作ります。この行列のサイズは$(m,n)$型です。つまり、$f$による$\bm{e}_1,\ldots,\bm{e}_n$の行先を並べて行列を作るわけです。

そして$\mathbb{R}^n$のベクトル$\bm{v}$を成分表示し$\bm{v} = {}^t (a_1, a_2, \ldots, a_n) = \sum_{i = 1}^n a_i \bm{e}_i$と書くと
\[
f\biggl(\sum_{i = 1}^n  a_i \bm{e}_i \biggr) = \sum_{i = 1}^n  f( a_i \bm{e}_i ) = \sum_{i = 1}^n  a_i f(\bm{e}_i) = \sum_{i = 1}^n  a_i \bm{v}_i
\]
が成り立ち、その一方で
\[
A\sum_{i = 1}^n  a_i \bm{e}_i = \sum_{i = 1}^n  a_i A\bm{e}_i = \sum_{i = 1}^n  a_i \bm{v}_i
\]
も成り立ちます。よって$f(\bm{v}) = A\bm{v}$です。$\bm{v}$の取り方は任意だったので、これで線型写像$f$と$A$倍写像が一致することが示せました。この$A$を\textbf{線型写像$f$の行列表示}といいます。ナイーブには「線型写像$=$行列」と思ってよい、ということになります。

\subsection{平面$\mathbb{R}^2$上の線型変換}

線型写像$f\colon \mathbb{R}^n\rightarrow\mathbb{R}^m$において、特に$n = m$が満たされる場合、$f$を\textbf{線型変換}と呼ぶことがあります。$m \neq n$なら$\mathbb{R}^n$と$\mathbb{R}^m$は異なるので、$f$は空間$\mathbb{R}^n$の点を別の空間$\mathbb{R}^m$の点に対応させるだけです。ところが$m = n$なら定義域と値域が同じ集合ですから、$f$は「空間$\mathbb{R}^n$の中で点を移動させる操作」と思うことができます。そこで$m = n$の場合、線型写像$f$は特別に「線型変換」と呼ばれます。

\paragraph{等長写像と等角写像} 

線型写像や線型変換は、一般に図形の「形」を保ちません。線型性のおかげでそこまで大幅に形が崩れたりはしないのですが、たとえば「長方形をつぶして平行四辺形にする」くらいの変形が起きます。ところが運が良いと、線型写像が「任意の$2$点間の距離を保つ」とか「任意の$2$本のベクトルのなす角を保つ」といった良い性質を持つことがあります。これらの典型例は、回転行列です。

こうした写像はしばしば登場するので、名前が付いています。任意の$2$点間の距離を保つ写像を\textbf{等長写像}といい、任意の$2$本のベクトルのなす角を保つ写像を\textbf{等角写像}といいます。また等角写像であるような線型変換は、あまり「等角変換」とは呼ばれず、\textbf{共形変換}と呼ばれることが多いです。

線型変換の場合、等長変換ならば必ず共形変換になります。これは三角形の合同条件に「三辺相等」があるからです。どの$2$点間の距離も保たれるなら三角形の形がそのまま保たれるので、したがって角度も保たれるというわけです。この事実をベクトルを用いて証明するのが、問4の(1)$\Rightarrow$(2)です。ところが共形変換であっても、必ずしも等長変換になるとは限りません。$2$つの三角形の内角が揃っても、その$2$つの三角形は一般に合同にはならず、相似にしかならないからです。ですから線型な共形変換は相似変換ですが、等長性までは保証されません。

\paragraph{問4の訂正} というわけで、問4の\textbf{問題が半分間違っていました}。ごめんなさい。(1)$\Rightarrow$(2)は正しいですが、(2)$\Rightarrow$(1)は正しくありません。反例はたとえば$2$倍写像$f(\bm{u}) := 2\bm{u}$です。

ベクトル$\bm{u}$, $\bm{v}$のなす角は$\bm{u}\cdot\bm{v}/|\bm{u}||\bm{v}|$で与えられます。いま$f$を$2$倍写像とすると、$f(\bm{u})\cdot f(\bm{v})/|f(\bm{u})||f(\bm{v})| = 2\bm{u}\cdot2\bm{v}/|2\bm{u}||2\bm{v}| = \bm{u}\cdot\bm{v}/|\bm{u}||\bm{v}|$なので、任意のベクトルのなす角は保存されます。しかし$f$は長さを保ちません。

\paragraph{問4 (1)$\Rightarrow$(2)の解答} 任意のベクトル$\bm{u}, \bm{v} \in \mathbb{R}^2$に対し
\begin{align*}
|\bm{u} + \bm{v}|^2 - |\bm{u} - \bm{v}|^2
&= (\bm{u} + \bm{v})\cdot(\bm{u} + \bm{v}) - (\bm{u} - \bm{v})\cdot(\bm{u} - \bm{v}) = 4\bm{u}\cdot\bm{v}
\end{align*}
が成り立つ。よって$F\colon\mathbb{R}^2\rightarrow\mathbb{R}^2$が等長線型変換なら、$|F(\bm{u}) + F(\bm{v})| = |\bm{u} + \bm{v}|$, $|F(\bm{u}) - F(\bm{v})| = |\bm{u} - \bm{v}|$なので
\[
F(\bm{u})\cdot F(\bm{v}) = \frac{|F(\bm{u}) + F(\bm{v})|^2 - |F(\bm{u}) - F(\bm{v})|^2}{4}
= \frac{|\bm{u} + \bm{v}|^2 - |\bm{u} - \bm{v}|^2}{4} = \bm{u}\cdot\bm{v}
\]
となる。つまり$F$は内積を保っている。 そして$F$は等長写像だから、角度$\bm{u}\cdot\bm{v}/|\bm{u}||\bm{v}|$も保つ。 \qed

\paragraph{問5の解答}

(1) $SR(\theta) = R(\theta)S$とすると
\begin{align*}
SR(\theta) &= 
\begin{pmatrix}
1 & 0 \\
0 & -1
\end{pmatrix}
\begin{pmatrix}
\cos\theta & -\sin\theta \\
\sin\theta & \cos\theta
\end{pmatrix}
=
\begin{pmatrix}
\cos\theta & -\sin\theta \\
-\sin\theta & -\cos\theta
\end{pmatrix}\\
R(\theta) S &= 
\begin{pmatrix}
\cos\theta & -\sin\theta \\
\sin\theta & \cos\theta
\end{pmatrix} =
\begin{pmatrix}
1 & 0 \\
0 & -1
\end{pmatrix}
\begin{pmatrix}
\cos\theta & \sin\theta \\
\sin\theta & -\cos\theta
\end{pmatrix}
\end{align*}
これらが等しいので$\sin\theta = -\sin\theta$、つまり$\sin\theta = 0$である。よって$\theta = n\pi$ ($n\in\mathbb{Z}$)である。

\noindent (2) 次の計算により、$SR(\theta)S$は$(-\theta)$回転であると分かる。
\[
SR(\theta)S = 
\begin{pmatrix}
1 & 0 \\
0 & -1
\end{pmatrix}
\begin{pmatrix}
\cos\theta & -\sin\theta \\
\sin\theta & \cos\theta
\end{pmatrix}
\begin{pmatrix}
1 & 0 \\
0 & -1
\end{pmatrix}
=
\begin{pmatrix}
\cos\theta & \sin\theta \\
-\sin\theta & \cos\theta
\end{pmatrix}
= R(-\theta)
\]

\paragraph{問5の訂正と線型空間の向き} いま見たように、どう見ても$SR(\theta)S$は「折り返し」ではなく、回転を表します。これも問題が間違えていました。すいません。

ちなみに「向き」を考えると、$SR(\theta)S$が折り返しでないという確証が得られます。平面上の$1$次独立な$2$本のベクトル$\bm{u}$, $\bm{v}$について、始点を揃えたとき$\bm{u}$から$\bm{v}$へ向かう方向が正の向き (反時計回り) になることを「$(\bm{u}, \bm{v})$は正の向きである」と呼びましょう。たとえば$(\bm{e}_1, \bm{e}_2)$は正の向きです。このとき線型変換$A$に対し「正の向きであるベクトルのペアを正の向きにうつす」ということと、$\det A>0$が同値になります。たとえば回転行列と折り返しは、それぞれ$\det R(\theta) = 1 > 0$, $\det S = -1 <0$を満たします。そして実際、正の向きのベクトルのペアを回転させたものは正の向きだし、折り返したものは負の向きになります。確かに「向きを保つかどうか」と$\det$の符号に対応がついていますね。

一方、行列式は$\det AB = \det A \det B$という性質を持つのでした。ですから$\det S R(\theta) S = \det S \det R(\theta) \det S = (-1)\cdot 1\cdot (-1) = 1$となり、$S R(\theta) S$は向きを保ちます。一方でどんな折り返し変換も向きを反転させますから、その行列式は負でないといけません。したがって、$S R(\theta) S$は絶対に折り返しにならないと分かりました。

\section{線型空間}

これまで数ベクトル空間上の線型写像を調べました。ここでもう一歩、線型写像の考察を進めてみましょう。僕たちは数ベクトルの性質を使って線型写像を定義したわけですが、その発想を逆転させてみます。「線型写像が定義できるには、定義域と値域にどんな性質があれば良いでしょうか?」\footnote{ちなみに、この手の発想は数学ではよく見かけます。「○○は××の性質を持つ」ということが分かったら、逆に「××という性質が上手く定まるのはどういう場合か」と考え、抽象化を図るのです。たとえば函数の「連続性」の概念から位相空間の考え方に到達します。}

その答えが「線型空間」と呼ばれるものです。線型空間は数ベクトル空間の性質を一部だけ抜き出したもので、これを使うと、数ベクトル空間でなくても線型写像の概念を定義することができます。

\subsection{線型空間の定義}

後でちゃんと線型空間の定義を理解する必要がありますが、今は軽く眺め「数ベクトル空間だったら当たり前に成り立つ性質だよなあ」と納得してみてください。

\paragraph{線型空間}

次の条件を満たす集合$V$を\textbf{線型空間}あるいは\textbf{ベクトル空間}と言います。

\begin{itemize}
\item $V$上に加法が定義されており
\begin{itemize}
\item 任意の$\bm{u}, \bm{v}, \bm{w}\in V$に対し$(\bm{u} + \bm{v}) + \bm{w} = \bm{u} + (\bm{v} + \bm{w})$が成り立つ
\item 任意の$\bm{u}\in V$に対し$\bm{u} + \bm{0} = \bm{0} + \bm{u} = \bm{0}$となる$\bm{0}\in V$が唯一存在する
\item 任意の$\bm{u}\in V$に対し、$\bm{u} + \bm{v} = \bm{0}$となる$\bm{v}\in V$が唯一存在する (この$\bm{v}$を$-\bm{u}$と書く)
\item 任意の$\bm{u}, \bm{v}\in V$に対し$\bm{u} + \bm{v} = \bm{v} + \bm{u}$が成り立つ
\end{itemize}
\item $V$上にスカラー倍が定義されている
\begin{itemize}
\item 任意の$\alpha, \beta\in\mathbb{R}$と$\bm{v}\in V$に対し$\alpha(\beta\bm{v})=(\alpha\beta)\bm{v}$が成り立つ
\item 任意の$\bm{v}\in V$に対し、$1\bm{v}=\bm{v}$が成り立つ
\end{itemize}
\item スカラー倍と和が分配法則を満たす
\begin{itemize}
\item 任意の$\bm{u}, \bm{v}\in V$と$\alpha\in\mathbb{R}$に対し$(\alpha + \beta)\bm{v} = \alpha\bm{v} + \beta\bm{v}$
\item 任意の$\bm{u}\in V$と$\alpha,\beta\in\mathbb{R}$に対し、$\alpha(\bm{u} + \bm{v}) = \alpha\bm{u} + \alpha\bm{v}$
\end{itemize}
\end{itemize}

これらの性質から、$0\bm{v}=\bm{0}$, $(-1)\bm{v} = -\bm{v}$が成り立つことが次のように示せます。
\begin{itemize}
\item $0 \bm{v} = (0 + 0) \bm{v} = 0\bm{v} + 0\bm{v}$の両辺に$-(0\bm{v})$を足して、$\bm{0} = 0\bm{v} + \bm{0} = 0\bm{v}$
\item $\bm{0} = 0\bm{v} = \bigl(1 + (-1)\bigr)\bm{v} = \bm{v} + (-1) \bm{v}$より、$-\bm{v} = (-1)\bm{v}$
\end{itemize}

ちなみにスカラーが実数体$\mathbb{R}$のとき、\uline{実}線型空間という言い方をします。スカラーを複素数体$\mathbb{C}$にすると、今と全く同じようにして\uline{複素}線型空間が定義されます。

\subsection{例: 微分方程式の解空間}

線型空間の定義を抽象化したことによって、我々は「数ベクトル空間でない線型空間」を扱えるようになりました。その中で最も重要なものが、微分方程式の解空間と呼ばれるものです。どのような分野に進んでも、おそらく微分方程式と全く無縁な人生を過ごす人はいないでしょう。たとえば「時々刻々と変化する量」を扱えば、確実に時間を変数とする微分方程式が出ます。そして単振動の方程式や波動方程式、熱方程式などの基本的な微分方程式の多くが、これから説明する「線型微分方程式」と呼ばれるクラスに属します。また解くのが難しい微分方程式も、線型微分方程式で近似して解くことが良くあります。こうした線型微分方程式を解くとき、線型代数が極めて重要な役割を果たします。

まずは例を見てみましょう。

\paragraph{問3の解答} 微分方程式$y'' - y' - 6y = 0$を考える。

\noindent (1), (2) 一般に$y = e^{\alpha x}$に対し、$y' = \alpha y$を満たす。よって$y'' - y' - 6y = (\alpha^2 - \alpha - 6) y = (\alpha - 3)(\alpha + 2) y$である。これより$y = e^{\alpha x}$が$y'' - y' - 6y =0$を満たすことは、$\alpha = -2$または$\alpha = 3$と同値である。

\noindent (3) $y$を微分方程式の解、$C\in \mathbb{R}$とする。このとき$(Cy)' = Cy'$, $(Cy)'' = Cy''$であるから、$(Cy)'' - (Cy)' - 6(Cy) = C(y'' - y' - 6) = 0$である。よって$Cy$も解である。

また$y_1, y_2$を微分方程式の解とする。このとき$(y_1 + y_2)'' - (y_1 + y_2)' - 6(y_1 + y_2) = (y_1'' - y_1' - 6 y_1) + (y_2'' - y_2' - 6 y_2) = 0$であるから、$y_1 + y_2$もまた微分方程式の解である。

\noindent (4) は (1), (2), (3) から直ちに従う。

\noindent (5) 微分方程式の解$y(x) = C_1 e^{3x} + C_2 e^{-2x}$が初期条件$y(0) = \alpha$, $y'(0) = \beta$を満たすとする。このとき$\alpha = y(0) = C_1 + C_2$, $\beta = y'(0) = 3C_1 - 2C_2$なので、$C_1 = (2\alpha + \beta)/5, C_2 = (3\alpha - \beta)/5$である。

\paragraph{線型微分方程式の解空間}

今の問題の(3)で示したことを一言で言えば「$y'' - y' - 6y = 0$の解全体は線型空間をなす」ということです。この証明で本質的なのは「微分の線型性」と「$y,y',\ldots$の項が斉次$1$次式\footnote{ここで「斉次」というのは、定数項が$0$という意味です。}」という点でした。

一般に$a_n(x) y^{(n)} + a_{n - 1}(x) y^{(n-1)} + \cdots + a_1(x) y' + a_0(x) y = 0$のように、微分方程式であって$y, y', y'',\ldots$に関する斉次$1$次式であるものを\textbf{線型微分方程式}といいます。問3と全く同様にして、この方程式の解全体は線型空間をなすことが示せます。そこで線型微分方程式の場合、解全体の集合を\textbf{解空間}といいます。

線型微分方程式の場合、$2$つの解を足したら新しい解を作ることができます。すなわち\textbf{解の重ね合わせ}ができます。たとえば「波」を表す函数は波動方程式という線型偏微分方程式の解として与えられます。波については重ね合わせの原理が成り立つことはよく知られていますが、微分方程式のレベルで見ると、これは波の満たす方程式が線型だという事実に対応しています。

\paragraph{線型空間の基底}

線型空間$V$のベクトルの組$(\bm{u}_1, \ldots, \bm{u}_n)$であって「任意の$\bm{v}\in V$が$\bm{v} = a_1 \bm{v}_1 + a_2 \bm{v}_2 + \cdots + a_n \bm{u}_n$の形に一意的に表せる」という性質を持つものを、$V$の\textbf{基底}と言います。たとえば第$i$成分だけが$1$で他の成分が$0$であるようなベクトルを$\bm{e}_i$と書くと、$\mathbb{R}^n$の基底として$(\bm{e}_1, \bm{e}_2, \ldots, \bm{e}_n)$が取れます。

実は一般の場合でも、線型空間には常に基底が取れることが知られています。ですから基底を$1$つ取ってしまえば、どんなベクトルもその$1$次結合で表せます。特に線型微分方程式の解空間の場合、\textbf{基底をなす解を見つけてしまえば、他の全ての解は基底の重ね合わせで得られる}ということです。今の方程式$y'' - y' - 6y = 0$では、$(e^{3x}, e^{-2x})$が解空間の基底であることが示せます。だから線型代数の議論によって、$C_1 e^{3x} + C_2 e^{-2x}$ ($C_1, C_2\in\mathbb{R}$) が\textbf{全ての解を尽くすこと}が保証されるのです。

また方程式$y'' - y' - 6y = 0$については、初期条件として$y(0)$と$y'(0)$を決めるごとに解が$1$つ定まります。ですから解の自由度は$2$です。そして (5) の問題で見たように、初期条件$y(0) = \alpha$, $y'(0) = \beta$に対応する解を求めることは、${}^t(\alpha, \beta) = C_1{}^t(1, 3) + C_2{}^t(1, -2)$を満たすスカラー$C_1, C_2$を求めることと同じでした。つまり初期条件を満たす解、$e^{3x}$と$e^{-2x}$とが、それぞれベクトル${}^t(\alpha, \beta)$, ${}^t (1,3)$と${}^t (1,2)$とにぴったり対応しているわけです\footnote{線型写像のことをもう少しちゃんと学ぶと、この「微分方程式の解と数ベクトルとが対応する」という事実は「解空間と$\mathbb{R}^2$の間の線型同型」という言葉できちんと定式化できます。}。そして「解の自由度が$2$である」という事実は、解空間の次元が$2$であるという事実にちょうど対応しているのです。「自由度」という言葉は必ずしも厳密な定式化が易しくありませんが、線型微分方程式の場合はこのように「解空間」という概念を導入することで、その次元としてきちんと定式化ができるのです。このようにして、線型代数の議論がいかに強力であるかを垣間見ることができます。
