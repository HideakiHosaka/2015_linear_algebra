\chapter{複素数と代数学の基本定理}

\begin{flushright}
担当教員: 植野 義明 / TA: 穂坂 秀昭 \\
講義日時: 2015年4月15日
\end{flushright}

\section{はじめに}

\paragraph{ごあいさつ}

みなさん、はじめまして。この授業のTA (ティーチング・アシスタント)をすることになりました、数理科学研究科博士課程の穂坂といいます。これから$1$セメスターの間、よろしくお願いします。

この授業では基本的に毎回レポート問題が課され、それをTAの穂坂が添削して返却することになっています。また答案返却に合わせて、この文書のような解説プリントを配付していく予定です。何か疑問要望等があれば、随時質問をしてください。

\paragraph{プリントのオンライン公開}

このプリントは毎回教室で配布するのに加え、インターネット上でも配布されます。そのURLは下記の通りですので、必要な方はアクセスしてください。

\paragraph{このプリントの作り方について}

せっかくなので、このプリントをどう作っているかについて説明しておきます。

ふつう「コンピュータで文書を作る」というと、大抵の人がMicrosoft Wordとか一太郎といったワープロソフトを連想すると思います。ところが残念なことに、市販のワープロソフトでは数式を入力するのに大変な苦労を強いられてしまいます。そこで数学が専門の人はどうするかというと、そういったワープロソフトの代わりに``\LaTeX''というソフトウェアを使います。これはD.~E.~Knuthという非常に有名な数学者・計算機科学者が作った``\TeX''というソフトウェアを、色々な人が改良してできあがったものです。

\LaTeX はワープロソフトとはちょっと違い、文字のスタイルを変えたり見出しをつけたりするのに「コマンド」というものを使います。ですから\LaTeX を使うにはコマンドの使い方を覚えなければいけません。加えてキーボードが打ち込んだものが、見た目通りに出てくるわけでもありません。一旦コマンドも含めて打ち込んだ文書を\LaTeX というプログラムに処理させることによって、やっと整形された文書がでてきます。ですから使い始めるにはちょっとハードルが高いのですが、使い慣れるとワープロソフトよりも手際よく文書が書けるし、また数式を中心として文書の仕上がりが美しいというメリットもあります。

もしかしたら皆さんの中には\LaTeX を既に知っている人がいるかもしれませんし、また将来\LaTeX を使う必要に迫られる人がいるかもしれません。そこでこの文書の\LaTeX ソースコードを
\begin{center}
\url{https://github.com/HideakiHosaka/2015_linear_algebra}
\end{center}
に置いておきます。もし\LaTeX の方に興味がある人は、こちらを見てください。また東京大学が持つ情報処理システムのオンライン自習教材「はいぱーワークブック」の第27章\footnote{\url{http://hwb.ecc.u-tokyo.ac.jp/current/applications/latex/}}に、\LaTeX の説明があります。\LaTeX を使う人は、一度読んでおくと良いと思います。

ちなみにソースコードの公開には``GitHub''というサービス\footnote{もしあなたが既に``GitHub''を知っているなら、きっと``pull request''の機能も知っているはずです。プリントに対して何か意見があれば、積極的にpull requestを送ってください。\textsf{\raisebox{1pt}{:}D}}を利用しています。上に貼ったURLを開くと、古いバージョンのプリントや、そうしたプリントがどう更新されていったかも見ることができます。授業自体の役には立たないと思いますが、興味があれば見てみてください。

\section{複素(数)平面の幾何}

\subsection{複素数と複素平面}

\paragraph{複素数の定義}

まず、最初に複素数の定義をおさらいしましょう。$i^2=-1$という規則で$i$という「数」\footnote{誰もが一度は「$-1$の平方根を数と呼んでいいのか」という疑問を抱いたことがあると思います。その疑問に対する答えをまだ書いていなかったので、ここでは括弧つきで「数」と書きました。でも一々こう書くと面倒なので、以下では括弧をつけないことにします。}を定めます。このとき、$2$つの実数$x,y\in\mathbb{R}$を用いて$z=x+yi$と表される数を複素数と言うのでした。また$z=x+yi$の$x$を実部、$y$を虚部と言うことも知っているはずです。足し算と掛け算はそれぞれ、分配法則などが上手く成り立つよう
\begin{align*}
(x+yi) + (x'+y'i) &= (x+x')+(y+y')i, &  (x+yi)(x'+y'i) &= (xx' - yy') + (xy'+x'y)i
\end{align*}
と決めていました。また、これらのルールがあれば
\begin{align*}
\frac{x+yi}{x'+y'i}
= \frac{(x+yi)(x'-y'i)}{(x'+y'i)(x'-y'i)} = \frac{(xx'+yy')+(x'y-xy')i}{(x')^2+(y')^2}
= \frac{xx'+yy'}{(x')^2+(y')^2} + \frac{x'y-xy'}{(x')^2+(y')^2}i
\end{align*}
というように割り算もできます。こうして複素数では四則演算が全部できると分かりました。

\paragraph{複素平面}

さて、全ての複素数は$x+yi$の形に、$(x,y)$という二つの実数$x,y$のペアを用いて表せます。また複素数を二つの実数$x,y$で$x+yi$の形に表す方法がただ一通りなことも明らかでしょう。これらの事実\footnote{集合と写像の言葉できちんと書くと、$(x,y)\in\mathbb{R}^2$に$x+yi\in\mathbb{C}$を対応させる写像$\mathbb{R}^2\rightarrow\mathbb{C}$が全単射、ということです。}から、\textbf{複素数$z=x+yi$と座標平面上の点$(x,y)$とを$1:1$に対応させられる}ことが分かります。このように平面$\mathbb{R}^2$上の点一つ一つを複素数と見なしたとき、平面$\mathbb{R}^2$のことを\textbf{複素(数)平面}\footnote{複素平面と複素数平面は、どっちの言葉も同じ意味です。複素平面の方を使う人が多いですが、複素数平面と言っても誤解を招くことはないですし、昔は複素数平面という言葉も割と良く使われていたそうです。関西学院大学の示野信一先生のブログに詳しい事情が書いてありますので、気になる人は読んでみてください: \url{http://mathsci.blog41.fc2.com/blog-entry-60.html}}と呼びます。

\begin{figure}[h!tbp]
\begin{center}
\begin{picture}(200,100)
\put(15,0){\vector(0,1){100}}
\put(0,15){\vector(1,0){200}}
\put(15,15){\dashbox{1.2}(100,50)}
\put(7,6){$0$}
\put(112,8){$x$}
\put(7,63){$y$}
\put(115,65){\circle*{3}}
\put(118,67){$x+yi$}
\end{picture}
\caption{複素平面における数と点の対応}
\end{center}
\end{figure}

少し大げさに言うと、我々は「複素数という代数的なもの」と「平面という幾何的なもの」を対応付けました。このことには、非常に重要な意味があります。なぜなら\textbf{代数の観点と幾何の観点を行ったりきたりすることで、色々なことが分かるようになる}からです。たとえば長さなどの幾何的な情報を代数的な操作で捉えたり、逆に掛け算などの代数的な操作を図形的に捉えたりというように。これから、それをやってみましょう。

\subsection{幾何を代数で捉える}

\paragraph{複素数の大きさと共役}

幾何的な情報の最も典型的なものとして「$2$点間の距離」が挙げられます。複素平面の場合、原点$0$と$z\in\mathbb{C}$との距離を$z$の\textbf{絶対値}または\textbf{大きさ}と言い、$|z|$で表します。三平方の定理から、すぐに$|x+yi|=\sqrt{x^2+y^2}$が従います。またベクトルのときと同様、複素数$z,z'\in\mathbb{C}$の間の距離は$|z-z'|$となります。

幾何的な観点からは、「線/点対称移動」といった操作を考えられるというメリットもあります。たとえば「実軸に対する線対称移動」で$z$が写る点を$\overline{z}$と書くと、$\overline{x+iy}=x-iy$です。この$\overline{z}$を$z$の\textbf{共役}と言います。共役は$\overline{z+w}=\overline{z}+\overline{w}$, $\overline{zw}=\overline{z}\overline{w}$という性質を満たすことが、計算で確かめられます。また共役を用いて、複素数の大きさを表すことができます。

\begin{comment}
\paragraph{複素数と複素数平面: 問1 の解答}
$z=x+iy$のとき、$z\overline{z}=(x+iy)(x-iy)=x^2+y^2=|z|^2$となる。$z=0$であることは$z$と原点$0$との距離が$0$であることと同値なので、直ちに$z=0\Leftrightarrow|z|=0$を得る。

\paragraph{複素数と複素数平面: 問4 の解答}

(2)は(1)を使えば$|z_1-z_2| = |z_1 + (-z_2) | \leq |z_1|+|{-z_2}| = |z_1| + |z_2|$と分かる。
\end{comment}

\subsection{代数を幾何で捉える}

続いて、四則演算という代数的な操作を複素平面で見てみましょう。複素数の足し算や引き算がベクトルの足し算や引き算と全く同じであることは、すぐに分かると思います。非自明なのは\textbf{平面上の点やベクトルには掛け算が定義されていないのに対し、複素数には掛け算がある}という点です\footnote{「ベクトルにも内積や外積があるじゃん」という声が聞こえてきそうですが、内積や外積は、いわゆる普通の「積」とは違う性質を持ちます。$2$つのベクトルの内積は数になってしまい、ベクトルにはなりません。また$2$つのベクトルの外積はベクトルになりますが、ベクトルの外積は順序を入れ替わると結果が変わります。この辺が数の掛け算と全然違うところです。}。そこで「$2$つの複素数を掛け算した結果は、複素平面上ではどのように見えるか」が問題となってきます。まずは問題を$1$つ解いてみましょう。

\paragraph{複素数と複素数平面: 問2の解答} $z=2+i$とおくと
\begin{align*}
z^2 &= 3+4i, & z^3 &= 2 + 11i, & z^4 &= -7+24i, & z^5 &= -38+41i, & z^6 &= -117+44i
\end{align*}
である。これらをプロットした結果は次の通り。

% TODO: ここに図をはさむ

\paragraph{極形式表示}

いま複素数$z\neq 0$に対し$z^0=1$, $z^1=1$, $z^2$, $z^3$, $z^4$, $z^5$, $z^6$を平面上にプロットし、かつこれらの点を原点と結んだ結果、隣り合う角が全て同じ大きさであるように見えます。それを実際に確かめてみましょう。

実軸からの角度を計算したいので、極座標を使うのが筋がよさそうです。そこで$z=x+yi$の点$(x,y)$を、極座標$(r,\theta)$を用いて$x=r\cos\theta$, $y=r\sin\theta$と表します。これを対応する複素数の方で表すと
\begin{align*}
z =x+yi = r\cos\theta + (r\sin\theta)i = r(\cos\theta+i\sin\theta)
\end{align*}
となります。この書き方を、複素数の\textbf{極形式表示}と呼びます。この書き方のもとで$r=|z|$です。また$\theta$は、複素平面の半直線$0z$と実軸の非負の部分がなす角を、実軸から反時計回りに測った角度となっています。この$\theta$を複素数$z$の\textbf{偏角}と言い、$\theta=\arg z$と書きます。$\arg z$は一通りでなく$2\pi$の整数倍だけずらせますが、今は気にしないでおきます。

さて、極形式で表された$2$つの複素数を掛け算すると
\begin{align*}
r(\cos\theta+i\sin\theta) \times r'(\cos\theta'+i\sin\theta')
&= rr'\bigr\{ (\cos\theta\cos\theta' - \sin\theta\sin\theta') + i(\sin\theta\cos\theta'+\cos\theta\sin\theta') \bigl\} \\
&= rr'\bigl(\cos(\theta+\theta') + i\sin(\theta+\theta')\bigr)
\end{align*}
となります。最後の式変形は、もちろん三角函数\footnote{函数と関数は同じ意味です。少々古臭い言い回しですが、好みの問題でこちらを使います。}の加法定理を使っています。この式は非常に重要なことを示唆しています。それは複素数$z$, $z'$の積$zz'$について
\begin{itemize}
\item 大きさは、$|zz'|=|z||z'|$で与えられる
\item 偏角は$\arg zz' = \arg z + \arg z'$で与えられる
\end{itemize}
ということです。言い換えれば、複素数$z$に対して別の複素数$z'$を掛け算する操作は
\begin{itemize}
\item $z$の大きさを$|z'|$倍し
\item $z$の偏角に$\arg z'$を足し算する
\end{itemize}
ということに他ならないからです。このように極形式を使うことによって、複素数の掛け算が「拡大縮小」と「回転」の組み合わせという図形的意味を持つことが読み取れるのです。

% TODO: ここに図を挟む

なお、一々$\cos\theta+i\sin\theta$と書いているのは長ったらしくて大変なので、以下では実数$\theta$に対して$e^{i\theta} = \cos\theta + i \sin\theta$や$\exp i\theta = \cos\theta + i \sin\theta$と表すことにします。たとえば
\[
e^{i\pi} = \cos\pi + i\sin\pi = -1
\]
という感じです。指数函数$e^x$と同じ記法を用いることには実は意味がある\footnote{そのうち微分積分学の授業で、函数のTaylor展開というものを習うはずです。その後で巾(べき)級数で指数函数$e^x$を定義し直すと、元々の「$e$のなんとか乗」という意味を越えて、$e^x$の$x$に複素数を代入できるようになります。そうして初めて$e^{i\theta}=\cos\theta+i\sin\theta$という式に正しく意味を与えることができます。}のですが、今は「単なる記号」だと思っていてください。この記号を使うと、極形式表示での掛け算は
\[
re^{i\theta} \times r'e^{i\theta'} = rr'e^{i(\theta+\theta')}
\]
と書けます。スッキリしてていいですね。

極形式表示は計算面でも、非常に威力を発揮することがあります。

\begin{comment}
\paragraph{複素数と複素数平面: 問3の解答} $z=e^{2k\pi i/n}$ ($k=0,1,\ldots,n-1$)とおくと、
\[
z^n = \bigl(e^{\frac{2k\pi i}{n}}\bigr)^n = \exp\Bigl(\frac{2k\pi i}{n}+\frac{2k\pi i}{n}+\cdots+\frac{2k\pi i}{n}\Bigr) = \exp \Bigl(\frac{2k\pi i}{n} \times n\Bigr) = e^{2k\pi i} = (e^{2\pi i})^k = 1
\]
となる。ここで$k$が$k=0,1,\ldots,n-1$を動けば、$n$個の異なる複素数が得られる\footnote{複素平面上にプロットすれば、異なることが直ちに分かります。}。また多項式$z^n-1$は$n$次式だから、$n+1$個以上の根を持つことはない。ゆえに$z=e^{2k\pi i/n}$ ($k=0,1,\ldots,n-1$)が全ての根を与える。

\paragraph{複素数と複素数平面: 問6の解答} 

$x,y\in M$とすると$x=a^2+b^2$, $y=c^2+d^2$となる整数$a,b,c,d\in\mathbb{Z}$が取れる。このとき$x=|a+bi|^2$, $y=|c+di|^2$なので$xy=|a+bi|^2|c+di|^2=|(a+bi)(c+di)|^2=|(ac-bd)+(ad+bc)i|^2=(ac-bd)^2+(ad+bc)^2$となる。$a,b,c,d\in\mathbb{Z}$より$ac-bd,ad+bc\in\mathbb{Z}$である。よって$xy\in M$である。

\paragraph{複素数と複素数平面: 問7の解答} 
次の式の$t$に好きな有理数を代入すれば、いくらでも有理点が得られる。
\[
\Bigl(\frac{2t}{1+t^2}\Bigr)^2 + \Bigl(\frac{1-t^2}{1+t^2}\Bigr)^2 = 1
\]
\end{comment}

\section{多項式の性質}

\subsection{多項式の割り算}

\paragraph{多項式の割り算と剰余}

\begin{comment}
\paragraph{多項式: 問1の解答}
$f(x)$が$x-\alpha$で割り切れるので、$f(x)=(x-\alpha)g(x)$と書ける。また$f(x)$は$x-\beta$でも割り切れるので、$f(\beta)=0$である。よって$0=f(\beta)=(\beta-\alpha)g(\beta)$となるが、$\alpha\neq\beta$より$g(\beta)=0$でないといけない。よって$g(x)$は$(x-\beta)$で割り切れる。これより$f(x)$は$(x-\alpha)(x-\beta)$で割り切れる。

\paragraph{多項式: 問2の解答}
$f(x)=x^3+10x^2+ax-2$に$x=2,-3$を代入した値が等しい。よって$f(2)=46+2a$と$f(-3)=-3a+61$が等しいのだから、$a=3$が得られる。求める余りは$52$である。

\paragraph{多項式: 問3の解答}
$f(x)=mx^3+nx^2-5$とおく。$f(-\frac{1}{2})=0$より$-\frac{m}{8}+\frac{n}{4}-5=0$である。また$f(\frac{2}{3})=7$より$\frac{8}{27}m+\frac{4}{9}n-5=7$である。これより$m=6$, $n=23$である。

\paragraph{多項式: 問4の解答} 	$F(x)$を$2x^2+x-1$で割った余りを$px+q$と書くと、何か多項式$P(x)$を用いて
\[
F(x) = (2x^2+x-1)P(x) + px + q
\]
と書ける。$F(x)$を$x+1$で割った余りが$6$なので$F(-1)=6$である、よって上式に$x=-1$を代入して$6=-p+q$を得る。同様に$F(x)$を$2x-1$で割った余りが$3$なので、$F(\frac{1}{2})=3$である。これより$3=\frac{1}{2}p+q$を得る。こうして$p,q$の連立$1$次方程式が得られたので、解くと$p=-2$, $q=4$が得られる。よって余りは$-2x+4$である。

\paragraph{多項式: 問6の解答}
$3$次多項式$f(x)$は、適当な多項式$g(x)$と$h(x)$によって$f(x)=g(x)(x^2-1)+5x-8=h(x)(x^2-x-6)+17x+4$と書ける。これに$x=1,-1,-2,3$を代入すると、それぞれ$f(1)=-3$, $f(-1)=-13$, $f(-2)=-30$, $f(3)=55$が得られる。そこで$f(x)=ax^3+bx^2+cx+d$とおくと
\begin{align*}
\begin{cases}
a+b+c+d &= -3 \\
-a+b-c+d &= -13 \\
-8a+4b-2c+d &= -30 \\
27a + 9b + 3c + d &= 55
\end{cases}
\end{align*}
という連立一次方程式が得られる。これを$a,b,c,d$について解けば$f(x) = 2x^3+3x-8$と分かる。

\subsection{有名な式変形}

\paragraph{円分多項式と基本対称式}

\paragraph{複素数と複素数平面: 問5の解答} $f(z)=z^4+z^3+z^2+z+1$である。

\noindent (1) $z$を$f(z)=0$の根とする。$z^5-1=(z-1)f(z)=0$なので、$z\neq 0$である。よって$f(z)/z^2=0$である。また$f(z)/z^2=z^2+z+1+z^{-1}+z^{-2} = (z+z^{-1})^2 + (z+z^{-1}) - 1 = t^2+t-1$である。よって$t^2+t-1=0$となるので、これを解いて$t=\frac{-1\pm\sqrt{5}}{2}$を得る。

\noindent (2) $t=z+z^{-1}$より$z^2-tz+1=0$である。$t=\frac{-1+\sqrt{5}}{2}$のとき、この方程式を解くと
\[
z = \frac{t\pm\sqrt{t^2-4}}{2} = \frac{-1+\sqrt{5}\pm\sqrt{(1-\sqrt{5})^2-16}}{4}
= \frac{-1+\sqrt{5}\pm\sqrt{-10-2\sqrt{5}}}{4}
= \frac{-1+\sqrt{5}\pm\sqrt{10+2\sqrt{5}}\,i}{4}
\]
となる。同様に$t=\frac{-1-\sqrt{5}}{2}$のとき
\[
z = \frac{t\pm\sqrt{t^2-4}}{2} = \frac{-1-\sqrt{5}\pm\sqrt{(-1-\sqrt{5})^2-16}}{4}
= \frac{-1-\sqrt{5}\pm\sqrt{-10+2\sqrt{5}}}{4}
= \frac{-1-\sqrt{5}\pm\sqrt{10-2\sqrt{5}}\,i}{4}
\]
が得られる。これらが全ての解である。

\noindent (3) $z = e^{2k\pi i/5}$ ($k=0,1,2,3,4$)が$z^5=1$の全ての解である。(2)で求めた解のうち実部と虚部がともに正なものが$e^{2\pi i/5}$である。よって$\sin(\frac{\pi}{2}-\theta) = \cos\theta$, $\cos(\frac{\pi}{2}-\theta) = \sin\theta$より
\[
\frac{-1+\sqrt{5}+\sqrt{10+2\sqrt{5}}\,i}{4} = e^{2\pi i/5}
= \cos\frac{2\pi}{5} + i \sin \frac{2\pi}{5} = \sin\frac{\pi}{10} + i \cos \frac{\pi}{10}
\]
となる。この式の実部と虚部を見ればよい。

\paragraph{多項式: 問5の解答}
\noindent (1) $x^n-a^n = (x-a)(x^{n-1}+ax^{n-2}+a^2x^{n-3}+\cdots+a^{n-1})$

\noindent (2) もし$x^n-a^n$が$x+a$で割り切れることは、$x$に$-a$を代入した結果が$0$になることと同値である。すなわち$0=(-a)^n-a^n=\bigl((-1)^n-1\bigr)a^n$より、$(-1)^n=1$が必要十分条件である。これは$n$が偶数であることに他ならない。

\noindent (3) $x^n+a^n$に$x=-a$を代入すると$(-a)^n+a^n = \bigl((-1)^n+1\bigr)a^n$となる。これが$0$になることは$n$が奇数であることと同値である。よって$n$が奇数なら$x^n+a^n$は$x+a$で割り切れる。

\paragraph{多項式: 問7の解答}
\begin{itemize}
\item[(1)] \ \\[-4zw]
\begin{align*}
4(ab+cd)^2-(a^2+b^2-c^2-d^2)^2 ]
&= (2ab+2cd+a^2+b^2-c^2-d^2)(2ab+2cd-a^2-b^2+c^2+d^2) \\
&= \bigl((a+b)^2-(c-d)^2\bigr)\bigl((c+d)^2-(a-b)^2\bigr) \\
&= (a+b+c-d)(a+b-c+d)(a-b+c+d)(-a+b+c+d)
\end{align*}
\item[(2)] $x^3-(a+b+c)x^2+(ab+bc+cd)x-abc = (x-a)(x-b)(x-c)$
\end{itemize}

\end{comment}

\subsection{複素数について}
\begin{citation}
{}``Was sind und was sollen die Zahlen?''
\end{citation}

これは、ドイツの数学者Richard Dedekindが記した本\footnote{ちなみに、1893年に出版されたドイツ語原著の第2刷が、東大の数理科学図書室にあります。}のタイトルです。『数とは何か、また何であるべきか?』この問題を、考えてみましょう。

\paragraph{複素数は数なのか?}

我々が日常生活の中で出会う「数」にはどんなものがあるでしょうか?たとえば物の個数を数えるときは自然数を使いますし、お金の計算をするときは収入と支出を表すのに正の数と負の数を使います。また料理をすればレシピの中に分数が出てきますし、円周の長さを測ろうとしたら$\pi$のような無理数も現れます。これらに登場する数は、いずれも実数の範囲に収まっていますね。

一方、複素数は実数の範囲を超えるものです。そのため$i^2=-1$となる数$i$を我々の世界で目にすることはありません。たとえばおつかいに行った子供が「ママー!\negthinspace おつりで$300+250\,i$円もらったよ!」なんて言うわけないですね。複素数を「気持ち悪い」と感じる主たる理由は、おそらくここにあるのではないでしょうか。英語では空想上の数``imaginary number''と呼ばれますし、日本語ではさらにネガティブな含みを持つ「虚」数という呼び名もあります。

ですが「我々の身の回りに見当たらないから」というだけの理由で、複素数を数と呼ぶべきではないのでしょうか。ここで一度、我々の身近にある数について「何が数たらしめているのか」を考えてみましょう。数を考える上で何よりも大事なことは「計算」です。たとえば自然数だったら足し算と掛け算ができます。整数なら、引き算がいつでもできます。有理数や実数なら、$0$以外の数による割り算もできます。また計算とは別に「大小の比較」ができることも、数の大きな特徴でしょう。

複素数は残念なことに「大小の比較」をすることはできません。ですが既に見てきたとおり、複素数では四則演算の全てを行うことができます。これをもって「数」と呼んでも良いのではないでしょうか。また「$-1$の平方根」というと気味が悪いかもしれませんが、「形式的に$i$という数を付け加え、$i^2=-1$というルールで計算を行う」という風に思えば、$i$の存在も受け入れられる気がします。

実際、現代数学では今のような方法で複素数を捉えています。より具体的には
\begin{itemize}
\item 四則演算ができる集合を\textbf{体(たい)}と呼び
\item 体が与えられたとき、その中にない元を付け加えて大きい体を作る\textbf{拡大}という操作を定義する
\end{itemize}
ことによって、きちんと複素数を定義しています。

\paragraph{複素数: 問8の解答}

\subsection{代数学の基本定理}

今回最後の話題は「代数学の基本定理」という、重要な定理です。

多項式$P(x)$に対し、方程式$P(x)=0$の解を根と言うのでした。多項式が根を持つかどうかは、「考える数の範囲」によって変わってきます。たとえば$P(x)=x^2+1$は実数の範囲で根を持ちませんが、複素数の範囲に広げると$x=\pm i$という根を持ちます。


\begin{comment}
\paragraph{代数学の基本定理: 問1の解答}

$f(z) = a_0 + a_1 z + \cdots + a_n z^n$とおく。このとき$a_0,a_1,\ldots,a_n\in\mathbb{R}$である。よって$\alpha\in\mathbb{C}$が$f(z)$の解であるとき、$f(\alpha)=0$の共役を取ると
\begin{align*}
0 &= \overline{f(\alpha)}
= \overline{a_0 + a_1 \alpha + \cdots + a_n \alpha^n}
= \overline{a_0} +\overline{a_1 \alpha} + \cdots + \overline{a_n \alpha^n}
= a_0 + a_1 \overline{\alpha} + \cdots + a_n \overline{\alpha}^n
= f(\overline{\alpha})
\end{align*}
となる。よって$\overline{\alpha}$も解である。

\paragraph{代数学の基本定理: 問2の解答}
$f$の次数に関する帰納法で示す。まず$\deg f=1$のときは$1$次式$1$個の積である。次に、$n-1$次以下の全ての複素係数多項式が$1$次式の積に分解すると仮定する。このとき$f(z)$を$n$次多項式とすると、代数学の基本定理より$f(z)$の根が存在する。それを$\alpha$とすると$f(z)=(z-\alpha)g(z)$と書け、$g(z)$は$n-1$次多項式となる。帰納法の過程から$g(z)$は$1$次式の積に分解するので、$f(z)$全体も$1$次式の積となる。


\paragraph{代数学の基本定理: 問3の解答}

$|f(0)|=1$より、$f(z)$の定数項は$0$でない。定数項の次に低い次数の項が$a_k z^k$だとして、$f(z) = a_0 + a_k z^k + a_{k+1} z^{k+1} + \cdots + a_n z^n$とおく。いま$a_0 = r_0 e^{i\theta_0}$, $a_k = r_k e^{i\theta_k}$と書ける。これらを用いて$z_0:=re^{i(\theta_0-\theta_k+\pi)/k}$と定めると
\begin{align*}
f(z_0) &= a_0 + a_k z^k \Bigl( 1 + \frac{a_{k+1}}{a_k} z + \cdots + \frac{a_n}{a_k} z^{n-k}\Bigr)
= r_0e^{i\theta_0} + r_k e^{i\theta_k} \Bigl(re^{(\theta_0-\theta_k+\pi)/k}\Bigr)^k \Bigl( 1 + \frac{a_{k+1}}{a_k} z + \cdots + \frac{a_n}{a_k} z^{n-k}\Bigr)\\
&= r_0e^{i\theta_0} + r_k r^k e^{i(\theta_0+\pi)} \Bigl( 1 + \frac{a_{k+1}}{a_k} z + \cdots + \frac{a_n}{a_k} z^{n-k}\Bigr)
= (r_0-r_kr^k) e^{\theta_0} - r_k r^k e^{i\theta_0} \Bigl( \frac{a_{k+1}}{a_k} z + \cdots + \frac{a_n}{a_k} z^{n-k}\Bigr)
\end{align*}
となる。よって$r$を十分小さく取れば
\begin{align*}
|f(z_0)| \leq |r_0-r_k r^k | + r_k r^k \Bigl| \frac{a_{k+1}}{a_k} z + \cdots + \frac{a_n}{a_k} z^{n-k}\Bigr|
= r_0-r_k r^k + r_k r^k \Bigl| \frac{a_{k+1}}{a_k} z + \cdots + \frac{a_n}{a_k} z^{n-k}\Bigr|
\end{align*}
となる。ここで第$3$項は$O(r^{k+1})$だから、$r$を十分小さく取っておけば$r_kr^k$より絶対値を小さくできる。これより$|f(z_0)|<r_0=|f(0)|$となることが分かった。
\end{comment}

\paragraph{代数学の基本定理の証明}

せっかく問$3$の解答を与えたので、代数学の基本定理の証明\footnote{この証明には微分積分学の知識が必要なので、今は読み解くのが難しいかもしれません。連続函数の性質を習った後で読むと、手頃な勉強になるでしょう。}を与えておきます。以下、$f(z)$を複素係数の多項式とします。


\noindent \underline{step 1.} $|z|>R$なる全ての複素数$z$に対して$|f(z)|>|f(0)|$が成り立つような正の実数$R>0$が取れることを示す。

$f(z)=a_n z^n + a_{n-1} z^{n-1} + \cdots + a_0$とおく。このとき
\[
|f(z)| = |a_n||z|^n\Bigl|1+\frac{a_{n-1}}{z} + \cdots + \frac{a_0}{z^n}\Bigr|
\]
である。ここで三角不等式$|z_1|-|z_2|\leq|z_1-z_2|$を使うと
\begin{align*}
\Bigl|1+\frac{a_{n-1}}{z} + \cdots + \frac{a_0}{z^n}\Bigr|
&\geq \Bigl|1+\frac{a_{n-1}}{z} + \cdots + \frac{a_1}{z^{n-1}}\Bigr| - \Bigl|\frac{a_0}{z^n}\Bigr| \geq \cdots
\geq 1-\Bigl|\frac{a_{n-1}}{z}\Bigr| - \cdots -\Bigl|\frac{a_{0}}{z^n}\Bigr|
\end{align*}
となる。ここで
\[
R_0:=\max\Biggl\{\frac{n|a_{n-1}|}{0.1},\Biggl(\frac{n|a_{n-2}|}{0.1}\Biggr)^{\frac{1}{2}},\ldots,\Biggl(\frac{n|a_{0}|}{0.1}\Biggr)^{\frac{1}{n}}\Biggr\}
\]
とおく\footnote{$0.1$に本質的な意味はありません。$1$未満の任意の正の数$\varepsilon$に対し、ここでの議論が成り立ちます。}と、$|z|>|R_0|$のとき、全ての$1\leq k\leq n-1$に対し
\[
\Bigl|\frac{a_{n-k}}{z^k}\Bigr| = \frac{|a_{n-k}|}{|z|^k} < \frac{|a_{n-k}|}{R_0^k} \leq |a_{n-k}|\Biggl(\frac{0.1}{n|a_{n-k}|}\Biggr)^{\frac{1}{k}\cdot k} = \frac{0.1}{n}
\]
となる。よって
\[
\Bigl|1+\frac{a_{n-1}}{z} + \cdots + \frac{a_0}{z^n}\Bigr| \geq
1-\Bigl|\frac{a_{n-1}}{z}\Bigr| - \cdots -\Bigl|\frac{a_{0}}{z^n}\Bigr|
\geq 1-\frac{0.1}{n}-\cdots-\frac{0.1}{n} = 0.9
\]
となる。さらに正の実数$R$を
\[
R:=\max\Biggl\{\Biggl(\frac{|f(0)|}{0.9|a_n|}\Biggr)^{\frac{1}{n}},R_0\Biggr\}
\]
とおくと、$|z|>R$のとき
\[
|f(z)| = |a_n||z|^n\Bigl|1+\frac{a_{n-1}}{z} + \cdots + \frac{a_0}{z^n}\Bigr| \geq |a_n|\Biggl(\frac{|f(0)|}{0.9|a_n|}\Biggr)^{\frac{1}{n}\cdot n} \cdot 0.9 = |f(0)|
\]
となる。

\noindent \underline{step 2.} 複素平面上の実数値函数$|f(z)|$が最小値を持つことを示す。

半径$R$の閉円板$\Delta_R:=\{z\in\mathbb{C}\mid |z|\leq R\}$を考える。$|f(z)|$は$\mathbb{C}=\mathbb{R}^2$上の連続函数である。そして$\Delta_R$は平面$\mathbb{R}^2$内の有界閉集合だから、$|f(z)|$は$\Delta_R$上で最小値を取る。その点を$z_0$とする。$0\in\Delta_R$より$|f(z_0)|\leq |f(0)|$である。また$|z|>R$なら$R$の取り方から$|f(z)|>|f(0)|\geq|f(z_0)|$が従う。かくして$|f(z_0)|$は複素平面全体における$|f(z)|$の最小値である。

\noindent \underline{step 3.} 方程式$f(z)=0$が解を持つことを示す。

もし$|f(z_0)|\neq 0$だとすると、$f(z_0)\neq 0$である。そこで新しい多項式$g(z)$を$g(z):=f(z+z_0)/f(z_0)$で定めることができる。このとき$g(0)=1$なので、問3の結果より、適当な複素数$z_1\in\mathbb{C}$を取ると$|g(z_1)|<1$とできる。ところが$|g(z_1)| = |f(z_1+z_0)| / |f(z_0)|$と合わせると$|f(z_1+z_0)| < |f(z_0)|$が導かれてしまう。これは$|f(z_0)|$が複素平面全体で$|f(z)|$の最小値になることに矛盾する。ゆえに$|f(z_0)|=0$である。つまり、$z_0$は$f(z)$の根である。

