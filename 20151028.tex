\chapter{固有値と固有ベクトル}
\lectureinfo{2015年10月28日 1限}

\section{固有値と固有ベクトル}

$A$を$n$次正方行列とします。このとき$\bm{u} \in \mathbb{R}^n$が$A$の固有値$\lambda \in \mathbb{R}$に属する\textbf{固有ベクトル}であるとは、$A\bm{u} = \lambda \bm{u}$かつ$\bm{u} \neq \bm{0}$となることをいいます。また$\lambda \in \mathbb{R}$が$A$の固有値であるとは、固有値$\lambda$に属する$A$の固有ベクトル$\bm{u}$が存在することをいいます。

今回、固有値と固有ベクトルの求め方を少し詳しく調べます。

\subsection{連立$1$次方程式の非自明な解}

$n$次正方行列$A \in \Mat_n(\mathbb{R})$が定める連立$1$次方程式$A \bm{x} = \bm{0}$を考えてみます。この時、解$\bm{x} = \bm{0}$を自明な解、それ以外の解を非自明な解と呼ぶのでした。そして非自明な解が存在するための必要十分条件は、$\det A = 0$です。実は$3$回前の演習の際、プリントの\pageref{subsection:non-trivial_answer}ページでこの話が$1$度出てきています。復習しましょう。

採点をしていたら、「同値性」がきちんと証明できておらず、片側の証明だけに終わっている人がそこそこいました。間違えた人は「自分がどっち側を示していたのか」を確認した上で、逆方向の証明にも取り組んでください。

\paragraph{問1の解答}

$\det A \neq 0$なら逆行列$A^{-1}$が存在するので、$A\bm{x} = \bm{0}$の両辺に$A^{-1}$をかければ$\bm{x} = \bm{0}$が得られる。対偶を取れば、$A\bm{x} = \bm{0}$が非自明な解を持つとき$\det A = 0$だと分かる。

逆に$\det A = 0$とすれば、$A$は正則ではないから全単射でない。さらに$A$が正方行列なので、$A$は全射でも単射でもない。$A$が単射でないことから$\Ker A \neq \{\bm{0}\}$が従う。これは$A\bm{x} = \bm{0}$が非自明な解を持つことを意味する。 \qed

\subsection{固有値と固有多項式}

さて固有値の求め方を考えましょう。$A \in \Mat_n(\mathbb{R})$を$n$次正方行列とします。このとき$A\bm{u} = \lambda \bm{u}$という式は$(A - \lambda I) \bm{u} = \bm{0}$と書き直せます。したがって$\lambda$が$A$の固有値であることは、連立$1$次方程式$(A - \lambda I)\bm{x} = \bm{0}$が非自明な解を持つことと同値で、さらに問1の結果を使えば$\det (A - \lambda I) = 0$とも同値になります。ですから結局、固有値を求めるには
\[
\varphi_A(t) := \det(t I - A)
\]
という$t$の多項式の根を探せばよいのです\footnote{$\det(A - tI)$と$\det (tI - A)$は$(-1)^n$倍の違いしかありません。$(-1)^n$倍があってもなくても多項式の根に影響はしないので、$t^n$の係数が$+1$で見栄えの良い$\det(tI - A)$の方を、固有多項式と定義します。}。この$\varphi_A(t)$を$A$の\textbf{固有多項式}といいます。

$A$を成分で$A = (a_{ij})_{1 \leq i, j \leq n}$と書くと
\[
\varphi_A(x) =
\det
\begin{pmatrix}
t - a_{11} & - a_{12} & \cdots & -a_{1n} \\
-a_{21} & t - a_{22} & \cdots & -a_{2n} \\
\vdots & \vdots & \ddots & \vdots \\
-a_{n1} & -a_{n2} & \cdots & t - a_{nn}
\end{pmatrix}
\]
です。いきなり$n$次でやるのが難しければ、$3$次くらいでガリガリ計算してみましょう。$3$次の場合は
\begin{align*}
\varphi_A(t) &= \det
\begin{pmatrix}
t - a_{11} & -a_{12} & -a_{13} \\
-a_{21} & t - a_{22} & -a_{23} \\
-a_{31} & -a_{32} & t - a_{33}
\end{pmatrix} \\
&= (t - a_{11})(t - a_{22})(t - a_{33}) - (t - a_{11}) a_{23} a_{32} - a_{13} (t - a_{22}) a_{31} - a_{12} a_{21} (t - a_{33}) - a_{12} a_{23} a_{31} - a_{13} a_{21} a_{32} \\
&= t^3 - (a_{11} + a_{22} + a_{33})t^2 - (a_{11} a_{22} + a_{22} a_{33} + a_{11} a_{33} + a_{12} a_{21} + a_{13} a_{31} + a_{23} a_{32}) t \\
& \qquad - (a_{11} a_{22} a_{33} - a_{12} a_{21} a_{33} - a_{11} a_{23} a_{32} + a_{12} a_{23} a_{31} + a_{13} a_{21} a_{32} - a_{13} a_{22} a_{31})
\end{align*}
となっています。自分でも手を動かして確かめてください。

実際の計算は個々の行列でやれば良いのですが、一般の場合であっても、$t^{n - 1}$の項と定数項は割と簡単に計算できます。問2の一般論が$3$次の場合に正しいことを、式を照らし合わせ確認してください。

\paragraph{問2の解答}

$n$次正方行列$A = (a_{ij})_{1 \leq i, j \leq n}$に対し、$tI - A = \bigl(\tilde{a}_{ij}(t)\bigr)_{1 \leq i, j \leq n}$と表すと
\begin{align*}
\tilde{a}_{ij}(t) :=
\begin{cases}
-a_{ij} & (i \neq j) \\
t - a_{ij} & (i \neq j)
\end{cases}
\end{align*}
である。これを用いて、固有多項式$\varphi_A(t)$は
\[
\varphi_A(t) = \sum_{\sigma \in \mathfrak{S}_n} \sgn(\sigma) \tilde{a}_{1 \sigma(1)}(t)\, \tilde{a}_{2 \sigma(2)}(t) \cdots \tilde{a}_{n \sigma(n)}(t)
\]
と書ける。ここで、全ての$\sigma \in \mathfrak{S}_n$と$1 \leq i \leq n$について$\tilde{a}_{i \sigma(i)}(t)$は$1$次以下である。よって$\sum$で足される項が$(n - 1)$次以上の式になるためには、$\{1, 2, \ldots, n\}$のうち$(n - 1)$個の$i$について$\deg \tilde{a}_{i\sigma(i)}(t) = 1$、つまり$i = \sigma(i)$が成り立たなければいけない。ところが$1, 2, \ldots, n$のうち$(n - 1)$個について$\sigma(i) = i$が成り立てば、残りの$1$個についても$\sigma(i) = i$となるから、このような$\sigma$は恒等置換$\sigma = \id$しかない。故に、$\varphi_A(t)$の$(n - 1)$次の項は
\[
\tilde{a}_{11}(t)\, \tilde{a}_{22}(t) \cdots \tilde{a}_{nn}(t) = (t - a_{11})(t - a_{22}) \cdots (t - a_{nn})
\]
からしか現れない。これを展開すると、
\[
(t - a_{11})(t - a_{22}) \cdots (t - a_{nn}) = t^n - (a_{11} + a_{22} + \cdots + a_{nn}) t^{n - 1} + \cdots
= t^n - (\tr A) t^{n - 1} + \cdots
\]
となる。よって$\varphi_A(t) = t^n - (\tr A) t^{n - 1} + \cdots$と書ける。

また$\varphi_A(t)$の定数項は$\varphi_A(0)$である。これと$\tilde{a}_{ij}(0) = -a_{ij}$より
\begin{align*}
\varphi_A(0)
&= \sum_{\sigma \in \mathfrak{S}_n} \sgn(\sigma) \tilde{a}_{1 \sigma(1)}(0)\, \tilde{a}_{2 \sigma(2)}(0) \cdots \tilde{a}_{n \sigma(n)}(0)
= \sum_{\sigma \in \mathfrak{S}_n} \sgn(\sigma) (-a_{1 \sigma(1)}) (-a_{2 \sigma(2)}) \cdots (-a_{n \sigma(n)}) \\
&= (-1)^n \sum_{\sigma \in \mathfrak{S}_n} \sgn(\sigma) a_{1 \sigma(1)} a_{2 \sigma(2)} \cdots a_{n \sigma(n)}
= (-1)^n \det A
\end{align*}
となる。\qed

\paragraph{$2$次正方行列の固有多項式とCayley--Hamiltonの定理}

今の計算で、正方行列$A$の固有多項式$\varphi_A(t)$の最高次の項、その$1$つ下の次数の項と定数項は簡単に計算できることが分かりました。ここで特に$A$が$2$次正方行列なら、これらの情報だけで$\varphi_A(t)$は決まってしまいます。問$2$の結果を見れば
\[
\varphi_A(t) = t^2 - (\tr A)t + \det A
\]
が、$A$の固有多項式だと分かります。$2$次の場合はよく使うので覚えておきましょう。

ところでこの式、どこかで見覚えのある式ですよね。そう、Cayley--Hamiltonの定理で出てきた式と同じです。実は一般の$n$次正方行列であっても、$\varphi_A(A) = 0$という式は常に成り立ち、この事実を\textbf{Cayley--Hamiltonの定理}と呼びます\footnote{「$\det(tI - A)$に$t = A$を代入したら$\det(A - A) = 0$だろ」と早とちりしてはいけません。$tI - A$は「成分に$t$が混ざった行列」なので、この各成分中にある$t$を$A$に置き換えることはできないからです。}。次回以降の授業のどこかで、きっと証明をします。

\subsection{固有値$0$と固有空間の例}

固有値の意味はこれから追々調べていきますが、その中でも「固有値$0$」は、実は今までに散々使ってきました。$\bm{u}$が固有値$0$に属する$A$の固有ベクトルであることは、$A\bm{u} = 0\bm{u} = \bm{0}$となることです。つまり$\bm{u} \in \Ker A$に他なりません。$\Ker A$は、固有値$0$に属する$A$の固有ベクトルを全て集めてきた集合と言えるのです。

これが分かっていれば、問$3$は当たり前になります。

\paragraph{問3の解答} $A$が固有値$0$に属する固有ベクトルを持つことと$\Ker A \neq \{\bm{0}\}$は同値である。$\Ker A \neq \{\bm{0}\}$は$A$が単射でないことと同値である。そして$A$は正方行列なので、$A$が単射であることと正則であることが同値である。これより求める結果を得る。 \qed

\paragraph{固有空間}

$\Ker A$はただベクトルを集めた集合ではなく、$\mathbb{R}^n$の部分空間になるという性質を持っていました。そして$\Ker A$が$0$固有値に対応する固有ベクトル (と$\bm{0}$) を集めた集合であることを確かめました。こうすると一般の場合にも、固有値が$\lambda$に属する固有ベクトルを集めて来れば部分空間ができる気がします。実際これは部分空間になり、固有値$\lambda$に対応する\textbf{固有空間}と呼ばれます。

すぐ後でする「行列の対角化」の話は、基底に依らない抽象的な線型代数の言葉で言うと、線型空間を固有空間の直和に分解する操作 (\textbf{固有空間分解}) に対応します。

\section{対角化に関連する簡単な例}

\subsection{行列の対角化}

$n$次正方行列$A \in \Mat_n(\mathbb{R})$について、うまく$n$次正方行列$P$を取って$P^{-1} A P$が対角行列になるようにできるとき、$A$は\textbf{対角化可能}であるといいます。

結論だけ先に言ってしまうと、$n$次正方行列$A$を対角化するには、$A$の固有ベクトルを$n$本見つければ良いのです。それを並べた行列を$P$とすれば、$P^{-1} A P$が対角行列になります。しかし答えだけ丸暗記しても仕方がないので、どうして固有ベクトルを並べれば良いのか、考えてみましょう。

\paragraph{行列の列ベクトルの意味}

$n$次正方行列$A$を、列ベクトルを並べて$A = (\bm{a}_1 \ \bm{a}_2 \ \cdots \ \bm{a}_n)$と表しましょう。このとき、$A\bm{e}_i = \bm{a}_i$となっています。つまり\textbf{$A$の列ベクトルは、標準基底をなすベクトルの行き先}に他なりません。

ここで特に、$A$が対角行列の場合を考えましょう。
\[
A =
\begin{pmatrix}
a_1 \\
& a_2 \\
& & \ddots \\
& & & a_n
\end{pmatrix}
\]
のとき、$A\bm{e}_i = a_i \bm{e}_i$なので、$\bm{e}_i$は固有値$a_i$に属する$A$の固有ベクトルです。つまり「$A$が対角行列であること」は、「標準基底$\bm{e}_1, \bm{e}_2, \ldots, \bm{e}_n$が全て$A$の固有ベクトルであること」が理由だと言えます。

このことを考えると、$A$が対角とは限らない一般の行列の場合に、どうすれば対角化できるか見当が付きます。$A$の固有ベクトルからなる基底を作って基底変換をすれば、基底をなすそれぞれのベクトルが$A$によってスカラー倍されます。だから新しい基底に関する$A$の行列表示が対角行列になるわけです。

\paragraph{対角化のための行列}

今のことを踏まえれば、行列の対角化をするには、固有ベクトルを集めて基底を作れば良いことが分かります。仮に$n$次正方行列$A \in \Mat_n(\mathbb{R})$の、$1$次独立な固有ベクトル$\bm{p}_1, \bm{p}_2, \ldots, \bm{p}_n$が取れたとします。これらを並べて、$n$次正方行列$P$を$P := (\bm{p}_1 \  \bm{p}_2 \ \cdots \bm{p}_n)$で定めます。また、各$\bm{p}_i$に対応する固有値を$\lambda_i$としましょう。つまり$A\bm{p}_i = \lambda_i \bm{p}_i$です。

$(\bm{p}_1, \bm{p}_2, \ldots, \bm{p}_n)$が$\mathbb{R}^n$の基底なので、$P$は正則です。よって$P^{-1}$が存在します。このとき
\[
\begin{pmatrix}
\bm{e}_1 & \bm{e}_2 & \cdots & \bm{e}_n
\end{pmatrix}
=
I = P^{-1} P
= P^{-1}
\begin{pmatrix}
\bm{p}_1 & \bm{p}_2 & \cdots & \bm{p}_n
\end{pmatrix}
=
\begin{pmatrix}
P^{-1}\bm{p}_1 & P^{-1}\bm{p}_2 & \cdots & P^{-1}\bm{p}_n
\end{pmatrix}
\]
なので、$P^{-1} \bm{p}_i = \bm{e}_i$が成り立ちます。これを踏まえた上で$P^{-1} A P$を計算すると
\begin{align*}
P^{-1} A P
&= P^{-1}
\begin{pmatrix}
A\bm{p}_1 & A\bm{p}_2 & \cdots & A\bm{p}_n
\end{pmatrix}
= P^{-1}
\begin{pmatrix}
\lambda_1 \bm{p}_1 & \lambda_2 \bm{p}_2 & \cdots & \lambda_n \bm{p}_n
\end{pmatrix} \\
&=
\begin{pmatrix}
\lambda_1 P^{-1} \bm{p}_1 & \lambda_2 P^{-2} \bm{p}_2 & \cdots & \lambda_n P^{-n} \bm{p}_n
\end{pmatrix}
=
\begin{pmatrix}
\lambda_1 \bm{e}_1 & \lambda_2 \bm{e}_2 & \cdots & \lambda_n \bm{e}_n
\end{pmatrix}\\
&=
\begin{pmatrix}
\lambda_1 \\
& \lambda_2 \\
& & \ddots \\
& & & \lambda_n
\end{pmatrix}
\end{align*}
となります。確かに$P^{-1} A P$は対角行列になっていますね。

こんな感じで、行列の固有ベクトルを集めてくれば対角化ができます。そして行列$A$の固有値が$\lambda$のとき、固有ベクトルは$A\bm{u} = \lambda \bm{u}$という連立$1$次方程式の解なので、頑張れば計算できます。このようにして、対角化が実現されるのです。

\subsection{対角化できない例}

さて、ここまで自信ありげに「こうすれば対角化できます」という話をしてきました。でも残念ながら、\textbf{全ての行列が対角化可能なわけではありません}。詳しい話は次回以降に回すことにして、今回は簡単に「対角化が上手くできないケース」を眺めてみましょう。

\paragraph{例1: 固有多項式の根が取れない場合}

$1$つ目は、固有多項式の根が存在しない場合です。前回もちょこっと話したように、回転行列は$\mathbb{R}$の中に固有値を持ちません。たとえば$90^{\circ}$回転の行列は
\[
A :=
\begin{pmatrix}
0 & -1 \\
1 & 0
\end{pmatrix}
\]
です。これの固有多項式は
\[
\varphi_A(t) =
\det
\begin{pmatrix}
t & 1 \\
-1 & t
\end{pmatrix}
= t^2 + 1
\]
です。方程式$\varphi_A(t) = 0$は実数解を持たないので、$A$は実数の固有値を持ちません。このことが、$\mathbb{R}^2$の中に$A$の固有ベクトルが存在しないことと対応しています。

ただし数の範囲を複素数まで広げれば、方程式$\varphi_A(t) = 0$は解$t = \pm\sqrt{-1}$を持ちます。そして連立$1$次方程式
\[
\begin{pmatrix}
0 & -1 \\
1 & 0
\end{pmatrix}
\begin{pmatrix}
x \\
y
\end{pmatrix}
=
\pm\sqrt{-1}
\begin{pmatrix}
x \\
y
\end{pmatrix}
\]
を解くと
\[
\begin{pmatrix}
x \\
y
\end{pmatrix}
=
k
\begin{pmatrix}
1 \\
\mp \sqrt{-1}
\end{pmatrix}
\]
となり、固有値$\pm\sqrt{-1}$に対応する固有ベクトルが取れます。実際
\[
\begin{pmatrix}
0 & -1 \\
1 & 0
\end{pmatrix}
\begin{pmatrix}
1 \\
\mp\sqrt{-1}
\end{pmatrix}
=
\begin{pmatrix}
\pm\sqrt{-1} \\
1
\end{pmatrix}
= \pm\sqrt{-1}
\begin{pmatrix}
1 \\
\mp \sqrt{-1}
\end{pmatrix}
\]
となっていますね。

このように固有値問題を考えるときは「方程式の根がどこに存在するか」が問題になってきます。考える数の範囲を狭く取ると、方程式の根が存在せず、対角化できない場合というのが出てきます。複素数の場合は代数学の基本定理があるおかげで、どんな多項式も必ず$1$次式の積に分解できるので、このような問題は起きません\footnote{前に言ったことの繰り返しですが、根が存在することと、実際に計算で求まるかどうかは別の問題です。}。

\paragraph{例2: 固有ベクトルが足りない場合}

さらに深刻なことに、固有多項式の根が存在しても対角化ができない場合があったりします。たとえば
\[
A =
\begin{pmatrix}
0 & 1 \\
0 & 0
\end{pmatrix}, 
P = 
\begin{pmatrix}
a & b \\
c & d
\end{pmatrix}
\]
とし、$A$を対角化する$P$を探してみましょう。$P$が正則だと仮定して$P^{-1} A P$を計算すると
\begin{align*}
\frac{1}{ad - bc}
\begin{pmatrix}
d & -b \\
-c & a
\end{pmatrix}
\begin{pmatrix}
0 & 1 \\
0 & 0
\end{pmatrix}
\begin{pmatrix}
a & b \\
c & d
\end{pmatrix}
&=
\frac{1}{ad - bc}
\begin{pmatrix}
d & -b \\
-c & a
\end{pmatrix}
\begin{pmatrix}
c & d \\
0 & 0
\end{pmatrix}
=
\frac{1}{ad - bc}
\begin{pmatrix}
cd & d^2 \\
-c^2 & -cd
\end{pmatrix}
\end{align*}
です。これが対角行列になるためには$c = d = 0$である必要がありますが、$c = d = 0$だったら行列$P$がそもそも正則になりません。ですから$A$は対角化できません。そして$A$の固有多項式は
\[
\varphi_A(x)
=
\det
\begin{pmatrix}
x & -1 \\
0 & x
\end{pmatrix}
= x^2
\]
です。さっきの例1とは違い、今回は$\varphi_A(x)$の根がちゃんと$\mathbb{R}$内に存在します。でも対角化ができません。

この原因は\textbf{固有ベクトルが足りない}ところにあります。$A$の固有値は$0$しかないので、固有ベクトルは
\[
\begin{pmatrix}
0 & 1 \\
0 & 0
\end{pmatrix}
\begin{pmatrix}
x \\
y
\end{pmatrix}
=
\begin{pmatrix}
0 \\
0
\end{pmatrix}
\]
という方程式を満たします。でもこの方程式を解くと$y = 0$しか出てこないので、固有ベクトルは
\[
\alpha
\begin{pmatrix}
1 \\
0
\end{pmatrix} \quad (\alpha \in \mathbb{R})
\]
だけです。$\dim \Ker A = 1$です。つまり$A$の固有ベクトルをいくら探しても$1$次元分しか取ってこられないので、$\mathbb{R}^2$の基底は作れません。だから対角化できないのです。

こういう問題の処方箋もきちんとあって、僕たちは
\begin{itemize}
\item 固有ベクトルが足りなくなるのはどんなときか
\item 固有ベクトルが足りないとき、「$P^{-1} A P$を見やすくする基底を探す問題」に対する次善の策は何か
\end{itemize}
を知ることができます。これは\textbf{Jordan標準形の理論}という名前が付いています。対角化問題を取り扱うときに余裕があれば、少しお話したいと思います。

\subsection{おまけ: 係数体について}

さっきの回転行列が対角化できない話で「数の範囲をどう取るか」という問題が出てきました。せっかくなので少し、この話に補足しておきます。

\paragraph{環と体}

僕たちは日頃「数」というと、大体、整数$\mathbb{Z}$、有理数$\mathbb{Q}$、実数$\mathbb{R}$や複素数$\mathbb{C}$を考えます。こうした数の中で
\begin{itemize}
\item 整数$\mathbb{Z}$の中では足し算、引き算と掛け算ができる
\item 有理数$\mathbb{Q}$、実数$\mathbb{R}$と複素数$\mathbb{C}$の中では、足し算、引き算と掛け算に加え、$0$以外の数で割り算ができる
\end{itemize}
という性質があります。大体こういう性質を指して
\begin{itemize}
\item 整数全体の集合$\mathbb{Z}$は\textbf{環} (かん) である
\item 有理数全体の集合$\mathbb{Q}$、実数全体の集合$\mathbb{R}$と複素数全体の集合$\mathbb{C}$は\textbf{体} (たい) である
\end{itemize}
と言います。体は環よりも強い条件を満たすものなので、体は環でもあります。

これまで行列の成分はつねに実数だとしてきましたが、よくよくこれまでの議論を振り返ると、実は「実数であること」はそこまで重要でありません。行列の足し算、引き算とかけ算を考えるだけなら、成分同士の足し算、引き算とかけ算ができれば十分です。たとえば整数を成分とする$2$つの行列を足したり引いたりかけたりした結果は、再び整数成分の行列になりますね。また連立$1$次方程式を掃き出し法で解いたり、逆行列を求めるときは、$1/a$の形の式が現れることがあります。ですので割り算ができないと困りますが、一方で割り算ができれば、それで十分です。

つまり体$K$を$1$つ決めるとき、「$K$を成分とする行列の理論」が展開できるのです。これまで僕たちは実数成分の行列ばかり扱ってきましたが、実数が実数たる所以である完備性\footnote{「Cauchy列がいつも極限を持つ」という事実です。多分、微分積分学の授業で出てきているはずです。}は使っていません。四則演算さえできれば良いので実数体$\mathbb{R}$に限らず、複素数体$\mathbb{C}$や有理数体$\mathbb{Q}$、さらにその他どんな体であっても、僕たちがこれまで積み上げてきた線型代数の議論は、ほとんど同様に成り立ちます。

\paragraph{代数閉体}

ところが今回のような「固有値問題」を考えると、どの体を考えるかによって状況が色々変わってきます。実数体$\mathbb{R}$の場合には、固有多項式の根が存在しないケースがありました。一方で複素数体$\mathbb{C}$では、代数学の基本定理が成り立つおかげで、いかなる多項式も$1$次式の積に分解できます。だから固有値問題を考えるときは、実数体よりも複素数体で考える方が、議論がやりやすそうだと分かります。

このように代数学の基本定理が成り立つような体を、\textbf{代数閉体}といいます。複素数体$\mathbb{C}$は代数閉体の典型例です。

\paragraph{係数拡大}

さて実数体$\mathbb{R}$は複素数体$\mathbb{C}$に含まれているので、実数成分の行列は複素数成分の行列の特別な場合とみなせます。この$\mathbb{R} \subset \mathbb{C}$のように、行列の成分をある体からもっと大きい体へ広げて考えることを\textbf{係数拡大}といいます。

「多項式の根がなくて対角化できない行列」を考えるときは、係数拡大が一つの処方箋です。たとえば$90^{\circ}$回転の行列は実数の範囲だと固有値が取れませんが、複素数まで係数拡大をすれば$\pm i$という固有値が出てきます。また実数体$\mathbb{R}$に限らず、Steinitzの定理「どんな体であっても、それを含むような代数閉体が存在する」という事実が知られています。だから多項式の根が取れない場合は、根が得られるまで係数拡大すれば、その体の中で対角化ができます。

係数拡大をして嬉しいかどうかは、時と場合によって変わってきます。問題によっては、与えられた体の中でできるだけ行列を見やすくすることが大事な場合もあります。ただ係数拡大をすれば原理的に固有多項式が解けるという事実は、知っておいて損はないでしょう。

