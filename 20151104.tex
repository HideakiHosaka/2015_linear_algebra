\chapter{対角化の計算練習}
\lectureinfo{2015年11月4日 1限}

今回のテーマは「対角化の計算」です。計算問題だったからなのか、多くの問題に取り組んだ人が結構いました。今回の演習を通して、対角化の計算に慣れ親しんでください。

\section{行列の対角化の計算}

\subsection{対角化のレシピ}

前回「行列の固有ベクトルで$\mathbb{R}^n$の基底を作れば、対角化ができる」ということを確認しました。今回は実際にその計算を色々やってみます。まずは正方行列$A$を対角化する計算方法を確認しておきましょう。

\begin{enumerate}
\item $A$の固有多項式$\varphi_A(t) := \det(tI - A)$を計算する。
\item 方程式$\varphi_A(t) = 0$を解いて、固有値を全て求める。ここで方程式が解けなかったら諦める\footnote{$4$次方程式までは「解の公式」が知られているので、どんな場合でも解けます。しかし$5$次以上の方程式に対しては、Galois理論というものを使って「解の公式が存在しないこと」を示せます。ですので厳密に解けるのは、運が良い場合に限られてしまうのです。

ちなみに$3$次方程式と$4$次方程式は、Cardanoの著書 ``Artis Magnae sive de Regulis Algebraicis'' (代数規則についての大技術, 1545) で初めて発表されました。金沢工業大学がこれの初版本を所有しており、大阪で行われた展覧会「世界を変えた書物」展にて展示していました。もしかしたら近いうち、東京にもこの展覧会が回ってきて、貴重な初版本を目にする機会ができるかもしれません。}。
\item それぞれの固有値$\lambda$に対し、連立$1$次方程式$A \bm{x} = \lambda\bm{x}\ (\Leftrightarrow (A - \lambda I)\bm{x} = \bm{0})$を解いて、$\Ker (A - \lambda I)$の基底をつくる。
\item 各$\lambda$毎に$\Ker (A - \lambda I)$の基底を集めてきて、幸いなことに$n$本のベクトルが得られていたら、それを一列に並べて正方行列$P$を作る。このとき$P^{-1} A P$は、$P$の$1$列目から$n$列目までの列ベクトルに対応する固有値が順番に並んだ対角行列になる。もし$n$本の固有ベクトルが集まらなかったら諦める。
\end{enumerate}

この方針に従って、計算をやってみましょう。その後で「諦める」という箇所について、いつ諦めるべきなのか、諦める場合はどうするのかをもう少し深く考察します。

\subsection{計算ミスについて}

固有値を求めるには、行列式を計算し、連立一次方程式を何回も解かなければいけません。計算を間違えた人が結構いたので、計算ミスをしないための方策を少し補足しておきます。
\begin{itemize}
\item 固有ベクトルは$\bm{0}$ではありません。対角化する行列に何故か$1$列丸ごと$\bm{0}$を書いてしまった人は、定義を復習しましょう。
\item 固有値の計算さえ間違えなければ、$A - \lambda I$は正則ではないので、$1$次方程式$(A - \lambda I)\bm{x} = \bm{0}$は必ず$1$つは解を持ちます。解が見つからないときは、固有値の計算が正しいかを確かめてから、落ち着いて方程式を解きなおしましょう。
\end{itemize}

\subsection{計算例}

$2$次と$3$次の場合に$1$つずつ、計算例を見てみましょう。

\paragraph{問1 (1) の解答} 固有多項式は
\[
\det
\begin{pmatrix}
t - 1 & -2 \\
-2 & t - 1
\end{pmatrix}
= (t - 1)^2 - 4 = t^2 - 2t -3 = (t + 1)(t - 3)
\]
なので、固有値は$3, -1$だと分かる。このそれぞれに対応する固有ベクトルを求める。

まず
\[
\begin{pmatrix}
1 & 2 \\
2 & 1
\end{pmatrix}
\begin{pmatrix}
x \\
y
\end{pmatrix}
=
3
\begin{pmatrix}
x \\
y
\end{pmatrix}
\]
を解くと、$(x, y) = \alpha(1, 1)$ ($\alpha \in \mathbb{R}$)と分かる。次に
\[
\begin{pmatrix}
1 & 2 \\
2 & 1
\end{pmatrix}
\begin{pmatrix}
x \\
y
\end{pmatrix}
=
-1
\begin{pmatrix}
x \\
y
\end{pmatrix}
\]
を解くと、$(x, y) = \alpha(-1, 1)$ ($\alpha \in \mathbb{R}$)となる。よって固有ベクトル$2$本で$\mathbb{R}^2$の基底が張れるので、この問題は対角化可能である。固有ベクトル$2$本を並べて
\[
P :=
\begin{pmatrix}
1 & 1 \\
1 & -1
\end{pmatrix}
\]
と定めると
\[
P^{-1}
\begin{pmatrix}
1 & 2 \\
2 & 1
\end{pmatrix}
P
=
\frac{1}{-2}
\begin{pmatrix}
-1 & -1 \\
-1 & 1
\end{pmatrix}
\begin{pmatrix}
3 & -1 \\
3 & 1
\end{pmatrix}
=
\begin{pmatrix}
3 & 0 \\
0 & -1
\end{pmatrix}
\]
となる。 \qed

\paragraph{問4 (3) の解答} 固有多項式は
\begin{align*}
\varphi(t) &=
\det
\begin{pmatrix}
t - 1 & -1 & 1 \\
-2 & t + 3 & -4 \\
-1 & 2 & t - 3
\end{pmatrix} \\
&= (t - 1)(t + 3)(t - 3) + (-2) \cdot 2 \cdot 1 + (-1) (-4) (-1) - 1 (t + 3) (-1) - (-1) (-2) (t - 3) - (t - 1)\cdot 2 \cdot (-4) \\
&= (t - 1)(t^2 - 9) - 4 - 4 + (t + 3) - 2(t - 3) + 8(t - 1) \\
&= (t^3 - t^2 - 9t + 9) - 8 + 7t + 1 = t^3 - t^2 - 2t + 2 \\
&= (t - 1)(t^2 - 2) = (t - 1)(t + \sqrt{2})(t - \sqrt{2})
\end{align*}
なので、固有値は$1$と$\pm\sqrt{2}$である。このそれぞれに対応する固有ベクトルを求める。

まず
\[
\begin{pmatrix}
1 & 1 & -1 \\
2 & -3 & 4 \\
1 & -2 & 3
\end{pmatrix}
\begin{pmatrix}
x \\
y \\
z
\end{pmatrix}
=
\begin{pmatrix}
x \\
y \\
z
\end{pmatrix}
\]
を解くと$(x, y, z) = \alpha(0, 1, 1)$ ($\alpha \in \mathbb{R}$) となる。次に
\[
\begin{pmatrix}
1 & 1 & -1 \\
2 & -3 & 4 \\
1 & -2 & 3
\end{pmatrix}
\begin{pmatrix}
x \\
y \\
z
\end{pmatrix}
=
\sqrt{2}
\begin{pmatrix}
x \\
y \\
z
\end{pmatrix}
\]
を解くと$(x, y, z) = \alpha(1 + \sqrt{2}, 2, 1)$ ($\alpha \in \mathbb{R}$) となる。最後に
\[
\begin{pmatrix}
1 & 1 & -1 \\
2 & -3 & 4 \\
1 & -2 & 3
\end{pmatrix}
\begin{pmatrix}
x \\
y \\
z
\end{pmatrix}
=
-\sqrt{2}
\begin{pmatrix}
x \\
y \\
z
\end{pmatrix}
\]
を解くと$(x, y, z) = \alpha(1 - \sqrt{2}, 2, 1)$ ($\alpha \in \mathbb{R}$) である。よって$3$本の固有ベクトルを並べて
\[
P :=
\begin{pmatrix}
0 & 1 + \sqrt{2} & 1 - \sqrt{2} \\
1 & 2 & 2 \\
1 & 1 & 1
\end{pmatrix}
\]
と定めると
\[
P^{-1} A P 
=
\frac{1}{2\sqrt{2}}
\begin{pmatrix}
0 & 1 & -1 \\
-2\sqrt{2} & -1 + \sqrt{2} & 1 + \sqrt{2} \\
4\sqrt{2} & 1 - \sqrt{2} & -1 - \sqrt{2}
\end{pmatrix}
\begin{pmatrix}
1 & 1 & -1 \\
2 & -3 & 4 \\
1 & -2 & 3
\end{pmatrix}
\begin{pmatrix}
0 & 1 + \sqrt{2} & 1 - \sqrt{2} \\
1 & 2 & 2 \\
1 & 1 & 1
\end{pmatrix}
=
\begin{pmatrix}
1 \\
& \sqrt{2} \\
& & -\sqrt{2}
\end{pmatrix}
\]
となり、対角化ができる。 \qed

\subsection{問題の解答}

他の問題に対する解答を載せておきます。一々細かい計算を書くのは省きますが、固有値と対応する固有ベクトル、それに対角化する行列$P$の逆行列$P^{-1}$を書いておきます。間違えた人は、どこで計算ミスをしたのかチェックしてください。

また、対角化の結果は固有ベクトルの並べ方に依存します。$P^{-1} A P$の対角成分の順番が入れ替わっていても、$P$の列ベクトルの並びと固有値の並び順が対応していれば、もちろん正解です。また固有ベクトルは、$0$以外の定数倍だけ任意性があります。自分の解答とここに書かれた答えが見た目違っても正しいことはあるので、そういう細かいところは各自で確かめてください。

\paragraph{問1の解答} (2) 固有値は$\pm2$である。対応する固有ベクトルと、対角化行列$P$の逆行列は
\[
A =
\begin{pmatrix}
0 & 4 \\
1 & 0
\end{pmatrix}, 
\bm{v}_2 = 
\begin{pmatrix}
2 \\
1
\end{pmatrix}, 
\bm{v}_{-2} = 
\begin{pmatrix}
-2 \\
1
\end{pmatrix}, \quad
P^{-1} = 
\frac{1}{4}
\begin{pmatrix}
1 & 2 \\
-1 & 2
\end{pmatrix}, \quad
P^{-1} AP =
\begin{pmatrix}
2 & 0 \\
0 & -2
\end{pmatrix}
\]

(3) 固有値は$\pm\sqrt{2}$である。対応する固有ベクトルと、対角化行列$P$の逆行列は
\[
\bm{v}_{\sqrt{2}} = 
\begin{pmatrix}
1 + \sqrt{2} \\
1
\end{pmatrix}, 
\bm{v}_{-\sqrt{2}} = 
\begin{pmatrix}
1 - \sqrt{2} \\
1
\end{pmatrix}, \quad
P^{-1} = 
\frac{1}{2\sqrt{2}}
\begin{pmatrix}
1 & -1 + \sqrt{2} \\
-1 & 1 + \sqrt{2}
\end{pmatrix}, \quad
P^{-1} AP =
\begin{pmatrix}
\sqrt{2} & 0 \\
0 & -\sqrt{2}
\end{pmatrix}
\]

\paragraph{問2の解答}

(1) 固有値は$2, 0, -1$である。対応する固有ベクトルと、対角化行列$P$の逆行列は
\[
\bm{v}_{2} = 
\begin{pmatrix}
1 \\
3 \\
3
\end{pmatrix}, 
\bm{v}_{0} = 
\begin{pmatrix}
1 \\
-1 \\
1
\end{pmatrix},
\bm{v}_{-1} = 
\begin{pmatrix}
1 \\
0 \\
0
\end{pmatrix},  \quad
P^{-1} = 
\frac{1}{6}
\begin{pmatrix}
0 & 1 & 1 \\
0 & -3 & 3 \\
6 & 2 & -4
\end{pmatrix}, \quad
P^{-1} AP =
\begin{pmatrix}
2 \\
 & 0 \\
 & & -1
\end{pmatrix}
\]

\noindent (2) 固有値は$2, 1, -1$である。対応する固有ベクトルと、対角化行列$P$の逆行列は
\[
\bm{v}_{2} = 
\begin{pmatrix}
0 \\
1 \\
0
\end{pmatrix}, 
\bm{v}_{1} = 
\begin{pmatrix}
2 \\
0 \\
3
\end{pmatrix}, 
\bm{v}_{-1} = 
\begin{pmatrix}
3 \\
0 \\
4
\end{pmatrix}, \quad
P^{-1} = 
\begin{pmatrix}
0 & 1 & 0 \\
-4 & 0 & 3 \\
3 & 0 & -2
\end{pmatrix}, \quad
P^{-1} AP =
\begin{pmatrix}
2 \\
 & 1 \\
 & & -1
\end{pmatrix}
\]

\noindent (3) 固有値は$3, 2, 1$である。対応する固有ベクトルと、対角化行列$P$の逆行列は
\[
\bm{v}_{3} = 
\begin{pmatrix}
1 \\
1 \\
1
\end{pmatrix}, 
\bm{v}_{2} = 
\begin{pmatrix}
1 \\
1 \\
0
\end{pmatrix}, 
\bm{v}_{1} = 
\begin{pmatrix}
1 \\
0 \\
0
\end{pmatrix}, \quad
P^{-1} = 
\begin{pmatrix}
0 & 0 & 1 \\
0 & 1 & -1 \\
1 & -1 & 0
\end{pmatrix}, \quad
P^{-1} AP =
\begin{pmatrix}
3 \\
 & 2 \\
 & & 1
\end{pmatrix}
\]

\paragraph{問3の解答}
(1) 固有値は$6, 1$である。対応する固有ベクトルと、対角化行列$P$の逆行列は
\[
\bm{v}_{6} = 
\begin{pmatrix}
1 \\
2
\end{pmatrix}, 
\bm{v}_{1} = 
\begin{pmatrix}
-2 \\
1
\end{pmatrix}, \quad
P^{-1} = 
\frac{1}{5}
\begin{pmatrix}
1 & 2 \\
-2 & 1
\end{pmatrix}, \quad
P^{-1} AP =
\begin{pmatrix}
6 & 0 \\
0 & 1
\end{pmatrix}
\]

\noindent (2) 固有値は$7, 0$である。対応する固有ベクトルと、対角化行列$P$の逆行列は
\[
\bm{v}_{7} = 
\begin{pmatrix}
1 \\
3
\end{pmatrix}, 
\bm{v}_{0} = 
\begin{pmatrix}
-2 \\
1
\end{pmatrix}, \quad
P^{-1} = 
\frac{1}{7}
\begin{pmatrix}
1 & 2 \\
-3 & 1
\end{pmatrix}, \quad
P^{-1} AP =
\begin{pmatrix}
7 & 0 \\
0 & 0
\end{pmatrix}
\]

\noindent (3) 固有値は$7, -1$である。対応する固有ベクトルと、対角化行列$P$の逆行列は
\[
\bm{v}_{7} = 
\begin{pmatrix}
1 \\
2
\end{pmatrix}, 
\bm{v}_{-1} = 
\begin{pmatrix}
-3 \\
2
\end{pmatrix}, \quad
P^{-1} = 
\frac{1}{8}
\begin{pmatrix}
2 & 3 \\
-2 & 1
\end{pmatrix}, \quad
P^{-1} AP =
\begin{pmatrix}
7 & 0 \\
0 & -1
\end{pmatrix}
\]

\noindent (4) 固有値は$\pm i$である。対応する固有ベクトルと、対角化行列$P$の逆行列は
\[
\bm{v}_{i} = 
\begin{pmatrix}
i \\
1
\end{pmatrix}, 
\bm{v}_{-i} = 
\begin{pmatrix}
-i \\
1
\end{pmatrix}, \quad
P^{-1} = 
\frac{1}{2}
\begin{pmatrix}
-i & 1 \\
i & 1
\end{pmatrix}, \quad
P^{-1} AP =
\begin{pmatrix}
i & 0 \\
0 & -i
\end{pmatrix}
\]

\paragraph{問4の解答}
(1) 固有値は$2, 1, -1$である。対応する固有ベクトルと、対角化行列$P$の逆行列は
\[
\bm{v}_{2} = 
\begin{pmatrix}
0 \\
0 \\
1
\end{pmatrix}, 
\bm{v}_{1} = 
\begin{pmatrix}
-1 \\
-1 \\
3
\end{pmatrix}, 
\bm{v}_{-1} = 
\begin{pmatrix}
1 \\
-1 \\
1
\end{pmatrix}, \quad
P^{-1} = 
\frac{1}{2}
\begin{pmatrix}
2 & 4 & 2 \\
-1 & -1 & 0 \\
1 & -1 & 0
\end{pmatrix}, \quad
P^{-1} AP =
\begin{pmatrix}
2 \\
 & 1 \\
 & & -1
\end{pmatrix}
\]

\noindent (2) 固有値は$2, 1, 0$である。対応する固有ベクトルと、対角化行列$P$の逆行列は
\[
\bm{v}_{2} = 
\begin{pmatrix}
1 \\
2 \\
2
\end{pmatrix}, 
\bm{v}_{1} = 
\begin{pmatrix}
1 \\
2 \\
1
\end{pmatrix}, 
\bm{v}_{0} = 
\begin{pmatrix}
1 \\
1 \\
0
\end{pmatrix}, \quad
P^{-1} = 
\begin{pmatrix}
1 & -1 & 1 \\
-2 & 2 & -1 \\
2 & -1 & 0
\end{pmatrix}, \quad
P^{-1} AP =
\begin{pmatrix}
2 \\
 & 1 \\
 & & 0
\end{pmatrix}
\]

\section{対角化可能性の判定条件}

今回の問題はいずれも、上手く対角化ができるものばかりでした。しかし先週見た通り、世の中には対角化不可能な行列も存在します。そこでもう少し、対角化可能性について考えてみましょう。

なお「固有多項式が解を持たなくて対角化できない場合」を除外したいので、この章では\textbf{行列の成分は全て複素数}とします\footnote{代数閉体でさえあれば、複素数体でなくてもよいです。}。

\subsection{対角化可能なための十分条件}

前回
\[
\begin{pmatrix}
0 & 1 \\
0 & 0
\end{pmatrix}
\]
という行列が対角化できないことを確認しました。理由を復習しましょう。

この行列の固有多項式は$t^2$だから、固有値は$0$のみです。しかし固有値$0$に属する固有ベクトルが$1$本しかないため、固有ベクトルだけからなる$\mathbb{R}^2$の基底が作れず、対角化ができないのでした。そこで「固有ベクトルが足りなくなる状況を回避できるのは、どういうときか」という問題が生じます。

\paragraph{異なる固有値に属する固有ベクトルの$1$次独立性}

これから「固有ベクトルで基底を作る」という話をするにあたり、一つ基本的な事実を示しておきます。それは、\textbf{異なる固有値に属する固有ベクトルは$1$次独立になる}ということです。

$n$次正方行列$A \in \Mat_n(\mathbb{C})$の固有ベクトル$\bm{p}_1, \bm{p}_2, \ldots, \bm{p}_k$が、それぞれ異なる固有値$\lambda_1, \lambda_2, \ldots, \lambda_k$に属しているとします。このとき
\[
\alpha_1 \bm{p}_1 + \alpha_2 \bm{p}_2 + \cdots + \alpha_n \bm{p}_n = \bm{0}
\]
を仮定して、$\alpha_1 = \alpha_2 = \cdots = \alpha_n = 0$を示しましょう。

固有ベクトルの定義から、$1 \leq i \leq k$に対し$A\bm{p}_i = \lambda_i \bm{p}_i$が成り立ちます。よって$\alpha_1 \bm{p}_1 + \alpha_2 \bm{p}_2 + \cdots + \alpha_n \bm{p}_n = \bm{0}$の両辺に$A^l$をかけると
\begin{align*}
\bm{0} &= A^l \bm{0} = A^l (\alpha_1 \bm{p}_1 + \alpha_2 \bm{p}_2 + \cdots + \alpha_n \bm{p}_n)
= \lambda_1^l \alpha_1 \bm{p}_1 +  \lambda_2^l \alpha_2 \bm{p}_2 + \cdots + \lambda_k^l \alpha_n \bm{p}_n
\end{align*}
が得られます。この$l$を$l = 0, 1, \ldots, k - 1$と動かした式を並べると
\begin{align*}
\begin{array}{r@{\,}c@{\,}r@{\,}c@{\,}c@{\,}c@{\,}r@{\,}c}
\alpha_1 \bm{p}_1 & + & \alpha_2 \bm{p}_2 & + & \cdots & + & \alpha_n \bm{p}_n  & = \bm{0} \\
\lambda_1 \alpha_1 \bm{p}_1 & + & \lambda_2 \alpha_2 \bm{p}_2 & + & \cdots & + & \lambda_k \alpha_n \bm{p}_n  &= \bm{0} \\
\lambda_1^2 \alpha_1 \bm{p}_1 & + & \lambda_2^2 \alpha_2 \bm{p}_2 & + & \cdots & + & \lambda_k^2 \alpha_n \bm{p}_n  &= \bm{0} \\
& & \vdots \\
\lambda_1^{k - 1} \alpha_1 \bm{p}_1 & + & \lambda_2^{k - 2} \alpha_2 \bm{p}_2 & + & \cdots & + & \lambda_k^{k - 1} \alpha_n \bm{p}_n  &= \bm{0}
\end{array}
\end{align*}
となります。行列の形で書くと
\[
\begin{pmatrix}
\alpha_1 \bm{p}_1 & \alpha_2 \bm{p}_2 & \cdots & \alpha_n \bm{p}_n
\end{pmatrix}
\begin{pmatrix}
1 & \lambda_1 & \lambda_1^2 & \cdots & \lambda_1^{k - 1} \\
1 & \lambda_2 & \lambda_2^2 & \cdots & \lambda_2^{k - 1} \\
\vdots & \vdots & \vdots & \ddots & \vdots \\
1 & \lambda_n & \lambda_n^2 & \cdots & \lambda_n^{k - 1} \\
\end{pmatrix}
=
\begin{pmatrix}
\bm{0} & \bm{0} & \cdots & \bm{0}
\end{pmatrix}
\]
となっています。この左辺に出てきた$k$次正方行列の行列式は、Vandermonde行列式なので
\[
\det
\begin{pmatrix}
1 & \lambda_1 & \lambda_1^2 & \cdots & \lambda_1^{k - 1} \\
1 & \lambda_2 & \lambda_2^2 & \cdots & \lambda_2^{k - 1} \\
\vdots & \vdots & \vdots & \ddots & \vdots \\
1 & \lambda_n & \lambda_n^2 & \cdots & \lambda_n^{k - 1} \\
\end{pmatrix}
=
\pm \prod_{1 \leq i, j \leq k} (\lambda_j - \lambda_i)
\]
と計算できます\footnote{ここでは$\det \neq 0$さえ示せれば後はどうでもよいです。落ち着いて考えれば符号が$\pm$のどっちになるかまで分かりますが、今はその情報が要らないので、どっちか分からない$\pm$をそのまま残しています。}。ここで行列$A$の固有値$\lambda_1, \lambda_2, \ldots, \lambda_k$は全て異なるとしたので、今出てきた行列式の値は$0$ではなく、行列が正則だと分かります。したがって右から逆行列をかけて
\[
\begin{pmatrix}
\alpha_1 \bm{p}_1 & \alpha_2 \bm{p}_2 & \cdots & \alpha_n \bm{p}_n
\end{pmatrix}
=
\begin{pmatrix}
\bm{0} & \bm{0} & \cdots & \bm{0}
\end{pmatrix}
\begin{pmatrix}
1 & \lambda_1 & \lambda_1^2 & \cdots & \lambda_1^{k - 1} \\
1 & \lambda_2 & \lambda_2^2 & \cdots & \lambda_2^{k - 1} \\
\vdots & \vdots & \vdots & \ddots & \vdots \\
1 & \lambda_n & \lambda_n^2 & \cdots & \lambda_n^{k - 1} \\
\end{pmatrix}^{-1}
=
\begin{pmatrix}
\bm{0} & \bm{0} & \cdots & \bm{0}
\end{pmatrix}
\]
を得ます。つまり全ての$1 \leq i \leq k$に対し$\alpha_i \bm{p}_i = \bm{0}$が分かります。ここで固有ベクトルは$\bm{p}_i \neq \bm{0}$を満たしているので、$\alpha_i = 0$が従います。これで異なる固有値に属する固有ベクトルたちの$1$次独立性が言えました。

\paragraph{固有多項式が重根を持たない場合}

さて、話を対角化問題に戻します。固有値があれば、それに対応する固有ベクトルが少なくとも$1$本は存在します。実際$\lambda$が$n$次正方行列$A \in \Mat_n(\mathbb{C})$の固有値なら$\Ker (A - \lambda I) \neq \{\bm{0}\}$なので、$1$本は固有ベクトルが取れます。ということは、$A$が$n$個の異なる固有値$\lambda_1, \lambda_2, \ldots, \lambda_n$を持っていれば、確実に$n$本の固有ベクトルが得られます。各$\lambda_i$毎に少なくとも$1$本は固有ベクトルが存在するのだから、それを$1$本ずつ集めて来れば自動的に$n$本になってしまうわけです。そして異なる固有値に属する固有ベクトルが$1$次独立になることをさっき示しましたから、これで固有ベクトルからなる基底を作れたことになります。よって$A$は対角化可能です。

また$A$が異なる$n$個の固有値を持つことは、固有多項式が重根を持たずに
\[
\varphi_A(t) = (t - \lambda_1)(t - \lambda_2) \cdots (t - \lambda_n)
\]
と因数分解されることと同値です。というわけで、\textbf{固有多項式が重根を持たなければ、固有ベクトルからなる基底が作れて、必ず対角化可能}になります。

\paragraph{対称行列の場合}

対角化問題を考える上で、もう一つ都合のよいケースがあります。それは行列が実対称な場合です。この場合は固有式が全て実数になります。しかも固有多項式が重根を持ってしまったとしても、必ずうまく対角化できることが知られています。しかも対角化するための行列として、直交行列が取って来られるのです。これと同様にして複素数成分の場合、Hermite行列 ($A^* = A$を満たす行列\footnote{$A^* := \overline{{}^t\!A}$です。}) がいつもユニタリ行列 ($U^{-1} = U^*$を満たす行列) で対角化できることが知られています。

この計算には\textbf{Gram--Schmidtの正規直交化法}という手法を使います。きっと残りの授業で現れるはずなので、その時に詳しいことを説明します。	対称行列という限定された状況でしか使えない手法ですが、応用上は対称行列だけで事足りることもままあります。たとえば二次曲線の様子を調べるには、対称行列の対角化ができれば十分です。

\subsection{対角化不可能な可能性}

今度は、対角化不可能な場合に目を向けましょう。「固有値が全て異なれば対角化可能」という話の対偶を取れば、対角化不可能なのは「固有多項式が重根を持つ場合」に限られます。固有多項式が重根を持って初めて「固有ベクトルが足りなかったらどうしよう」という心配をする必要に迫られるのです。

ただ「固有多項式が重根を持ったら、必ず対角化不可能か」というと、そうでもありません。これがややこしいところです。たとえばさっきの対角化不可能な行列
\[
A =
\begin{pmatrix}
0 & 1 \\
0 & 0
\end{pmatrix}
\]
は固有多項式が$t^2$で、重根$0$を持っていました。また$2$次のゼロ行列$O$の固有多項式は$\varphi_O(t) = \det(tI - O) = t^2$なので、これも重根$0$を持ちます。でも$O$は既に対角化されているから、対角化可能です。

こんな感じで、固有多項式が重根を持つからといって、常に対角化できないとは限らないのです。しかもさっきの$A$と$O$は、固有多項式がどちらも$t^2$であるにも関わらず、一方は対角化可能で他方は対角化不可能でした。したがって\textbf{固有多項式を見ているだけでは、対角化可能かどうかの判定はできない}ことも分かります。固有多項式が重根を持ってしまったときは、対角化可能性を判定するのにもっと詳しい情報が必要なのです。この問題を考えるにあたっては
\begin{itemize}
\item どういう行列の場合、固有多項式が重根を持つか
\item 同じ固有多項式を持つ行列について、対角化可能か不可能かは何によって左右されるのか
\end{itemize}
を順番に突き止める必要があります。そこで固有多項式が重根を持つ行列についても、次回以降詳しく調べてみましょう。

