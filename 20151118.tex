\chapter{2次正方行列の対角化とJordan標準形}
\lectureinfo{2015年11月18日 1限}

\section{2次正方行列のJordan標準形}

既に僕たちは「正方行列の固有値が全て異なれば、その行列は対角化可能である」という事実を証明しています。そこで次に問題になるのは「固有値が重複する場合」の扱いです。$2$次正方行列の場合は話が簡単なので、まずは$2$次に限定して状況を分析しましょう。

\subsection{固有値が重複しても対角化可能な場合}

一般には固有多項式が重根を持ったからといって、対角化不可能とは限りません。ですが$2$次の場合「対角化可能かつ固有多項式が重根を持つ」という場合は極めて限定的で、「単位行列の定数倍」しかありません。そのことを確認しましょう。

\paragraph{問2の解答} $2$次正方行列$A$が対角化可能で、かつ固有多項式が$\varphi_A(t) = (t - \alpha)^2$であったとする。このとき$A$が対角化可能で、しかも固有値が$\alpha$しかない。したがって正則行列$P$で
\[
P^{-1} A P = 
\begin{pmatrix}
\alpha & 0 \\
0 & \alpha
\end{pmatrix}
= \alpha I
\]
を満たすものが存在する。すると$A = P (\alpha I) P^{-1} = \alpha I$となる。 \qed

\subsection{対角化不可能な場合}

\paragraph{状況の分類}

いま問$2$で示した「$2$次正方行列の固有多項式が重根を持ち、かつ対角化可能であれば、それは単位行列の定数倍」という結果から、対角化不可能なのはどういう場合なのかが分かります。リストでまとめておきましょう。
\begin{itemize}
\item 固有多項式が重根を持たない\hspace{2zw}→ 対角化可能
\item 固有多項式が重根を持つ
\begin{itemize}
\item そもそも対角化されていた\hspace{1zw}→ 対角化可能
\item 対角化されていなかった\hspace{2zw}→ 対角化不可能
\end{itemize}
\end{itemize}
これを踏まえた上で、対角化不可能な行列を調べてみます。

\paragraph{巾零行列への帰着}

いま、$2$次正方行列$A$が対角化不可能だったとしましょう。このとき固有多項式$\varphi_A(t)$は重根を持ちます。そこで$\varphi_A(t) = (t - \lambda)^2$と書いておきます。するとCayley--Hamiltonの定理から、$\varphi_A(A) = (A - \lambda I)^2 = O$が成り立つと分かります。ここでもし$A - \lambda I = O$だったら$A = \lambda I$は対角化されているので、対角化不可能という仮定に反します。したがって$A - \lambda I \neq O$です。

かくして対角化不可能な$2$次正方行列$A$は、その唯一の固有値$\lambda$に関して$A - \lambda I \neq O$かつ$(A - \lambda I)^2 = O$という性質を満たすことが分かりました。ここで$N := A - \lambda I$とおくと、$N \neq O$かつ$N^2 = O$です。したがって「$N$の見やすい形」が分かれば$A = \lambda I + N$という式によって「$A$の見やすい形」も求まります。対角化不可能な$2$次正方行列は、固有値をずらすことにより、$N^2 = O$となる行列の見やすい形を求める問題に帰着されるのです。

\paragraph{巾零行列の標準形}

$2$次正方行列$N$が$N \neq O$かつ$N^2 = O$を満たすとします。この$N$の「見やすい形」を求めましょう。

まず、このような$N$が常に対角化不可能なことを確かめておきます。$N^2 = O$より$0 = \det O = \det (N^2) = (\det N)^2$です。よって$\det N = 0$となり、$N$の固有多項式は$\varphi_N(t) = t^2 - (\tr N)t$となります。ここでCayley--Hamiltonの定理を使えば、$O = \varphi_N(N) = N^2 - (\tr N)N = -(\tr N)N$です。これと$N \neq O$より$\tr N = 0$が得られ、$\varphi_N(t) = t^2$と分かります。$N$は固有値として$0$のみを持ち、しかも$N \neq O$なので、さっきの状況リストから対角化不可能だと分かります。

次に$N$の固有ベクトルを取ります。$\det N = 0$より$N$は正則ではありません。したがって$\Ker N \neq \{\bm{0}\}$で、言い換えれば$\bm{v} \neq \bm{0}$かつ$N\bm{v} = \bm{0}$となるベクトル$\bm{v} \in \mathbb{R}^2$が取れます。これは$N$の固有値$0$に属する固有ベクトルでもあります。ですから「$N$の見やすい形」を与える基底の中に、$\bm{v}$は含まれるべきでしょう。

そして、この先が問題です。$N$は対角化不可能だから、$N$の固有ベクトルは$\bm{v}$の定数倍しか取ってこられません。「$N$が見やすくなる基底を作るには、$\bm{v}$ともう一つ、何を持って来ればよいか」を考えなければいけないのです。ここで$N \neq O$かつ$N^2 = O$という性質を使います。どんなベクトルに対しても$N^2$を当てたら$\bm{0}$になるので、$\Im N \subset \Ker N$です。一方で$N \neq O$だから$\Im N \neq \{\bm{0}\}$で、$\dim \Im N \geq 1$が分かります。これと$\dim \Ker N = 1$を合わせると$1 \leq \dim \Im N \leq \dim \Ker N = 1$となるので、$\Im N = \Ker N$が得られます。これより$N\bm{w} = \bm{v}$を満たすベクトル$\bm{w} \neq \bm{0}$の存在が言えます。この$(\bm{v}, \bm{w})$を基底として採用してみます。

$N$の表示を計算する前に、$\bm{v}$と$\bm{w}$が$1$次独立なことを確認しておきましょう。$\alpha \bm{v} + \beta \bm{w} = \bm{0}$とおきます。この両辺に$N$をかけると、$N\bm{w} = \bm{v}$かつ$N\bm{v} = \bm{0}$より、$\beta \bm{v} = \bm{0}$が得られます。$\bm{v} \neq \bm{0}$より$\beta = 0$です。したがって$\alpha \bm{v} = \bm{0}$となり、$\bm{v} \neq \bm{0}$と合わせて$\alpha = 0$が分かります。これで$\bm{v}$と$\bm{w}$が$1$次独立で、$\mathbb{R}^2$の基底を張ることが言えました。

そして$P = (\bm{v} \ \bm{w})$とおき、$P^{-1}NP$を計算してみましょう。
\[
NP = 
N
\begin{pmatrix}
\bm{v} & \bm{w}
\end{pmatrix}
=
\begin{pmatrix}
N\bm{v} & N\bm{w}
\end{pmatrix}
= 
\begin{pmatrix}
\bm{0} & \bm{v}
\end{pmatrix}
=
\begin{pmatrix}
\bm{v} & \bm{w}
\end{pmatrix}
\begin{pmatrix}
0 & 1 \\
0 & 0
\end{pmatrix}
=
P
\begin{pmatrix}
0 & 1 \\
0 & 0
\end{pmatrix}
\]
です。したがって
\[
P^{-1} N P =
\begin{pmatrix}
0 & 1 \\
0 & 0 
\end{pmatrix}
\]
と分かります。この形を巾零行列$N$の\textbf{Jordan標準形}といいます。

\paragraph{一般の場合}

最後に、一般の対角化不可能な行列の標準形を与えましょう。$2$次正方行列$A$の固有多項式が重根$\lambda$を持ち、しかも対角化不可能な場合、$A - \lambda I \neq O$かつ$(A - \lambda I)^2 = O$を満たすのでした。そして、この$A - \lambda I$のJordan標準形をいま求めたばかりです。まとめると、上手く基底を取ることで
\[
A - \lambda I = 
\begin{pmatrix}
0 & 1 \\
0 & 0
\end{pmatrix}
\quad \text{i.e.} \quad
A =
\begin{pmatrix}
\lambda & 1 \\
0 & \lambda
\end{pmatrix}
\]
と表せることが分かります。これが対角化不可能な場合の、一つの「見やすい形」です。

これで僕たちは、$2$次正方行列の場合に限り「見やすい形」は何かという問題に答えを与えることができました。対角化可能な場合は、対角化したものが見やすい形です。そうでない場合は対角線に固有値が並び、かつその右上に$1$、左下に$0$というのが見やすい形です。これらをまとめて、$2$次正方行列の\textbf{Jordan標準形}といいます。

また実対称行列の場合は、固有値が重根を持ったとしても常にJordan標準形が対角行列になることも知られています。$2$次の場合なら簡単に示せるので、やってみましょう。

\paragraph{問6の解答}

$2$次実対称行列
\[
A = 
\begin{pmatrix}
a & b \\
b & c
\end{pmatrix} \in \Mat_2(\mathbb{R})
\]
の固有多項式$\varphi_A(t) = t^2 - (a + c)t + (ac - b^2)$が重根を持ったとする。このとき、固有多項式の判別式は$0$だから
\[
0 = (a + c)^2 - 4(ac - b^2) = a^2 - 2ac + c^2 + 4b^2 = (a - c)^2 + (2b)^2
\]
となる。よって$a - c = 2b = 0$でなければいけない。かくして$a = c$かつ$b = 0$を得る。 \qed

\subsection{$2$次正方行列のJordan標準形の計算方法}

ここまでの話から、$2$次正方行列のJordan標準形の計算手順をまとめておきます。$A$を$2$次正方行列とします。
\begin{enumerate}
\item そもそも$A$が最初から対角行列だったら、それがJordan標準形である。
\item $A$が対角行列でなければ、固有多項式$\varphi_A(t)$を計算し、根を求める。
\begin{itemize}
\item $\varphi_A(t)$が$2$個の異なる根を持ったら、$2$個の固有ベクトルを並べた$2$次正方行列を$P$とする。
\item $\varphi_A(t)$が重根$\lambda$を持つときは、$A$の固有ベクトル$\bm{v}$と$(A - \lambda I)\bm{w} = \bm{v}$となるベクトル$\bm{w}$を並べて、$P := (\bm{v} \  \bm{w})$と定める。
\end{itemize}
\item $P^{-1} A P$が$A$のJordan標準形である。
\end{enumerate}

実際にこの手順に従って、計算をしてみましょう。

\paragraph{問4の解答}
\[
A = 
\begin{pmatrix}
1 & 2 \\
0 & 1
\end{pmatrix}, \quad
P = 
\begin{pmatrix}
1 & 0 \\
0 & \frac{1}{2}
\end{pmatrix}
\]
とする。$A$の固有多項式は$\varphi_A(t) = t^2 - 2t + 1 = (t - 1)^2$で、重根を持つ。そして$A$は対角行列でないから、対角化不可能である。また$P^{-1} A P$は
\[
P^{-1} A P = 
\begin{pmatrix}
1 & 0 \\
0 & 2
\end{pmatrix}
\begin{pmatrix}
1 & 2 \\
0 & 1
\end{pmatrix}
\begin{pmatrix}
1 & 0 \\
0 & \frac{1}{2}
\end{pmatrix}
=
\begin{pmatrix}
1 & 0 \\
0 & 2
\end{pmatrix}
\begin{pmatrix}
1 & 1 \\
0 & \frac{1}{2}
\end{pmatrix}
=
\begin{pmatrix}
1 & 1 \\
0 & 1
\end{pmatrix}
\]
である。 \qed

\paragraph{問5の解答}

\[
A = 
\begin{pmatrix}
4 & -1 \\
9 & -2
\end{pmatrix}, \quad
\bm{q} = 
\begin{pmatrix}
1 \\
0 
\end{pmatrix}
\]
とおく。$\varphi_A(t) = t^2 - 2t + 1 = (t - 1)^2$なので、固有値は$1$のみである。そして$A$は対角化されていないので、対角化可能でない。

Cayley--Hamiltonの定理を使うと$(A - I)^2 = \varphi_A(A) = O$となる。そして
\[
\bm{p} := (A - I) \bm{q}
=
\begin{pmatrix}
3 & -1 \\
9 & -3
\end{pmatrix}
\begin{pmatrix}
1 \\
0 
\end{pmatrix}
=
\begin{pmatrix}
3 \\
9
\end{pmatrix}
\]
とおくと、
\[
A\bm{p} =
\begin{pmatrix}
4 & -1 \\
9 & -2
\end{pmatrix}
\begin{pmatrix}
3 \\
9
\end{pmatrix}
=
\begin{pmatrix}
3 \\
9
\end{pmatrix}, \quad
A\bm{q} = 
\begin{pmatrix}
4 & -1 \\
9 & -2
\end{pmatrix}
\begin{pmatrix}
1 \\
0 
\end{pmatrix}
=
\begin{pmatrix}
4 \\
9
\end{pmatrix}
= \bm{p} + \bm{q}
\]
となっている。そこで$P := (\bm{p} \ \bm{q})$とおくと
\[
A
\begin{pmatrix}
\bm{p} & \bm{q}
\end{pmatrix}
=
\begin{pmatrix}
A\bm{p} & A\bm{q}
\end{pmatrix}
=
\begin{pmatrix}
\bm{p} & \bm{p} + \bm{q}
\end{pmatrix}
=
\begin{pmatrix}
\bm{p} & \bm{q}
\end{pmatrix}
\begin{pmatrix}
1 & 1 \\
0 & 1
\end{pmatrix}
\]
となる。これよりJordan標準形が
\[
P^{-1} A P =
\begin{pmatrix}
1 & 1 \\
0 & 1
\end{pmatrix}
\]
と求まる。 \qed

\section{対角化に関する計算問題の解答}

対角化の問題に関しては、答えだけ載せておきます。やり方は前回までと同じなので、各自で確かめてください。

\paragraph{問1の解答}
(1) 固有値は$5, -3$である。対応する固有ベクトルと、対角化行列$P$の逆行列は
\[
\bm{v}_5 = 
\begin{pmatrix}
1 \\
2
\end{pmatrix}, \quad
\bm{v}_{-3} = 
\begin{pmatrix}
-1 \\
2
\end{pmatrix}, \quad
P^{-1} = 
\frac{1}{4}
\begin{pmatrix}
2 & 1 \\
-2 & 1
\end{pmatrix}, \quad
P^{-1} AP =
\begin{pmatrix}
5 & 0 \\
0 & -3
\end{pmatrix}
\]

(2) 固有値は$(1 \pm \sqrt{17})/2$である。対応する固有ベクトルと、対角化行列$P$の逆行列は
\begin{align*}
&\bm{v}_{\frac{1 + \sqrt{17}}{2}} = 
\begin{pmatrix}
5 - \sqrt{17} \\
4
\end{pmatrix}, \quad
\bm{v}_{\frac{1 - \sqrt{17}}{2}} = 
\begin{pmatrix}
5 + \sqrt{17} \\
4
\end{pmatrix}, \\
& P^{-1} = 
\frac{1}{8\sqrt{17}}
\begin{pmatrix}
-4 & 5 + \sqrt{17} \\
4 & -5 + \sqrt{17}
\end{pmatrix}, \quad
P^{-1} AP =
\begin{pmatrix}
\frac{1 + \sqrt{17}}{2} & 0 \\
0 & \frac{1 - \sqrt{17}}{2}
\end{pmatrix}
\end{align*}

(3) 固有値は$2, 1$である。対応する固有ベクトルと、対角化行列$P$の逆行列は
\[
\bm{v}_2 = 
\begin{pmatrix}
1 \\
2
\end{pmatrix}, \quad
\bm{v}_{1} = 
\begin{pmatrix}
1 \\
1
\end{pmatrix}, \quad
P^{-1} = 
\begin{pmatrix}
-1 & 1 \\
2 & -1
\end{pmatrix}, \quad
P^{-1} AP =
\begin{pmatrix}
2 & 0 \\
0 & 1
\end{pmatrix}
\]

(4) 固有値は$3, 0$である。対応する固有ベクトルと、対角化行列$P$の逆行列は
\[
\bm{v}_3 = 
\begin{pmatrix}
1 \\
1
\end{pmatrix}, \quad
\bm{v}_0 = 
\begin{pmatrix}
-2 \\
1
\end{pmatrix}, \quad
P^{-1} = 
\frac{1}{3}
\begin{pmatrix}
1 & 2 \\
-1 & 1
\end{pmatrix}, \quad
P^{-1} AP =
\begin{pmatrix}
3 & 0 \\
0 & 0
\end{pmatrix}
\]

\paragraph{問3の解答}

(1) 固有値は$-2, -3$である。対応する固有ベクトルと、対角化行列$P$の逆行列は
\[
\bm{v}_{-2} = 
\begin{pmatrix}
2 \\
1
\end{pmatrix}, \quad
\bm{v}_{-3} = 
\begin{pmatrix}
1 \\
1
\end{pmatrix}, \quad
P^{-1} = 
\begin{pmatrix}
1 & -1 \\
-1 & 2
\end{pmatrix}, \quad
P^{-1} AP =
\begin{pmatrix}
-2 & 0 \\
0 & -3
\end{pmatrix}
\]

(2) 固有値は$5, 0$である。対応する固有ベクトルと、対角化行列$P$の逆行列は
\[
\bm{v}_5 = 
\begin{pmatrix}
1 \\
2
\end{pmatrix}, \quad
\bm{v}_{0} = 
\begin{pmatrix}
-1 \\
3
\end{pmatrix}, \quad
P^{-1} = 
\frac{1}{5}
\begin{pmatrix}
3 & 1 \\
-2 & 1
\end{pmatrix}, \quad
P^{-1} AP =
\begin{pmatrix}
5 & 0 \\
0 & 0
\end{pmatrix}
\]

(3) 固有値は$4, 2$である。対応する固有ベクトルと、対角化行列$P$の逆行列は
\[
\bm{v}_4 = 
\begin{pmatrix}
-1 \\
1
\end{pmatrix}, \quad
\bm{v}_2 = 
\begin{pmatrix}
-2 \\
1
\end{pmatrix}, \quad
P^{-1} = 
\frac{1}{4}
\begin{pmatrix}
1 & 2 \\
-1 & -1
\end{pmatrix}, \quad
P^{-1} AP =
\begin{pmatrix}
4 & 0 \\
0 & 2
\end{pmatrix}
\]

(4) 固有値は$4, 1$である。対応する固有ベクトルと、対角化行列$P$の逆行列は
\[
\bm{v}_4 = 
\begin{pmatrix}
1 \\
2
\end{pmatrix}, \quad
\bm{v}_1 = 
\begin{pmatrix}
-1 \\
1
\end{pmatrix}, \quad
P^{-1} = 
\frac{1}{3}
\begin{pmatrix}
1 & 1 \\
-2 & 1
\end{pmatrix}, \quad
P^{-1} AP =
\begin{pmatrix}
4 & 0 \\
0 & 1
\end{pmatrix}
\]

\paragraph{問7の解答}

(1) 固有値は$\pm i$である。対応する固有ベクトルと、対角化行列$P$の逆行列は
\[
\bm{v}_i = 
\begin{pmatrix}
i \\
1
\end{pmatrix}, \quad
\bm{v}_{-i} = 
\begin{pmatrix}
-i \\
1
\end{pmatrix}, \quad
P^{-1} = 
\frac{1}{2}
\begin{pmatrix}
-i & 1 \\
i & 1
\end{pmatrix}, \quad
P^{-1} AP =
\begin{pmatrix}
i & 0 \\
0 & -i
\end{pmatrix}
\]

(2) 固有値は$\pm\sqrt{2}$である。対応する固有ベクトルと、対角化行列$P$の逆行列は
\[
\bm{v}_{\sqrt{2}} = 
\begin{pmatrix}
1 \\
\sqrt{2}
\end{pmatrix}, \quad
\bm{v}_{-\sqrt{2}} = 
\begin{pmatrix}
-1 \\
\sqrt{2}
\end{pmatrix}, \quad
P^{-1} = 
\frac{1}{2\sqrt{2}}
\begin{pmatrix}
\sqrt{2} & 1 \\
-\sqrt{2} & 1
\end{pmatrix}, \quad
P^{-1} AP =
\begin{pmatrix}
\sqrt{2} & 0 \\
0 & -\sqrt{-2}
\end{pmatrix}
\]

(3) 固有値は$\pm \sqrt{2} - 1$である。対応する固有ベクトルと、対角化行列$P$の逆行列は %要修正
\[
\bm{v}_{\sqrt{2} - 1} = 
\begin{pmatrix}
1 \\
\sqrt{2}
\end{pmatrix}, \quad
\bm{v}_{-\sqrt{2} - 1} = 
\begin{pmatrix}
-1 \\
\sqrt{2}
\end{pmatrix}, \quad
P^{-1} = 
\frac{1}{2\sqrt{2}}
\begin{pmatrix}
\sqrt{2} & 1 \\
-\sqrt{2} & 1
\end{pmatrix}, \quad
P^{-1} AP =
\begin{pmatrix}
\sqrt{2} - 1 & 0 \\
0 & - \sqrt{2} - 1
\end{pmatrix}
\]

(4) 固有値は$1 \pm i$である。対応する固有ベクトルと、対角化行列$P$の逆行列は
\[
\bm{v}_{1 + i} = 
\begin{pmatrix}
i \\
1
\end{pmatrix}, \quad
\bm{v}_{1 - i} = 
\begin{pmatrix}
-i \\
1
\end{pmatrix}, \quad
P^{-1} = 
\frac{1}{2}
\begin{pmatrix}
-i & 1 \\
i & 1
\end{pmatrix}, \quad
P^{-1} AP =
\begin{pmatrix}
1 + i & 0 \\
0 & 1 - i
\end{pmatrix}
\]

\section{おまけ: 一般の$n$次正方行列のJordan標準形}

それではいよいよ、$n$次正方行列に対するJordan標準形を考えましょう。中々大変なので、余力があるときに読んでみてください。また、次回に$3$次のJordan標準形の話をしますから、先にそっちを読んでも良いでしょう。

これの基本的なアイデアは、$n$次正方行列$A \in \Mat_n(\mathbb{C})$の見やすい形を求める問題を「全ての多項式$f(t) \in \mathbb{C}[t]$に対し、$f(A)$の見やすい形を求める」という問題に広げて考えることです。たとえば$A$が対角行列だったら、どんな多項式$f(t) \in \mathbb{C}[t]$を持ってきても$f(A)$が対角行列になるので、このままの形で$f(A)$は十分見やすくなっています。そこで一般の場合にも「$A$を多項式に代入して得られる行列」をひっくるめて、どうやったら見やすくなるか考えてみるのです\footnote{ちょっと専門的になりますが、もし読めたら今後の展開が一発で分かるので、補足を書いておきます。

ここでする話は、行列$A$の代入で得られる環の準同型写像$\Phi_A\colon \mathbb{C}[t] \rightarrow \Mat_n(\mathbb{C})$を考えるということです。固有多項式$\varphi_A(t)$が$\Phi_A(\varphi_A(t)) = O$を満たすので、$\Phi_A$は商環からの写像$\overline{\Phi_A}\colon \mathbb{C}[t]/(\varphi_A(t)) \rightarrow \Mat_n(\mathbb{C})$を誘導します。そして$\mathbb{C}$は代数閉体だから$\varphi_A(t)$は$1$次式の積に分解し、$\varphi_A(t) = \prod_{i = 1}^l (t - \lambda_i)^{m_i}$と書けます。よって中国剰余定理を使うと、$\mathbb{C}[t]/(\varphi_A(t)) \simeq \bigoplus_{i = 1}^l \mathbb{C}[t]/((t - \lambda)^{m_i})$のように環が分解します。この直和分解に応じて$1 \in \mathbb{C}[t]/(\varphi_A(t))$を分解するのが直交巾等元分解です。また$1, t \in \mathbb{C}[t]/(\varphi_A(t))$を直和因子毎にバラして$\Mat_n(\mathbb{C})$における像を見たものが、単位行列の分解と$A$のスペクトル分解になっています。

ここで肝要なのは、商環$\mathbb{C}[t]/(\varphi_A(t))$からの写像を誘導するところです。ところが商環からの写像を誘導するだけなら、何も固有多項式$\varphi_A(t)$を使わずとも、$\psi_A(A) = O$を満たす多項式$\psi_A(t) \in \mathbb{C}[t]$を$1$つ取って来れば十分です。そして$\Phi_A \colon \mathbb{C}[t] \rightarrow \Mat_n(\mathbb{C})$の定義域と値域の次元の比較をすれば$\Ker \Phi_A \neq 0$が直ちに分かるので、$\mathbb{C}[t]$のイデアル$\Ker \Phi_A$の生成元$\psi_A(t)$が取れることが分かります。これが最小多項式に他なりません。このように簡単な環論を使えば、Cayley--Hamiltonの定理を迂回していきなり最小多項式$\psi_A(t)$を作って、スペクトル分解の議論をすることもできます。実際の計算方法を無視してJordan標準形の存在だけを示すなら、こっちの方が楽です。

なお、固有多項式$\varphi_A(t)$と最小多項式$\psi_A(t)$のどちらを使っても、得られる$A$の分解は同じです。定義から$\psi_A(t)$は$\varphi_A(t)$を割り切るので、商環からの写像$\mathbb{C}[t]/(\varphi_A(t)) \rightarrow \Mat_n(\mathbb{C})$は$\mathbb{C}[t]/(\psi_A(t)) \rightarrow \Mat_n(\mathbb{C})$を経由します。このとき$\mathbb{C}[t]/(\varphi_A(t))$と$\mathbb{C}[t]/(\psi_A(t))$を中国剰余定理で分解すれば、$\overline{\Phi}$を$\mathbb{C}[t]/(\varphi_A(t)) \rightarrow \mathbb{C}[t]/(\psi_A(t)) \rightarrow \Mat_n(\mathbb{C})$と書いたとき、$\mathbb{C}[t]/(\varphi_A(t))$の直和因子それぞれが$\mathbb{C}[t]/(\psi_A(t))$の直和因子をきちんと経由します。なので像$\Mat_n(\mathbb{C})$で見た結果は、どちらも同じです。
}。

\subsection{中間目標}

最初に中間目標を書いておきましょう。$n$次正方行列$A$の固有多項式が
\[
\varphi_A(t) = \prod_{i = 1}^{l} (t - \lambda_i)^{m_i}
\]
と因数分解されていたとします。このとき上手く正則行列$P$を取ると、実は$P^{-1} A P$を
\begin{itemize}
\item $A_i \in \Mat_{m_i}(\mathbb{C})$
\item $A_i$の固有多項式は$(t - \lambda_i)^{m_i}$
\end{itemize}
を満たす正方行列$A_1, A_2, \ldots, A_l$によって
\[
P^{-1} A P =
\begin{pmatrix}
A_1 & O & \cdots & O \\
O & A_2 & \cdots & O \\
\vdots & \vdots & \ddots & \vdots \\
O & O & \cdots & A_l
\end{pmatrix}
\]
と表すことができます。つまり完全な対角化までは叶わなくても、「固有値が全部$\lambda_1$の正方行列」「固有値が全部$\lambda_2$の正方行列」……が対角線上に並んだブロック対角行列の形にまでは、いつでも変形ができるのです。この形にまでしてしまえば、元々の問題が「固有値が全部同じである行列の、見やすい形を求める問題」にまで細分化できます。

この事実の証明にCayley--Hamiltonの定理を使います。

\subsection{Cayley--Hamiltonの定理の帰結}

以下、$n$次正方行列$A$を$1$つ固定します。$A$の「見やすい形」を求める問題に取り組みましょう。

\paragraph{問題を細分するアイデア}

僕たちは「多項式に$A$を代入して得られる行列」を全部ひっくるめて考えようとしていたのでした。ですが「全部の多項式」を考える必要は、実はありません。というのもCayley--Hamiltonの定理があるからです。

$A$の固有多項式を$\varphi_A(t)$とします。このとき任意の多項式$f(t) \in \mathbb{C}[t]$について、$f(t)$を$\varphi_A(t)$で割った商と余り
\[
f(t) = q(t) \varphi_A(t) + r(t)
\]
を考えることができます。両辺に$A$を代入すると、Cayley--Hamiltonの定理より$\varphi_A(A) = O$だから、$f(A) = r(A)$が分かります。$f(t)$に$A$を代入しても、$f(t)$を$\varphi_A(t)$で割った余りに$A$を代入しても、結果は同じなのです。僕たちは\textbf{全ての多項式を考える代わりに、多項式を$\varphi_A(t)$で割った余りだけを考えていれば十分}なのです。以下$f(t) = q(t) \varphi_A(t) + r(t)$のとき、$f(t) \equiv r(t) \mod \varphi_A(t)$と表すことにします\footnote{この記号の使い方は、整数の余りを考えるときの$\mathrm{mod}$と全く同じです。整数の$\mathrm{mod}$を知っている人は、それと同じように扱うと良いでしょう。}。

\paragraph{固有多項式を用いた直交巾等元分解}

$n$次正方行列$A$の固有多項式$\varphi_A(t) \in \mathbb{C}[t]$が、$1$次式の積として
\[
\varphi_A(t) = \prod_{i = 1}^l (t - \lambda_i)^{m_i}
\]
と書けていたとします。これを$p_i(t) := (t - \lambda_i)^{m_i}$を用いて$\varphi_A(t) = p_1(t) p_2(t) \cdots p_l(t)$と書きましょう。$\lambda_1, \ldots, \lambda_l$は互いに異なるから、$p_1(t), \ldots, p_l(t)$は互いに素です。このとき$\!\!\!\mod \varphi_A(t)$のもと、多項式を上手いこと分解できるのです。そのことを調べます。

$1/\varphi_A(t)$を部分分数分解することで
\[
\frac{1}{\varphi_A(t)} = \frac{q_1(t)}{p_1(t)} + \frac{q_2(t)}{p_2(t)} + \cdots + \frac{q_l(t)}{p_l(t)}
\]
と書けます。これの両辺に$\varphi_A(t) = p_1(t) p_2(t) \cdots p_l(t)$をかけることで
\begin{align*}
1 &= \frac{q_1(t)\varphi_A(t)}{p_1(t)} + \frac{q_2(t)\varphi_A(t)}{p_2(t)} + \cdots + \frac{q_l(t)\varphi_A(t)}{p_l(t)} \\
&= q_1(t) p_2(t) \cdots p_l(t) + p_1(t) q_2(t) p_3(t) \cdots p_l(t) + p_1(t) p_2(t) \cdots p_{l - 1}(t) q_l(t)
\end{align*}
が得られます。これを
\[
1 = e_1(t) + e_2(t) + \cdots + e_l(t)
\]
と書いておきます。この式が要になります。

まず$p_i(t) e_i(t) \equiv 0 \mod \varphi_A(t)$に気を付けます。実際$p_i(t) e_i(t) = q_i(t) p_1(t) \cdots p_l(t) = q_i(t) \varphi_A(t)$となっています。よって$p_i(t) e_i(t) \equiv 0 \mod \varphi_A(t)$です。

次に$i \neq j$なら$e_i(t) e_j (t) \equiv 0 \mod \varphi_A(t)$です。$i \neq j$のとき、$e_j(t) = p_1(t) \cdots p_{j - 1}(t) q_j(t) p_{j + 1}(t) \cdots p_l(t)$は$p_i(t)$という因子を持っています。これを$e_i(t)$にかけると$e_i(t) e_j(t) = \varphi_A(t) \times (\text{その他})$という格好になるから、$\mod \varphi_A(t)$で$e_i(t) e_j(t) \equiv 0$となります。

最後に$e_i(t)^2 \equiv e_i(t) \mod \varphi_A(t)$です。実際
\[
e_i(t) = e_i(t) e_1(t) + e_i(t) e_2(t) + \cdots + e_i(t) e_l(t)
\]
の両辺で$\!\!\!\mod \varphi_A(t)$を取ると、右辺で残るのは$e_i(t)^2$の項のみです。よって$e_i(t) \equiv e_i(t)^2 \mod \varphi_A(t)$です。

かくして僕たちは$\!\!\!\mod \varphi_A(t)$のもと、$e_i(t) e_j(t) \equiv \delta_{ij} e_i(t)$を満たす多項式によって、$1$という多項式を$1 = e_1(t) + e_2(t) + \cdots + e_l(t)$と分解できました。これを$1 \mod \varphi_A(t)$の\textbf{直交巾等元分解}といいます。$e_i(t)^2 \equiv e_i(t)$が巾等、$i \neq j$のとき$e_i(t) e_j(t) \equiv 0$となることが直交と呼ばれるので、$e_1(t), \ldots, e_l(t)$が直交巾等元と呼ばれます。

多項式に$1$をかけても結果は変わりませんから、いまの直交巾等元分解を使うと、どんな多項式$f(t) \in \mathbb{C}[t]$も
\[
f(t) = f(t) e_1(t) + f(t) e_2(t) + \cdots + f(t) e_l(t)
\]
と分解できます。

\paragraph{単位行列の直交巾等元分解}

さて、そもそも僕たちは$n$次正方行列$A \in \Mat_n(\mathbb{C})$の見やすい形が知りたくて
\begin{itemize}
\item 全ての多項式$f(t) \in \mathbb{C}[t]$に対する$f(A)$を、一斉に考える
\item 多項式に$f(t)$に$A$を代入するときは、$f(t) \mod \varphi_A(t)$を考えれば良い
\end{itemize}
という話をしていたのでした。この「$\!\!\!\mod \varphi_A(t)$で考えれば良い」という事実は非常に重要です。というのも、さっき得られた$e_1(t), \ldots, e_l(t)$が直交巾等元であるという性質は$\!\!\!\mod \varphi_A(t)$の元で成り立つものでした。ですから、これらの多項式に$A$を代入して$e_1(A), \ldots, e_l(A)$を考えると、直交巾等元の性質がそのまま引き継がれるのです。

$1 = e_1(t) + e_2(t) + \cdots + e_l(t)$の$t$に行列$A$を代入してみましょう。$1 \leq i \leq l$に対し$P_i := e_i(A)$と書けば
\[
I = P_1 + P_2 + \cdots + P_l
\]
となり、単位行列が分解されます。そして今の分解で出てきた行列$P_1, P_2, \ldots, P_l$たちは
\begin{itemize}
\item いずれも$A$の多項式なので、積について$A$と可換
\item $e_i(t) e_j(t) \equiv 0 \mod \varphi_A(t)$より$P_i P_j = O$
\item $e_i(t)^2 \equiv e_i(t) \mod \varphi_A(t)$より、$P_i^2 = P_i$
\end{itemize}
という性質を満たしています。これも単位行列の直交巾等元分解といいます。

\paragraph{一般固有空間分解}

今の直交巾等元分解$I = P_1 + P_2 + \cdots + P_k$を使うと、上手く空間$\mathbb{R}^n$を分解することができます。

まず$I = P_1 + P_2 + \cdots + P_k$より、全てのベクトル$\bm{u} \in \mathbb{R}^n$は$\bm{u} = P_1\bm{u} + P_2\bm{u} + \cdots + P_l\bm{u}$と書けます。つまり$\mathbb{R}^n$の全てのベクトルは$\Im P_1, \Im P_2, \ldots, \Im P_l$に属するベクトルの和として書けます。一方、$i \neq j$なら$\Im P_i \cap \Im P_j = \{\bm{0}\}$です。実際$\bm{u} \in \Im P_i \cap \Im P_j$とすると、$\bm{u} = P_i \bm{v} = P_j \bm{w}$と書けます。すると$\bm{u} = P_i \bm{u} = P_i^2 \bm{u} = P_i P_j \bm{w} = \bm{0}$となってしまいます。

かくして$\mathbb{R}^n = \Im P_1 + \Im P_2 + \cdots + \Im P_l$という式において、右辺に現れる空間はどの$2$つを取っても共通部分がなく、この分解は無駄がないことが分かります。過去に定義した直和記号を使えば$\mathbb{R}^n = \Im P_1 \oplus \Im P_2 \oplus \cdots \oplus \Im P_l$です。この$\Im P_i$は、後で説明する\textbf{一般固有空間}と呼ばれるものになっています。以下$\Im P_i$のことを$\tilde{V}_{\lambda_i}$と書き、各$\tilde{V}_{\lambda_i}$の基底を集めて$\mathbb{R}^n$全体の基底を作りましょう。

いま$\bm{v} \in \tilde{V}_{\lambda_i}$だったとします。そうすると$\bm{v} = P_i \bm{w}$と書けるわけですが、$P_i^2 = P_i$だったので$P_i \bm{v} = P_i^2 \bm{w} = P_i \bm{w} = \bm{v}$です。$\tilde{V}_{\lambda_i}$の上に制限すれば、$P_i$は単位行列として振る舞うわけです。したがって$m_i' := \dim \tilde{V}_{\lambda_i}$とすれば\footnote{実際には、一般固有空間$\tilde{V}_{\lambda_i}$の次元は固有多項式$\varphi_A(t)$の根に$\lambda_i$が現れる個数と一致します。ただし今の時点では$\dim \tilde{V}_{\lambda_i} = m_i$を示せていないので、$m'_i$という別の文字で表しました。後で$m'_i = m_i$を示します。}、新しい基底に関する$P_i$の行列表示は
\[
P^{-1} P_i P =
\begin{pmatrix}
O_{m'_1} \\
& \ddots \\
& & O_{m'_{i - 1}} \\
& & & I_{m'_i} \\
& & & & O_{m'_{i + 1}} \\
& & & & & \ddots \\
& & & & & & & O_{m'_{i_l}}
\end{pmatrix}
\]
と分かります\footnote{$O$や$I$の右下にある数は、行列のサイズを表すものです。}。

この$P^{-1} P_i P$を見れば、$P_i$が「$\tilde{V}_{\lambda_i}$に対応する成分だけを抜きだす」操作を表すことが一目瞭然です。実際$P^{-1} P_i P$をベクトルに当てると、途中にある単位行列がかかる箇所だけがそのまま生き残り、この生き残る箇所は$\tilde{V}_{\lambda_i}$成分に他なりません。つまり$P_i$は\textbf{一般固有空間$\tilde{V}_{\lambda_i}$への射影}です。$I = P_1 + P_2 + \cdots + P_l$は「各$\tilde{V}_{\lambda_i}$成分への射影を全部集めると、元のベクトルが復元できる」という事実を表しています。

\paragraph{行列のスペクトル分解}

さらに、いま作った直交巾等元分解$I = P_1 + P_2 + \cdots + P_l$は、実は$A$ととても相性が良いのです。$P_i = e_i(A)$だったので、$P_i$と$A$の積が入れ替えられるからです。これを使うと、射影$P_i$たちによる空間の分解に応じて、行列$A$を分解することができてしまうのです。

$\bm{v} \in \Im P_i$とします。このとき$P_i \bm{v} = \bm{v}$と$A P_i = P_i A$より$A \bm{v} = A P_i \bm{v} = P_i A \bm{v} \in \Im P_i$となります。$\tilde{V}_{\lambda_i}$のベクトルを$A$で動かしたものは再び$\tilde{V}_{\lambda_i}$に入り、他の$j \neq i$に対する$\tilde{V}_{\lambda_j}$成分を持たないのです。よって、新しい基底に関する$A$の行列表示は
\[
P^{-1} A P = 
\begin{pmatrix}
A_1 \\
& A_2 \\
& & \ddots \\
& & & A_l
\end{pmatrix}
\quad \bigl(A_i \in \Mat_{m'_i}(\mathbb{C})\bigr)
\]
と、ブロック対角行列に分解されます。これを$A$の\textbf{スペクトル分解}といいます。

最後に、スペクトル分解の各ブロック$A_i$からちょうど固有多項式$(t - \lambda_i)^{m_i}$が出てくることを示しましょう。いまの$P^{-1} A P$の表式から
\[
\varphi_{A}(t) = \varphi_{P^{-1} A P}(t) = \det
\begin{pmatrix}
tI_{m'_1} - A_1 \\
& tI_{m'_2} - A_2 \\
& & \ddots \\
& & & tI_{m'_l} - A_l
\end{pmatrix}
= \varphi_{A_1}(t) \varphi_{A_2}(t) \cdots \varphi_{A_l}(t)
\]
が分かります。一方、最初に
\[
\varphi_A(t) = \prod_{i = 1}^l (t - \lambda_i)^{m_i}
\]
と仮定していました。実はこの分解がぴったり一致し、$\varphi_{A_i}(t) = (t - \lambda_i)^{m_i}$となるのです。

これも多項式を$\!\!\!\mod \varphi_A(t)$して考えます。$P^{-1} A P$、$P^{-1} P_i P$をかけて$\tilde{V}_{\lambda_i}$成分だけを抜き出すと、$A_i$と、それ以外の対角ブロックに$O$を並べた行列が得られます。これを
\[
\tilde{A}_i = 
\begin{pmatrix}
O_{m'_1} \\
& \ddots \\
& & O_{m'_{i - 1}} \\
& & & A_i \\
& & & & O_{m'_{i + 1}} \\
& & & & & \ddots \\
& & & & & & & O_{m'_{i_l}}
\end{pmatrix}
\]
と書くと、$\tilde{A}_i = (P^{-1} A P)(P^{-1} P_i P) = P^{-1} A e_i(A) P$です。ここで$p_i(t) e_i(t) \equiv 0 \mod \varphi_A(t)$を思い出しましょう。すると
\[
(t - \lambda_i)^{m_i} e_i(t) = (t - \lambda_i)^{m_i} e_i(t)^{m_i} = \bigl(t e_i(t) - \lambda_i e_i(t)\bigr)^{m_i} \equiv 0 \mod \varphi_A(t)
\]
なので、両辺に$A$を代入して$(A P_i - \lambda_i P_i)^{m_i} = O$が得られます\footnote{ここに出てくるのは$m'_i$ではなく$m_i$です。$m'_i = m_i$を示そうとしているので、議論の中に$m'_i$と$m_i$が混在します。注意深く読んでください。}。これを$P^{-1}$と$P$で挟むと、$P^{-1} A P_i P = \tilde{A}_i$より$(\tilde{A}_i - \lambda_i P^{-1} P_i P)^{m_i} = O$となります。巾零行列の固有値は$0$しかない\footnote{巾零行列$N$の固有値$\lambda$に属する固有ベクトル$\bm{v}$を取ると、$N^k \bm{v} = \lambda^k \bm{v}$です。この式で$k$を十分大きく取れば$N^k = O$となるので、$\lambda^k \bm{v} = \bm{0}$となります。$\bm{v} \neq \bm{0}$なので$\lambda = 0$です。}ので、$\tilde{A}_i - \lambda_i P^{-1} P_i P$は全て固有値が$0$です。そして$\tilde{A}_i - \lambda_i P^{-1} P_i P$における$i$番目の対角ブロックにはちょうど$A_i - \lambda_i I_{m'_i}$があるので、$A_i$の固有多項式は$(t - \lambda_i)^{m'_i}$となります。

こうして、それぞれのブロックに出る小行列$A_i$の固有多項式は$(t - \lambda_i)^{m'_i}$と分かったので、
\[
\prod_{i = 1}^{m_i} (t - \lambda_i)^{m_i} = \varphi_A(t) = \varphi_{P^{-1} A P}(t) = \prod_{i = 1}^l (t - \lambda_i)^{m'_i}
\]
です。この式の一番左と一番右が一致するには、$m_i = m'_i$でなければいけません。これで$\dim \tilde{V}_{\lambda_i} = m_i$が従い、$A$のブロック対角行列への分解が固有多項式の因数分解とぴったり対応していることが分かりました。そして対角線上に並ぶ個々の$A_i - \lambda_i I$は巾零行列です。Jordan標準形の問題は、ようやく巾零行列の標準形を求めるところまで帰着できました。

\paragraph{一般固有空間の意味}

最後に、一般固有空間の意味を説明しておきましょう。

$n$次正方行列$A \in \Mat_n(\mathbb{C})$の固有値$\lambda$に属する固有空間$V_{\lambda}$は、$V_{\lambda} := \Ker (A - \lambda I) = \{\bm{v} \in \mathbb{C}^n \mid (A - \lambda I)\bm{v} = \bm{0}\}$と定義されていました。一方、上で定義した一般固有空間$\tilde{V}_{\lambda}$は、実は$\tilde{V}_{\lambda} = \{\bm{v} \in \mathbb{C}^n \mid \exists k \in \mathbb{N}\  (A - \lambda I)^k \bm{v} = \bm{0}\}$と書くことができます。$(A - \lambda I)$を当てたら$\bm{0}$になるベクトルを集めたのが固有空間であったのに対し、$(A - \lambda I)$を\uline{何回か}当てたら$\bm{0}$になるベクトルを集めたのが一般固有空間です。定義から$V_{\lambda_i} \subset \tilde{V}_{\lambda_i}$です。

$\tilde{V}_{\lambda} = \{\bm{v} \in \mathbb{C}^n \mid \exists k \in \mathbb{N}\  (A - \lambda I)^k \bm{v} = \bm{0}\}$は、次のようにして示せます。まず$\tilde{V}_i = \Im P_i$の上で、$(A_i - \lambda_i I)^{m_i} = O$が成り立つのでした。これで$\tilde{V}_{\lambda}$の元に$A_i - \lambda_i I$を何回か当てたら消えることが分かります。逆に$(A_i - \lambda_i I)^k \bm{v} = \bm{0}$かつ$(A_i - \lambda_i I)^{k - 1} \bm{v} \neq \bm{0}$のとき、$(A_i - \lambda_i I)^{k - 1} \bm{v}$は固有値$\lambda_i$に属する$A$の固有ベクトルです。これと、元々$\mathbb{C}^n$が$\tilde{V}_i$たちの直和に分解していたことを合わせると、$\bm{v}$が$\tilde{V}_i$以外の成分を持ちえないことが分かります。

\subsection{巾零行列のJordan標準形}

行列のスペクトル分解によって、行列の見やすい形を求める問題は「固有値が全て$\lambda$であるような行列$A$」の見やすい形を求める問題に帰着されました。そしてこのような行列は、固有多項式が$\varphi_A(t) = (t - \lambda)^n$なので、Cayley--Hamiltonの定理より$(A - \lambda I)^n = O$を満たします。ここで$N := A - \lambda I$とおくと、$N$の見やすい形さえ求まれば$A = N + \lambda I$によって$A$の見やすい形が求まります。そこで以下、巾零行列$N$の見やすい形を調べましょう
\footnote{
これについても、ちょっと専門的な補足を書いておきます。

スペクトル分解を使うと、元の行列$A$の代入から誘導される$\overline{\Phi_A}\colon \mathbb{C}[t]/(\varphi_A(t)) \rightarrow \Mat_n(\mathbb{C})$を、直和因子への制限$\mathbb{C}[t]/((t - \lambda_i)^{m_i}) \rightarrow \Mat_n(\mathbb{C})$にバラせるのでした。そして$A$がブロック対角行列になるような基底を取れば、各直和因子の像は$i$番目のブロックになるので、結局$\mathbb{C}[t]/((t- \lambda_i)^{m_i}) \rightarrow \Mat_{m_i}(\mathbb{C})$を調べれば良いことになります。さらに固有値をシフトさせれば、調べるべき行列は巾零行列にできます。

さて$m$次巾零行列$N \in \Mat_m(\mathbb{C})$が$N^k = O$を満たすとします。このとき代入準同型$\Phi_N \colon \mathbb{C}[t] \rightarrow \Mat_m(\mathbb{C})$は、形式巾級数環からの代入準同型$\tilde{\Phi}_N \colon \mathbb{C}[\![t]\!] \rightarrow \Mat_m(\mathbb{C})$に拡張されます。どうせ$N$は$k$乗したら$O$なので、巾級数にも$N$を代入できるからです。こうして$\mathbb{C}^m$を$\mathbb{C}[\![t]\!]$加群と思って単因子論を使うと、$\dim \mathbb{C}^m$が有限次元である一方$\mathbb{C}[\![t]\!]$は無限次元なので、自由部分は存在しえないと分かります。よって$\mathbb{C}^m$は$\mathbb{C}[\![t]\!]$の商として得られる加群いくつかの直和です。そして$\mathbb{C}[\![t]\!]/(t^k)$のイデアルは$(t^l)\ (1 \leq l \leq k - 1)$しかないので、結局$\mathbb{C}^m$が$\mathbb{C}[\![t]\!]/(t^l)$の形の加群の直和となります。$\mathbb{C}[\![t]\!]/(t^l)$の基底として$(1, t, t^2, \ldots, t^{l - 1})$を取ることで、巾零行列の標準形が求まります。	
}。

\paragraph{最小多項式}

巾零行列の固有値は全て$0$です。よってCayley--Hamiltonの定理を使うと、$n$次の巾零行列$N \in \Mat_n(\mathbb{C})$は必ず$N^n = O$を満たします。ですが$n$乗までいかなくても、$n$より小さい自然数$k$で$N^k = O$となってしまうこともあり得ます。たとえば
\[
N_1 :=
\begin{pmatrix}
0 & 1 & 0 \\
0 & 0 & 0 \\
0 & 0 & 0
\end{pmatrix}, \quad
N_2 :=
\begin{pmatrix}
0 & 1 & 0 \\
0 & 0 & 1 \\
0 & 0 & 0
\end{pmatrix}
\]
とおくと、$N_1^3 = N_2^3 = O$ですが$N_1^2 = O$, $N_2^2 \neq O$となっています。そこで「何乗の時点で$O$になるか」を考えてみます。$N^k \neq O$かつ$N^{k + 1} = O$となる最小の自然数$m (\leq n)$をとり、$\psi_N(t) := t^m$とおきます。これを$N$の\textbf{最小多項式}と呼びます。

\paragraph{基底を作る戦略}

最小多項式を利用して、$N$が見やすくなるような$\mathbb{R}^n$の基底を作りましょう。

まず$2$次の場合を復習します。$2$次の巾零行列$N \neq O$については、固有値$0$に属する$N$の固有ベクトルが$1$次元分しか取って来られないのでした。そこで固有値$0$の固有ベクトル$\bm{v}$を$1$つ取り、さらに$N\bm{v} = \bm{w}$を満たす$\bm{w}$を取って、基底$(\bm{v} \ \bm{w})$を作りました。この基底によって$N$は
\[
\begin{pmatrix}
0 & 1 \\
0 & 0
\end{pmatrix}
\]
で表示されました。一般の場合も「$N$の積で連なるベクトルの列」をたくさん作ることが、$N$の見やすい形を作る鍵になります。

部分空間の言葉で言い換えてみましょう。$N$を何乗かしていくと零行列$O$に近づくので、$k$が増えるにつれて$\Ker N^k$は大きくなり、$\Im N^k$は小さくなっていきます。すなわち、次のような部分空間の列が得られます。
\begin{align*}
& \{\bm{0}\} = \Ker N^0 \subset \Ker N^1 \subset \Ker N^2 \subset \cdots \subset \Ker N^{m - 1} \subset \Ker N^m = \mathbb{R}^n \\
& \{\bm{0}\} = \Im N^m \subset \Im N^{m - 1} \subset \Im N^{m - 2} \subset \cdots \subset \Im N^1 \subset \Im N^0 = \mathbb{R}^n
\end{align*}
これを利用して、$N$が見やすくなるような$\mathbb{R}^n$の基底を、小さい空間から順に積み重ねて作りましょう。

\paragraph{$\Im N^{m - 1}$の基底について}

まず$\Im N^{m - 1}$の基底$(\bm{w}^{(1)}_1, \bm{w}^{(1)}_2, \ldots, \bm{w}^{(1)}_{l_1})$を取ります。このとき$\Im N^{m - 1}$の定義から、$\bm{w}^{(1)}_i = N^{m - 1} \bm{v}^{(1)}_i$となるベクトル$\bm{v}^{(1)}_i$を各$i$毎に取ることができます。すると
\begin{align*}
&\bm{v}^{(1)}_1 \mapsto N \bm{v}^{(1)}_1 \mapsto N^2 \bm{v}^{(1)}_1 \mapsto \cdots \mapsto N^{m - 1} \bm{v}^{(1)}_1 \mapsto N^m \bm{v}^{(1)}_1 = \bm{0} \\
&\bm{v}^{(1)}_2 \mapsto N \bm{v}^{(1)}_2 \mapsto N^2 \bm{v}^{(1)}_2 \mapsto \cdots \mapsto N^{m - 1} \bm{v}^{(1)}_2 \mapsto N^m \bm{v}^{(1)}_2 = \bm{0} \\
&\qquad \vdots \\
&\bm{v}^{(1)}_{l_1} \mapsto N \bm{v}^{(1)}_{l_1} \mapsto N^2 \bm{v}^{(1)}_{l_1} \mapsto \cdots \mapsto N^{m - 1} \bm{v}^{(1)}_{l_1} \mapsto N^m \bm{v}^{(1)}_{l_1} = \bm{0}
\end{align*}
という、$N$で繋がれたベクトルの列が$l_1$本得られます。これらは全て一次独立です。実際
\[
\sum_{j = 0}^{m - 1} \sum_{i = 1}^{l_1} \alpha_{ij} N^i \bm{v}_j = \bm{0}
\]
とおいて、係数$\alpha_{ij}$が全て$0$になることを示しましょう。

両辺に$N^{m - 1}$を当てると、左端に並んでいるベクトル以外は全部$\bm{0}$になるので、
\[
\alpha_{10} N^{m - 1}\bm{v}^{(1)}_1 + \alpha_{20} N^{m - 1} \bm{v}^{(1)}_2 + \cdots + \alpha_{l_1 0} N^{m - 1}\bm{v}^{(1)}_{l_1} = \bm{0}
\]
が得られます。ところが$\bm{w}^{(1)}_i = N^{m - 1} \bm{v}^{(i)}_i$たちは$\Im N^{m - 1}$の基底になるように作っていたので、$1$次独立です。これで$\alpha_{10} = \alpha_{20} = \cdots = \alpha_{l_1 0} = 0$が言え
\[
\sum_{j = 1}^{m - 1} \sum_{i = 1}^{l_1} \alpha_{ij} N^i \bm{v}_j = \bm{0}
\]
という式が残りました。$j$の始まりが$1$増えたことに気を付けてください。この式に$N^{m - 2}$を当てれば、さっきと同じように$j = 1$の項のみが生き残ります。そして$(\bm{w}^{(1)}_1, \ldots, \bm{w}^{(1)}_{l_1})$が$\Im N^{m - 1}$の基底だったことを使うと、$j = 1$の項の係数が全て$0$だと分かります。こんな風に$N$の巾を上手くかけて「一番左の列だけを残す」という操作を繰り返すと、全ての係数が$0$になることが言えるのです。

\paragraph{$\Im N^{m - 2}$の基底の作り方}

次に、$\Im N^{m - 1}$の基底を延長して$\Im N^{m - 2}$の基底を作ります。上に現れた$1$次独立なベクトルたちのうち
\[
N^{m - 2} \bm{v}^{(1)}_{1}, N^{m - 2} \bm{v}^{(1)}_{2}, \ldots, N^{m - 2} \bm{v}^{(1)}_{l_1}, N^{m - 1} \bm{v}^{(1)}_{1}, N^{m - 1} \bm{v}^{(1)}_{2}, \ldots, N^{m - 1} \bm{v}^{(1)}_{l_1}
\]
は、全て$\Im N^{m - 2}$にも入っています。そこで、これらを延長して基底が作れるように$\tilde{\bm{w}}^{(2)}_1, \tilde{\bm{w}}^{(2)}_2, \ldots, \tilde{\bm{w}}^{(2)}_{l_2} \in \Im N^{m - 2}$を取ります。このとき$N \tilde{\bm{w}}^{(2)}_i \in \Im N^{i - 1}$で、かつ$N^{m - 1}\bm{v}^{(1)}_1, N^{m - 1}\bm{v}^{(1)}_2, \ldots, N^{m - 1}\bm{v}^{(1)}_{l_1}$は$\Im N^{m - 1}$の基底だったので
\[
N \tilde{\bm{w}}^{(2)}_i = \beta_{i1} N^{m - 1}\bm{v}^{(1)}_1 + \beta_{i2} N^{m - 1}\bm{v}^{(1)}_2 + \cdots + \beta_{i l_1} N^{m - 1}\bm{v}^{(1)}_{l_1}
\]
となる$\beta_{i1}, \beta_{i2}, \ldots, \beta_{i l_1}$が存在します。これらの$\beta_{ij}$たちを用いて$\bm{\tilde{w}}^{(2)}_i$を修正し、改めて
\[
\bm{w}^{(2)}_i := \tilde{\bm{w}}^{(2)}_i - (\beta_{i1} N^{m - 2}\bm{v}^{(1)}_1 + \beta_{i2} N^{m - 2}\bm{v}^{(1)}_2 + \cdots + \beta_{i l_1} N^{m - 2}\bm{v}^{(1)}_{l_1})
\]
という式で$\bm{w}^{(2)}_i$を定めてみます。すると$N^{m - 2} \bm{v}^{(1)}_{1}, N^{m - 2} \bm{v}^{(1)}_{2}, \ldots, N^{m - 2} \bm{v}^{(1)}_{l_1}, N^{m - 1} \bm{v}^{(1)}_{1}, N^{m - 1} \bm{v}^{(1)}_{2}, \ldots, N^{m - 1} \bm{v}^{(1)}_{l_1}$たちに$\bm{w}^{(2)}_1, \bm{w}^{(2)}_2, \ldots, \bm{w}^{(2)}_{l_2}$を付け加えたものも、再び$\Im N^{m - 2}$の基底となります。さらに$N \bm{w}^{(2)}_i = \bm{0}$です。

そして$\bm{w}^{(2)}_i \in \Im N^{m - 2}$を使うと、各$i$毎に$\bm{w}^{(2)}_i = N^{m - 2} \bm{v}^{(2)}_i$となるベクトル$\bm{v}^{(2)}_i$を取ることができます。これによって
\begin{align*}
&\bm{v}^{(2)}_1 \mapsto N \bm{v}^{(2)}_1 \mapsto N^2 \bm{v}^{(2)}_1 \mapsto \cdots \mapsto N^{m - 2} \bm{v}^{(2)}_1 \mapsto N^m \bm{v}^{(2)}_1 = \bm{0} \\
&\bm{v}^{(2)}_2 \mapsto N \bm{v}^{(2)}_2 \mapsto N^2 \bm{v}^{(2)}_2 \mapsto \cdots \mapsto N^{m - 2} \bm{v}^{(2)}_2 \mapsto N^m \bm{v}^{(2)}_2 = \bm{0} \\
&\vdots \\
&\bm{v}^{(2)}_{l_2} \mapsto N \bm{v}^{(2)}_{l_2} \mapsto N^2 \bm{v}^{(2)}_{l_2} \mapsto \cdots \mapsto N^{m - 2} \bm{v}^{(2)}_{l_2} \mapsto N^m \bm{v}^{(2)}_{l_2} = \bm{0}
\end{align*}
というベクトルの列が得られました。ここに現れたベクトルたちを、さっき上で作った$N^j \bm{v}^{(1)}_i$たちと合わせると、やはり$1$次独立です。この証明は先ほどと同様「$N$のべき乗をかけて、ベクトルをハコの右の方に追いやる」という方法で示せます。

\newpage

\paragraph{帰納法を走らせる}

ここまでの結果を踏まえ、得られたベクトルをハコに詰めて並べてみましょう。

\begin{figure}[h!]
\centering
\begin{picture}(280, 155)
\put(0, 0){\line(0, 1){120}}
\put(20, 0){\line(0, 1){120}}
\put(50, 0){\line(0, 1){120}}
\put(85, 0){\line(0, 1){120}}
\put(105, 0){\line(0, 1){120}}
\put(155, 0){\line(0, 1){120}}
\put(205, 0){\line(0, 1){120}}
\put(255, 60){\line(0, 1){60}}
\put(0, 0){\line(1, 0){205}}
\put(0, 20){\line(1, 0){205}}
\put(0, 40){\line(1, 0){205}}
\put(0, 60){\line(1, 0){255}}
\put(0, 80){\line(1, 0){255}}
\put(0, 100){\line(1, 0){255}}
\put(0, 120){\line(1, 0){255}}
\put(2, 106){$\bm{v}^{(1)}_1$}
\put(8, 86){$\vdots$}
\put(2, 66){$\bm{v}^{(1)}_{l_1}$}
\put(2, 46){$\bm{v}^{(2)}_1$}
\put(8, 26){$\vdots$}
\put(2, 6){$\bm{v}^{(2)}_{l_2}$}
\put(23, 106){$N \bm{v}^{(1)}_1$}
\put(34, 86){$\vdots$}
\put(23, 66){$N \bm{v}^{(1)}_{l_1}$}
\put(23, 46){$N \bm{v}^{(2)}_1$}
\put(34, 26){$\vdots$}
\put(23, 6){$N \bm{v}^{(2)}_{l_2}$}
\put(53, 106){$N^2 \bm{v}^{(1)}_1$}
\put(64, 86){$\vdots$}
\put(53, 66){$N^2 \bm{v}^{(1)}_{l_1}$}
\put(53, 46){$N^2 \bm{v}^{(2)}_1$}
\put(64, 26){$\vdots$}
\put(53, 6){$N^2 \bm{v}^{(2)}_{l_2}$}
\put(89, 108){$\cdots$}
\put(94, 86){$\vdots$}
\put(89, 68){$\cdots$}
\put(89, 48){$\cdots$}
\put(94, 26){$\vdots$}
\put(89, 8){$\cdots$}
\put(108, 106){$N^{m - 3} \bm{v}^{(1)}_1$}
\put(129, 86){$\vdots$}
\put(108, 66){$N^{m - 3} \bm{v}^{(1)}_{l_1}$}
\put(108, 46){$N^{m - 3} \bm{v}^{(2)}_1$}
\put(129, 26){$\vdots$}
\put(108, 6){$N^{m - 3} \bm{v}^{(2)}_{l_2}$}
\put(158, 106){$N^{m - 2} \bm{v}^{(1)}_1$}
\put(179, 86){$\vdots$}
\put(158, 66){$N^{m - 2} \bm{v}^{(1)}_{l_1}$}
\put(158, 46){$N^{m - 2} \bm{v}^{(2)}_1$}
\put(179, 26){$\vdots$}
\put(158, 6){$N^{m - 2} \bm{v}^{(2)}_{l_2}$}
\put(208, 106){$N^{m - 1} \bm{v}^{(1)}_1$}
\put(229, 86){$\vdots$}
\put(208, 66){$N^{m - 1} \bm{v}^{(1)}_{l_1}$}
\put(205, 125){\line(0, 1){10}}
\put(205, 135){\vector(1, 0){50}}
\put(155, 125){\line(0, 1){25}}
\put(155, 150){\vector(1, 0){100}}
\put(260, 131){$\Im N^{m - 1}$}
\put(260, 148){$\Im N^{m - 2}$}
\end{picture}
\end{figure}

この箱で注目すべき点は、以下の$3$つです。
\begin{itemize}
\item $N$を当てると、ハコの中身が一つ右に移動します。ハコからはみ出たら$\bm{0}$になります。
\item 並んでいるベクトルたちは全て$1$次独立です。
\item 一番右の列にあるベクトルが$\Im N^{m - 1}$の基底、右$2$列にあるベクトルが$\Im N^{m - 2}$の基底です。
\end{itemize}
そして、ここまでの操作と全く同様のことが繰り返せます。すなわち
\begin{itemize}
\item 上で得られたベクトルを延長するように$\Im N^{m - 3}$の基底$\tilde{\bm{w}}^{(3)}_1, \tilde{\bm{w}}^{(3)}_2, \ldots, \tilde{\bm{w}}^{(3)}_{l_3}$を作る
\item $\tilde{\bm{w}}^{(3)}_i$に$N^{m - 3}\bm{v}^{(1)}_1, \ldots, N^{m - 3}\bm{v}^{(1)}_{l_1}, N^{m - 3}\bm{v}^{(2)}_1, \ldots, N^{m - 3}\bm{v}^{(2)}_{l_2}$の定数倍を適当に足して$\Ker N$に入るよう修正し、新しくベクトル$\bm{w}^{(3)}_i$を作る
\item $\bm{w}^{(3)}_i \in \Im N^{m - 3}$を使って、$\bm{w}^{(3)}_i = N^{m - 3} \bm{v}^{(3)}_i$となるベクトル$\bm{v}^{(3)}_i$と、$N$でつながったベクトルの列を作る
\end{itemize}
という方法により、$1$次独立なベクトルをさらに増やせます。これを$N^{m - 4}, N^{m - 5}, \ldots$と繰り返すことによって、最終的に$\mathbb{R}^n = \Im N^0$の基底全体が構成できるというわけです。

\begin{figure}[h!tbp]
\centering
\scalebox{0.8}{
\begin{picture}(280, 295)
\put(0, 0){\line(0, 1){100}}
\put(22, 0){\line(0, 1){60}}
\put(0, 0){\line(1, 0){22}}
\put(0, 20){\line(1, 0){22}}
\put(0, 40){\line(1, 0){22}}
\put(0, 60){\line(1, 0){22}}
\put(2, 46){$\bm{v}^{(m)}_1$}
\put(8, 26){$\vdots$}
\put(2, 6){$\bm{v}^{(m)}_{l_m}$}
\put(22, 60){\line(133, 40){133}}
\put(8, 65){$\vdots$}
\put(8, 85){$\vdots$}
\put(28, 65){\rotatebox{90}{$\ddots$}}
\put(58, 85){\rotatebox{90}{$\ddots$}}
\put(0, 100){\line(0, 1){120}}
\put(20, 100){\line(0, 1){120}}
\put(50, 100){\line(0, 1){120}}
\put(85, 100){\line(0, 1){120}}
\put(105, 100){\line(0, 1){120}}
\put(155, 100){\line(0, 1){120}}
\put(205, 100){\line(0, 1){120}}
\put(255, 160){\line(0, 1){60}}
\put(0, 100){\line(1, 0){205}}
\put(0, 120){\line(1, 0){205}}
\put(0, 140){\line(1, 0){205}}
\put(0, 160){\line(1, 0){255}}
\put(0, 180){\line(1, 0){255}}
\put(0, 200){\line(1, 0){255}}
\put(0, 220){\line(1, 0){255}}
\put(2, 206){$\bm{v}^{(1)}_1$}
\put(8, 186){$\vdots$}
\put(2, 166){$\bm{v}^{(1)}_{l_1}$}
\put(2, 146){$\bm{v}^{(2)}_1$}
\put(8, 126){$\vdots$}
\put(2, 106){$\bm{v}^{(2)}_{l_2}$}
\put(23, 206){$N \bm{v}^{(1)}_1$}
\put(34, 186){$\vdots$}
\put(23, 166){$N \bm{v}^{(1)}_{l_1}$}
\put(23, 146){$N \bm{v}^{(2)}_1$}
\put(34, 126){$\vdots$}
\put(23, 106){$N \bm{v}^{(2)}_{l_2}$}
\put(53, 206){$N^2 \bm{v}^{(1)}_1$}
\put(64, 186){$\vdots$}
\put(53, 166){$N^2 \bm{v}^{(1)}_{l_1}$}
\put(53, 146){$N^2 \bm{v}^{(2)}_1$}
\put(64, 126){$\vdots$}
\put(53, 106){$N^2 \bm{v}^{(2)}_{l_2}$}
\put(89, 208){$\cdots$}
\put(94, 186){$\vdots$}
\put(89, 168){$\cdots$}
\put(89, 148){$\cdots$}
\put(94, 126){$\vdots$}
\put(89, 108){$\cdots$}
\put(108, 206){$N^{m - 3} \bm{v}^{(1)}_1$}
\put(129, 186){$\vdots$}
\put(108, 166){$N^{m - 3} \bm{v}^{(1)}_{l_1}$}
\put(108, 146){$N^{m - 3} \bm{v}^{(2)}_1$}
\put(129, 126){$\vdots$}
\put(108, 106){$N^{m - 3} \bm{v}^{(2)}_{l_2}$}
\put(158, 206){$N^{m - 2} \bm{v}^{(1)}_1$}
\put(179, 186){$\vdots$}
\put(158, 166){$N^{m - 2} \bm{v}^{(1)}_{l_1}$}
\put(158, 146){$N^{m - 2} \bm{v}^{(2)}_1$}
\put(179, 126){$\vdots$}
\put(158, 106){$N^{m - 2} \bm{v}^{(2)}_{l_2}$}
\put(208, 206){$N^{m - 1} \bm{v}^{(1)}_1$}
\put(229, 186){$\vdots$}
\put(208, 166){$N^{m - 1} \bm{v}^{(1)}_{l_1}$}
\put(205, 225){\line(0, 1){10}}
\put(205, 235){\vector(1, 0){50}}
\put(155, 225){\line(0, 1){25}}
\put(155, 250){\vector(1, 0){100}}
\put(260, 231){$\Im N^{m - 1}$}
\put(260, 248){$\Im N^{m - 2}$}
\put(200, 263){$\vdots$}
\put(270, 263){$\vdots$}
\put(20, 225){\line(0, 1){60}}
\put(20, 285){\vector(1, 0){235}}
\put(260, 280){$\Im N$}
\end{picture}
}
\end{figure}

これで$N$の「見やすい形」を作るための基底が作れました。自然数$j \in \mathbb{N}$と複素数$\alpha \in \mathbb{C}$に対し
\[
J_{j}(\alpha) :=
\begin{pmatrix}
\alpha & 1 \\
& \alpha & 1 \\
& & \ddots & \ddots \\
& & & \alpha & 1 \\
& & & & \alpha
\end{pmatrix}
\in \Mat_j(\mathbb{C})
\]
とおきます。すると
\[
\bm{v}^{(k)}_i \mapsto N \bm{v}^{(k)}_i \mapsto N^2 \bm{v}^{(k)}_i \mapsto \cdots \mapsto N^{m - k} \bm{v}^{(k)}_i
\]
という$N$でつながった一連のベクトルがあることに対応し、$N$の中に$J_{m - k + 1}(0)$が対角線上に現れるのです。したがって上で得られたベクトルの列たちを基底に取れば、$N$は$J_{m}(0)$が$l_1$個、$J_{m - 1}(0)$が$l_2$個、$\ldots$、$J_{1}(0)$が$l_m$個並び、他の成分が全て$0$である行列として表されることが分かります。この、$J_{*}(0)$の形の行列を対角線上に並べて得られる行列を\textbf{巾零行列のJordan標準形}といいます。また$1$つ$1$つの$J_*(0)$を\textbf{Jordan細胞}といいます。これまでの議論で、僕たちはあらゆる巾零行列がJordan標準形に変形できることを示しました。

\paragraph{$\Im$ではなく$\Ker$に注目した標準形の導出}

いま上で得られた基底の列を、右寄せで並べてみましょう。こうすると今度は$\Ker N^k$が見やすくなります。

\begin{figure}[h!tbp]
\centering
\scalebox{0.8}{
\begin{picture}(280, 290)
\put(255, 0){\line(0, 1){100}}
\put(205, 0){\line(0, 1){60}}
\put(205, 0){\line(1, 0){50}}
\put(205, 20){\line(1, 0){50}}
\put(205, 40){\line(1, 0){50}}
\put(205, 60){\line(1, 0){50}}
\put(222, 46){$\bm{v}^{(m)}_1$}
\put(230, 26){$\vdots$}
\put(222, 6){$\bm{v}^{(m)}_{l_m}$}
\put(205, 60){\line(-155, 40){155}}
\put(229, 65){$\vdots$}
\put(229, 85){$\vdots$}
\put(190, 65){$\ddots$}
\put(130, 85){$\ddots$}
\put(0, 160){\line(0, 1){60}}
\put(20, 100){\line(0, 1){120}}
\put(50, 100){\line(0, 1){120}}
\put(85, 100){\line(0, 1){120}}
\put(105, 100){\line(0, 1){120}}
\put(155, 100){\line(0, 1){120}}
\put(205, 100){\line(0, 1){120}}
\put(255, 100){\line(0, 1){120}}
\put(20, 100){\line(1, 0){235}}
\put(20, 120){\line(1, 0){235}}
\put(20, 140){\line(1, 0){235}}
\put(0, 160){\line(1, 0){255}}
\put(0, 180){\line(1, 0){255}}
\put(0, 200){\line(1, 0){255}}
\put(0, 220){\line(1, 0){255}}
\put(2, 206){$\bm{v}^{(1)}_1$}
\put(8, 186){$\vdots$}
\put(2, 166){$\bm{v}^{(1)}_{l_1}$}
\put(27, 146){$\bm{v}^{(2)}_1$}
\put(33, 126){$\vdots$}
\put(27, 106){$\bm{v}^{(2)}_{l_2}$}
\put(23, 206){$N \bm{v}^{(1)}_1$}
\put(34, 186){$\vdots$}
\put(23, 166){$N \bm{v}^{(1)}_{l_1}$}
\put(55, 146){$N \bm{v}^{(2)}_1$}
\put(66, 126){$\vdots$}
\put(55, 106){$N \bm{v}^{(2)}_{l_2}$}
\put(53, 206){$N^2 \bm{v}^{(1)}_1$}
\put(64, 186){$\vdots$}
\put(53, 166){$N^2 \bm{v}^{(1)}_{l_1}$}
\put(89, 208){$\cdots$}
\put(94, 186){$\vdots$}
\put(89, 168){$\cdots$}
\put(89, 148){$\cdots$}
\put(94, 126){$\vdots$}
\put(89, 108){$\cdots$}
\put(108, 206){$N^{m - 3} \bm{v}^{(1)}_1$}
\put(129, 186){$\vdots$}
\put(108, 166){$N^{m - 3} \bm{v}^{(1)}_{l_1}$}
\put(116, 146){$N^2 \bm{v}^{(2)}_1$}
\put(129, 126){$\vdots$}
\put(116, 106){$N^2 \bm{v}^{(2)}_{l_2}$}
\put(158, 146){$N^{m - 3} \bm{v}^{(2)}_1$}
\put(179, 126){$\vdots$}
\put(158, 106){$N^{m - 3} \bm{v}^{(2)}_{l_2}$}
\put(158, 206){$N^{m - 2} \bm{v}^{(1)}_1$}
\put(179, 186){$\vdots$}
\put(158, 166){$N^{m - 2} \bm{v}^{(1)}_{l_1}$}
\put(208, 146){$N^{m - 2} \bm{v}^{(2)}_1$}
\put(229, 126){$\vdots$}
\put(208, 106){$N^{m - 2} \bm{v}^{(2)}_{l_2}$}
\put(208, 206){$N^{m - 1} \bm{v}^{(1)}_1$}
\put(229, 186){$\vdots$}
\put(208, 166){$N^{m - 1} \bm{v}^{(1)}_{l_1}$}
\put(205, 225){\line(0, 1){10}}
\put(205, 235){\vector(1, 0){50}}
\put(155, 225){\line(0, 1){25}}
\put(155, 250){\vector(1, 0){100}}
\put(260, 231){$\Ker N$}
\put(260, 248){$\Ker N^2$}
\put(200, 263){$\vdots$}
\put(270, 263){$\vdots$}
\put(20, 225){\line(0, 1){60}}
\put(20, 285){\vector(1, 0){235}}
\put(260, 280){$\Ker N^{m - 1}$}
\end{picture}
}
\end{figure}

$N$を$1$個当てると右の箱に移動し、かつ箱からはみ出たら$\bm{0}$になるのはさっきの図と同じです。ですから一番右の列が$\Ker N$、右から$2$つの列が$\Ker N^2$をそれぞれ張ることが分かります。

このことに注目しても、Jordan標準形の導出に必要な基底を求められます。
\begin{itemize}
\item $\Ker N^{m - 1}$の基底に付け加えて$\mathbb{R}^n$全体の基底が得られるよう、$\bm{v}^{(1)}_1, \bm{v}^{(1)}_2, \ldots, \bm{v}^{(1)}_{l_1}$を選ぶ
\item $\Ker N^{m - 2}$の基底に$N\bm{v}^{(1)}_1, N\bm{v}^{(1)}_2, \ldots, N\bm{v}^{(1)}_{l_1}$を付け加え、さらに$\Ker N^{m - 1}$の基底が得られるようベクトル$\bm{v}^{(2)}_1, \bm{v}^{(2)}_2, \ldots, \bm{v}^{(2)}_{l_2}$を選ぶ
\item 以下繰り返し
\end{itemize}
というようにして、上の図で左の列から順番に$\mathbb{R}^n$の基底を作るのです。たとえば佐武一郎『線型代数学』(裳華房) では、こちらの方法で巾零行列のJordan標準形を求めています。

\paragraph{Jordan標準形の一意性}

Jordan標準形は一通りには決まりません。というのも基底を並べる順番を入れ替えられるからです。$N$を新しい基底で表す際、基底をなすベクトルの列
\[
\bm{v}^{(k)}_i \mapsto N \bm{v}^{(k)}_i \mapsto N^2 \bm{v}^{(k)}_i \mapsto \cdots \mapsto N^{m - k} \bm{v}^{(k)}_i
\]
に応じてJordan細胞$J_{m - k + 1}(0)$が現れるのでした。ですからこの列を一塊にして順番を入れ替えれば、対角線上にJordan細胞が並ぶ順序が入れ替わります。

ですが順番の話を除き「どんな大きさのJordan細胞が、どれだけの個数並ぶか」は$N$だけで完全に決まります。さっきの基底の作り方を思い出すと、Jordan細胞の大きさは$N$で繋がれたベクトルの列の長さに対応しています。そして長さが$m - k + 1$の列
\[
\bm{v}^{(k)}_i \mapsto N \bm{v}^{(k)}_i \mapsto N^2 \bm{v}^{(k)}_i \mapsto \cdots \mapsto N^{m - k} \bm{v}^{(k)}_i
\]
の本数$l_k$たちは
\[
\dim (\Im N^{m - k} \cap \Ker N) = l_1 + l_2 + \cdots + l_k
\]
という関係式を満たします。ですから$l_1 = \dim (\Im N^{m - 1} \cap \Ker N), l_2 = \dim (\Im N^{m - 2} \cap \Ker N) - l_1, \ldots$というようにして$l_1, l_2, \ldots, l_m$は決まるのです。基底の取り方とかに関係せず、$N$の情報だけで完璧に$l_1, l_2, \ldots, l_m$が決まっているので、大きさ$m - k + 1$のJordan細胞が何個現れるかは基底の取り方に依存しません。

\subsection{巾零行列のJordan標準形の計算法}

さっき僕たちはJordan標準形を与える基底の作り方を、具体的に与えました。ですからその通りに頑張れば、原理的には巾零行列のJordan標準形を計算することができるはずです。とは言っても、ひたすら地道にJordan標準形を求めるのは大変です。そこで、もう少し楽してJordan標準形を求める方法を考えてみましょう。

\paragraph{Young図形}

さっきJordan標準形を作るとき、基底をハコに並べました。このハコの中身ではなくフレームだけを考えると、次のような図形が得られています。
\begin{figure}[h!tbp]
\centering
\begin{picture}(100, 100)
\put(0, 0){\line(1 ,0){10}}
\put(0, 10){\line(1 ,0){10}}
\put(0, 20){\line(1 ,0){30}}
\put(0, 30){\line(1 ,0){40}}
\put(0, 40){\line(1 ,0){60}}
\put(0, 50){\line(1 ,0){80}}
\put(0, 60){\line(1 ,0){90}}
\put(0, 70){\line(1 ,0){100}}
\put(0, 80){\line(1 ,0){100}}
\put(0, 0){\line(0 ,1){80}}
\put(10, 0){\line(0 ,1){80}}
\put(20, 20){\line(0 ,1){60}}
\put(30, 20){\line(0 ,1){60}}
\put(40, 30){\line(0 ,1){50}}
\put(50, 40){\line(0 ,1){40}}
\put(60, 40){\line(0 ,1){40}}
\put(70, 50){\line(0 ,1){30}}
\put(80, 50){\line(0 ,1){30}}
\put(90, 60){\line(0 ,1){20}}
\put(100, 70){\line(0 ,1){10}}
\end{picture}
\end{figure}

この図形は、ハコを
\begin{itemize}
\item 各行ごとに左揃えにして
\item 下の行に行くにつれ、ハコの数は広義単調減少する
\end{itemize}
というルールで並べたものです。このような図形を\textbf{Young図形}といいます。Young図形は、$1$列目から順にハコの数を並べても表せます。たとえば上の例だったら、$1$行目に$10$個、$2$行目に$9$個、$\cdots$という風になっているので、$(10, 9, 8, 6, 4, 3, 1, 1)$と表されます。

そしてJordan標準形の構成から得られるYoung図形のそれぞれの行は「$N$で繋がれたベクトルの列」を表していて、これが$1$つのJordan細胞に対応するのでした。ですからJordan標準形から現れるYoung図形を$\lambda = (\lambda_1, \lambda_2, \ldots, \lambda_l)$とすれば、$N$には大きさ$\lambda_1, \lambda_2, \ldots, \lambda_l$のJordan細胞がそれぞれ現れます。すなわちJordan標準形は、Young図形で決まっているのです。このYoung図形を、Jordan標準形の\textbf{型}と呼ぶことにしましょう。巾零行列のJordan標準形そのものを求める問題は、型を求める問題に他なりません。

\paragraph{Young図形と自然数の分割の関係}

さてYoung図形は、自然数の分割と対応しています。そのことを見ておきましょう。

自然数$n$の分割とは、$n$を自然数の和として$n = p_1 + p_2 + \cdots + p_k$のように表す方法のことをいいます。たとえば$(3, 2, 1)$や$(4, 2)$などは$6$の分割です。話をするときは「$n$がどう分割されるか」だけを問題にし、$p_1, p_2, \ldots, p_k$の並び順は問題にしません。ですので$n$の分割を表すときは、最初から$p_1 \geq p_2 \geq \cdots \geq p_k$としておきます。

こうすると$n$の分割$(p_1, p_2, \ldots, p_k)$があるとき、「$i$行目に$p_i$個のハコを置く」というルールによって、ハコの数が$n$個のYoung図形ができます。逆にYoung図形$\lambda = (\lambda_1, \lambda_2, \ldots, \lambda_l)$が与えられれば、これを$\lambda_1 + \lambda_2 + \cdots + \lambda_l$の分割と思うことができます。

このように、Young図形を考えることと整数の分割を考えることは同じなのです。したがって巾零行列のJordan標準形の型を求めるとき、型の候補は$n$の分割です。$n$の分割の中から、どの型が正解かを探す問題になるのです。

\paragraph{Young図形の縦と横の長さ}

次に、Jordan標準形の型を表すYoung図形をもう少し詳しく調べてみましょう。$N$のJordan標準形の型を$\lambda = (\lambda_1, \lambda_2, \ldots, \lambda_l)$とします。

まず最初に分かるのは「Young図形の$1$行目は、最小多項式の次数に一致する」という事実です。
\begin{itemize}
\item Young図形の各行は、$N$で繋がったベクトルの列を表していた
\item Young図形の行の数は、下に行くにつれて広義単調減少
\end{itemize}
ということを考えると、全てのベクトルは$N^{\lambda_1}$を当てると$\bm{0}$になることが分かります。一方$\lambda$の$1$行目を見れば、$N^{\lambda_1 - 1}$を当てても$\bm{0}$にならないベクトルが存在することも分かります。したがって$N$の最小多項式は$\psi_N(t) = t^{\lambda_1}$と分かります。逆に言えば、$N$の最小多項式を求めれば$\lambda$の$1$行目が決定するのです。

次に、$\lambda$の縦の長さを考えてみます。ハコに入ったベクトルたちを右寄せにした図を見れば、$\lambda$の縦の長さは$\dim \Ker N$だと分かります。ですから連立$1$次方程式$N\bm{x} = \bm{0}$を解いてパラメータの個数を調べれば、$\lambda$の縦の長さが分かります。

\paragraph{$2$次と$3$次の場合}

$2$次と$3$次の巾零行列がどのJordan標準形を持つかは、最小多項式を見れば一発で分かってしまいます。

まず大きさ$2$のYoung図形は
\[
\Yvcentermath1 \yng(2),\ \yng(1,1)
\]
の$2$つしかありません。そして$\Yvcentermath1\yng(2)$の型に対応する$2$次正方行列は$O$しかないので、$O$でない巾零行列のYoung図形は自動的に$\Yvcentermath1\yng(1,1)$の型を持つと分かります。次に、大きさ$3$のYoung図形は
\[
\Yvcentermath1 \yng(3),\ \yng(2,1),\ \yng(1,1,1)
\]
の$3$つしかありません。これら$3$つのYoung図形は、$1$行目の長さだけで判別できます。つまり最小多項式の次数を見れば、Jordan標準形の型が
\[
\begin{pmatrix}
0 & 0 \\
 & 0 & 0 \\
 & & 0
\end{pmatrix}, \quad
\begin{pmatrix}
0 & 1 \\
 & 0 & 0 \\
 & & 0
\end{pmatrix}, \quad
\begin{pmatrix}
0 & 1 \\
 & 0 & 1 \\
 & & 0
\end{pmatrix}
\]
のどれになるか、答えだけは分かります。

さらに縦の情報も組み合わせると、もっと高次の場合でもJordan標準形をあっさり求められます。最小多項式$\psi_N(t)$の次数と$\dim \Ker N$が求まれば、$\lambda$の$1$行目と$1$列目の長さが決定するのでした。$n = 6$のときは、$6$の分割は
\[
\Yvcentermath1 \yng(6),\ \yng(5,1),\  \yng(4,2),\  \yng(4,1,1),\ 
\yng(3,3),\  \yng(3,2,1),\  \yng(2,2,2),\  \yng(2,2,1,1),\  \yng(2,1,1,1,1),\  \yng(1,1,1,1,1,1)
\]
のいずれかです。これらのYoung図形は、$1$行目と$1$列目の情報だけで完全に決定できてしまいます。Jordan標準形だけを求めるのに、一々基底を求める必要はありません。

ただし$7$次になってしまうと、$1$行目と$1$列目の情報だけでは
\[
\Yvcentermath1 \yng(3,3,1),\  \yng(3,2,2)
\]
が識別できません。こうなってくると、部分的にでも地道に基底を求める必要性が出てきます。

\subsection{一般のJordan標準形の計算法}

ここまでの話をまとめると、次の手順で行列のJordan標準形が計算できます。
\begin{enumerate}
\item 固有多項式$\varphi_A(t)$を求め、その根を全て計算し$\varphi_A(t) = \prod_{i = 1}^l (t - \lambda_i)^{m_i}$と因数分解する。
\item 固有多項式が重根を持ったなら、最小多項式$\psi_A(t)$を
\begin{itemize}
\item 最小多項式が固有多項式$\varphi_A(t)$を割り切ること
\item 固有多項式の根は、必ず最小多項式の根にもなっていること
\end{itemize}
を利用して求める。
\item 最小多項式を$\psi_A(t) = \prod_{i = 1}^l (t - \lambda_i)^{n_i}$と書き、各$\lambda_i$について
\begin{itemize}
\item Jordan細胞の最大サイズが$n_i$であること
\item Jordan細胞のサイズの合計が$m_i$であること
\item Jordan細胞の個数が$\dim (\Ker A - \lambda_i I)$と一致すること
\end{itemize}
を手掛かりに、Jordan細胞を全て求める。
\end{enumerate}

実務上は、固有値が$7$重に重複した行列のJordan標準形を求める場面はほとんどないと思います。せいぜい$3$次くらいが分かっていればで十分でしょう。今回は$2$次に特化した扱いを述べたので、次回は$3$次の場合を扱います。

