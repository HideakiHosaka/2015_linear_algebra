\chapter{1年間のまとめ}
\lectureinfo{2015年12月9日 1限}

いよいよ$1$年分の長い授業が終わりましたね。お疲れ様でした。最後に配布するこのプリントでは
\begin{itemize}
\item これまでに説明しそびれた「向き」の話
\item Aセメスター期末試験の解説
\item 今後の展望
\end{itemize}
について書きます。

\section{事務連絡}

\paragraph{印刷された解答について}

前回、期末試験の前に解説プリントをレポートボックスで配布したとき、持って行った人がすごく少なかったです。プリントがなくてもいいやと思った人や、あるいはネットで見ればいいやと思った人が結構いたのでしょう。ですので今回は、印刷部数をかなり減らしています。もし印刷したプリントが品切れになってしまった場合は、\texttt{hosaka@ms.u-tokyo.ac.jp} に連絡をもらえれば用意します。もちろん、いつも通り
\begin{center}
\url{https://github.com/HideakiHosaka/2015_linear_algebra/raw/master/2015linear_algebra.pdf}
\end{center}
からも落とせるようにしておきます。

\paragraph{昔の答案の扱い}

前回「お友達の分を持ってってください」と書いたおかげか、回収されずに残っていた答案が結構減りました。ご協力ありがとうございます。ただ、まだ未回収の答案が結構あります。引き続き、自分の答案を持っていくとき、近くのお友達の分を合わせて持って行ってください。ご協力お願いします。

なおレポートボックスが閉まった後、3月末までは穂坂の手元に答案を置いておきます。もし後から取りに来る場合は、メールで連絡してください。

\section{実線型空間の向き}

ここまでの解説の中で、一つ議論をほったらかしていたことがあります。それは「向き」の話です。たとえば$2$次元空間$\mathbb{R}^2$の場合、基底をなす$2$本のベクトルを並べると、時計回りか反時計回りかを考えることができます。また$3$次元空間$\mathbb{R}^3$の場合、基底をなす$3$本のベクトルを並べると、右手系か左手系かを考えることができます。これらが基底の「向き」というものです。

さて、基底の向きは行列式で判断できます。$3$次元空間$\mathbb{R}^3$の場合、基底$(\bm{f}_1, \bm{f}_2, \bm{f}_3)$の向きは$\det(\bm{f}_1, \bm{f}_2, \bm{f}_3)$が正のときに右手系、負のときに左手系という事実があります\footnote{正確には「$\det > 0$を正の向きと定義すると、右手系が正の向きになる」です。}。一般に$n$次元空間$\mathbb{R}^n$の場合も、基底を並べた行列の行列式の正負に応じて、正の向きと負の向きを定義します。

ただ、行列式を定義すれば「向き」の定義はできるものの、この定義が良いかどうかは別問題です。実は「右手系と左手系が混ざらないかどうか」とか、あるいは「右手系と左手系を、もっと細かく分けるべきなのかどうか」といった問題は、ちゃんと考えないと分かりません。特に$3$次元の場合、右手系か左手系かといった話は物理などで良く用いられる、重要なものです。ここまでの授業で準備は整っているので、それらを活用して「向き」の議論をしましょう。

\subsection{「向き」とは何か}

「向き」とは、基底の並びを指定するものです。このことをまず感覚的に捉えるため、$3$次元の場合を例にしてみましょう。よく$3$本のベクトル$\bm{u}, \bm{v}, \bm{w} \in \mathbb{R}^3$が右手系であるとは「右手の親指を$\bm{u}$に、人差し指を$\bm{v}$に重ねた時、$\bm{w}$が中指の側に来ること」と説明されます。実際には手の形とか指の関節とかによって、親指、人差し指、中指をぴったり$\bm{u}, \bm{v}, \bm{w}$に重ねられないこともありますが、大体の意味は通じるでしょう。

さて今の「右手系」の説明から分かる、非常に重要なことがあります。それは「右手系の基底をちょっとだけ動かしても、右手系のままだ」という事実です。右手の親指、人差し指、中指を$\bm{u}, \bm{v}, \bm{w}$にそれぞれ重ねた時、どの指も少しは動かすことができるはずです。よって基底をなす$3$本のベクトルのどれを動かしても、十分小さい動かし方であれば右手系のままになります。実は、この「ちょっと動かしても向きが変わらない」ということだけで、向きに関する議論を全て行うことができるのです。

%もうちょっと数学的に言えば、「ちょっと動かす」というのは「函数の連続性」の話をしていることになります。基底全体の集合は$GL_n(\mathbb{R})$と同一視でき、それを含む$\Mat_n(\mathbb{R})^2$は$\mathbb{R}^{n^2}$と同一視できます。よって$\gamma\colon [0, 1] \rightarrow GL_n(\mathbb{R})$が連続写像であることが定義できます。

\subsection{行列式の連続性}

以下の話の鍵になるのは、行列式を写像と思った時の連続性です。たとえば
\[
\det
\begin{pmatrix}
x & y \\
z & w
\end{pmatrix}
= xw - yz
\]
は、$x, y, z, w$という$4$つの変数の多項式です。よってこれは連続函数です。また一般に$n$次元の場合も
\[
\det (x_{ij})_{1 \leq i, j \leq n} = \sum_{\sigma \in \mathfrak{S}_n} \sgn(\sigma) x_{1 \sigma(1)} x_{2 \sigma(2)} \cdots x_{n \sigma(n)}
\]
であり、見た目はややこしくなるものの
\begin{itemize}
\item $x_{1 \sigma(1)} x_{2 \sigma(2)} \cdots x_{n \sigma(n)}$は、$n^2$個の変数$x_{11}, x_{12}, \ldots, x_{1n}, x_{21}, x_{22}, \ldots, x_{2n}, \ldots, x_{n1}, x_{n2}, \ldots, x_{nn}$の多項式
\item $\sgn(\sigma)$は$\pm1$のどっちか
\item 多項式をいくつか足し算しても多項式
\end{itemize}
という簡単な事実を組み合わせるだけで、$\det X$は$X$の成分の多項式だと分かります。よって$X$が$n$次正方行列の全体を動き回るとき、$\det\colon \Mat_n(\mathbb{R}) \rightarrow \mathbb{R}$は連続函数になります\footnote{多項式が連続写像であることは、連続写像の積 / 和が連続写像であることを何回か使えば証明できます。}。

\paragraph{正の向きと負の向きの分離}
さて$\mathbb{R}^n$の基底$(\bm{f}_1, \bm{f}_2, \ldots, \bm{f}_n)$を、これらを並べてできる行列$F := (\bm{f}_1 \  \bm{f}_2 \  \cdots \  \bm{f}_n)$と同一視します。このとき基底の条件は$\det F \neq 0$と同値でした。よって行列式が$0$でない$n$次正方行列全体の集合を$GL_n(\mathbb{R})$と書くことにすれば、$GL_n(\mathbb{R})$は$\mathbb{R}^n$の基底全体のなす集合と思えます。そして行列式を取る写像を$GL_n(\mathbb{R})$に制限した写像$\det\colon GL_n(\mathbb{R}) \rightarrow \mathbb{R}$を考えると、値域からは$0$が抜け落ちます。イメージはこんな図です。$\det$の行き先の$0$の部分に対応した「穴」が、$GL_n(\mathbb{R})$に開いています。

\begin{figure}[h!tbp]
\centering
\begin{picture}(100, 100)
\put(0, 25){\framebox(38, 70){}}
\put(42, 25){\framebox(38, 70){}}
\put(0, 10){\line(1, 0){38}}
\put(40, 10){\circle{4}}
\put(40, 20){\dashbox(0, 80){}}
\put(42, 10){\line(1, 0){38}}
\put(92, 7){$\mathbb{R}$}
\put(37.5, 0){$0$}
\put(85, 62){$GL_n(\mathbb{R})$}
\put(98, 37){$\det$}
\put(95, 59){\vector(0, -1){40}}
\end{picture}
\end{figure}

この図を見ながら$\det$の連続性を考えると、正の向きと負の向きが混ざらないことが証明できます。いま行列$F_+, F_- \in GL_n(\mathbb{R})$がそれぞれ$\det F_+ > 0$, $\det F_- < 0$を満たしたとします。もし$F_+$の成分を連続的にちょっとずつ変えて、「基底に対応する」という条件を保ったまま$F_-$に変形できたとしましょう。つまり連続函数$\gamma\colon [0, 1] \rightarrow GL_n(\mathbb{R})$で、$\gamma(0) = F_+$, $\gamma(1) = F_-$を満たすものが存在したとします。このとき$\det F_+ > 0$かつ$\det F_- < 0$ですから、$\det \circ \gamma(0) > 0$かつ$\det \circ \gamma(1) < 0$です。よって中間値の定理より、$F_+$を$F_-$に変える途中で必ず$\det F = 0$となる行列を経由しなければいけません。ところが$F \not \in GL_n(\mathbb{R})$は基底に対応しません。これは、$F_+$をちょっとずつ変形して$F_-$に変形できることに矛盾します。このように行列式の連続性と中間値の定理を組み合わせることで、正の向きの基底を連続的に変形して負の向きの基底にすることはできないと分かります。特に$3$次元の場合、右手系と左手系は決して混ざりません。

\subsection{正規直交基底の向き}

「正の向きと負の向きが決して混ざらないこと」の次に問題になるのは「正の向きと負の向きを、これ以上細かく分ける必要はないのか」という問題です。もし正の向きより細かく「連続に変形していったときに移り合う基底のかたまり」が分けられるのであれば、細かい分け方を調べるのに意味があるでしょう。

ですが実際は、正の向きと負の向きより細かい区分はしようがありません。これを確かめるため、まずは扱いやすい「正規直交基底全体の集合」に限定して考えます。正規直交基底と直交行列は$1$対$1$に対応するのでした。そして$\mathbb{R}^n$における正の向きの正規直交基底は、$\det > 0$の直交行列全体の集合$SO(n)$と対応します。このことから「向きを細分化できるかどうか」という問題は「行列式が正の直交行列全体の集合$SO(n)$のつながり方」を調べる問題へと言い換えられます。

\paragraph{$2$次元の場合}

たとえば$2$次元の場合、$SO(2)$は回転行列全体のなす集合と一致するのでした。そうすると
\[
SO(2) = 
\biggl\{
\begin{pmatrix}
\cos \theta & -\sin \theta \\
\sin \theta & \cos \theta
\end{pmatrix}
\mid 0 \leq \theta < 2\pi
\biggr\}
\]
は半開区間$[0, 2\pi)$の両端$0$と$2\pi$をつなげた図形、つまり円周と$1$対$1$に対応します。
\[
SO(2) \ni
\begin{pmatrix}
\cos \theta & -\sin \theta \\
\sin \theta & \cos \theta
\end{pmatrix}
\longleftrightarrow
\begin{pmatrix}
\cos \theta \\
\sin \theta
\end{pmatrix}
\in \mathbb{R}^2
\]
この対応により、$SO(2)$の形は「円周」というひとつながりの図形だと分かります\footnote{このことを指して、$SO(2)$は\textbf{連結である}といいます。}。よって$\mathbb{R}^2$の場合、正の向きの正規直交基底は全て連続的な変形で移り合います。

\subsection{$3$次の特殊直交群の決定}

では$3$次元の場合はどうでしょうか。この場合も、やはり$\det > 0$の直交行列全体の集合$SO(3) := \{ A \in GL_3(\mathbb{R}) \mid {}^t\!A A = I, \det A = 1\}$が$\mathbb{R}^3$の正の向きの正規直交基底と対応します。よって$SO(3)$がひとつながりかどうかが問題になります。このことを調べるには「そもそも$3$次の直交行列にはどんなものがあるか」ということを知る必要がです。

結論から言ってしまうと$2$次元のときと同様、$SO(3)$は$3$次元空間における回転を表す行列全体の集合と一致しています。たとえば$x, y, z$軸それぞれに関する回転行列は
\[
R_x(\theta) :=
\begin{pmatrix}
1 & 0 & 0 \\
0 & \cos \theta & -\sin \theta \\
0 & \sin \theta & \cos \theta \\
\end{pmatrix}, \quad
R_y(\theta) :=
\begin{pmatrix}
\cos \theta & 0 & \sin \theta \\
0 & 1 & 0 \\
-\sin \theta & 0 & \cos \theta
\end{pmatrix}, \quad
R_z(\theta) :=
\begin{pmatrix}
\cos \theta & -\sin \theta & 0 \\
\sin \theta & \cos \theta & 1 \\
0 & 0 & 1
\end{pmatrix}
\]
で与えられ\footnote{$R_y$の定義式でマイナスが付く位置に気を付けましょう。$xyz$座標系を$y$軸の正の側から眺めると、$z$軸から$x$軸へ回る向きが正の向きになります。これに合わせるため、左下にマイナスがつきます。}、いずれも$3$次直交行列になっています。これに限らず、$SO(3)$の全ての元は「何らか回転」を表すことを示しましょう。ただし$SO(2)$の元は「回転する角度」だけで決まったのに対し、$SO(3)$の元は回転角に加えて「回転軸の向き」という情報を指定しないと決まりません。なので$SO(3)$の方が、話はややこしくなります。

\paragraph{回転軸の存在}

まず全ての$A \in SO(3)$が「回転軸」にあたるものを持つことを示しましょう。ベクトル$\bm{u} \in \mathbb{R}^3$が回転軸の方向を向いているとは、$A\bm{u} = \bm{u}$であること、つまり$\bm{u}$が$A$の固有値$1$に属する固有ベクトルであることに他なりません。だから示すべきは、$A$が$1$を固有値に持つことです。そこで固有多項式を$\varphi_A(t)$とおきます。ここで
\begin{itemize}
\item $A$の実固有値は$\pm1$のいずれかしかないこと
\item $\det A = 1$
\end{itemize}
を思い出しましょう。いま$\varphi_A(t)$は$3$次多項式なので、グラフを考えれば必ず$1$個以上の実根を持ちます。そして直交行列の実固有値は$\pm1$しかないので、$\varphi_A(t)$の実根は$\pm1$のいずれかに限られます。もし$\varphi_A(t)$が実根$1$を持てば、話はそこで終わりです。

そこで$\varphi_A(t)$が、実根$-1$を持つと仮定しましょう。このとき$\det A = 1$なので、$\varphi_A(t)$の定数項は$(-1)^3 \times 1 = -1$です。これと$\varphi_A(t)$における$t^3$の係数が$+1$であることを合わせると、$\varphi_A(t) = (t + 1)(t^2 - at - 1)$と書けることが分かります。すると後ろの$2$次式$t^2 - at - 1$の判別式は$a^2 + 4 > 0 $ですから、$a$の値が何であっても必ず実根を持ちます。ここで再び$\varphi_A(t)$の実根が$\pm1$以外に存在しないことを考えると、$t^2 - at - 1$は$\pm1$のいずれかを実根に持ちます。そこで$t = \pm1$を代入すると、$t$の符号に関わらず$a = 0$が得られます。すると結局$\varphi_A(t) = (t + 1)(t^2 - 1) = (t - 1) (t + 1)^2$になって、$\varphi_A(t)$が$1$を実根に持つことが分かります。どんな場合でも、必ず$\varphi_A(t)$は$t = 1$を実根に持つのです。これで$A$が固有値$1$に属する固有ベクトル$\bm{u}$を持つことが分かりました。

\paragraph{$SO(3)$の元が回転を表すこと}

この固有値$1$に属する固有ベクトル$\bm{u}$を使って、$A$が回転を表すことを示します。いま$\bm{u}$が球面極座標で$(\sin\varphi \cos\theta, \sin\varphi \sin\theta, \cos\theta)$と書けていたとします。すると$\bm{u}$を$z$軸回りに$-\varphi$回転し、さらに$y$軸回りに$-\theta$回転させると$z$軸の方向を向きます。故に$R_y(-\theta) R_z(-\varphi) \bm{u}$は$z$軸と平行なベクトルです。そして$A\bm{u} = \bm{u}$なので、
\[
A R_z(\varphi) R_y(\theta) \bm{e}_3 = A \frac{\bm{u}}{|\bm{u}|} = \frac{\bm{u}}{|\bm{u}|} = R_z(\varphi) R_y(\theta) \bm{e}_3
\]
となります。これより、行列$R_y(-\theta) R_z(-\varphi) A R_z(\varphi) R_y(\theta)$は
\[
R_y(-\theta) R_z(-\varphi) A R_z(\varphi) R_y(\theta) = 
\begin{pmatrix}
* & * & 0 \\
* & * & 0 \\
* & * & 1
\end{pmatrix}
=
\begin{pmatrix}
A' & \bm{0} \\
{}^t\bm{a} & 1
\end{pmatrix} \quad (A' \in \Mat_2(\mathbb{R}), \bm{a} \in \mathbb{R}^2)
\]
という格好をしていないといけません。そして$R_y(-\theta) R_z(-\varphi) A R_z(\varphi) R_y(\theta)$は直交行列の積だから、再び直交行列です。よって
\[
\begin{pmatrix}
1 & 0 & 0 \\
0 & 1 & 0 \\
0 & 0 & 1
\end{pmatrix}
=
\begin{pmatrix}
A' & \bm{0} \\
{}^t\bm{a} & 1
\end{pmatrix}
{}^t
\begin{pmatrix}
A' & \bm{0} \\
{}^t\bm{a} & 1
\end{pmatrix}
=
\begin{pmatrix}
A' & \bm{0} \\
{}^t\bm{a} & 1
\end{pmatrix}
\begin{pmatrix}
{}^t\!A' & \bm{a} \\
{}^t\bm{0} & 1
\end{pmatrix}
=
\begin{pmatrix}
A'\,{}^t\!A' & A'\bm{a} \\
{}^t\bm{a}\,{}^t\!A' & 1
\end{pmatrix}
\]
を満たします。これよりまず$A'\,{}^t\!A' = I$で、$A'$は$2$次直交行列と分かります。特に$A'$は正則なので、これと$A' \bm{a} = \bm{0}$より$\bm{a} = \bm{0}$も分かります。よって結局
\[
R_y(-\theta) R_z(-\varphi) A R_z(\varphi) R_y(\theta)
=
\begin{pmatrix}
A' & \bm{0} \\
{}^t\bm{0} & 1
\end{pmatrix}
\]
となります。この両辺の$\det$を取れば$1 = \det A = \det A'$となるので、$A' \in SO(2)$です。よって$A'$は$xy$平面内の回転を表す行列だから
\[
A' = 
\begin{pmatrix}
\cos \psi & - \sin \psi \\
\sin \psi & \cos \psi
\end{pmatrix}, \quad
R_y(-\theta) R_z(-\varphi) A R_z(\varphi) R_y(\theta)
=
\begin{pmatrix}
\cos \psi & - \sin \psi & 0 \\
\sin \psi & \cos \psi & 0 \\
0 & 0 & 1
\end{pmatrix}
= R_z(\psi)
\]
と書けます。$A$の固有ベクトル$\bm{u}$を$z$軸に持ってくれば回転を表すので、元々の$A$は$\bm{u}$軸回りの回転を表すことが分かります。また今の式から$A = R_z(\varphi) R_y(\theta) R_z(\psi) R_y(-\theta) R_z(-\varphi)$となり、$A \in SO(3)$は$x, y, z$軸回りの回転行列の積で表せる\footnote{$SO(3)$の元を軸回りの回転行列の積で表すやり方には他にも色々あります。}ことも分かりました。

\paragraph{$SO(3)$がひとつながりなこと}

いまの式を使うと、$SO(3)$がひとつながりなことが分かります。全ての$A \in SO(3)$は$A = R_z(\varphi) R_y(\theta) R_z(\psi) R_y(-\theta) R_y(-\varphi)$の形に書けるのでした。よって$R_z(\varphi) R_y(\theta) R_z(\psi) R_y(-\theta) R_y(-\varphi)$という式で、$(\psi, \theta, \varphi) = (0, 0, 0)$から始めて$\psi$を$\psi_0$まで連続的に変え、その後$\theta$を$\theta_0$まで連続的に変え、最後に$\varphi$を$\varphi_0$まで連続的に変えれば、単位行列を連続的に$A$に変えることができます\footnote{ちゃんと書くと、連続写像$\gamma \colon [0, 1] \rightarrow SO(3)$を$[0, 1/3]$で$\gamma(t):= R_z(3t\psi)$, $[1/3, 2/3]$で$\gamma(t):= R_y(3(t - 1/3)\theta) R_z(\psi) R_y(-3(t - 1/3)\theta)$, $[2/3, 1]$で$R_z(3(t - 2/3)\varphi) R_y(\theta) R_z(\psi) R_y(-\theta) R_z(-3(t - 2/3)\varphi)$と定める、という意味です。連続写像に関するテクニカルな扱いについてはこれ以上立ち入らないでおきます。感覚的に「連続的に変形できること」が分かっていれば十分です。}。これで$SO(3)$のどんな行列も単位行列の連続変形で得られることが分かるので、$SO(3)$はひとつながりです。

基底の話に直せば、今示したことは「全ての正規直交基底は連続的に回転させることで移り合う」という事実に他なりません。

\paragraph{$4$次元以上の場合}

一言だけ、$4$次元以上の場合について補足しておきます。今の$SO(3)$の議論は$4$次元以上では上手くいきません。というのも$4$次元の場合、たとえば
\[
\begin{pmatrix}
\cos \theta & - \sin \theta & 0 & 0 \\
\sin \theta & \cos \theta & 0 & 0 \\
0 & 0 & \cos \varphi & -\sin \varphi \\
0 & 0 & \sin \varphi & \cos \varphi
\end{pmatrix}
\]
のような「$2$軸に関する独立な回転を同時に行う行列」が登場します。この行列は「回転軸」に当たるものを持たないので、さっきの議論が使えなくなります。


\subsection{向きの分類}

こうして正規直交基底の向きが正の向きと負の向きしかないことが分かると、残りの基底を含めて考えても、基底の向きは$2$通りしかないことが言えます。この証明には、Gram--Schmidtの正規直交化法を使います。

Gram--Schmidtの正規直交化法を復習しておきましょう。何か基底$(\bm{u}_1, \bm{u}_2, \ldots, \bm{u}_n)$が与えられたとき、新しい基底$(\bm{f}_1, \bm{f}_2, \ldots, \bm{f}_n)$を
\begin{enumerate}
\item $\bm{u}_i$から$\bm{f}_1, \bm{f}_2, \ldots, \bm{f}_{i - 1}$方向の成分を消すため、$\bm{f}_i' := \bm{u}_i - (\bm{u}_i \cdot \bm{f}_1) \bm{f}_1 - (\bm{u}_i \cdot \bm{f}_2) \bm{f}_2 - \cdots - (\bm{u}_i \cdot \bm{f}_{i - 1}) \bm{f}_{i - 1}$とおく。
\item $\bm{f}_i'$が$\bm{f}_1, \bm{f}_2, \ldots, \bm{f}_{i - 1}$と直交するようになったので、長さを$1$に揃えるため$\bm{f}_i := \bm{f}_i' / |\bm{f}_i|$と定める。
\end{enumerate}
という操作を繰り返すことによって作ると、これが正規直交基底になるのでした。

\paragraph{正規直交基底の向き}

さて、各変形のステップは行列で書くことができます。簡単のためまず$2$次元で確かめてみましょう。$(\bm{u}_1, \bm{u}_2)$が$\mathbb{R}^2$の基底のとき
\[
\bm{f}_1 := \frac{\bm{u}_1}{|\bm{u}_1|}, \bm{f}_2' := \bm{u}_2 - (\bm{u}_2 \cdot \bm{f}_1)\bm{f}_1, \bm{f}_2 := \frac{\bm{f}_2'}{|\bm{f}_2'|}
\]
によって正規直交基底$(\bm{f}_1, \bm{f}_2)$が得られています。これを行列で書くと
\[
\begin{pmatrix}
\bm{f}_1 & \bm{u}_2
\end{pmatrix}
=
\begin{pmatrix}
\bm{u}_1 & \bm{u}_2
\end{pmatrix}
\begin{pmatrix}
\frac{1}{|\bm{u}_1|} & 0 \\
0 & 1
\end{pmatrix}, \ 
\begin{pmatrix}
\bm{f}_1 & \bm{f}_2'
\end{pmatrix}
=
\begin{pmatrix}
\bm{f}_1 & \bm{u}_2
\end{pmatrix}
\begin{pmatrix}
1 & -\bm{u}_2 \cdot \bm{f}_1 \\
0 & 1
\end{pmatrix}, \ 
\begin{pmatrix}
\bm{f}_1 & \bm{f}_2
\end{pmatrix}
=
\begin{pmatrix}
\bm{f}_1 & \bm{f}_2'
\end{pmatrix}
\begin{pmatrix}
1 & 0 \\
0 & \frac{1}{|\bm{f}_2'|}
\end{pmatrix}
\]
であり、まとめて
\[
\begin{pmatrix}
\bm{f}_1 & \bm{f}_2
\end{pmatrix}
=
\begin{pmatrix}
\bm{u}_1 & \bm{u}_2
\end{pmatrix}
\begin{pmatrix}
\frac{1}{|\bm{u}_1|} & 0 \\
0 & 1
\end{pmatrix}\begin{pmatrix}
1 & -\bm{u}_2 \cdot \bm{f}_1 \\
0 & 1
\end{pmatrix}
\begin{pmatrix}
1 & 0 \\
0 & \frac{1}{|\bm{f}_2'|}
\end{pmatrix}
\]
と書けます。よって
\[
\begin{pmatrix}
\bm{u}_1 & \bm{u}_2
\end{pmatrix}
\begin{pmatrix}
a & 0 \\
0 & 1
\end{pmatrix}\begin{pmatrix}
1 & b \\
0 & 1
\end{pmatrix}
\begin{pmatrix}
1 & 0 \\
0 & c
\end{pmatrix}
\]
という式において、最初$(a, b ,c) = (1, 0, 1)$とし、そこから連続的に$(a, b, c)$を$(1/|\bm{u}_1|, -\bm{u}_2 \cdot \bm{f}_1, 1/|\bm{u}_2|)$という値に変化させることによって、元の基底$(\bm{u}_1, \bm{u}_2)$を連続的に正規直交基底$(\bm{f}_1, \bm{f}_2)$へと変形できます。イメージとしては「親指と人差し指を開いて、直角を作る操作」という感じです。

そして$n$次元の場合でも、話は同じです。基底$(\bm{u}_1, \bm{u}_2, \ldots, \bm{u}_n)$にGram--Schmidtの正規直交化法を施して$(\bm{f}_1, \bm{f}_2, \ldots, \bm{f}_n)$の正規直交基底を作るとき、$i$番目のステップは
\begin{align*}
\bigl(\bm{f}_1 \ \  \cdots \ \  \bm{f}_{i - 1} \ \  \bm{f}_i' \ \  \bm{u}_{i + 1} \ \  \cdots \ \  \bm{u}_n \bigr)
&=
\bigl(\bm{f}_1 \ \  \cdots \ \  \bm{f}_{i - 1} \ \  \bm{u}_i \ \  \bm{u}_{i + 1} \ \  \cdots \ \  \bm{u}_n \bigr)
\bordermatrix{
& & & & i^{\text{th}} \\
& 1 & & & a_1 \\
& & \ddots & & \vdots \\
& & & 1 & a_{i - 1} \\
& & & & 1 \\
& & & & & 1 \\
& & & & & & \ddots \\
& & & & & & & 1
}
\\
\bigl(\bm{f}_1 \ \  \cdots \ \  \bm{f}_{i - 1} \ \  \bm{f}_i \ \  \bm{u}_{i + 1} \ \  \cdots \ \  \bm{u}_n \bigr)
&=
\bigl(\bm{f}_1 \ \  \cdots \ \  \bm{f}_{i - 1} \ \  \bm{f}_i' \ \  \bm{u}_{i + 1} \ \  \cdots \ \  \bm{u}_n \bigr)
\bordermatrix{
& & & & i^{\text{th}} \\
& 1 & \\
& & \ddots \\
& & & 1 \\
& & & & b \\
& & & & & 1 \\
& & & & & & \ddots \\
& & & & & & & 1
}
\end{align*}
において$a_j = - \bm{u}_i \cdot \bm{f}_j$ ($1 \leq j \leq i$) および$b = 1/|\bm{f}_j'|$とした式で与えられます。ですから各$a_j$を$- \bm{u}_i \cdot \bm{f}_j$に、$b$を$1$から$1/|\bm{f}_j'|$に連続的に変える操作を考えれば、Gram--Schmidtの正規直交化法は基底の連続的な変形で実現できることが分かります。また、この過程で向きは変わりません。

\paragraph{向きが$2$種類しかないこと}

かくして、
\begin{itemize}
\item どんな基底も、連続的な変形で向きを保ったまま正規直交基底にできること
\item 正の向きのどんな正規直交基底も、連続的な変形で標準基底にできること
\end{itemize}
が分かりました。これで、正の向きの基底は全て標準基底に連続変形できることが分かりました。正の向きの基底全体はひとつながりです。また、負の向きの基底もひとつながりです。というのも、負の向きの基底は最初の$1$本を逆向きにすれば正の向きになります。こうすれば「正の向き基底である」という条件を保ったまま連続的な変形で標準基底に移せます。その後、このプロセス全体で最初の$1$本を$-1$倍すれば、負の向き基底が全て$(-\bm{e}_1, \bm{e}_2, \ldots, \bm{e}_n)$に移せることが分かります。

「正の向き / 負の向きの基底は、連続的にちょっとだけ変えても正の向き / 負の向きであるべき」という発想と組み合わせれば、正の向きの基底は、これ以上分割しようがないことが分かります。

\section{試験の解答例と解説}

期末試験の解答例と解説を載せます。解答例はあくまで一例に過ぎず、他の解き方も存在し得るということに注意してください。また配点についてはTAの穂坂は一切知りませんので、このプリントには書いていません。

\subsection{問題1: Gram--Schmidtの正規直交化法}

\noindent (1)
\[
|\bm{a}_0|^2 = \bm{a}_0 \cdot \bm{a}_0 = \int_{-1}^1 x^{0 + 0} dx = \int_{-1}^1 dx = 2
\]
より、$|\bm{a}_0| = \sqrt{2}$である。

\noindent (2) $\alpha \bm{a}_0 + \beta \bm{a}_1 + \gamma \bm{a}_2 = \bm{0}$とする。この両辺と$\bm{a}_0, \bm{a}_1, \bm{a}_2$との内積をそれぞれ取ると
\[
\begin{array}{c@{\,}c@{\,}c@{\,}c@{\,}c@{\,}c@{\,}c@{\,}c@{\,}c}
0 & = & \bm{a}_0 \cdot (\alpha \bm{a}_0 + \beta \bm{a}_1 + \gamma \bm{a}_2) & = & (\bm{a}_0 \cdot \bm{a}_0) \alpha & + & (\bm{a}_0 \cdot \bm{a}_1) \beta & + & (\bm{a}_0 \cdot \bm{a}_2) \gamma \rule{0zw}{1.2zw}\\
0 & = & \bm{a}_1 \cdot (\alpha \bm{a}_0 + \beta \bm{a}_1 + \gamma \bm{a}_2) & = & (\bm{a}_1 \cdot \bm{a}_0) \alpha & + & (\bm{a}_1 \cdot \bm{a}_1) \beta & + & (\bm{a}_1 \cdot \bm{a}_2) \gamma \rule{0zw}{1.2zw}\\
0 & = & \bm{a}_2 \cdot (\alpha \bm{a}_0 + \beta \bm{a}_1 + \gamma \bm{a}_2) & = & (\bm{a}_2 \cdot \bm{a}_0) \alpha & + & (\bm{a}_2 \cdot \bm{a}_1) \beta & + & (\bm{a}_2 \cdot \bm{a}_2) \gamma \rule{0zw}{1.2zw}
\end{array}
\]
となる。ここで
\[
\bm{a}_m \cdot \bm{a}_n = \int_{-1}^1 x^{m + n} dx = \frac{1 - (-1)^{m + n + 1}}{m + n + 1}
\]
を代入すると、$\alpha, \beta, \gamma$は
\[
\begin{pmatrix}
2 & 0 & \frac{2}{3} \\
0 & \frac{2}{3} & 0 \\
\frac{2}{3} & 0 & \frac{2}{5}
\end{pmatrix}
\begin{pmatrix}
\alpha \\
\beta \\
\gamma
\end{pmatrix}
= \bm{0}
\]
という連立$1$次方程式を満たすことが分かります。この係数行列の行列式を計算すると$\frac{2}{3} \cdot (\frac{4}{5} - \frac{4}{9}) \neq 0$と分かるので、係数行列は正則です。よって$\alpha = \beta = \gamma = 0$が従い、$\bm{a}_0, \bm{a}_1, \bm{a}_2$が正則だと分かります。

\noindent (3) まず(1)の結果から、$\bm{e}_0 = \frac{1}{\sqrt{2}} \bm{a}_0$です。次に$\bm{a}_1 \cdot \bm{a}_1 = \frac{2}{3}$と$\bm{a}_0 \cdot \bm{a}_1 = 0$なので、$\bm{e}_1 = \sqrt{\frac{2}{3}}\bm{a}_1$と置きます。これで$|\bm{e}_1| = 1$かつ$\bm{e}_0 \cdot \bm{e}_1 = 1$です。最後に$\bm{a}_0 \cdot \bm{a}_2 = \frac{2}{3}$, $\bm{a}_1 \cdot \bm{a}_2 = 0$より$\bm{e}_0 \cdot \bm{a}_2 = \frac{\sqrt{2}}{3}$, $\bm{e}_1 \cdot \bm{a}_2 = 0$です。そこで$\bm{e}'_2 := \bm{a}_2 - \frac{\sqrt{2}}{3} \bm{e}_0 = \bm{a}_2 - \frac{1}{3}\bm{a}_0 $とおくと$\bm{e}_0 \cdot \bm{e}_2' = \bm{e}_1 \cdot \bm{e}_2' = 0$となります。最後に
\[
|\bm{e}_2'|^2 = \Bigl|\bm{a}_2 - \frac{1}{3}\bm{a}_0\Bigr|^2 = |\bm{a}_2|^2 - \frac{2}{3}(\bm{a}_2 \cdot \bm{a}_0) + \frac{1}{9} |\bm{a}_0|^2 = \frac{2}{5} - \frac{4}{9} + \frac{2}{9} = \frac{8}{45}
\]
です。そこで$\bm{e}_2 := \sqrt{\frac{45}{8}} (\bm{a}_2 - \frac{1}{3} \bm{a}_0) = \frac{3}{2}\sqrt{\frac{5}{2}} \bm{a}_2 - \frac{1}{2}\sqrt{\frac{5}{2}} \bm{a}_0$とおけば、正規直交基底が得られます。 \qed

\paragraph{Legendre多項式}

問題中では$\bm{a}_0, \bm{a}_1, \bm{a}_2, \ldots$という記号を使っていましたが、要はこの問題で言いたいのは、多項式全体のなす線型空間$\mathbb{R}[x]$における「内積」を
\[
f(x) \cdot g(x) := \int_{-1}^1 f(x)g(x) dx
\]
で定義したという話です。この内積は対称正定値という性質、つまり
\begin{itemize}
\item 任意の多項式$f(x), g(x) \in \mathbb{R}[x]$について$f(x) \cdot g(x) = g(x) \cdot f(x)$が成り立つ
\item 任意の多項式$f(x) \in \mathbb{R}[x]$について$f(x) \cdot f(x) \geq 0$が成り立つ
\item $f(x) \cdot f(x) = 0$となる多項式$f(x) \in \mathbb{R}[x]$は$f(x) = 0$に限る
\end{itemize}
という性質を持っています。内積がこれらの性質を満たすことを使えば、Gram--Schmidtの正規直交化法によって正規直交基底を求めることができます。このような、多項式のなす線型空間における「内積」に関する直交基底として得られる多項式たちを\textbf{直交多項式}といいます。特に、今回求めた多項式たちを適当に定数倍したものは\textbf{Legendre多項式}と呼ばれます。

\subsection{問題2: $3$直線の共点条件と$3$点の共線条件}

\noindent (1) 平面上の$3$点$\mathrm{A}(a_1, a_2)$, $\mathrm{B}(b_1, b_2)$, $\mathrm{P}(x, y)$を考えます。この平面を空間$\mathbb{R}^3$内の平面$z = 1$上と同一視し、$\mathrm{A}$, $\mathrm{B}$, $\mathrm{P}$に対応する点をそれぞれ$\mathrm{A}'(a_1, a_2, 1)$, $\mathrm{B}'(b_1, b_2, 1)$, $\mathrm{P}'(x, y, 1)$とします。すると点$\mathrm{A}'$, $\mathrm{B}'$, $\mathrm{P}'$が同一直線上にあることは、ベクトル$\overrightarrow{\,\mathrm{OA}'\,\,}$, $\overrightarrow{\,\mathrm{OB}'\,\,}$, $\overrightarrow{\,\mathrm{OP}'\,\,}$の張る部分空間が$2$次元以下になることと同値で、したがってベクトル$\overrightarrow{\,\mathrm{OA}'\,\,}$, $\overrightarrow{\,\mathrm{OB}'\,\,}$, $\overrightarrow{\,\mathrm{OP}'\,\,}$が$1$次従属なこととも同値になります。この条件は行列式を用いて
\[
\det
\begin{pmatrix}
x & a_1 & b_1 \\
y & a_2 & b_2 \\
1 & 1 & 1
\end{pmatrix}
= 0
\]
と書けます。行を並び替えても行列式が$0$かどうかは変わらないので、求める結果が得られます。

\noindent (2) 平面上の$3$直線$a_i x + b_i y + c_i = 0$ ($i = 1, 2, 3$) は、どれも平行でなく、かつ一致もしないと仮定します。この$3$直線が点$(x_0, y_0)$で交わるとき
\[
\begin{pmatrix}
a_1 & b_1 & c_1 \\
a_2 & b_2 & c_2 \\
a_3 & b_3 & c_3 \\
\end{pmatrix}
\begin{pmatrix}
x_0 \\
y_0 \\
1
\end{pmatrix}
=
\begin{pmatrix}
0 \\
0 \\
0
\end{pmatrix}
\]
が成り立ちます。行列を用いてこの式を$A\bm{x}_0 = \bm{0}$と書き直せば、連立$1$次方程式$A\bm{x} = \bm{0}$が非自明な解$\bm{x} = \bm{x}_0$を持つことになります。よって$\det A = 0$です。

逆に$\det A = 0$が成り立っていれば、連立$1$次方程式$A\bm{x} = \bm{0}$は非自明な解$\bm{x} = \bm{x_0}$を$1$つは持ちます。$\bm{x}_0$の$z$成分が$0$でなければ、定数倍で$z$成分が$1$になるように調整したベクトルを$\alpha \bm{x}_0 = (x_0, y_0, 1)$とおくことで、$3$直線が$(x_0, y_0)$で交わることが分かります。もし$\bm{x}_0$の$z$成分が$0$であれば、$\bm{x}_0 = (x_0, y_0, 0)$とおくと
\[
\begin{pmatrix}
a_1 & b_1 & c_1 \\
a_2 & b_2 & c_2 \\
a_3 & b_3 & c_3 \\
\end{pmatrix}
\begin{pmatrix}
x_0 \\
y_0 \\
0
\end{pmatrix}
=
\begin{pmatrix}
a_1 & b_1 \\
a_2 & b_2 \\
a_3 & b_3 \\
\end{pmatrix}
\begin{pmatrix}
x_0 \\
y_0 \\
\end{pmatrix}
=
\begin{pmatrix}
0 \\
0 \\
0
\end{pmatrix}
\]
より、特に
\[
\begin{pmatrix}
a_1 & b_1 \\
a_2 & b_2 \\
\end{pmatrix}
\begin{pmatrix}
x_0 \\
y_0 \\
\end{pmatrix}
=
\begin{pmatrix}
0 \\
0 \\
\end{pmatrix}
\]
が成り立ちます。これは直線$a_1 x + b_1 y + c_1 = 0, a_2 x + b_2 y + c_2 = 0$の法線ベクトルが平行であることに他なりません。しかしそもそも、与えられた$3$直線は平行でもなく一致もしないと仮定されていました。これは矛盾です。したがって連立$1$次方程式$A \bm{x} = \bm{0}$の非自明な解$\bm{x} = \bm{x}_0$の$z$成分は決して$0$になりません。

以上より、$3$直線が$1$点で交わることと行列式が$0$でないことが同値だと言えまし。

\noindent (3) 初等幾何で解けるので、初等幾何で解いてみます\footnote{もちろんベクトルに持ち込んで(2)の結果を使うこともできますが、スマートに計算する方法が思いつかなかったので、初等幾何で解くやり方を紹介することにしました。ベクトルの方でも、上手くやれば綺麗に解けるかもしれません。}。まず、次の左の図を見てください。直線$\mathrm{AP}$と直線$\mathrm{BQ}$の交点を$\mathrm{X}$とします。このとき$\bigtriangleup\mathrm{ACP}$と$\bigtriangleup\mathrm{QCB}$について
\begin{align*}
\begin{cases}
\mathrm{AC} = \mathrm{QC} & \text{($\because\bigtriangleup{ACQ}$は正三角形)} \\
\mathrm{CP} = \mathrm{CB} & \text{($\because\bigtriangleup{ACQ}$は正三角形)} \\
\angle\mathrm{ACP} = \angle\mathrm{QCB} & \text{($\because$どちらも$\angle\mathrm{ACB} + \pi/3$)} \\
\end{cases}
\end{align*}
なので、二辺夾角相等で$\bigtriangleup\mathrm{ACP} \equiv \bigtriangleup\mathrm{QCB}$です。よって円周角の定理の逆から、$\angle\mathrm{PAC} = \angle\mathrm{BQC}$となるので、$4$点$\mathrm{A}$, $\mathrm{Q}$, $\mathrm{C}$, $\mathrm{X}$は同一円周上に乗ると分かります。したがって円周角の定理から$\angle\mathrm{AQX} = \angle\mathrm{ACX}$となります。

\begin{figure}[h!tbp]
\centering
\includegraphics[viewport = 70 100 350 370, width = .45\textwidth, angle = 270, clip]{20151224-fig5.pdf}
\begin{picture}(0, 0)
\put(-32.5, -28){\circle*{3}}
\put(-106, -53){\circle*{3}}
\put(-145, -120){\circle{3}}
\put(-81.5, -198){\circle{3}}
\end{picture} \hfil
\includegraphics[viewport = 70 100 350 370, width = .45\textwidth, angle = 270, clip]{20151224-fig6.pdf}
\begin{picture}(0, 0)
\put(-32.5, -28){\circle*{3}}
\put(-106, -53){\circle*{3}}
\put(-145, -120){\circle{3}}
\put(-81.5, -198){\circle{3}}
\put(-39, -26){\framebox(2,2){}}
\put(-60.5, -96){\framebox(2,2){}}
\end{picture}
\end{figure}

今度は直線$\mathrm{BQ}$と直線$\mathrm{CR}$の交点を$\mathrm{Y}$とします。先ほどと同様に$\bigtriangleup\mathrm{RAC} \equiv \bigtriangleup\mathrm{BAQ}$が示せます。これによって$4$点$\mathrm{A}$, $\mathrm{C}$, $\mathrm{Q}$, $\mathrm{Y}$が同一円周上にあることが分かり、円周角の定理から$\angle\mathrm{AQY} = \angle\mathrm{ACY}$が分かります。

\begin{figure}[h!tbp]
\centering
\includegraphics[viewport = 70 100 350 370, width = .45\textwidth, angle = 270, clip]{20151224-fig7.pdf}
\begin{picture}(0, 0)
\put(-39, -26){\framebox(2,2){}}
\put(-60.5, -96){\framebox(2,2){}}
\put(-185, -47){\line(1, -1){3}}
\put(-185, -50){\line(1, 1){3}}
\put(-144.5, -107){\line(1, -1){3}}
\put(-144.5, -110){\line(1, 1){3}}
\end{picture} \hfil
\includegraphics[viewport = 70 100 350 370, width = .45\textwidth, angle = 270, clip]{20151224-fig8.pdf}
\begin{picture}(0, 0)
\put(-39, -26){\framebox(2,2){}}
\put(-60.5, -96){\framebox(2,2){}}
\put(-185, -47){\line(1, -1){3}}
\put(-185, -50){\line(1, 1){3}}
\put(-144.5, -107){\line(1, -1){3}}
\put(-144.5, -110){\line(1, 1){3}}
\put(-32.5, -28){\circle*{3}}
\put(-106, -53){\circle*{3}}
\end{picture}
\end{figure}

これらをまとめると、結局$\angle\mathrm{ACX} = \angle\mathrm{AQB} = \angle\mathrm{ACY}$と$\angle\mathrm{CAX} = \angle\mathrm{CQB} = \angle\mathrm{CAY}$が分かります。よって$2$角夾辺相等で$\bigtriangleup\mathrm{ACX} = \bigtriangleup\mathrm{ACY}$が従います。そして$\mathrm{X}$と$\mathrm{Y}$は共に$\bigtriangleup{ABC}$の内部の点なので、$\mathrm{X} = \mathrm{Y}$でないといけません。これで示すべきことが言えました\footnote{ちなみにこの点は、Fermat点と呼ばれるものです。}。 \qed

\paragraph{図形的な解釈}

この問題の(1)と(2)は、なんとなく似ています。(1)は「$3$点が$1$直線上に乗る条件」で、(2)は「$3$直線が$1$点を通る条件」を考えています。そしてどちらについても、必要十分条件が行列式で書けています。これは決して偶然ではなく、必然的なものです。その理由を考えてみましょう。

キーポイントは\textbf{平面の問題を空間に持ち上げて考える}ことです。$3$次元空間内で$xy$平面に平行な平面$z = 1$を考えてみます。この平面に$\Pi$と名前を付けます。すると
\begin{itemize}
\item 空間$\mathbb{R}^3$内で原点を通り、方向ベクトルの$z$成分が$0$でない直線
\item 平面$\Pi$上の点
\end{itemize}
が$1:1$に対応します。実際$\Pi$上の点を取れば、原点とその点を通る直線がただ$1$つ存在します。逆に原点を通り方向ベクトルの$z$成分が$0$でない直線は、必ず平面$\Pi$と$1$点で交わります。同様にして
\begin{itemize}
\item 空間$\mathbb{R}^3$内で原点を通る、$xy$平面以外の平面
\item 平面$\Pi$上の直線
\end{itemize}
が$1:1$に対応します。実際、原点を通る$xy$平面以外の平面は必ず$\Pi$と交わり、その交わりが直線になります。逆に平面$\Pi$上に直線があれば、その直線と原点とを含む$\mathbb{R}^3$の平面がただ$1$つ存在します。かくして、元の問題を空間内で考えると、雑に言えば
\begin{itemize}
\item 平面$\Pi$上の点と、空間$\mathbb{R}^3$の原点を通る直線
\item 平面$\Pi$上の直線と、空間$\mathbb{R}^3$の原点を通る平面
\end{itemize}
とが対応することが分かりました\footnote{要は、原点を通る直線や平面を、原点から放射状に出た光源によって平面$\Pi$に射影している感じです。}。

こうすると嬉しいことがあります。$3$次元空間の場合、$\mathbb{R}^3$の原点を通る直線と原点を通る平面とが$1:1$に対応するのです。どう対応付けるのかは後回しにして、ここまでの事実を図式でまとめましょう。
\[
\begin{tikzcd}
\text{平面$\Pi$上の点} \arrow{d}{} & \text{平面$\Pi$上の直線} \arrow{d}{} \\
\text{$\mathbb{R}^3$の原点を通る直線} \arrow{u}{} \arrow{r}{} & \text{$\mathbb{R}^3$の原点を通る平面} \arrow{u}{} \arrow{l}{}
\end{tikzcd}
\]
この図式をぐるっと回ることで、平面$\Pi$上の点と平面$\Pi$上の直線が対応つきます。そうすると、この問題の(1)と(2)の条件が似ていることが自然な気が徐々にしてきます。

さて肝心の「$\mathbb{R}^3$の原点を通る直線と原点を通る平面との対応」ですが、これは難しくありません。
\begin{itemize}
\item 原点を通る直線には、それと垂直で原点を通る平面を
\item 原点を通る平面には、それと垂直で原点を通る直線を
\end{itemize}
対応させるだけです。この$2$つの対応が互いに逆向きなことは、明らかでしょう。$1 + 2 = 3$なので、$3$次元空間の場合では直線の次元と平面の次元を足したらちょうど空間全体の次元になります。このおかげで、直線と平面とが綺麗に対応づきます。

そして、この対応のすごいところは、単に「直線と平面が$1:1$に対応する」だけではありません。いま平面$\Pi$上で、異なる$2$直線が$1$点で交わっていたとします。このとき、今作った対応によって
\begin{itemize}
\item 交わる$2$直線のそれぞれを、対応する$\Pi$の点に
\item $2$直線の交点を、対応する$\Pi$上の直線に
\end{itemize}
それぞれ移すと、きちんと「$2$点が$1$直線上に乗っている図」が出てくるのです。その様子を絵にすると、こうなります。
\begin{figure}[h!tbp]
\centering
\includegraphics[width = .35\textwidth]{20151224-fig3.pdf}
 \hfil
\includegraphics[width = .35\textwidth]{20151224-fig4.pdf}
\end{figure}

登場人物が多くてややこしい図なので、見方を説明します\footnote{\url{https://github.com/HideakiHosaka/2015_linear_algebra} に \texttt{duality\_between\_lines\_and\_planes.nb} という名前で、Mathematicaで	作った動くおもちゃを載せました。$2$枚の平面を色々な方向に動かしながら絵を見ると、何をやっているか分かりやすいはずです。大学のECCS端末にあるMathematicaで開いてみてください。}。まずは左の図からです。
\begin{itemize}
\item うっすらと半透明で描かれた平面が、$\Pi$です。
\item 左にある緑っぽい色\footnote{印刷はモノクロで行っていますが、元のPDFではカラーの図面になっています。「$3$枚の平面を左の$2$枚と右の$1$枚に分けて考える」ということさえ分かっていれば、原理的には分かるはず。ですが、できれば原本を見てください。}の$2$平面と平面$\Pi$との交わりにできる黄色い$2$直線が、今考えたいターゲットです。
\item 右側にある青い平面は、原点を通り、緑色の$2$平面の交線と垂直です。この青い平面の上に描かれた黒い$2$直線は、緑色の平面の法線です。
\end{itemize}
これを踏まえて、先ほどの対応を追いかけましょう。
\begin{itemize}
\item ターゲットの$2$直線の交点は、緑色の$2$平面の交線、それと垂直な青い平面を経由して、青い平面と半透明の平面$z = 1$の交わりにできる黄色い直線に対応する
\item ターゲットの$2$直線は、それぞれ緑色の平面およびその法線である黒い直線を経由して、黒い直線と平面$z = 1$の交点に対応する
\end{itemize}
というわけです。この対応を見れば、交わる$2$直線が$2$点を通る直線へとうつることが分かります。その理由も「緑色の$2$平面の交線と垂直な青い平面は、$2$平面それ自体とも垂直だから」だと分かります。また、右の図は平面$z = 1$の透け具合を減らして強調したものです。この白い平面の左側で交わる黄色い$2$直線が、青い平面上の$2$直線と白い平面の交点にそれぞれ移ります。

結局この対応を通せば、平面$\Pi$上で交わる$3$直線は、同一直線上にある$3$点に移ることが分かります。だから問題の(1)と(2)が現れるのは、必然だと言えます。問題(1)と(2)のどちらかは真面目に解かないといけません。ですが(1)が解けていれば「(2)の条件は、今の対応で移し替えた後の$3$点が同一直線上に乗るための条件と同じ」という理由で、(1)に帰着できるのです。

\paragraph{双対原理と射影空間}

ちなみに、この問題のように「点と直線を入れ替えても正しい結果が得られる」という事実を\textbf{双対原理}といいます。S2タームの最終回、p.~\pageref{section:dual_space}で双対空間の話を扱いました。双対空間を使うと、今の双対原理は「$\mathbb{R}^3$の原点を通る直線と、$(\mathbb{R}^3)^*$の原点を通る平面が$1:1$に対応する」と述べることができます。

さらに元々の問題では、原点を通る直線や平面を別の平面$\Pi$に射影していました。ですが射影をすると実はちょっとだけ不具合があります。$\Pi$上の点と直線は大体の場合は$1:1$に対応するものの、たまに「直線に対応する点が無限遠に飛んでいく」という不具合が生じるのです。この不具合を回避するには、射影をしないで物事を考える必要があります。このときに登場する「$\mathbb{R}^3$の原点を通る直線全体が作る空間」のことを、\textbf{射影空間}と呼びます。射影空間を用いることで、例外を一切なくした形で双対原理を述べることができるようになります。

こうした双対原理と射影空間の話は、西山亨『数学のかんどころ 19 射影幾何学の考え方』(共立出版) の6章を通じて丁寧に説明されています。初等幾何に限らず「双対」というアイデアは色々な場面で登場するものです。時間があったらぜひ読んでみてください。

\subsection{問題3: 行列の対角化}

行列を対角化する計算問題です。幸いなことに固有多項式の根が全てばらけるので、難しい処理は必要ありません。固有値と対応する固有ベクトルを求めるだけです。

\noindent (1) 行列
\[
A :=
\begin{pmatrix}
-3 & 2 & 1 \\
0 & 2 & 2 \\
0 & -1 & -1
\end{pmatrix}
\]
の固有多項式は
\[
\varphi_A(t) = \det(tI - A) =
\det
\begin{pmatrix}
t + 3 & -2 & -1 \\
0 & t - 2 & -2 \\
0 & 1 & t + 1
\end{pmatrix}
=
(t + 3)
\det
\begin{pmatrix}
t - 2 & -2 \\
1 & t + 1
\end{pmatrix}
= t(t + 3)(t - 1)
\]
と求まります。よって固有値は$-3, 0, 1$と分かります。$A$の形を見れば、固有値$-3$に属する固有ベクトルとして$\bm{v}_{-3} := {}^t(1, 0, 0)$が取れます。また固有値$0, 1$に属する固有ベクトルは、連立$1$次方程式$A\bm{x} = \lambda\bm{x}$ ($\lambda = 0, 1$)を解いて
\[
\bm{v}_0 =
\begin{pmatrix}
1 \\
3 \\
-3
\end{pmatrix}, 
\bm{v}_1 = 
\begin{pmatrix}
3 \\
8 \\
-4
\end{pmatrix}
\]
と求まります。かくして
\[
P:= 
\begin{pmatrix}
1 & 1 & 3 \\
0 & 3 & 8 \\
0 & -3 & -4
\end{pmatrix}
\]
とおけば、$P^{-1} A P = \diag(-3, 0, 1)$と求まります。

\noindent (2) 回転行列
\[
R(\theta) :=
\begin{pmatrix}
\cos \theta & -\sin \theta \\
\sin \theta & \cos \theta
\end{pmatrix}
\]
の固有多項式は$\varphi_{R(\theta)}(t) = t^2 - 2\cos \theta + 1$です。解の公式で解くと、根は
\[
t = \cos \theta \pm \sqrt{\cos^2\theta - 1} = \cos \theta \pm i \sin \theta
\]
だと分かります\footnote{本当は$\theta$の値に応じて$\sqrt{\cos^2\theta - 1}$が$\sin \theta$になるのか$-\sin\theta$のどっちになるのか変わります。ですが根号を外すときに$\pm$の符号がどっちになっても、「$2$つの根をまとめて$t = \cos \theta \pm i \sin \theta$と書ける」という事実に変わりはありません。なので少し横着した書き方をしました。}。これで固有値が求まりました。また対応する固有ベクトルは、連立$1$次方程式$R(\theta)\bm{x} = \lambda \bm{x}$ ($\lambda = \cos \theta \pm i \sin \theta$) を解いて
\[
\begin{pmatrix}
\cos \theta & -\sin \theta \\
\sin \theta & \cos \theta
\end{pmatrix}
\begin{pmatrix}
i \\
1
\end{pmatrix}
=
(\cos\theta + i \sin \theta)
\begin{pmatrix}
i \\
1
\end{pmatrix}, \quad
\begin{pmatrix}
\cos \theta & -\sin \theta \\
\sin \theta & \cos \theta
\end{pmatrix}
\begin{pmatrix}
-i \\
1
\end{pmatrix}
=
(\cos\theta - i \sin \theta)
\begin{pmatrix}
-i \\
1
\end{pmatrix}
\]
と求まります\footnote{固有多項式が重根、つまり$\theta = 0, \pi$の場合もこの式は正しいです。}。そこで
\[
P :=
\begin{pmatrix}
i & -i \\
1 & 1
\end{pmatrix}
\]
とおけば、$P^{-1} R(\theta) P = \diag(\cos\theta + i\sin \theta, \cos\theta - i \sin\theta)$が得られます。 \qed

\subsection{問題4: 実対称行列の対角化}
実対称行列
\[
A =
\begin{pmatrix}
0 & 0 & 1 \\
0 & -1 & 0 \\
1 & 0 & 0 
\end{pmatrix}
\]
を対角化する問題です。実対称行列であることから、直交行列によって対角化できることが分かります。

まず固有多項式は、$\det$を$2$行目について余因子展開すると
\[
\varphi_A(t) = \det(tI - A)
\det
\begin{pmatrix}
t & 0 & -1 \\
0 & t + 1 & 0 \\
-1 & 0 & t 
\end{pmatrix}
=
(t + 1)
\det
\begin{pmatrix}
t & -1 \\
-1 & t 
\end{pmatrix}
= (t + 1)(t^2 - 1)
= (t - 1)(t + 1)^2
\]
となります。よって$A$の固有値は$1, -1, -1$です。

次に固有ベクトルを求めましょう。$(\bm{e}_1, \bm{e}_2, \bm{e}_3)$を$\mathbb{R}^3$の標準基底とします。固有値$-1$に属する$A$の固有ベクトルとして、最初から$\bm{e}_2 = {}^t(0, 1, 0)$が見えています。すると残り$2$本の固有ベクトルは$\bm{e}_2$と直交するよう取れるはずなので、第$2$成分を持ちません。このことに注意すると、固有値$1$に属する固有ベクトルとして$\bm{v}_1 = {}^t(1, 0, 1)$が見つかります。最後に、外積で$\bm{e}_2, \bm{v}_1$の両方に直交するベクトルを作ると
\[
\bm{e}_2 \times \bm{v}_1 = \bm{e}_2 \times (\bm{e}_1 + \bm{e}_3) = \bm{e}_2 \times \bm{e}_1 + \bm{e}_2 \times \bm{e}_3
= -\bm{e}_3 + \bm{e}_1 = 
\begin{pmatrix}
1 \\
0 \\
-1
\end{pmatrix}
\]
となります。計算すると、予定通りこれが$A$の固有値$-1$に属するもう$1$本の固有ベクトルになっています。あとは、固有ベクトルの長さが$1$になるよう調整してから並べて
\[
P := 
\begin{pmatrix}
\frac{1}{\sqrt{2}} & 0 & \frac{1}{\sqrt{2}} \\
0 & 1 & 0 \\
\frac{1}{\sqrt{2}} & 0 & -\frac{1}{\sqrt{2}}
\end{pmatrix}
\]
とおけば、$P^{-1} A P = \diag(1, -1, -1)$となり、対角化が完了します。 \qed

\subsection{問題5: 平面上の$2$次曲線}

平面上で方程式$3x^2 - 2xy + 3y^2 = 4$が表す方程式を求める問題です。$xy$の項があるため、これは標準形の$2$次曲線を回転したものになっています。うまく回転させて、どういう形なのかをはっきりさせましょう。

\noindent (1) 実対称行列$A$を
\[
A =
\begin{pmatrix}
3 & -1 \\
-1 & 3
\end{pmatrix}
\]
で定めると、${}^t\bm{x} A \bm{x} = 3x^2 - 2xy + 3y^2$となります。よって方程式が${}^t\bm{x} A \bm{x} = 4$と書き換えられます。

\noindent (2) $A$の固有多項式は$\varphi_A(t) = t^2 - 6t + 8 = (t - 2)(t - 4)$なので、$A$の固有値は$2, 4$です。それぞれに対応する固有ベクトルは、連立$1$次方程式$A\bm{x} = \lambda\bm{x}$ ($\lambda = 2, 4$)を解いて
\[
\bm{v}_2 = 
\begin{pmatrix}
1 \\
1
\end{pmatrix}, 
\bm{v}_4 = 
\begin{pmatrix}
-1 \\
1
\end{pmatrix}
\]
と求まります。

\noindent (3) いま求めた固有ベクトル$\bm{v}_2, \bm{v}_4$は直交します。そこで長さを$1$に揃え、かつ$\bm{v}_2$から$\bm{v}_4$への向きが正の向きになるよう調整すると
\[
P := 
\frac{1}{\sqrt{2}}
\begin{pmatrix}
1 & -1 \\
1 & 1
\end{pmatrix}
\]
とおけば良いことが分かります。この$P$は$\pi/4$回転を表す回転行列で、$P^{-1} A P = \diag(2, 4)$を満たします。

\noindent (4) 元の方程式の左辺を、${}^tP = P^{-1}$を使って${}^t\bm{x} A \bm{x} = {}^t \bm{x} P P^{-1} A P P^{-1} \bm{x} = {}^t({}^tP \bm{x}) P^{-1} A P ({}^tP \bm{x})$と変形します。${}^tP \bm{x} = {}^t(u, v)$とおくと、$P^{-1} A P = \diag(2, 4)$より
\[
4 = {}^t\bm{x} A \bm{x} = 
\begin{pmatrix}
u & v
\end{pmatrix}
\begin{pmatrix}
2 & 0 \\
0 & 4
\end{pmatrix}
\begin{pmatrix}
u \\
v
\end{pmatrix}
= 2u^2 + 4v^2
\]
となります。よって$uv$座標系では、この曲線は$(u/\sqrt{2})^2 + v^2 = 1$と書けるので、$u$軸方向の長半径が$\sqrt{2}$, $v$軸方向の短半径が$1$の楕円だと分かります。

あとは$uv$座標系から$xy$座標系に変換すれば終わりです。$x, y$座標系を$u, v$座標系に変換するのに${}^tP = P^{-1}$を使ったので、$xy$座標系の基底を$uv$座標系の基底に変換する行列は$P$です。そして$P$が$\pi/4$回転を表す行列なので、$u$軸は${}^t(1, 1)$方向を、$v$軸は${}^t(-1, 1)$方向を向いていると分かります。これで曲線の概形が決まります。

\begin{figure}[h!tbp]
\centering
\begin{picture}(0,0)
\put(0, 86.1){\vector(1, 0){180}}
\put(88.8, 0){\vector(0, 1){175}}
\put(182, 83){$u$}
\put(92, 170){$v$}
\put(79, 78){$O$}
\put(147, 89){$\sqrt{2}$}
\put(-13, 90){$-\sqrt{2}$}
\put(90, 144){$1$}
\put(90, 22){$-1$}
\end{picture}
\includegraphics[width = .35\textwidth]{20151224-fig1.pdf}
 \hfil
\begin{picture}(0,0)
\put(0, 86.1){\vector(1, 0){180}}
\put(88.8, 0){\vector(0, 1){175}}
\put(88.8, 86.1){\vector(-1, 1){80}}
\put(88.8, 86.1){\line(1, -1){80}}
\put(88.8, 86.1){\vector(1, 1){80}}
\put(88.8, 86.1){\line(-1, -1){80}}
\put(88.8, 86.1){\dashbox(54.7, 54.7){}}
\put(73, 78){$O$}
\put(182, 83){$x$}
\put(92, 170){$y$}
\put(170, 166){$u$}
\put(3, 167){$v$}
\put(137, 88){$1$}
\put(90, 133){$1$}
\end{picture}
\includegraphics[width = .35\textwidth]{20151224-fig2.pdf}
\end{figure}

なお、回転の向きを逆向きにしていないかどうか心配になるかもしれないので、チェックしておきましょう。上の議論で$u$軸上の点$(\sqrt{2}, 0)$は曲線上に乗っています。もし回転のさせ方が正しければ、この点は$xy$座標系で$(1, 1)$という点に対応するはずです。そして元の方程式$3x^2 - 2xy + 3y^2 = 4$に$x = y = 1$を代入すると式が成り立ちます。これで、回転のさせ方が正しいと分かります。 \qed

\section{$1$年間のまとめ}

\subsection{これまでにやったこと}

この$1$年間で何をやったのか、簡単にまとめておきましょう。

\paragraph{行列の演算}

数を縦に$m$個、横に$n$個並べたものを$(m, n)$型の行列といい
\begin{itemize}
\item $(m, n)$型の行列同士の足し算と引き算
\item $(m, n)$型行列と$(n, l)$型行列の掛け算
\end{itemize}
を定義しました。掛け算の規則は一見すると変わっていますが、後で線型写像のことを学んだ暁には「行列を線型写像と思った時の合成」がぴったり掛け算と対応すると分かるのでした。また行列では、$O$でない行列の積が$O$になることがあったり、「逆数」に当たる逆行列が必ずしも存在しないことを確認しました。

\paragraph{行列を用いた連立$1$次方程式の解法}

連立$1$次方程式は行列の積を用いて表すことができます。このとき消去法で連立$1$次方程式を解くことは、係数行列に対して基本変形を行うことと等価です。そして行列の階数を定義すると、係数行列と拡大係数行列の階数比較によって解の存在判定ができるようになります。

\paragraph{線型空間と線型写像}

行列の理論を展開するにあたり、必要なのは「数ベクトル空間$\mathbb{R}^n$の上で、足し算とスカラー倍が定義されている」ということだけでした。そこでこの性質だけを抜き出して抽象化し、線型空間を定義しました。その後「線型空間の構造と相性のいい写像」を考えることで線型写像を定義し、これが数ベクトル空間の場合に行列と一致することを確かめました。

また線型空間の基底を定義すると、どんな基底も同じ本数のベクトルからなることが証明できます。これが線型空間の次元というものです。

\paragraph{行列式の計算}

行列式とは$n$次正方行列に対して定義される数です。$n$次正方行列を$n$本の数ベクトルと思ったとき、行列式は「$n$本のベクトルが張る平行$2n$面体の体積」と一致するものです。そして$n$本のベクトルが$1$次従属なことと、それらが張る平行$2n$面体の体積が$0$なことは同値です。したがって行列式が$0$かどうかで、$n$本のベクトルの$1$次独立性が判定できます。

また行列式には余因子展開を繰り返して計算する方法をはじめ、その他色々な計算公式があります。それらを証明しました。

\paragraph{固有値と対角化問題}

$n$次正方行列を$\mathbb{R}^n$のベクトルに当てると、	同じ空間の中で別のベクトルへと変化します。このとき普通はベクトルの向きが変わりますが、運が良いと向きは変わらず、長さだけが変化することがあります。こういうベクトルを固有ベクトルといいます。そして固有ベクトルからなる基底を作ることができると、新しい基底に関して元の行列が対角行列で表示されます。この操作を行列の対角化といいます。

行列の固有値が全部バラバラであれば、行列は対角化可能になります。しかし固有値に重複が出てくると、対角化ができない場合があります。この問題の本質的な理由は「固有値が全て$0$である巾零行列は、必ずしも行列のサイズと同じ本数の固有ベクトルを持つとは限らない」という事実です。そのような場合は次善の策として、対角線の上にいくつか$1$が並んだJordan標準形という形にまで変形ができます。

\paragraph{Hermite行列のユニタリ対角化}

一般に行列は対角化可能とは限らないわけですが、実対称行列やHermite行列の場合、必ず対角化可能という事実があります。この理由は一言で言えば「標準内積との相性が良い」ということに尽きます。しかも対角化のときに挟む行列として、実対称行列が相手なら直交行列が、Hermite行列が相手ならユニタリ行列が取れるのでした。対角化するだけなら別に直交行列やユニタリ行列を使わなくてもいいですが、実対称行列やHermite行列の対角化可能性を示す際にはGram--Schmidtの正規直交化法が必要で、その証明を見ると「直交行列やユニタリ行列が持ってこれる」ということが分かります。

\subsection{参考書案内}

ここまでまとめた内容を復習するのに、また一歩踏み込んだ内容を勉強するために、いくつか線型代数の本を紹介します。

\begin{itemize}
\item 佐武 一郎『線型代数学』(裳華房)
\item 足助 太郎『線型代数学』(東京大学出版会)
\item 長谷川 浩司『線型代数』(日本評論社)
\item 竹山 美宏『線形代数 行列と数ベクトル空間』(日本評論社)
\item 斎藤 毅『線形代数の世界』(東京大学出版会)
\item 室田 一雄・杉原 正顯『線形代数I』(丸善)
\end{itemize}

TAの個人的な好みでは、佐武一郎先生の『線型代数学』(裳華房) を一押しします。最近新装版が出たものの、初版の出版年は1974年と、中々古い本です。線型代数の基本的なことが一通り書かれているのはもちろんですが、各章の終わりに「研究課題」という節がついており、発展的なテーマが扱われているのが嬉しいところです。

斎藤毅『線形代数の世界』の著者の斎藤先生は、今も現役の数学者で、東大駒場の数理科学研究科にいらっしゃいます。この本は他の本と違い「大学$1$年生で線型代数を習った人が、より高度な線型代数の理論を学ぶための本」です。「高度な線型代数」をうたった本の大半は、線型代数よりさらに話を広げた「環上の加群の理論」というものを扱うのですが、この本は珍しく線型代数の範囲で収まっています。

最後に挙げた室田 一雄・杉原 正顯『線形代数I』は、東京大学工学教程というシリーズのうちの$1$冊です。工学部の先生方が執筆されただけあって、書き方はかなり工学寄りです。諸々の定理が成り立つ原理に関する記述が深くない一方、工学の中で線型代数がどう使われるかについては、豊富に例が扱われています。工学での例を知るには非常に良い$1$冊でしょう。他の本と組み合わせて勉強することをお勧めします。

\subsection{この後に続く数学}

最後に、$1$年生の授業に続く数学をちょっとだけ紹介します。「もう数学勉強するの嫌だ」という気持ちの人もいるかもしれませんが、少しだけ聞いてください。

実は数学を専門としない限り、この先の人生で必要になる数学は大体$1$年生の勉強で事足ります。希望する専門分野があるなら、その分野の大学院の入試問題を見てみると良いでしょう。多分、一般教養科目として「数学」がありますが、そこに並んでいる問題の多くは微積分か線型代数です。実際に必要になる計算テクニックの多くは、既に皆さんが勉強した内容でカバーできるのです。

もちろん理論物理や電気工学など、ジャンルによっては高度な数学が必要になることもあるでしょう。そこで以下、大事と思われる順に、この先に出てくる数学を紹介したいと思います。

\paragraph{微分方程式}

これはほとんど全ての人にとって必修です。自然科学ならまず間違いなく、「時間が経つにつれて連続的に変化する量」を分析する必要性に迫られます。たとえば地球と太陽の距離とか、化学反応の進行する速度とか、生体内で代謝される特定のホルモン量とか。そして、こうした連続量が従う法則は、ほぼ確実に微分方程式で記述されます。こういう事情で、微分方程式が必要になるのです。

微分方程式にも色々難しい理論がありますが、高度なことを知る必要はないでしょう。現実問題として、積分で解ける微分方程式はごくわずかしかなく、解けない大多数の方程式は計算機によって近似的に処理されます。ですから変数分離形など、いくつか有名な方程式を解けるようになっていけば当面事足ります。$2$年生になったら微分方程式の授業があるはずですので、頻出問題の解き方だけはぜひ学んでください。

\paragraph{ベクトル解析}

ベクトル解析は、平面や空間内のベクトル値函数を対象にした微積分の理論です。ベクトル値函数は単なる函数という以上に、空間の各点にベクトルが張り付いた「場」という意味を持ちます。たとえば物理学では、万有引力の場や電場・磁場といったものが典型的な例です。

\paragraph{複素函数論}

$1$年生で習う微積分では実変数の函数を扱いましたが、複素函数論やそれに類する名前の授業では、$1$変数の複素数変数の函数を扱います。複素変数$1$個は実変数$2$個と等価なので、ちょっと考えると「$2$変数のベクトル値函数と何が違うんだろう」と思うかもしれません。ですが「複素函数の微分」を上手く定義すると「微分可能」という条件が程よく強くなり、微分可能な函数が非常に良い性質を示すようになるのです。こういう函数を\textbf{正則函数}といいます。

そして複素函数に対しても、積分をすることができます。特に正則函数や、あるいはいくつか発散する点を許した有理型函数と呼ばれる函数は\textbf{留数定理}と呼ばれる定理を満たします。実はこの留数定理が、普通の実$1$変数函数の積分を計算するのに役立つことがあります。おそらく、今ある汎用的な積分公式の中で最後に学ぶのが留数定理でしょう。

\paragraph{群論}

図形や式の持つ「対称性」を抽出して得られる概念が、群と呼ばれるものです。線型代数の授業の中では、$n$次対称群$\mathfrak{S}_n$や、正則行列のなす群$GL_n(\mathbb{R})$などが登場しました。こういう群の性質をもっと深く探るのが群論です。

大学入試の問題とかでも、数学の問題を解くときに「対称性」を使ったことは何回かあるのではないかと思います。群論を使うという行為は、ある意味で「対称性」を使った問題の解き方を一段と推し進めることにあたります。ですから非常に強力です。たとえば量子力学では、水素原子の波動関数を求めるのに回転対称性をフル活用します。水素原子に限らずもっと大きな分子でも、対称性の解析で軌道エネルギーの縮退度を決定できたりします。またX線回折による結晶解析では、どういう対称性がどういうX線反射パターンを生み出すかという情報を用います。この「結晶の持つ対称性」の分析は、空間群と呼ばれる群を調べることに他なりません。

\paragraph{Fourier解析}

電気回路や音響など「波」を扱う上で必須の理論です。このプリント中でも一度だけ、直線$y = x$を三角函数の和で表す式をp.~\pageref{paragraph:Fourier_series}で紹介しました。この話をもっと深く追求し、あらゆる周期関数を簡単な三角函数の無限級数で表すのが、Fourier級数です。

