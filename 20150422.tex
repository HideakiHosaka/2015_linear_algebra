\chapter{種々の函数}

\begin{flushright}
担当教員: 植野 義明 / TA: 穂坂 秀昭 \\
講義日時: 2015年4月22日 1限
\end{flushright}

\section{逆函数について}

\subsection{逆函数の定義と例}

函数とは、数に対して何かしらの数を対応させる規則のことです。たとえば$f(x)=x^2$という函数は、実数$x\in\mathbb{R}$に$x^2\in\mathbb{R}$という数を対応させています。他にも三角函数、対数函数や指数函数といった有名な函数があり、そしてこれらの函数を合成するとさらに色々な函数を作ることができますが、どの函数も\textbf{数に数を対応させる}ことには間違いありません。別の言い方をすれば、僕たちが普段「函数」と呼んでいるものは、定義域と値域が実数全体$\mathbb{R}$の部分集合であるような写像です\footnote{ただし、この用語の使い方は慣用的なもので、厳密な取り決めがあるわけではありません。たとえば定義域や値域がベクトルの集合であっても、函数と呼ぶことはままあります。}。

さて函数$f(x)$が与えられると、数$x$が与えられた時に$f$の値$f(x)$が定まります。そしてしばしば「数$y$が与えられたとき、$y=f(x)$となる数$x$を求めたい」という状況が生じます。対応$x\mapsto f(x)$ではなく、その逆の対応を求めたいわけです。このように函数$f$が与えられたとき、「数$y$に対し、$y=f(x)$となる$x$を対応させる」という方法で定義される函数を$f$の\textbf{逆函数}と言い、$f^{-1}$と書きます。

\subsection{逆函数の主値}

\paragraph{逆函数の定義に必要な条件}

ただし、いつでも逆函数が無条件で定義できるわけではありません。上で述べたように、函数は写像ですから「$1$つの数に、$1$つの数を対応させる」ことが大原則です。だから「$1$つの数に$2$つの数を対応させる」という操作になってしまっては、函数とは呼べないのです。いま、函数$f$が相異なる実数$x_1$, $x_2$に対して$f(x_1)=f(x_2)$を満たしたとしましょう。そして$y=f(x_1)=f(x_2)$とおきます。すると$x_1$, $x_2$のどちらに対しても$y$が対応するので、逆函数を作ろうとすると「$y$に$x_1$, $x_2$の両方が対応する」というマズい状況が実際に起きます。こうならないためには「異なる実数$x_1$, $x_2$に対して必ず$f(x_1)\neq f(x_2)$が成り立つ」という条件が必要です。こういう性質を$\textbf{単射性}$というのでした。

また単射なら良いかというと、そうでもありません。たとえば指数函数$\exp x:=e^x$を考えましょう。指数函数を$\exp\colon \mathbb{R}\rightarrow\mathbb{R}$という写像と思えば、これは単射です。グラフを描けば狭義短調増加ですから、異なる値に対して同じ値が対応しようがありません。そして正の実数$y>0$に対しては、$e^x = y$となる$x$が$x=\log y$で与えられます。ですが$y\leq 0$のとき、$y=\exp x$となる$x\in\mathbb{R}$は存在しません。さっきは「$1$つの数に$2$個(以上)の数が対応してしまう」ことが問題でしたが、今度は「ある数に対して、対応させられる数がなくなってしまった」という問題が起きています。これももちろん写像の定義を満たさないので、指数函数$\exp$を$\mathbb{R}$から$\mathbb{R}$への写像と思うと、逆函数は作れません。ですが$\exp$の値域を正の整数全体の集合$\mathbb{R}_{>0}:=\{x\in\mathbb{R} \mid x>0 \}$だと思えば、どんな正の数$y>0$に対しても$x:=\log y$とおくことで$y=e^x$なる実数$x$を見つけられます。このような、函数$f$の「どんな値域の元$y$に対しても、$y=f(x)$となる$x$が存在する」という性質を、\textbf{全射性}と呼ぶのでした。

まとめると、$f$の逆函数が存在するための必要条件は、$f$が全単射であることです。また$f$の逆函数$g$が存在していれば、$f$は全単射になることが示せます。かくして$f$の逆函数が存在することと、$f$が全単射であることが同値だと分かります。

\paragraph{逆函数の主値}

ただし$f$が全単射でないときも、部分的には$f$の逆函数を作れることが多いです。さっきの指数函数の例では、値域を$\mathbb{R}$から$\mathbb{R}_{>0}$に取り換えれば逆函数が作れました。また$f$の定義域を制限してしまうのも一つの手です。たとえば$f(x)=x^2$という函数を考えましょう。正の実数$x>0$に対し$f(-x)=(-x)^2=x^2=f(x)$という式が成り立ってしまうので、$f$は単射ではありません。ですが$f$の定義域を非負実数全体の集合$\mathbb{R}_{\geq 0}:=\{x\in\mathbb{R}\mid x\geq 0\}$に制限してしまえば、$f$は単射になります。そしてどんな非負の実数$y\in\mathbb{R}_{\geq 0}$に対しても、$x:=\sqrt{y}$とおけば$y=x^2=f(x)$が成り立ちます。こうして$f$を制限して$\mathbb{R}_{\geq 0}\rightarrow\mathbb{R}_{\geq 0}$という写像にすれば、$f$の逆函数$y\mapsto \sqrt{y}$が得られます。$f$の定義域を制限して逆写像$f^{-1}$を作るとき、その値を$f^{-1}$の\textbf{主値}と言います。

\subsection{逆函数の微分}

\section{逆三角函数と円周率の近似公式}

\subsection{逆三角函数の定義}

三角函数の逆函数を逆三角函数と言います。そして$\cos$, $\sin$, $\tan$の逆函数は、それぞれ$\arccos$, $\arcsin$, $\arctan$と呼ばれます。ただし三角函数はすべて周期函数なので、ある意味「単射から最も遠い函数」です。たとえば$\sin x = 1/2$となる実数$x$は無限個存在してしまいます。ですから逆三角函数を考えるには、三角函数の定義域を制限して主値を選ぶ必要があります。通常、主値は次のように選びます。
\begin{table}[h!tbp]
\begin{center}
\begin{tabular}{ccc} \hline
函数 & 定義域 & 主値 \\ \hline
$\arcsin$ & $[-1,1]$ & $[-\pi/2,\pi/2]$ \\
$\arccos$ & $[-1,1]$ & $[0,\pi]$ \\
$\arctan$ & $\mathbb{R}$ & $(-\pi/2,\pi/2)$ \\ \hline
\end{tabular}
\end{center}
\end{table}

\subsection{$\arctan$の公式}

三角函数に色々な公式があるように、逆三角函数にも色々な公式があります。その中でも$\arctan$の計算公式は比較的有名です。というのも後で見るように、$\arctan$の計算ができると、円周率を求めることができるからです。$\pi/4 = \arctan 1$を、色々な方法で表してみましょう。

\paragraph{問2, 3} $\tan$の加法定理
\[
\tan(x+y) = \frac{\tan x + \tan y}{1 - \tan x\tan y}
\]
に$u=\tan x$, $v=\tan y$を代入すると
\begin{align*}
\arctan u + \arctan v
&= x + y = \arctan \tan(x+y)
= \arctan \frac{\tan x + \tan y}{1 - \tan x\tan y}
= \arctan \Bigl(\frac{u+v}{1-uv}\Bigr)
\end{align*}
を得る。この式で$u=1/2$, $v=1/3$とおくと
\[
\arctan \frac{1}{2} + \arctan \frac{1}{3} = \arctan \Bigl(\frac{u+v}{1-uv}\Bigr) = \arctan \Bigl(\frac{1/2+1/3}{1-1/6}\Bigr) = \arctan 1 = \frac{\pi}{4}
\]
が得られる。これはEulerの公式と呼ばれる。

また$u=v=1/5$とおくと、同じように計算することで
\[
2\arctan\frac{1}{5} = \arctan\frac{2/5}{1-(1/25)} = \arctan\frac{5}{12}
\]
となる。さらに$u=v=5/12$に対しては
\[
2\arctan\frac{5}{12} = \arctan\frac{10/12}{1-(25/144)} = \arctan\frac{120}{119}
\]
である。そして$u=120/119$に対し$(u+v)/(1-uv)=1$となる$v$を求めると、$v=(1-u)/(1+u)=-1/239$となる。これらをまとめると
\begin{align*}
\frac{\pi}{4}
&= \arctan \frac{\frac{120}{119}+\frac{-1}{239}}{1-\frac{120}{119}\frac{-1}{239}}
= \arctan \frac{120}{119} + \arctan\frac{-1}{239}
= 2\arctan\frac{5}{12} - \arctan\frac{1}{239}
= 4\arctan\frac{1}{5} - \arctan\frac{1}{239}
\end{align*}

\paragraph{問4}
\begin{enumerate}
\item[(1)] $(3+i)(7+i) = 20+10i$
\item[(2)] $(2+i)(3+i) = 5+5i$
\item[(3)] $(5+i)^4/(239+i) = 2+2i$
\end{enumerate}
$z=x+yi$の偏角が$\arctan (y/x)$で与えられることに注意せよ。これらの両辺の偏角を取れば、次の式が得られる。
\[
\arctan\frac{1}{3} + \arctan\frac{1}{7} = \arctan\frac{1}{2},
\arctan\frac{1}{2} + \arctan\frac{1}{3} = \arctan{1} = \frac{\pi}{4},
4\arctan\frac{1}{5} - \arctan\frac{1}{239} = \arctan{1} = \frac{\pi}{4}
\]

\paragraph{問5} 45度

\subsection{円周率の計算}

$\arctan$には次のような公式があります:
\[
\arctan x = x - \frac{x^3}{3} + \frac{x^5}{5} - \frac{x^7}{7} + \cdots
\]
この公式は、形式的には
\[
\arctan x = \int_0^x (\arctan t)' dt = \int_0^x \frac{1}{1+t^2} dt = \int_0^x (1-t^2+t^4-t^6+\cdots) dt
= x - \frac{x^3}{3} + \frac{x^5}{5} - \frac{x^7}{7} + \cdots
\]
のようにして求められます\footnote{この証明は全然厳密ではありません。まず、何も考えずに無限和と積分の順序を入れ替えてはいけません。また$1/(1+t^2)$の展開ができるのは$t^2<1$の範囲だけです。}。右辺は頑張ればいくらでも精度よく計算できますから、この公式をガリガリ計算することで$\arctan x$の値を求めることができます。たとえば$x=1$とすれば
\[
\frac{\pi}{4} = \arctan 1 = 1 - \frac{1}{3} + \frac{1}{5} - \frac{1}{7} + \cdots
\]
という風にして円周率が求められます。これはLeibnizの公式と呼ばれているようです。

ただし、今の$\pi/4$公式は非常に効率が悪いです。というのも第$50$番目の項が$-1/99 = 0.010101\cdots$ですから、$50$番目の項を計算すると小数第$2$位の値が変わります。円周率を求める上では「ちょこっと計算しただけで、上の方の桁が正確に分かる」ような公式が望ましいわけで、こんな「$50$項も計算してもまだ小数第$2$位が分からない」なんて公式は役に立ちません。そこで登場するのがEulerの公式やMachinの公式です。たとえばEulerの公式なら
\begin{align*}
\frac{\pi}{4}
&= \arctan \frac{1}{2} + \arctan \frac{1}{3}
= \Bigl(\frac{1}{2} - \frac{1}{3\cdot 2^3} + \frac{1}{5\cdot 2^5} - \cdots\Bigr)
+ \Bigl(\frac{1}{3} - \frac{1}{3\cdot 3^3} + \frac{1}{5\cdot 3^5} - \cdots\Bigr) \\
&= \Bigl(\frac{1}{2} - \frac{1}{24} + \frac{1}{160} - \cdots\Bigr)
+ \Bigl(\frac{1}{3} - \frac{1}{81} + \frac{1}{1215} - \cdots\Bigr)
\end{align*}
のように、分母にある$x^n$の形の項が効いてきて、足し引きされる項の値が急激に小さくなっていきます。今の場合、最初から$3$つずつの項を拾うだけで$\pi=3.14\cdots$が求まります。さっきより断然楽ですね。Machinの公式に至っては$1/239$という項がありますから、少ない手間でもっと精度よい値を求められます。William Jonesという数学者が1706年に著した``Synopsis Palmariorum Mathesos''という本\footnote{ちなみに本郷キャンパスにある総合図書館の書庫内に、この本があるそうです。OPACで検索すると出てきます。}に、Machinが求めたとされる円周率が100桁以上載っていますが、今の公式を使ったのでしょうか?

もちろん$x$に放り込む数が小さくなれば小さくなるほど、計算はどんどん楽になります。したがって「$\pi$が上手く出てくる$\arctan$の公式を見つけて、君だけのオリジナルの円周率近似公式をつくろう!」…という話になりそうですが、実はこの公式は頑張ってもそんなに精度が出ません。いまの公式では分母に$x^n$が出てくることがキモだったので「$1$つ先の項を計算したときに、何桁分だけ精度が良くなるか」は毎回同じです。もし$x$が$100$だったら、次の項を考えてもせいぜい$2$桁程度しか制度が良くならないわけです。ところが世の中には「$n$回目の計算で、それまでの桁数の倍の桁数だけ精度が良くなる」という、圧倒的に強い公式があります。Gauss--Legendreの公式といいますので、興味のある人は調べてみてください。ちなみに、プリントをアップロードしているGitHubのページ \url{https://github.com/HideakiHosaka/2015_linear_algebra} に、Excel で円周率の近似計算の実験をした結果を載せています。

\section{双曲線函数と逆双曲線函数}

\paragraph{双曲線函数}

\paragraph{問6}
(a) 点$(u_0,v_0)$を双曲線$u^2-v^2=1$上の、$u_0,v_0>0$を満たす点とする。このとき$\sinh\colon\mathbb{R}\rightarrow\mathbb{R}$は全単射だから、$v_0=\sinh t_0$となる$t_0$がただ一つ存在する。このとき$u_0^2=1+v_0^2=\cosh^2 t_0$となるので、$u_0>0$と合わせて$u_0=\cosh t_0$を得る。そして$u$軸、双曲線と直線$\mathrm{OP}$とで囲まれる部分の面積は
\begin{align*}
\int_0^{u_0} \frac{v_0}{u_0}u du - \int_1^{u_0} \sqrt{u^2-1} du
&= \frac{u_0v_0}{2} - \int_0^{t_0} \sqrt{\cosh^2 t -1} \sinh t \,dt 
= \frac{u_0v_0}{2} - \int_0^{t_0} \sinh^2 t \,dt  \\
&= \frac{u_0v_0}{2} - \int_0^{t_0} \frac{\cosh 2t - 1}{2} \,dt
= \frac{u_0v_0}{2} + \frac{t_0}{2} - \frac{\sinh 2t_0}{4} \\
&= \frac{\cosh t_0 \sinh t_0}{2} + \frac{t_0}{2} - \frac{\sinh t_0 \cosh t_0}{2} = \frac{t_0}{2}
\end{align*}
となる。

(b)
\begin{align*}
\cosh x \cosh y + \sinh x \sinh y
&= \frac{e^x + e^{-x}}{2} \frac{e^y + e^{-y}}{2} + \frac{e^x - e^{-x}}{2}\frac{e^y - e^{-y}}{2} \\
&= \frac{e^{x+y} + e^{x-y} + e^{-x+y} + e^{-(x+y)}}{4} + \frac{e^{x+y} - e^{x-y} - e^{-x+y} + e^{-(x+y)}}{4} \\
&= \cosh(x+y) \\
\sinh x \cosh y + \cosh x \sinh y
&= \frac{e^x - e^{-x}}{2} \frac{e^y + e^{-y}}{2} + \frac{e^x + e^{-x}}{2}\frac{e^y - e^{-y}}{2} \\
&= \frac{e^{x+y} + e^{x-y} - e^{-x+y} - e^{-(x+y)}}{4} + \frac{e^{x+y} + e^{x-y} - e^{-x+y} - e^{-(x+y)}}{4} \\
&= \sinh(x+y)
\end{align*}

\noindent (c) 
$y=\cosh x = (e^x+e^{-x})/2$とおくと、$(e^x)^2 + 1 = 2ye^x$である。よって$e^x$の$2$次方程式$(e^x)^2 - 2y e^x+1=0$を解いて
$e^x = y\pm\sqrt{y^2-1}$を得る。よって$\arccosh y = x = \log(y\pm\sqrt{y^2-1})$である。$\pm$の符号は上の枝と下の枝

同様に$y=\sinh x = (e^x - e^{-x})/2$を$e^x$について解くと$e^x=y\pm\sqrt{y^2+1}$を得る。ただし$e^x>0$なので$-$の符号は不適である。よって$x=\arcsinh x=\log(y+\sqrt{y^2+1})$が得られる。

\noindent (d)
\begin{align*}
(\cosh x)' &= \sinh x & 
(\arccosh x)' &= \frac{1+\frac{2x}{2\sqrt{x^2-1}}}{x+\sqrt{x^2-1}} = \frac{1}{\sqrt{x^2-1}} \\
(\sinh x)' &= \cosh x &
(\arcsinh x)' &= \frac{1+\frac{2x}{2\sqrt{x^2+1}}}{x+\sqrt{x^2+1}} = \frac{1}{\sqrt{x^2+1}}
\end{align*}
