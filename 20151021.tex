\chapter{直交行列の性質}
\lectureinfo{2015年10月21日 1限}

$n$次正方行列$P$が直交行列であるとは、${}^t\!P = P^{-1}$を満たすことをいうのでした。今回はその性質を詳しく調べてみましょう。

\section{「任意」の扱い}

今回の答案を採点してみたところ、「任意」というキーワードの扱いに不慣れな人が非常に多いようでした。最初に少し確認しておきましょう。

具体的に問題になったのは「正方行列$P$が、任意の$\bm{u}, \bm{v} \in \mathbb{R}^n$に対して$P\bm{u} \cdot P\bm{v} = \bm{u} \cdot \bm{v}$を満たすなら、$P$は直交行列である」という命題です。後で解答を書きますが、ちょっと計算すれば「${}^t\bm{u} ({}^tPP - I) \bm{v} = 0$が成り立つこと」が示せます。ですが、ここから${}^tP P = I$を導くには、\uline{全ての}$\bm{u}, \bm{v}$に対して今の式が成り立つことが本質的に重要です。

たとえば
\[
\bm{u} = \bm{v} =
\begin{pmatrix}
1 \\
0
\end{pmatrix}, \quad
P = 
\begin{pmatrix}
1 & 0 \\
0 & 2
\end{pmatrix}
\]
としてみます。このとき
\[
{}^t \bm{u} ({}^tPP - I) \bm{v}
= 
\begin{pmatrix}
1 & 0
\end{pmatrix}
\begin{pmatrix}
0 & 0 \\
0 & 3
\end{pmatrix}
\begin{pmatrix}
1 \\
0
\end{pmatrix}
= 0
\]
となっています。ですが${}^tPP \neq I$ですね。このように「上手いこと$\bm{u}, \bm{v} \in \mathbb{R}^n$を取って来れば${}^t\bm{u}({}^tP P - I)\bm{v} = 0$になることもあるが、必ずしもすべての$\bm{u}, \bm{v} \in \mathbb{R}^n$に対し${}^t\bm{u}({}^tP P - I)\bm{v} = 0$を満たすわけではない」という条件を満たす直交行列でない$P$は、他にもたくさん存在します。ですが「どんな$\bm{u}, \bm{v} \in \mathbb{R}^n$に対しても${}^t\bm{u}({}^tP P - I)\bm{v} = 0$が成り立つ」という条件が満たされれば、必ず$P$が直交行列になることが言えるのです。ここの違いを理解した上で、正しい論証ができるようになってください。

\section{$2$次直交群}

\subsection{$2$次直交群の決定} \index{ちょっこうぐん@直交群}

最初に、直交行列に対する感覚をつかむために具体例を見てみます。もっとも易しい$2$次の場合に、直交行列\index{ちょっこうぎょうれつ@直交行列}がどのような行列なのかを具体的に調べてみます。$O(2)$で$2$次直交行列全体の集合を表すことにし
\[
A = 
\begin{pmatrix}
a & b \\
c & d
\end{pmatrix}
\in O(2)
\]
とします。このとき$A$が満たす条件を調べてみましょう。

${}^tA A = I$なので
\[
\begin{pmatrix}
1 & 0 \\
0 & 1
\end{pmatrix}
=
\begin{pmatrix}
a & c \\
b & d
\end{pmatrix}
\begin{pmatrix}
a & b \\
c & d
\end{pmatrix}
= 
\begin{pmatrix}
a^2 + c^2 & ab + cd \\
ab + cd & b^2 + d^2
\end{pmatrix}
\]
です。よって$a = \cos \theta$, $c = \sin \theta$となる$\theta \in [0, 2\pi)$がただ$1$つ存在します。すると$b \cos \theta + d \sin \theta = 0$となるので、$b = k \sin \theta$, $d = -k \cos \theta$となる$k \in \mathbb{R}$が取れます。この式を$b^2 + d^2 = 1$に代入すると$1 = k^2 \sin^2 \theta + k^2 \cos^2 \theta = k^2$となり、$k = \pm1$が従います。これより、複合同順で
\[
A =
\begin{pmatrix}
\cos \theta & \mp \sin \theta \\
\sin \theta & \pm \cos \theta
\end{pmatrix}
\]
と書けることが分かりました。$2$次の直交行列は、必ずこのような形をしています。

逆に
\[
R(\theta) :=
\begin{pmatrix}
\cos \theta & - \sin \theta \\
\sin \theta & \cos \theta
\end{pmatrix}, \quad
R(\theta)
\begin{pmatrix}
1 & 0 \\
0 & -1
\end{pmatrix}
=
\begin{pmatrix}
\cos \theta & \sin \theta \\
\sin \theta & -\cos \theta
\end{pmatrix}
\]
はいずれも直交行列です。これは転置行列と自分との積を計算すればすぐに分かります。したがって
\[
O(2) =
\biggl\{
\begin{pmatrix}
\cos \theta & - \sin \theta \\
\sin \theta & \cos \theta
\end{pmatrix}
\mid
0 \leq \theta < 2\pi
\biggr\}
\cup
\biggl\{
\begin{pmatrix}
\cos \theta & \sin \theta \\
\sin \theta & -\cos \theta
\end{pmatrix}
\mid
0 \leq \theta < 2\pi
\biggr\}
\]
と分かりました。

なお前者は言うまでもなく、$2$次の回転行列です。また後者は、回転行列と折り返しを合成したものです。回転行列は$\det = 1$で、折り返しを合成した方は$\det = -1$なので、これら$2$つのタイプの直交行列に共通部分はありません。

\paragraph{問8の解答}
$2$次直交行列は、必ず
\[
\begin{pmatrix}
\cos \theta & - \sin \theta \\
\sin \theta & \cos \theta
\end{pmatrix},
\begin{pmatrix}
\cos \theta & \sin \theta \\
\sin \theta & -\cos \theta
\end{pmatrix} \quad (0 \leq \theta < 2\pi)
\]
のいずれかの形に表せる。\qed

\subsection{特殊直交群と直交群}

ここで、$O(2)$の中で行列式が$1$のものだけを集めてできる集合$SO(2) := \{ P \in O(2) \mid \det P = 1\}$に注目してみましょう。すぐ上で見たように、全ての$2$次直交行列は回転と折り返しの組合せで表されます。そして$\det = 1$になるのは回転行列だけなので
\[
SO(2) :=
\Biggl\{
\begin{pmatrix}
\cos \theta & - \sin \theta \\
\sin \theta & \cos \theta
\end{pmatrix}
\in O(2) \mid
0 \leq \theta < 2\pi
\Biggr\}
\]
です。したがって$SO(2)$は回転行列全体の集合と言うこともできます。$SO(2)$は後で出てくる群の公理を満たしており、\textbf{特殊直交群}\index{とくしゅちょっこうぐん@特殊直交群}あるいは\textbf{回転群}\index{かいてんぐん@回転群}と呼ばれます。一般に$n$次の場合でも、$\det = 1$となる直交行列全体の全体$SO(n)$は、$n$次元空間における回転操作に対応することが知られています。

\section{直交行列の基本的性質}

前回、$P$が直交行列であることは$P$の列ベクトルたちが正規直交基底をなすことと同値だと確認しました。これを踏まえた上で、問題に取り組みましょう。

\subsection{行列式の値}

正方行列の行列式には「符号付き体積」という意味がありました。もし$P$が直交行列なら、$P$の列ベクトルたちは正規直交基底なので、$n$次元空間における$1$辺の長さが$1$の「立方体」にあたるものを張っているはずです。そう考えれば$\det P = \pm1$となることはほとんど明らかでしょう。計算によっても、次のように示せます。

\paragraph{問7の解答} $\det {}^tP = \det P$, ${}^t P P = I$と$\det$の乗法性より、
\[
1 = \det I = \det{}^tP P = \det{}^t P \det P = (\det P)^2
\]
となる。これより$\det P = \pm1$である。\qed

\subsection{直交群の群構造}
$n$次直交行列全体の集合を
\[
O(n) := \{P \in \Mat_n(\mathbb{R}) \mid {}^t P = P^{-1}\}
\]
と書きます。このとき$O(n)$は
\begin{itemize}
\item 行列の積で閉じている
\item 任意の$P, Q, R \in O(n)$に対し、結合法則$(PQ)R = P(QR)$を満たす
\item 単位行列$I$を元に持つ
\item 任意の$P \in O(n)$について、$O(n)$の中に逆元$P^{-1}$が存在する
\end{itemize}
という性質を持っています。このような性質をもって、$O(n)$は\textbf{群}であるというのでした。$2$番目と$3$番目の条件は明らかなので、残り$2$つの条件をチェックしましょう。

\paragraph{問2の解答} $P$, $Q$を共に直交行列とする。

\noindent (1) ${}^t(PQ) = {}^tQ {}^tP = Q^{-1} P^{-1} = (PQ)^{-1}$なので、$PQ$も直交行列である。

\noindent (2) ${}^t(P^{-1}) = {}^t({}^tP) = P = (P^{-1})^{-1}$なので、$P^{-1}$も直交行列である。 \qed

\subsection{内積・長さと直交行列}

続いて、直交行列と内積の関係を調べてみます。直交行列は${}^t P = P^{-1}$という式で定義されましたが、「内積を保つ」という性質によっても特徴づけられます。このことを確認するため、まずは標準内積の性質を調べましょう。

\paragraph{問3の解答} $\mathbb{R}^n$における標準内積が、行列の積を用いて$\bm{u} \cdot \bm{v} = {}^t\bm{u} \bm{v}$と書けることに留意する。

\noindent (1) $\bm{u} \cdot \bm{v} = {}^t\bm{u} \bm{v} = {}^t({}^t\bm{u} \bm{v}) = {}^t\bm{v} \bm{u} = \bm{v} \cdot \bm{u}$である。

\noindent (2) $\bm{u} \cdot (\bm{v} + \bm{w}) = {}^t\bm{u}(\bm{v} + \bm{w}) = {}^t\bm{u} \bm{v} + {}^t\bm{u} \bm{w} = \bm{u} \cdot \bm{v} + \bm{u} \cdot \bm{w}$である。

\noindent (3) $(k\bm{u}) \cdot \bm{v} = {}^t(k\bm{u}) \bm{v} = k{}^t\bm{u} \bm{v} = k(\bm{u} \cdot \bm{v})$である。同様に$\bm{u} \cdot (k\bm{v}) = k(\bm{u} \cdot \bm{v})$も従う。

\noindent (4) $P$が$n$次正方行列のとき、$(P\bm{u}) \cdot \bm{v} = {}^t(P\bm{u}) \bm{v} = {}^t\bm{u} {}^tP \bm{v} = \bm{u} \cdot ({}^tP \bm{v})$である。 \qed

\paragraph{直交行列の特徴づけ}

続いて、直交行列の特徴付けをしましょう。一般に (直交とは限らない) $n$次正方行列$A$について
\begin{itemize}
\item 任意の$\bm{u}, \bm{v} \in \mathbb{R}^n$に対し$(A\bm{u}) \cdot (A\bm{v}) = \bm{u} \cdot \bm{v}$が成り立つとき、$A$は\textbf{標準内積を保つ}といいます。
\item 任意の$\bm{u} \in \mathbb{R}^n$に対し$|A\bm{u}| = |\bm{u}|$が成り立つとき、$A$は\textbf{長さを保つ}といいます。
\end{itemize}
そしてこれから示すのは、直交行列であること、標準内積を保つこと、長さを保つことが全て同値だということです。

まず、標準内積と直交行列の関係を調べておきましょう。$n$次正方行列$P$が、任意のベクトル$\bm{u}, \bm{v} \in \mathbb{R}^n$に対し$P\bm{u} \cdot P\bm{v} = \bm{u} \cdot \bm{v}$を満たすとします。このとき特に$\bm{u} = \bm{e}_i$, $\bm{v} = \bm{e}_j$とおくと
\[
\delta_{ij} = \bm{e}_i \cdot \bm{e}_j = P\bm{e}_i \cdot P\bm{e}_j = {}^t(P\bm{e}_i) P\bm{e}_j = {}^t\bm{e}_i {}^tP P \bm{e}_j = ({}^tP P)_{ij}
\]
となります。$1 \leq i, j\leq n$を動かすことで、${}^tPP = I$が得られます。よって標準内積を保つ行列は直交行列です。逆に$P$が直交行列なら、任意の$\bm{u}, \bm{v} \in \mathbb{R}^n$に対し$P\bm{u} \cdot P\bm{v} = {}^t(P\bm{u}) P\bm{v} = {}^t\bm{u} {}^tP P\bm{v} = {}^t\bm{u} \bm{v} = \bm{u} \cdot \bm{v}$となります。よって$P$は標準内積を保ちます。かくして$P$が内積を保つことと、$P$が直交行列であることは同値だと分かりました。

これを踏まえた上で、今度は「$P$が長さを保つこと」が直交行列の条件と同値であることを示します。

\paragraph{問4の解答} $P$を直交行列とする、$\bm{u} \in \mathbb{R}^n$とする。このとき$P$は内積を保つので$|P\bm{u}|^2 = P\bm{u} \cdot P\bm{u} = \bm{u} \cdot \bm{u} = |\bm{u}|^2$となる。これとベクトルの長さが非負であることから、$|P\bm{u}| = |\bm{u}|$が従う。 \qed

\paragraph{問5の解答}

$n$次正方行列$P$が、任意のベクトル$\bm{u} \in \mathbb{R}^n$に対し$|P\bm{u}| = |\bm{u}|$を満たしたとする。このとき、任意に$\bm{v}, \bm{w} \in \mathbb{R}^n$を取ると、$|P \bm{v}| = |\bm{v}|$, $|P \bm{w}| = |\bm{w}|$, $|P\bm{v} + P\bm{w}| = |P (\bm{v} + \bm{w})| = |\bm{v} + \bm{w}|$である。よって
\[
P\bm{v} \cdot P \bm{w}
= \frac{|P\bm{v} + P\bm{w}|^2 - |P\bm{v}|^2 - |P\bm{w}|^2}{2}
= \frac{|\bm{v} + \bm{w}|^2 - |\bm{v}|^2 - |\bm{w}|^2}{2}
= \bm{v} \cdot \bm{w}
\]
となる。$P$は標準内積を保つから、直交行列である。 \qed 

\subsection{固有値と固有ベクトル}

今回の問6には「行列の固有値」というものが登場します。次回以降で固有値を扱うので、それに先立ち、少し一般論を話しておきましょう。

$A \in \Mat_n(\mathbb{R})$を$n$次正方行列とします。このときベクトル$\bm{u} \in \mathbb{R}^n$が$A$の\textbf{固有値\index{こゆうち@固有値}$\lambda$に属する固有ベクトル\index{こゆうべくとる@固有ベクトル}}であるとは、$\bm{u} \neq \bm{0}$かつ$A\bm{u} = \lambda \bm{u}$を満たすことをいいます。普通、適当なベクトルに適当な行列をかけたら、向きは元と変わってしまいます。ですが$A$の固有ベクトルというのは、$A$をかけても向きが変わらないという性質を持ちます。ちょっぴり、他のベクトルよりも特別な感じがしますよね。

詳しい話は次回以降に送りますが、実は固有値は「行列を特徴づける本質的なデータ」と言うことができます。S2タームで「基底の取り換えに伴う行列表示の変換」の話をしたのを覚えていますか?$A$を$A\colon \mathbb{R}^n \rightarrow \mathbb{R}^n$という写像だと思います。このとき$\mathbb{R}^n$の基底を行列$P$で取り換えると、$A$は$P^{-1}AP$に化けるのでした。ですが基底を取り換えて$A$の見かけが$P^{-1}AP$に変わっても、実は固有値は元と同じままなのです。「固有値は、線型空間の座標の取り方に依らず決まっている」という意味で、線型写像にとって本質的な量なのです。

直交行列の場合、実数の固有値は$\pm1$しかありません。そのことはすぐにチェックできます。

\paragraph{問6の解答}

$P$を$n$次直交行列、$\bm{u} \in \mathbb{R}^n$, $\bm{u} \neq 0$, $P\bm{u} = \lambda \bm{u}$とする。このとき$|\bm{u}| = |P \bm{u}| = |\lambda \bm{u}| = |\lambda||\bm{u}|$なので、$|\bm{u}| \neq 0$より$|\lambda| = 1$となる。よって$\lambda = \pm1$である。 \qed

\paragraph{固有値の値の範囲}

さて、$2$次の直交行列の例として、回転行列
\[
R(\theta) 
=
\begin{pmatrix}
\cos \theta & - \sin \theta \\
\sin \theta & \cos \theta
\end{pmatrix}
\]
を考えてみましょう。これの固有ベクトルは、$\theta \neq 0$である限り、ないですよね。$\bm{0}$でないどんなベクトルも回転行列で回っちゃうんだから、$R(\theta)$をかけて向きが変わらないベクトルなんて存在しません。さっき示したのは「固有値が存在すれば$\pm1$のどっちか」という話であって、固有ベクトルの存在を主張してはいません。こんなことがあるので、気を付けてください。

ちなみに実数の範囲では固有ベクトルが存在しませんが、複素数にまで範囲を広げると固有ベクトルが出てきます。これは「実数係数の多項式であって、解が複素数のものが存在する」という事実に起因するものです。中々悩ましい問題ですよね。こういう話も、追々していきます。

\subsection{対称行列へ直交群の作用}

最後に、対称行列と直交群の関係を調べておきます。まず問題を解いてから、この問題の意味するところを探っていきましょう。

\paragraph{問1の解答}

$A$が実対称行列、$P$が直交行列なら、${}^t({}^tP AP) = {}^tP {}^tA {}^t({}^tP) = {}^tP AP$である。よって$P^{-1}AP = {}^tP AP$も実対称行列である。\qed

\paragraph{内積の一般化}

さて$\mathbb{R}^n$における標準内積は$\bm{u} \cdot \bm{v} := {}^t\bm{u} \bm{v}$で定義されていました。この間に正方行列$A = (a_{ij}) \in \Mat_n(\mathbb{R})$を挟んで$(\bm{u}, \bm{v}) := {}^t\bm{u} A \bm{v}$と定めると、内積の一般化ができます。実際、今回の問3で示したような
\begin{itemize}
\item $(\bm{u}, \bm{v} + \bm{w}) = (\bm{u}, \bm{v}) + (\bm{u}, \bm{w})$, $(\bm{u} + \bm{v}, \bm{w}) = (\bm{u}, \bm{v}) + (\bm{v}, \bm{w})$
\item $(k\bm{u}, \bm{v}) = k(\bm{u}, \bm{v}) = (\bm{u}, k\bm{v})$
\end{itemize}
という性質 (\textbf{双線型性}\index{そうせんけいせい@双線型性}) は、今の$(\bm{u}, \bm{v}) := {}^t\bm{u} A \bm{v}$でも成り立っています。そして$A = I$の場合が標準内積だから、$(\bm{u}, \bm{v}) = {}^t\bm{u} A \bm{v}$は標準内積の一般化と呼ばれています。


ただし標準内積の時に成り立っていたような対称性$(\bm{u}, \bm{v}) = (\bm{v}, \bm{u})$が成り立つためには、$A$に条件が必要です。$\mathbb{R}^n$の標準基底を持ってきて$\bm{u} = \bm{e}_i$, $\bm{v} = \bm{e}_j$としてみると、$(\bm{e}_i, \bm{e}_j) = {}^t\bm{e}_i A \bm{e}_j = a_{ij}$となります。したがって一般化された内積$(, )$が対称性を満たすなら、全ての$1 \leq i, j \leq n$に対し$(\bm{e}_i, \bm{e}_j) = (\bm{e}_j, \bm{e}_i)$、つまり$a_{ij} = a_{ji}$が成り立たないといけません。すなわち$A$は対称行列である必要があります。逆に$A$が対称行列なら、
\[
(\bm{v}, \bm{u}) = {}^t\bm{v} A \bm{u} = {}^t\bm{v} {}^t\!A \bm{u}
= {}^t({}^t\bm{v} {}^t\!A \bm{u}) = {}^t\bm{u} A \bm{v} = (\bm{u}, \bm{v})
\]
となるので、内積の対称性が成り立ちます。かくして一般化された内積が対称性を持つことと、行列$A$が対称行列であることは同値だと分かりました。

\paragraph{座標変換との関係}

ここで$P = (\bm{p}_1 \ \cdots \ \bm{p}_n)$を直交行列とします。すると
\[
(\bm{e}_1, \ldots, \bm{e}_n) P = (\bm{p}_1, \ldots, \bm{p}_n)
\]
なので、$P$によって基底$(\bm{e}_1, \ldots, \bm{e}_n)$は$(\bm{p}_1, \ldots, \bm{p}_n)$へ変換されます。このとき座標の方は、$P^{-1} = {}^tP$によって変換されるのでした。座標変換の前後で、一般化された内積がどう変化するかを見てみましょう。

$\bm{u}, \bm{v} \in \mathbb{R}^n$としましょう。これらのベクトルは、標準基底$(\bm{e}_1, \ldots, \bm{e}_n)$に基づいて座標表示されています。それを新しい基底$(\bm{p}_1, \ldots, \bm{p}_n)$で見ると、座標はそれぞれ$P^{-1} \bm{u}$, $P^{-1} \bm{v}$に変化します。この$P^{-1}\bm{u}$と$P^{-1}\bm{v}$から元の$(\bm{u}, \bm{v}) = {}^t\bm{u} A \bm{v}$の値を得るには、どうすればよいでしょうか?

ここで、次の式を見てみましょう。
\[
{}^t\bm{u} A \bm{v} = {}^t\bm{u}P (P^{-1} A P) P^{-1}\bm{v} = {}^t(P^{-1} \bm{u}) (P^{-1} A P) (P^{-1} \bm{v})
\]
$P$が直交行列なので、${}^t(P^{-1}\bm{u}) = {}^t({}^tP \bm{u}) = {}^t\bm{u} P$という式が成り立ちます。最後の変形ではこれを使いました。記号を見やすくするため新たに$(\bm{a}, \bm{b})' := {}^t\bm{a} (P^{-1} A P) \bm{b}$と定めると、結局
\[
(\bm{u}, \bm{v}) = (P^{-1}\bm{u}, P^{-1}\bm{v})'
\]
という式が得られたことになります。対称行列$A$によって定まる一般化された内積は、直交行列$P$で座標を変換すると、$P^{-1} A P$という行列によって表されるのです。さっきの問1は、座標変換後の$(,)'$がきちんと対称な内積になっているかどうかを、計算で確かめるものでした。

\paragraph{2次曲線}

一般化された内積$(\bm{u}, \bm{v}) = {}^t\bm{u} A \bm{v}$において$\bm{u} = \bm{v}$とすると、$2$次形式\index{にじけいしき@$2$次形式}という式が得られます\footnote{遠い昔、S1タームの$5$回目、\pageref{paragraph:example_of_inner_product}ページに出てきました。}。一番簡単な$2$次元の場合に、$2$次形式がどういうものなのか、またどういう場面で登場するのかを少しだけ紹介してみます。

一般化された内積の両方に同じベクトルを突っ込んだもの$(\bm{u}, \bm{u}) = {}^t\bm{u} A \bm{u}$は$2$次形式と呼ばれます。$2$次元で
\[
A = 
\begin{pmatrix}
a & b \\
b & c
\end{pmatrix}, \quad
\bm{u} = 
\begin{pmatrix}
x \\
y
\end{pmatrix}
\]
の場合に計算してみると
\[
{}^t\bm{u} A \bm{u} =
\begin{pmatrix}
x & y
\end{pmatrix}
\begin{pmatrix}
a & b \\
b & c
\end{pmatrix}
\begin{pmatrix}
x \\
y
\end{pmatrix}
= a x^2 + 2bxy + c y^2
\]
です。要は$2$次式ですね。本当に一般の場合は$x, y$について$1$次の項が登場しますが、平方完成をすればそれらの項は消去できます。なので$2$次形式のことを考える場合は、大体${}^t\bm{u} A \bm{u}$という形の式だけ考えていれば十分です。

そして$2$次形式が現れる典型的な場面といえば、$2$次曲線です。$2$次曲線\index{にじきょくせん@$2$次曲線}は「$2$次形式が一定」という形の式で表されます。たとえば$a, c \neq 0$かつ$b = 0$のとき
\[
a x^2 + c y^2 = 1
\]
という式は、$a$と$c$が同符号なら楕円、異符号なら双曲線を表すのでした。$b \neq 0$のときも、$a x^2 + 2b xy + c y^2 = d$という形の式は、大抵は楕円・双曲線・放物線のどれかを表すことが知られています。でも$b \neq 0$のときは、たとえば軸の向きだとか、軸と曲線の交わる位置などの情報は、ぼーっと式を見ているだけでも何も分かりません。

そんなときに登場するのが、さっきの$P^{-1} A P$という式です。目の前によく分からない$2$次曲線$a x^2 + 2b xy + cy^2 = d$が現れたとき、上手く基底を回して軸の向きに合わせれば、式の見た目が綺麗になる気がしますよね。これを行列の言葉で言えば「$P^{-1} A P$が見やすい形になるような直交行列$P$を探せ」という問題に化けます。そして$2$次形式は$b = 0$なら見やすいのだから、結局僕たちが解くべき問題は「$P^{-1} A P$が対角行列になるような直交行列$P$を見つけろ」という問題になるのです。

今ではこの問題をシステマティックに解くやり方が知られていて、実はそれがさっき紹介した「固有値と固有ベクトル」の計算と同じなのです。実対称な行列$A$の場合、上手く直交行列$P$を持って来ればいつでも$P^{-1} A P$が対角行列にできることが知られています。そして$P$を作るには、$A$の固有ベクトルを全て求めて、長さが$1$になるよう調節してから$1$列に並べます。

こんな感じで、行列を上手く扱うことで、僕たちは平面上の$2$次曲線を完全に扱いきることができるようになります。この「$2$次曲線の扱い」は、$1$年生で学ぶ線型代数の目標でもあります。それに向けて次回以降、固有値の話をしていきましょう。
