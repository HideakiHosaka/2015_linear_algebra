\chapter{2015年4月15日}

\begin{flushright}
担当教員: 植野 義明 / TA: 穂坂 秀昭
\end{flushright}

\section{はじめに}

\paragraph{ごあいさつ}

みなさん、はじめまして。この授業のTA (ティーチング・アシスタント)をすることになりました、数理科学研究科博士課程の穂坂といいます。これから$1$セメスターの間、よろしくお願いします。

この授業では基本的に毎回レポート問題が課され、それをTAの穂坂が添削して返却することになっています。また答案返却に合わせて、この文書のような解説プリントを配付していく予定です。何か疑問要望等があれば、随時質問をしてください。

\paragraph{プリントのオンライン公開}

このプリントは毎回教室で配布するのに加え、インターネット上でも配布されます。そのURLは下記の通りですので、必要な方はアクセスしてください。

\paragraph{このプリントの作り方について}

せっかくなので、このプリントをどう作っているかについて説明しておきます。

ふつう「コンピュータで文書を作る」というと、大抵の人がMicrosoft Wordとか一太郎といったワープロソフトを連想すると思います。ところが残念なことに、市販のワープロソフトでは数式を入力するのに大変な苦労を強いられてしまいます。そこで数学が専門の人はどうするかというと、そういったワープロソフトの代わりに``\LaTeX''というソフトウェアを使います。これはD.~E.~Knuthという非常に有名な数学者・計算機科学者が作った``\TeX''というソフトウェアを、色々な人が改良してできあがったものです。

\LaTeX はワープロソフトとはちょっと違い、文字のスタイルを変えたり見出しをつけたりするのに「コマンド」というものを使います。ですから\LaTeX を使うにはコマンドの使い方を覚えなければいけません。加えてキーボードが打ち込んだものが、見た目通りに出てくるわけでもありません。一旦コマンドも含めて打ち込んだ文書を\LaTeX というプログラムに処理させることによって、やっと整形された文書がでてきます。ですから使い始めるにはちょっとハードルが高いのですが、使い慣れるとワープロソフトよりも手際よく文書が書けるし、また数式を中心として文書の仕上がりが美しいというメリットもあります。

もしかしたら皆さんの中には\LaTeX を既に知っている人がいるかもしれませんし、また将来\LaTeX を使う必要に迫られる人がいるかもしれません。そこでこの文書の\LaTeX ソースコードを
\begin{center}
\url{https://github.com/HideakiHosaka/2015_linear_algebra}
\end{center}
に置いておきます。もし\LaTeX の方に興味がある人は、こちらを見てください。また東京大学が持つ情報処理システムのオンライン自習教材「はいぱーワークブック」の第27章\footnote{\url{http://hwb.ecc.u-tokyo.ac.jp/current/applications/latex/}}に、\LaTeX の説明があります。\LaTeX を使う人は、一度読んでおくと良いと思います。

ちなみにソースコードの公開には``GitHub''というサービス\footnote{もしあなたが既に``GitHub''を知っているなら、きっと``pull request''の機能も知っているはずです。プリントに対して何か意見があれば、積極的にpull requestを送ってください。\textsf{\raisebox{1pt}{:}D}}を利用しています。上に貼ったURLを開くと、古いバージョンのプリントや、そうしたプリントがどう更新されていったかも見ることができます。授業自体の役には立たないと思いますが、興味があれば見てみてください。

\newpage

hoge

