\chapter{3次正方行列の対角化とJordan標準形}
\lectureinfo{2015年11月25日 1限}


\section{計算問題について}

\subsection{特別な形の行列}

特別な行列の場合は、対角化に必要な計算が簡単にできることがあります。

\paragraph{三角行列}

三角行列 (上三角行列または下三角行列) の場合、その行列から単位行列の$t$倍$t I$を引いたものも三角行列です。その行列の$\det$は、対角成分を全てかけたものに一致します。ですから三角行列の固有値は、対角成分と一致します。たとえば問1の(1)は、すぐに固有値が見えます。

\paragraph{ブロック対角行列}

対角線上に正方行列をいくつか並べた形のブロック対角行列の場合、それぞれのブロックごとに対角化の計算をすれば全体の対角化ができます。たとえば問2(1)は、右下の$1 \times 1$ブロックを見れば${}^t(0, 0, 1)$が固有ベクトルになることが分かります。そうすれば対角化の問題は、左上$2 \times 2$ブロックの問題に落とし込めます。

\paragraph{置換行列}

問1の(2)で出てくる行列を$A$と書くと、$A$は置換行列と呼ばれるものになっています。$3$次の置換$\sigma \in \mathfrak{S}_3$を$\sigma := (132)$と表すと、$A$は$\bm{e}_i$ ($i = 1, 2, 3$)を$\bm{e}_{\sigma(i)}$へと移します。したがって$\sigma^3 = I$から$A^3 = I$となることが分かります。これより$A$の固有値$\lambda$に対応する固有ベクトルを$\bm{v}$とすると、$A\bm{v} = \alpha\bm{v}$から$\bm{v} = I\bm{v} = A^3 \bm{v} = \alpha^3 \bm{v}$となります。つまり固有多項式を計算する前から$\alpha$は$1$の三乗根のいずれかだと、最初から当たりが付くのです。

また置換行列に限らず「群の表現」と呼ばれるものとして登場する行列は、群の構造に着目して固有値等の計算ができることがあります。

\paragraph{対称行列}

対称行列の場合は、対角化するための行列$P$として直交行列を取ってくることができます。たとえば問1の(4)で問題になっている行列は対称行列です。これの固有ベクトルを全て長さが$1$になるよう調節しておくと、$P$が直交行列になります。$P$を直交行列にしておくと$P^{-1} = {}^tP$なので、逆行列の計算が楽になります。

\subsection{問題の解答}

\paragraph{問1の解答}

(1) 固有値は$1, 2, 3$である。対応する固有ベクトルと、対角化行列$P$の逆行列は
\[
\bm{v}_1 = 
\begin{pmatrix}
1 \\
0 \\
0
\end{pmatrix}, \quad
\bm{v}_2 = 
\begin{pmatrix}
1 \\
1 \\
0
\end{pmatrix}, \quad
\bm{v}_3 = 
\begin{pmatrix}
1 \\
1 \\
1
\end{pmatrix}, \quad
P^{-1} = 
\begin{pmatrix}
1 & -1 & 0 \\
0 & 1 & -1 \\
0 & 0 & 1
\end{pmatrix}, \quad
P^{-1} AP =
\begin{pmatrix}
1 & 0 & 0 \\
0 & 2 & 0 \\
0 & 0 & 3
\end{pmatrix}
\]

\noindent (2) $\omega := \frac{-1 + \sqrt{3}i}{2}$とすると、固有値は$1, \omega, \omega^2$である。対応する固有ベクトルと、対角化行列$P$の逆行列は
\[
\bm{v}_1 = 
\begin{pmatrix}
1 \\
1 \\
1
\end{pmatrix}, \quad
\bm{v}_{\omega} = 
\begin{pmatrix}
\omega \\
\omega^2 \\
1
\end{pmatrix}, \quad
\bm{v}_{\omega^2} = 
\begin{pmatrix}
\omega^2 \\
\omega \\
1
\end{pmatrix}, \quad
P^{-1} = 
\frac{1}{3}
\begin{pmatrix}
1 & 1 & 1 \\
\omega^2 & \omega & 1 \\
\omega & \omega^2 & 1
\end{pmatrix}, \quad
P^{-1} AP =
\begin{pmatrix}
1 & 0 & 0 \\
0 & \omega & 0 \\
0 & 0 & \omega^2
\end{pmatrix}
\]

\noindent (3) 固有値は$2, 0, -1$である。対応する固有ベクトルと、対角化行列$P$の逆行列は
\begin{align*}
& \bm{v}_2 = 
\frac{1}{\sqrt{6}}
\begin{pmatrix}
2 \\
1 \\
1
\end{pmatrix}, \quad
\bm{v}_0 = 
\frac{1}{\sqrt{2}}
\begin{pmatrix}
0 \\
-1 \\
1
\end{pmatrix}, \quad
\bm{v}_{-1} = 
\frac{1}{\sqrt{3}}
\begin{pmatrix}
-1 \\
1 \\
1
\end{pmatrix}, \\
& P^{-1} = 
\begin{pmatrix}
\frac{2}{\sqrt{6}} & \frac{1}{\sqrt{6}} & \frac{1}{\sqrt{6}} \\
0 & \frac{-1}{\sqrt{2}} & \frac{1}{\sqrt{2}} \\
\frac{-1}{\sqrt{3}} & \frac{1}{\sqrt{3}} & \frac{1}{\sqrt{3}}
\end{pmatrix}, \quad
P^{-1} AP =
\begin{pmatrix}
2 & 0 & 0 \\
0 & 0 & 0 \\
0 & 0 & -1
\end{pmatrix}
\end{align*}

\noindent (4) 固有値は$2, 1, -2$である。対応する固有ベクトルと、対角化行列$P$の逆行列は
\[
\bm{v}_2 = 
\begin{pmatrix}
1 \\
-1 \\
1
\end{pmatrix}, \quad
\bm{v}_1 = 
\begin{pmatrix}
1 \\
1 \\
0
\end{pmatrix}, \quad
\bm{v}_{-2} = 
\begin{pmatrix}
0 \\
-1 \\
1
\end{pmatrix}, \quad
P^{-1} = 
\begin{pmatrix}
1 & -1 & -1 \\
0 & 1 & 1 \\
-1 & 1 & 2
\end{pmatrix}, \quad
P^{-1} AP =
\begin{pmatrix}
2 & 0 & 0 \\
0 & 1 & 0 \\
0 & 0 & -2
\end{pmatrix}
\]

\qed

\paragraph{問2の解答}

(1) 固有値は$0, 1, 1$である。対応する固有ベクトルと、対角化行列$P$の逆行列は
\[
\bm{v}_1 = 
\begin{pmatrix}
1 \\
1 \\
0
\end{pmatrix}, \quad
\bm{v}_0 = 
\begin{pmatrix}
2 \\
1 \\
0
\end{pmatrix}, \quad
\bm{v}'_1 = 
\begin{pmatrix}
0 \\
0 \\
1
\end{pmatrix}, \quad
P^{-1} = 
\begin{pmatrix}
-1 & 2 & 0 \\
1 & -1 & 0 \\
0 & 0 & 1
\end{pmatrix}, \quad
P^{-1} AP =
\begin{pmatrix}
1 & 0 & 0 \\
0 & 0 & 0 \\
0 & 0 & 1
\end{pmatrix}
\]

\noindent (2) 固有値は$2, 2, 1$である。対応する固有ベクトルと、対角化行列$P$の逆行列は
\[
\bm{v}_2 = 
\begin{pmatrix}
1 \\
1 \\
0
\end{pmatrix}, \quad
\bm{v}_1 = 
\begin{pmatrix}
1 \\
2 \\
3
\end{pmatrix}, \quad
\bm{v}'_2 = 
\begin{pmatrix}
0 \\
0 \\
1
\end{pmatrix}, \quad
P^{-1} = 
\begin{pmatrix}
2 & -1 & 0 \\
-1 & 1 & 0 \\
3 & -3 & 1
\end{pmatrix}, \quad
P^{-1} AP =
\begin{pmatrix}
2 & 0 & 0 \\
0 & 1 & 0 \\
0 & 0 & 2
\end{pmatrix}
\]

\noindent (3) 固有値は$14, 0, 0$である。対応する固有ベクトルと、対角化行列$P$の逆行列は
\[
\bm{v}_{14} = 
\begin{pmatrix}
1 \\
2 \\
3
\end{pmatrix}, \quad
\bm{v}_0 = 
\begin{pmatrix}
-3 \\
0 \\
1
\end{pmatrix}, \quad
\bm{v}'_0 = 
\begin{pmatrix}
-2 \\
1 \\
0
\end{pmatrix}, \quad
P^{-1} = 
\frac{1}{14}
\begin{pmatrix}
1 & 2 & 3 \\
-3 & -6 & 5 \\
-2 & 10 & -6
\end{pmatrix}, \quad
P^{-1} AP =
\begin{pmatrix}
14 & 0 & 0 \\
0 & 0 & 0 \\
0 & 0 & 0
\end{pmatrix}
\]
\qed

\paragraph{問3の解答}

 (1) 固有値は$3, 0, 0$である。対応する固有ベクトルと、対角化行列$P$の逆行列は
\[
\bm{v}_3 = 
\begin{pmatrix}
1 \\
1 \\
1
\end{pmatrix}, \quad
\bm{v}_0 = 
\begin{pmatrix}
-1 \\
0 \\
1
\end{pmatrix}, \quad
\bm{v}'_0 = 
\begin{pmatrix}
-1 \\
1 \\
0
\end{pmatrix}, \quad
P^{-1} = 
\frac{1}{3}
\begin{pmatrix}
1 & 1 & 1 \\
-1 & -1 & 2 \\
-1 & 2 & -1
\end{pmatrix}, \quad
P^{-1} AP =
\begin{pmatrix}
3 & 0 & 0 \\
0 & 0 & 0 \\
0 & 0 & 0
\end{pmatrix}
\]

\noindent (2) 固有値は$1 + a^2 + b^2, 0, 0$である。$1 + a^2 + b^2 \neq 0$なら対角化可能で、対応する固有ベクトルと、対角化行列$P$の逆行列は
\begin{align*}
& \bm{v}_{1 + a^2 + b^2} = 
\begin{pmatrix}
1 \\
a \\
b
\end{pmatrix}, \quad
\bm{v}_1 = 
\begin{pmatrix}
-b \\
0 \\
1
\end{pmatrix}, \quad
\bm{v}'_2 = 
\begin{pmatrix}
-a \\
1 \\
0
\end{pmatrix} \\
& P^{-1} = 
\begin{pmatrix}
1 & a & b \\
-b & -ab & 1 + a^2 \\
-a & 1 + b^2 & -ab
\end{pmatrix}, \quad
P^{-1} AP =
\frac{1}{1 + a^2 + b^2}
\begin{pmatrix}
1 & a & b \\
-b & -ab & 1 + a^2 \\
-a & 1 + b^2 & -ab
\end{pmatrix}
\end{align*}
となる。$a, b \in \mathbb{C}$で$1 + a^2 + b^2 = 0$となる場合、対角化ができない。 \qed

\section{$3$次正方行列のJordan標準形}

前回のプリントで、一般の次数におけるJordan標準形の存在証明を行いました。今回はそれを、$3$次正方行列に特殊化した場合を扱います。既に「固有値が全て異なる行列は対角化可能」という事実を知っているので、問題になるのは固有多項式が$(t - \lambda)^2 (t - \mu)$あるいは$(t - \lambda)^3$という形をした行列の扱いです。これらの場合を調べましょう。

\subsection{部分分数分解}

$3$次正方行列$A \in \Mat_n(\mathbb{C})$の固有多項式が$\varphi_A(t) = (t - \lambda)^2 (t - \mu)$で$2$重根を持つとき、僕たちは$1/\varphi_A(t)$の部分分数分解を計算する必要があります。この計算は、原理的なことを言えば
\[
\frac{1}{(t - \lambda)^2 (t - \mu)} = \frac{\alpha t + \beta}{(t - \lambda)^2} + \frac{\gamma}{t - \mu}
\]
とおいた上で、両辺に$(t - \lambda)^2 (t - \mu)$をかけてから$1, t, t^2$の係数を比較し、$3$本の連立方程式を立てれば求まります。が、こうやって計算するのは面倒なので、もうちょっと楽にやってみましょう。

\paragraph{Laurent展開}

今回使うのは、Laurent展開と呼ばれる道具です。既に微積分の授業で、函数$f(z)$の$z = a$におけるTaylor展開
\[
f(z) = f(a) + f'(a) (z - a) + \frac{f''(a)}{2!} (z - a)^2 + \frac{f'''(a)}{3!}(z - a)^3 \cdots
\]
を習っていると思います\footnote{変数を$z$としたのは、複素数の函数を考えているからです。複素函数のTaylor展開はまだ習っていないと思いますが、良く知られた函数が相手なら実数の函数と同じように扱えます。}。これをちょっと拡張したのがLaurent展開です。

たとえば$f(z)$が$1/(z - a)$の場合、$z = a$を$f(z)$に代入することはできません。$0$による割り算になってしまうからです。このように$f(z)$が$z = a$で発散するとき、$a$を函数$f(z)$の\textbf{特異点}といいます。点$z = a$が$f(z)$の特異点であるときは、$f(z)$を$z = a$でTaylor展開することもできません。ですが、$f(z)$の特異点に$\log 0$のような形の式が絡んでこないときは、Taylor展開を負巾の項に伸ばした
\[
f(z) = \frac{a_{-n}}{(z - a)^n} + \frac{a_{-(n - 1)}}{(z - a)^{n - 1}} + \cdots + \frac{a_{-1}}{z - a} + a_0 + a_1 (z - a) + a_2 (z - a)^2 + \cdots
\]
のような形に展開できることが知られています。これを$f(z)$の$z = a$における\textbf{Laurent展開}といいます。またLaurent展開において$z - a$の負巾の項を全て集めた式
\[
\frac{a_{-n}}{(z - a)^n} + \frac{a_{- (n - 1)}}{(z - a)^{n - 1}} + \cdots + \frac{a_{-1}}{z - a}
\]
をLaurent展開の\textbf{主要部}といいます。主要部が$(z - a)^{-n}$から始まるとき、$z = a$は\textbf{$n$次の極}であるといいます。

\paragraph{極限を用いた主要部の決定}

さて、後で僕たちは正方行列$A$の固有多項式$\varphi_A(t)$が$\varphi_A(t) = (t - \lambda)^2 (t - \mu)$の形のときに$1/\varphi_A(t)$の部分分数分解を計算する必要があります。つまり
\[
\frac{1}{(t - \lambda)^2 (t - \mu)} = \frac{\alpha t + \beta}{(t - \lambda)^2} + \frac{\gamma}{t - \mu}
\]
という式を満たすように定数$\alpha, \beta, \gamma$を決定するわけです。この右辺を見ると、どうも左辺$1/\varphi_A(t)$の$t = \lambda$におけるLaurent展開を見れば初項が、$t = \mu$におけるLaurent展開を見れば第$2$項が求まりそうな気がしてきます。そこでLaurent展開の求め方を考えてみましょう。

いま、函数$f(z)$が
\[
f(z) = \frac{a_{-n}}{(z - a)^n} + \frac{a_{-(n - 1)}}{(z - a)^{n - 1}} + \cdots + \frac{a_{-1}}{z - a} + a_0 + a_1 (z - a) + a_2 (z - a)^2 + \cdots
\]
という形にLaurent展開されていたとします。これの$a_{-n}$を求めるには、どうすれば良いでしょうか?その答えは非常に簡単で、$(z - a)^n$をかけてから$z \rightarrow a$の極限を取ればOKです。実際$(z - a)^n$を両辺にかけてしまえば
\[
(z - a)^n f(z) = a_{-n} + a_{-(n - 1)}(z - a) + \cdots
\]
となるので、普通に$z \rightarrow a$の極限を取ってしまえば話が終わります。

次に$a_{-(n - 1)}$を求めるには、どうすればいいでしょうか。それには
\[
f(z) - \frac{a_{-n}}{(z - a)^n} = \frac{a_{-(n - 1)}}{(z - a)^{n - 1}} + \frac{a_{-(n - 2)}}{(z - a)^{n - 2}} + \cdots + \frac{a_{-1}}{z - a} + a_0 + a_1 (z - a) + a_2 (z - a)^2 + \cdots
\]
という式を使います。もう既に$a_{-n}$は分かっているのだから、その項を$f(z)$から引いてしまえば、極が$n$位から$n - 1$位へ変化します。そこでさっきと同様$(z - a)^{n - 1}$をかけて
\[
(z - a)^{n - 1} \biggl(f(z) - \frac{a_{-n}}{(z - a)^n}\biggr) = a_{-(n - 1)} + a_{-(n - 2)} (z - a) + \cdots
\]
としてから$z \rightarrow a$の極限を取れば、$a_{-(n - 1)}$が求まります。

こんな風に極の場所と位数が分かっていれば、$f(z)$のLaurent展開の主要部は極限を使って順番に計算できます。そして、一般の$f(z)$についてLaurent展開を求めるときは極の位数を頑張って調べなければいけないのですが、いま僕たちのターゲットは$1/(t - \lambda)^2 (t - \mu)$です。式の形から$t = \lambda$に$2$位の極、$t = \mu$に$1$位の極があることが見えています。だから今回は、Laurent展開の計算が簡単にできるわけです。

\paragraph{実際の計算}

それでは$1/\varphi_A(t) = 1/(t - \lambda)^2 (t - \mu)$について、極$t = \lambda, \mu$でのLaurent展開を計算してみましょう。

まず$t = \mu$でのLaurent展開を考えます。$(t - \mu)/\varphi_A(t) = 1/(t - \lambda)^2$は$t = \mu$で発散しないので、$1/\varphi_A(t)$は$t = \mu$を$1$位の極に持ちます。そして
\[
\lim_{t \rightarrow \mu} (t - \mu)\frac{1}{\varphi_A(t)} = \frac{1}{(\mu - \lambda)^2}
\]
なので、$1/\varphi_A(t)$を$t = \mu$でLaurent展開すると、主要部は$1/(t - \mu)(\mu - \lambda)^2$となるはずです。

次に、$t = \lambda$でのLaurent展開を計算します。$1/\varphi_A(t)$が$t \rightarrow \lambda$の極限で発散しないためには、$(t - \lambda)^2$をかける必要があります。よって$t = \lambda$は$1/\varphi_A(t)$の$2$位の極です。そこで$1/(t - \lambda)^2$と$1/(t - \lambda)$の係数を順番に決定しましょう。
\[
\lim_{t \rightarrow \lambda} (t - \lambda)^2 \frac{1}{\varphi_A(t)} = \frac{1}{(\lambda - \mu)}
\]
なので、$1/\varphi_A(t)$のLaurent展開の右辺は$1/(t - \lambda)^2(\lambda - \mu)$から始まるはずです。今現れた$1/(t - \lambda)^2$の項を、$1/\varphi_A(t)$から引き去ると
\[
\frac{1}{\varphi_A(t)} - \frac{1}{(t - \lambda)^2 (\lambda - \mu)}
= \frac{1}{(t - \lambda)^2} \Bigl(\frac{1}{t - \mu} - \frac{1}{\lambda - \mu}\Bigr)
= \frac{1}{(t - \lambda)^2}\frac{\lambda - t}{(t - \mu)(\lambda - \mu)}
= -\frac{1}{(t - \lambda)(t - \mu)(\lambda - \mu)}
\]
となります。この式は$t = \lambda$を$1$位の極に持ち、$(t - \lambda)$をかけて$t \rightarrow \lambda$の極限を取ると
\[
\lim_{t \rightarrow \lambda} (t - \lambda) \Bigl( \frac{1}{\varphi_A(t)} - \frac{1}{(t - \lambda)^2 (\lambda - \mu)} \Bigr)
= -\frac{1}{(\lambda - \mu)^2}
\]
となります。よって$\varphi_A(t)$の$t = \lambda$におけるLaurent展開の主要部は
\[
\frac{1}{\varphi_A(t)}
= \frac{1}{(t - \lambda)^2(\lambda - \mu)} - \frac{1}{(t - \lambda)(\lambda - \mu)^2} + \cdots
\]
と決定しました。

これで結局$t = \lambda, \mu$におけるLaurent展開の主要部が分かりました。そこで$1/\varphi_A(t)$から$t = \lambda, \mu$でのLaurent展開の主要部を両方とも差っ引くと
\[
\frac{1}{\varphi_A(t)}
- \frac{1}{(t - \mu)(\mu - \lambda)^2} - \frac{1}{(t - \lambda)^2(\lambda - \mu)} + \frac{1}{(t - \lambda)(\lambda - \mu)^2}
\]
は$t = \lambda, \mu$で発散しないことが分かります。

ここで、よく見ると
\[
\frac{1}{(t - \lambda)^2(t - \mu)}
- \frac{1}{(t - \mu)(\mu - \lambda)^2} - \frac{1}{(t - \lambda)^2(\lambda - \mu)} + \frac{1}{(t - \lambda)(\lambda - \mu)^2} = 0
\]
が成り立っています。実際、両辺に$(\lambda - \mu)^2\varphi_A(t)$をかけて移行すると
\[
(\lambda - \mu)^2 = (t - \lambda)^2 + (\lambda - \mu)(t - \mu) - (t - \lambda)(t - \mu)
\]
という式は正しいことが分かります。僕たちは$1/\varphi_A(t)$の$2$つの極でそれぞれLaurent展開の主要部を調べただけですが、実はそれだけで部分分数分解が得られてしまうのです。これは一般の場合でも正しく、\textbf{多項式$f(z)$の逆数$1/f(z)$の部分分数分解は、全ての極におけるLaurent展開の主要部を足したものに等しい}ことが知られています。ですからLaurent展開の計算を使えば、連立$1$次方程式を解く操作を経由しないで、極限計算のみで部分分数分解の計算ができるのです。知っておくと得するテクニックなので、ぜひ覚えておいてください。

\paragraph{Liouvilleの定理}

なぜLaurent展開をするだけで$1/\varphi_A(t)$の部分分数分解が求まるのか、補足しておきます。

鍵になるのは、Liouvilleの定理と呼ばれる定理です。この定理は「(複素函数の意味で) 微分可能な函数$f(z)$が、全ての$z \in \mathbb{C}$および$z = \infty$で発散しないならば、$f(z)$は定数である」ことを主張します。そしてさっきの部分分数分解の途中で出てきた
\[
\frac{1}{\varphi_A(t)}
- \frac{1}{(t - \mu)(\mu - \lambda)^2} - \frac{1}{(t - \lambda)^2(\lambda - \mu)} + \frac{1}{(t - \lambda)(\lambda - \mu)^2}
\]
は、$t \rightarrow \mu, \lambda, \infty$のいずれでも発散しません。だからLiouvilleの定理から、この式の値は定数$c$であることが分かります。そして$z \rightarrow \infty$で全ての項は$0$に収束するので、$c = 0$だと分かります。

このLiouvilleの定理の証明は、複素函数論と呼ばれる分野の話なので、今は述べません。興味がある人は、たとえばAhlfors『複素解析』(現代数学社) などを読んでください。「Liouvilleの定理があるから、Laurent展開の主要部だけ辻褄を合わせれば良い」といったたぐいの議論は楕円関数論など色々な場面で出てきます。

\subsection{スペクトル分解}

引き続き、$3$次正方行列$A \in \Mat_3(\mathbb{C})$が$\varphi_A(t) = (t - \lambda)^2 (t - \mu)$ ($\lambda \neq \mu$) と書けていたとします。

さっきの部分分数分解の式で、両辺に$\varphi_A(t)$をかけてから$(t - \lambda)$の絡む項をまとめ
\[
1 = - \frac{(t - 2\lambda + \mu)(t - \mu)}{(\lambda - \mu)^2} + \frac{(t - \lambda)^2}{(\lambda - \mu)^2}
\]
と変形しておきます。これを$1 = e_1(t) + e_2(t)$と書きましょう。このとき
\[
e_1(t) e_2(t) = -\frac{(t - \lambda)^2 (t - \mu) (t - 2\lambda + \mu)}{(\lambda - \mu)^4} = -\varphi_A(t)\frac{(t - 2\lambda + \mu)}{(\lambda - \mu)^4}
\]
です。よって$\varphi_A(A) = O$より$e_1(A)e_2(A) = O$となります。これを用いると、$e_1(A) + e_2(A) = I$の両辺に$e_1(A)$をかけると$e_1(A)^2 = e_1(A)$が、$e_2(A)$をかけると$e_2(A)^2 = e_2(A)$が得られます。

さらに$A e_1(A)$と$A e_2(A)$を考えてみましょう。
\[
\bigl(A e_1(A) - \lambda e_1(A)\bigr)^2 = (A - \lambda I)^2 e_1(A)^2 = (A - \lambda I)^2 e_1(A)
\]
は、多項式$(t - \lambda)^2 e_1(t)$に$A$を代入したものです。ところが$(t - \lambda)^2 e_1(t) = - \varphi_A(t) (t - 2\lambda + \mu)/(\lambda - \mu)^2$なので、これに$A$を代入したら$O$です。よって$\bigl(A e_1(A) - \lambda e_1(A)\bigr)^2 = O$です。同様にして$A e_2(A) - \mu e_2(A) = O$が従います。

% \Im e_i(A) が A によって保たれること
% e_2(A) \neq O なので \dim e_2(A) = 1
% 固有多項式でなく最小多項式を使えば、すぐに e_2(A) \neq O が分かること

\paragraph{固有値$\mu$の固有空間}

今までの式から、任意の$\bm{u} \in \mathbb{C}^3$に対し$(A - \mu I) e_2(A) \bm{u} = \bigl(A e_2(A) - \mu e_2(A)\bigr) e_2(A) \bm{u} = \bm{0}$となります。よって$\Im e_2(A)$は、$A$の固有値$\mu$に属する固有空間に含まれると分かります。$e_2(A) \neq O$より$\Im e_2(A) \neq \{\bm{0}\}$となることを合わせると、$\Im e_2(A) = \{ \bm{u} \in \mathbb{C}^3 \mid A\bm{u} = \mu \bm{u}\}$と分かります。

\paragraph{固有値$\lambda$の一般固有空間}

いま$\mathbb{C}^3 = \Im e_1(A) \oplus \Im e_2(A)$で、$\Im e_2(A)$は$1$次元なのでした。よって$\Im e_1(A)$は$2$次元です。そこで何でもいいので、$\Im e_1(A)$の基底$\bm{u}_{\lambda, 1}, \bm{u}_{\lambda, 2}$と$\Im e_2(A)$の基底$\bm{u}_{\mu}$を取り、$P := (\bm{u}_{\lambda, 1}\ \bm{u}_{\lambda, 2}\ \bm{u}_{\mu})$とおきます。。そうすると$A$は$\Im e_1(A)$, $\Im e_2(A)$の両方を保つので、
\[
P^{-1} A P = 
\begin{pmatrix}
* & * & 0 \\
* & * & 0 \\
0 & 0 & \mu
\end{pmatrix}
\]
という格好になります。これで元の$A$のJordan標準形を求める問題を、左上の$2$次正方行列のJordan標準形を求める問題にまで帰着できました。そして$2$次正方行列のJordan標準形は、先週既に見た通りです。よって
\[
\begin{pmatrix}
\lambda & 0 & 0 \\
0 & \lambda & 0 \\
0 & 0 & \mu
\end{pmatrix}, \quad
\begin{pmatrix}
\lambda & 1 & 0 \\
0 & \lambda & 0 \\
0 & 0 & \mu
\end{pmatrix}
\]
のいずれかに変形できます。これで固有多項式が$2$重根を持つ場合のJordan標準形が分かりました。

\subsection{巾零行列の標準形}

最後に、固有多項式が$3$重根を持つ場合を考えます。$3$次正方行列$A \in \Mat_3(\mathbb{C})$の固有多項式が$\varphi_A(t) = (t - \lambda)^3$と書けていたとします。このときCayley--Hamiltonの定理から$O = \varphi_A(A) = (A - \lambda I)^3$なので、$A - \lambda I$は巾零行列です。そこで$N := A - \lambda I$とおき、$N$のJordan標準形を求めましょう。

$N = O$なら既に話は終わっているので、$N \neq O$とします。$N^3 = O$も分かっています。よって$N^2 \neq O$か$N^2 = O$かによって、話が違ってきます。もし$N^2 \neq O$なら、$\Im N^2 \neq \{\bm{0}\}$です。よって$\bm{0}$でない$\bm{w} \in \Im N^2$および$\bm{w} = N^2 \bm{v}$となる$\bm{v} \in \mathbb{C}^3$が取れるはずです。このとき$\bm{v}, N\bm{v}, N^2 \bm{v}$は$1$次独立です。実際$\alpha \bm{v} + \beta N\bm{v} + \gamma N^2 \bm{v} = \bm{0}$とおくと、両辺に$N^2$をかけたら$\alpha N^2\bm{v} = \bm{0}$となります。よって$N^2 \bm{v} \neq \bm{0}$より$\alpha = 0$が分かります。これより$\beta N\bm{v} + \gamma N^2 \bm{v} = \bm{0}$となるので、両辺に$N$をかけて$\beta N^2 \bm{v} = \bm{0}$が得られます。再び$N^2 \bm{v} \neq \bm{0}$なので、$\beta = 0$です。最後に$\gamma N^2\bm{v} = \bm{0}$が残るので、$\gamma = 0$です。

そこで$P:= (\bm{v} \  N\bm{v} \ N^2\bm{v})$とおくと、$P$は正則で
\[
NP = 
\begin{pmatrix}
N\bm{v} & N^2\bm{v} & N^3 \bm{v}
\end{pmatrix}
=
\begin{pmatrix}
N\bm{v} & N^2\bm{v} & \bm{0}
\end{pmatrix}
=
\begin{pmatrix}
\bm{v} & N\bm{v} & N^2 \bm{v}
\end{pmatrix}
\begin{pmatrix}
0 & 1 & 0 \\
0 & 0 & 1 \\
0 & 0 & 0
\end{pmatrix}
\]
となります。よって
\[
P^{-1} NP =
\begin{pmatrix}
0 & 1 & 0 \\
0 & 0 & 1 \\
0 & 0 & 0
\end{pmatrix}
\]
が分かりました。これがJordan標準形です。

残るのは$N^2 = O$の場合です。この場合$N \neq O$と$N^2 = O$から、$\Ker N \supset \Im N \neq \{\bm{0}\}$が分かります。よって$\dim \Ker N \geq \dim \Im N \geq 1$です。かたや次元定理から$\dim \Ker N + \dim \Im N = 3$も従います。これらの次元の条件から$\dim \Im N = 1, \dim \Ker N = 2$でないといけないことが得られます。次元の情報を手掛かりに、良い基底を求めましょう。

$\Im N \neq \{\bm{0}\}$なので、$N \bm{u} \neq \bm{0}$なるベクトル$\bm{u} \in \mathbb{C}^3$が存在します。このとき$\bm{u}$と$N\bm{u}$は$1$次独立です。実際$\alpha \bm{u} + \beta N\bm{u} = \bm{0}$とおくと、両辺に$N$をかけて$\alpha N \bm{u} = \bm{0}$が得られます。これと$N\bm{u} \neq \bm{0}$より$\alpha = 0$です。このような基底の取り方は、$2$次巾零行列のJordan標準形を求めるときにもやったことです。$N\bm{u}$はちょうど$N$の固有値$0$に属する固有ベクトルになってますから、$N$の見やすい形を求めるにあたり、基底に$\bm{u}$と$N\bm{u}$が入ってくるのは自然なことでしょう。

そして$\dim \Ker N = 2$なので、$N\bm{u}$と合わせたときに$1$次独立になるベクトル$\bm{v}$で、$\Ker N$に入るものが存在します。そこで基底として$(\bm{u}, N\bm{u}, \bm{v})$を取りましょう。そうすると$P: = (\bm{u}\ N\bm{u}\ \bm{v})$とおけば
\[
NP =
\begin{pmatrix}
N\bm{u} & N^2\bm{u} & N\bm{v}
\end{pmatrix}
=
\begin{pmatrix}
N\bm{u} & \bm{0} & \bm{0}
\end{pmatrix}
=
\begin{pmatrix}
\bm{u} & N\bm{u} & \bm{v}
\end{pmatrix}
\begin{pmatrix}
0 & 1 & 0 \\
0 & 0 & 0 \\
0 & 0 & 0
\end{pmatrix}
\]
より
\[
P^{-1} N P =
\begin{pmatrix}
0 & 1 & 0 \\
0 & 0 & 0 \\
0 & 0 & 0
\end{pmatrix}
\]
となります。これが$N$のJordan標準形です。

\subsection{まとめ}

以上をまとめて、$3$次正方行列$A$のJordan標準形の分類が完了します。以下、$f(A) = O$となる多項式の中で最も次数が低く、かつ最高次の係数が$1$であるものを$\psi_A(t)$と書きます。これを$A$の\textbf{最小多項式}といいます。

まず固有多項式$\varphi_A(t)$が異なる$3$つの根$\lambda, \mu, \nu$を持つとき、その行列は対角化可能で、Jordan標準形は
\[
\begin{pmatrix}
\lambda & 0 & 0 \\
0 & \mu & 0 \\
0 & 0 & \nu
\end{pmatrix}
\]
になります。次に固有多項式$\varphi_A(t)$が$2$重根$\lambda$と別の根$\mu$を持つとき、Jordan標準形は
\[
\begin{pmatrix}
\lambda & 0 & 0 \\
0 & \lambda & 0 \\
0 & 0 & \mu
\end{pmatrix}, \quad
\begin{pmatrix}
\lambda & 1 & 0 \\
0 & \lambda & 0 \\
0 & 0 & \mu
\end{pmatrix}
\]
のいずれかになります。最小多項式$\psi_A(t)$が$(t - \lambda)(t - \mu)$のときに前者、$(t - \lambda)^2 (t - \mu)$のとき後者が出てきます。そして固有多項式$\varphi_A(t)$が$3$重根$\lambda$を持つとき、Jordan標準形は
\[
\begin{pmatrix}
\lambda & 0 & 0 \\
0 & \lambda & 0 \\
0 & 0 & \lambda
\end{pmatrix}, \quad
\begin{pmatrix}
\lambda & 1 & 0 \\
0 & \lambda & 0 \\
0 & 0 & \lambda
\end{pmatrix}, \quad
\begin{pmatrix}
\lambda & 1 & 0 \\
0 & \lambda & 1 \\
0 & 0 & \lambda
\end{pmatrix}
\]
のいずれかです。それぞれ、最小多項式$\psi_A(t)$が$t - \lambda, (t - \lambda)^2, (t - \lambda)^3$の場合に対応します。

